\chapter*{Agradecimentos}

Quero agradecer à Deus que me iluminou com as pessoas que colocou no meu caminho durante esta jornada e por toda minha vida.

Agradeço à minha esposa Izabel por todo apoio, carinho, companheirismo e amizade.

À minha mãe, pai e irmão e minha família que sempre foram os pilares na minha vida e realizações.

Aos meus orientadores, Igor e Marzulo, pelo empenho dedicado à elaboração deste trabalho e por confiarem em mim nesta empreitada.
E a todos os professores por todo o conhecimento proporcionado para minha formação e que contribuiu para este trabalho.

% ----------------------------------------------------------
% RESUMO
% ----------------------------------------------------------

\chapter*{Resumo}

% \refbibliografica
% Emprego de múltiplas estratégias de vizinhança com uso de GPU para solução de problemas de otimização
% BEGIN IGOR
Problemas de otimização são de grande importância para diversos setores da indústria, desde o planejamento de produção até escoamento e transporte dos produtos.
Diversos problemas de interesse se enquadram na classe NP-Difícil, sendo desconhecidos algoritmos eficientes para resolvê-los de forma exata em tempo polinomial.
Assim, estratégias heurísticas com capacidade de escapar de ótimos locais de baixa qualidade (meta-heurísticas) são geralmente empregadas. % FIM IGOR
A busca local é, em geral, a etapa mais custosa, em termos de tempo computacional, do processo de uma meta-heurística, desta forma torna-se muito importante fazer bom uso dos recursos nela utilizados.
Esta dissertação estuda o emprego de múltiplas estratégias de vizinhança utilizadas paralelamente para explorar um espaço de vizinhança maior e melhor aproveitar os recursos computacionais.
O processamento paralelo das estratégias de vizinhança é implementado em nível de grão fino, através de processamento em GPU, e grão grosso, por meio de processamento multi core e processamento em rede, sendo os dois níveis combinados num ambiente heterogêneo, % IGOR BEGIN
para arquiteturas von Neumann e dataflow. % IGOR END

\vspace{1em}
\noindent {Palavras-chave}: Meta-heurística, Busca Local, Dataflow, Graphics Processing Unit, Variable Neighborhood Descent.

% ----------------------------------------------------------
% Abstract
% ----------------------------------------------------------

\chapter*{Abstract}

% Melhorar ainda
Optimization problems have big importance in the industry field, from production management to production outflow and product transportation.
Many problems of interest are classified as NP-Hard, so there is no known algorithm to find the exact solution in a polinomial time.
Therefore heuristic strategies with the ability to escape from poor quality local optima (meta-heuristics) are generally employed.
In general, the local search is the most costly, in computational time, phase of a meta-heuristic, becoming mandatory a good use of the available resources.
The parallel processing of neighborhood strategies is implemented at the fine grain level through GPU processing and coarse grain through multi-core processing and network processing, the comibation of the two level parallelization in a heterogeneous environment for von Neumann architectures and dataflow.

\vspace{1em}
\noindent {Keywords}: Meta-heuristics, Local Search, Dataflow, Graphics Processing Unit, Variable Neighborhood Descent.