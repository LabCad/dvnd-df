\section{Dataflow DVND}

Foi apresentado em~\cite{df-dvnd2018} uma nova abordagem para resolver problemas de otimização, o \textit{desenho de algoritmos projetado para rodar paralelamente}(\textit{parallel thinking design}) pode prover novos algoritmos para resolver problemas otimização e expor algumas limitações para paralelizar os algoritmos tradicionais.
Esta técnica foi aplicara para um problema que busca minimizar a latência de entregas numa rede, a saber, o Problema da Mínima Latência, sendo uma variante do clássico Problema do Caixeiro Viajante que é NP-difícil.

Os experimentos computacionais foram realizados em uma implementação de um VND Distribuído como busca local, usando a biblioteca Sucuri (em Python).
Os resultados do DVND foram comparados com o resultado pela implementação em dataflow do RVND que apresenta hoje o estado da arte para esse problema, obtendo melhor qualidade no valor da solução e menores tempos para as maiores instâncias (mais de 500 cidades, a partir da instância 4).
É esperado que o a composição de movimentos implementada ao DVND (que não se aplicaria ao RVND) possa evitar uso desnecessário de recursos computacionais.
% Dúvida de como dizer essa próxima frase
Até onde se pode investigar, a implementação apresentada em~\cite{df-dvnd2018} é a primeira implementação de uma busca local em dataflow para resolver um problema de otimização.
