\section{LCRVND}

Conforme apresentado em~\cite{2017:Araujo} e publicado em~\cite{endm2018:araujo}, foi proposto um GRASP combinado com uma busca local insipirada no VND com uma lista de restrição de melhore soluções a serem expandidas para resolver o PMV, contudo esta pode ser aplicada a outros problemas de otimização.

Os testes foram realizados com 60 instâncias\footnote{Disponíveis em \url{http://cs.adelaide.edu.au/~optlog/research/combinatorial.php}.
Todas as características das instâncias foram descritas em~\cite{Polyakovskiy:2014}} tendo sido propostas por Bonyadi et al.~\cite{Bonyadi:2013} os resultados foram comparados com resultados recentes publicados obtidos por uma Busca Tabu~\cite{Oliveira:2015}.

Como extensão desse trabalho pode ser feita uma melhoria na eficiência do processo de busca e uma investigação mais aprofundada das vizinhanças usadas na busca local, de maneira a minimizar a enumeração de soluções redundantes.

Podem ser realizados testes comparando instâncias com maior número de cidades e itens, para esse problema foram propostas instâncias com 85.900 cidades e 429.495 itens, contudo para estas soluções será necessária uma abordagem diferenciada em termos da estrutura de dados utilizada devido ao enorme número de soluções vizinhas para cada iteração do LCRVND.
% %Despite of the initial state of this study, 
% The proposed method shows potential to reach results even better and may be used in other problems due to not being related to the TTP.