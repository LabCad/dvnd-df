\chapter{Resultados} \label{cap:resultados}

Este capítulo exibe os resultados computacionais dos algoritmos propostos no Capítulo~\ref{cap:metodologia} para o caso do PML, para cada instância foi gerado um conjunto com 100 soluções iniciais aleatórias que foram submetidas aos métodos para comparação dos resultados.

Quando há referência à melhoria na solução (\textit{Imp}), esta melhoria pode ser calculada pelo quociente do valor da solução inicial pela solução final, ou seja:
\begin{equation}\label{eq:calculateImprovement}
Imp = \frac{f(\textrm{solução inicial})}{f(\textrm{solução final})}
\end{equation}

Desta forma quanto maior for o valor da melhoria ($Imp$) mais o método melhorou o valor da solução inicial.

\section{Instâncias} \label{sec:instancias}

Todas as instâncias usadas nos testes computacionais e cujas configurações de lançamento foram descritas na Tabela~\ref{tab:neighborhoodsLaunchConfigurarion} são as mesmas usadas em~\cite{wamca2016}.
Para o RVND foi feita uma implementação do algoritmo clássico (Algoritmo~\ref{alg:rvnd}) e também a implementação dataflow mencionada na Figura~\ref{fig:rvndGraph} fazendo uso de uma máquina.
Para o caso do DVND foi utilizada a implementação clássica (Algoritmo~\ref{alg:dvnd}) e a implementação dataflow proposta (Figura~\ref{fig:dvndGraph}), os resultados foram obtidos com diferentes números de máquinas e os mesmos são indicados conforme o caso.

\section{Implementação e ambiente computacional}\label{sec:amb}

A implementação para cada algoritmo proposto no Capítulo~\ref{cap:metodologia} utiliza a linguagem de programação \textit{C++11} em conjunto com a API CUDA\texttrademark, para a implementação dos grafos e do ambiente dataflow foi utilizada a biblioteca Sucuri~\cite{sucuri-original}\footnote{Disponível em \url{https://github.com/tiagoaoa/Sucuri}} implementada em Python, para a integração entre o dataflow e o código CUDA foi utilizada a biblioteca SimplePyCuda~\cite{simple-pycuda}\footnote{Disponível em \url{https://github.com/igormcoelho/simple-pycuda}}. As implementações com múltiplas threads usaram a biblioteca OpenMP.
%As implementações foram compiladas através do \textit{GCC} \textit{(GNU Compiler Collection)}\footnote{O GCC está disponível no seguinte sítio eletrônico: \url{https://gcc.gnu.org/}.} com a \textit{flag} de otimização $-O3$.
O ambiente computacional utilizado em todos os testes neste trabalho consiste de 4 máquinas com a seguinte configuração:

\begin{itemize}
    \item Processador Intel\textregistered Core\texttrademark i7-4820K 3.7 GHz (4 núcleos);
    \item 8 GB de memória RAM;
    \item Sistema Operacional Ubuntu 14.10 (x64);
    \item NVIDIA GeForce GTX 780 com 2304 CUDA cores.
\end{itemize}

\newcommand{\figureDvndOrRvndDcDd}[7]{
% #1 {box, scatter}, #2 {count, imp, time}, #3 {instance number}, #4 {Tempo, Melhoria}, #5 tamanho instância, #6 {DVND, RVND}, #7 {dvnd, rvnd}
\begin{figure}%
    \centering
    \includegraphics[scale=0.9]{figuras/#7/dc_dd/#1/#7_#1100sol_#2_in#3.png}
    \caption{#4 do #6 para a instância #3 de tamanho #5. $m$ indica o número de máquinas, \textit{DC} refere-se ao #6 clássico e \textit{DD} ao #6 implementado em dataflow.}%
    \label{fig:#2_#7DcDd_in#3}%
\end{figure}
}

\newcommand{\figureDvndDcDd}[5]{
% #1 {box, scatter}, #2 {count, imp, time}, #3 {instance number}, #4 {Tempo, Melhoria}, #5 tamanho instância
    \figureDvndOrRvndDcDd{#1}{#2}{#3}{#4}{#5}{DVND}{dvnd}
}

\newcommand{\figureRvndDcDd}[5]{
% #1 {box, scatter}, #2 {count, imp, time}, #3 {instance number}, #4 {Tempo, Melhoria}, #5 tamanho instância
    \figureDvndOrRvndDcDd{#1}{#2}{#3}{#4}{#5}{RVND}{rvnd}
}

\newcommand{\tabelaEstatisticasGeral}[6]{
% #1 Descrição, #2 label, #3 {dvnd, rvnd}, #4 {DVND, RVND}, #6 DD/DC, #6 Conteúdo
\begin{table}[ht]
    \centering
    \begin{tabular}{c|ccc|cc|ccc|cc|c}
        \hline \hline
        \# & Tipo & $m$ & $n$ & $min$ & $max$ & 1Q & 2Q & 3Q & $\overline{x}$ & $\sigma$ & $p-valor$ \\ \hline
        #6
    \end{tabular}
    \caption{#1 #4 #5
        Instância (\#), tipo de implementação (Tipo), número de máquinas ($m$), tamanho da instância ($n$), valor mínimo ($min$), máximo ($max$), primeiro quartil (1Q), mediana (2Q), terceiro quartil (3Q), média ($\overline{x}$), desvio padrão ($\sigma$) e p-valor para o teste de Wilcox entre as versões (valores em negrito quando $p-valor > 0.05$).
    }
    \label{tab:#3DcDd#2}
\end{table}
}

\newcommand{\tabelaEstatisticas}[5]{
    \tabelaEstatisticasGeral{#1}{#2}{#3}{#4}{na implementação clássica (DC) e a proposta de implementação usando dataflow (DD).}{#5}
}

\newcommand{\figureDvndSogMog}[7]{
% #1 {box, scatter}, #2 {count, imp, time}, #3 {instance number}, #4 {Tempo, Melhoria}, #5 tamanho instância, #6 {DVND, RVND}, #7 {dvnd, rvnd}
\begin{figure}%
    \centering
    \includegraphics[scale=0.9]{figuras/#7/sog_mog/#1/#7_#1100sol_#2_in#3.png}
    \caption{#4 do #6 para a instância #3 de tamanho #5. \textit{SOG} refere-se a uma porta de saída e \textit{MOG} a múltiplas portas de saída.}%
    \label{fig:#2_#7SogMog_in#3}%
\end{figure}
}

\newcommand{\figureDvndGdvnd}[9]{
% #1 {box, scatter}, #2 {count, imp, time}, #3 {instance number}, #4 {Tempo, Melhoria}, #5 tamanho instância, #6 {DVND, RVND}, #7 {dvnd, rvnd}, #8 {man_time, full_time}, #9 {man, dvnd} #10 descricao
\begin{figure}%
    \centering
    \includegraphics[scale=0.9]{figuras/#7/#8/#1/#9_#1100sol_#2_in#3.png}
    \caption{#4 do #6 para a instância #3 de tamanho #5. \textit{DVND} refere-se ao tempo gasto pelo algoritmo de mesmo nome, para \textit{GDVND} é análogo ao anterior, no caso do \textit{GDVND-MAN} este se refere ao tempo do GDVND subtraido do tempo para gerenciar os movimentos.}%
    \label{fig:#2_#7_#8_in#3}%
\end{figure}
}

\newcommand{\figureDvndGdvndTime}[8]{
    \figureDvndGdvnd{#1}{#2}{#3}{#4}{#5}{#6}{#7}{man_time}{man}
}

\newcommand{\figureGdvndDvndRvnd}[9]{
% #1 {box, scatter}, #2 {count, imp, time}, #3 {instance number}, #4 {Tempo, Melhoria}, #5 tamanho instância, #6 {DVND, RVND}, #7 {dvnd, rvnd}, #8 {man_time, full_time}, #9 {man, dvnd} #10 descricao
\begin{figure}%
    \centering
    \includegraphics[scale=0.9]{figuras/#7/#8/#1/#9_#1100sol_#2_in#3.png}
    \caption{#4 do #6 para a instância #3 de tamanho #5. \textit{DVND}, \textit{GDVND} e \textit{RVND} referem-se ao tempo gasto pelos algoritmos de mesmo nome.}%
    \label{fig:#2_#7_#8_in#3}%
\end{figure}
}

% \subfloat[$m=#1$]{{ %scale=0.225
%         \includegraphics[scale=0.425]{figuras/dvnd/n#1/time#2.png}
%         \label{fig:timeDvndRvnd_n#1in#2}
%     }}%
% #1 {dvnd, rvnd, gdvnd}, #2 {sog_mog, dc_dd}, #3 {time, imp}, #4 in, #5 tamanho, #6 {box, scatter}
\newcommand{\subFig}[6]{
    \subfloat[][Instância #4, $n=#5$]{
        \includegraphics[width=.5\linewidth]{figuras/#1/#2/#6/#1_#6100sol_#3_in#4.png}
		\label{fig:#1_#2_#3_in#4}
    }
% 	\begin{subfigure}{0.45\textwidth} % dvnd_box100sol_imp_in0
% 		\includegraphics[scale=0.425]{figuras/#1/#2/#6/#1_#6100sol_#3_in#4.png}
% 		\caption{Instância #4, $n=#5$}
        % \label{fig:#1_#2_#3_in#4}
    % \end{subfigure}
}

\newcommand{\subFigBox}[5]{
	\subFig{#1}{#2}{#3}{#4}{#5}{box}
}

\newcommand{\subFigScatter}[5]{
	\subFig{#1}{#2}{#3}{#4}{#5}{scatter}
}

% #1 {dvnd, rvnd, gdvnd}, #2 {sog_mog, dc_dd}, #3 {time, imp}, #4 {box, scatter}, #5 {Tempo do DVND...}
\newcommand{\multiFigureInstanciasGeral}[5]{
	\begin{figure}[ht]
		\centering
		\subFig{#1}{#2}{#3}{0}{52}{#4}
		~
		\subFig{#1}{#2}{#3}{1}{100}{#4}
		
		\subFig{#1}{#2}{#3}{2}{226}{#4}
		~
		\subFig{#1}{#2}{#3}{3}{318}{#4}

        \includegraphics[width=.9\linewidth]{figuras/#1/#2/#4/#1_#4100sol_#3_legenda.png}
		\caption{#5 Instâncias 0 a 3.}
		\label{fig:#1_#2_#3_in0_4}
	\end{figure}
	
	\begin{figure}[ht]
		\centering
		\subFig{#1}{#2}{#3}{4}{501}{#4}
		~
		\subFig{#1}{#2}{#3}{5}{657}{#4}
		
		\subFig{#1}{#2}{#3}{6}{783}{#4}
		~
		\subFig{#1}{#2}{#3}{7}{1001}{#4}

		\includegraphics[width=.9\linewidth]{figuras/#1/#2/#4/#1_#4100sol_#3_legenda.png}
		\caption{#5 Instâncias 4 a 7.}
		\label{fig:#1_#2_#3_in5_7}
	\end{figure}
}

% #1 {dvnd, rvnd, gdvnd}, #2 {sog_mog, dc_dd}, #3 {time, imp}, #4 {Tempo do DVND...}
\newcommand{\multiFigureInstancias}[4]{
    \multiFigureInstanciasGeral{#1}{#2}{#3}{box}{#4}
}


\section{Múltiplas portas de saída} \label{sec:resultadosMultiplasPortas}

% \subsection{Tempo}

% A Tabela~\ref{tab:rvndDcDdImp} e as Figuras~\ref{fig:imp_rvndDcDd_in0}-\ref{fig:imp_rvndDcDd_in7} apresentam resultados para execuções utilizando apenas uma máquina ($m = 1$) pois pela construção naturalmente sequencial do RVND não utilizar paralelismo, logo o emprego de mais de uma máquina não traria ganhos em termos de desempenho tampouco no valor da solução.

Para validar a utilização de múltiplas portas de saída em cada nó do grafo dataflow, conforme discutido na Seção~\ref{subsec:multiplasSaidas}, foi realizado um experimento utilizando o dataflow DVND com apenas uma porta de saída (SOG) e comparado com outro experimento com o mesmo algoritmo DVND mas agora implementado utilizando múltiplas portas de saída para os nós (MOG).

O resultado para os tempos no experimento pode ser visto na Tabela~\ref{tab:sogMogDcDdTime}, ao aplicar o teste de Wilcox pode se perceber que exceto para a instância 2 (de tamanho 226) em todas as outras situações houve significância estatística para verificar a diferença entre as amostras.

\tabelaEstatisticasGeral{Tempos comparativos do}{Time}{sogMog}{SOG vs MOG}{onde SOG indica a execução com uma porta de saída e MOG com múltiplas portas de saída.}{
    \multirow{1}{*}{0} & MOG & 4 & \multirow{1}{*}{52} & 2,43 & 4,028 & 2,698 & 2,811 & 2,928 & 2,953 & 0,429 & 0,0271 \\
     & SOG & 4 &  & 2,329 & 3,786 & 2,68 & 2,761 & 2,87 & 2,802 & 0,257 &  \\ \hline
    \multirow{1}{*}{1} & MOG & 4 & \multirow{1}{*}{100} & 1,841 & 4,156 & 2,831 & 2,963 & 3,808 & 3,207 & 0,527 & 0,003571 \\
     & SOG & 4 &  & 1,858 & 3,982 & 2,781 & 2,876 & 3,018 & 2,911 & 0,305 &  \\ \hline
    \multirow{1}{*}{2} & MOG & 4 & \multirow{1}{*}{226} & 2,504 & 4,545 & 3,079 & 3,171 & 3,29 & 3,243 & 0,355 & \textbf{0,1708} \\
     & SOG & 4 &  & 2,643 & 4,481 & 3,021 & 3,263 & 3,405 & 3,242 & 0,299 &  \\ \hline
    \multirow{1}{*}{3} & MOG & 4 & \multirow{1}{*}{318} & 2,674 & 4,592 & 3,515 & 3,622 & 3,756 & 3,648 & 0,324 & 5,161e-06 \\
     & SOG & 4 &  & 3,175 & 4,414 & 3,646 & 3,83 & 4,015 & 3,817 & 0,269 &  \\ \hline
    \multirow{1}{*}{4} & MOG & 4 & \multirow{1}{*}{501} & 3,423 & 5,801 & 4,345 & 4,514 & 4,691 & 4,496 & 0,38 & 4,901e-08 \\
     & SOG & 4 &  & 3,781 & 6,382 & 4,553 & 4,783 & 4,974 & 4,783 & 0,38 &  \\ \hline
    \multirow{1}{*}{5} & MOG & 4 & \multirow{1}{*}{657} & 4,17 & 6,604 & 5,246 & 5,464 & 5,682 & 5,49 & 0,441 & 6,173e-07 \\
     & SOG & 4 &  & 4,566 & 7,11 & 5,559 & 5,775 & 6,007 & 5,778 & 0,421 &  \\ \hline
    \multirow{1}{*}{6} & MOG & 4 & \multirow{1}{*}{783} & 5,434 & 8,442 & 6,305 & 6,508 & 6,691 & 6,514 & 0,507 & 1,429e-16 \\
     & SOG & 4 &  & 5,95 & 8,881 & 6,782 & 7,04 & 7,356 & 7,087 & 0,471 &  \\ \hline
    \multirow{1}{*}{7} & MOG & 4 & \multirow{1}{*}{1001} & 7,871 & 10,88 & 8,836 & 9,09 & 9,493 & 9,164 & 0,566 & 7,009e-10 \\
     & SOG & 4 &  & 8,253 & 10,96 & 9,352 & 9,626 & 10,05 & 9,693 & 0,565 &  \\ \hline
    \hline
}

\multiFigureInstancias{dvnd}{sog_mog}{time}{Tempo do DVND, \textit{SOG} refere-se a uma porta de saída e \textit{MOG} a múltiplas portas de saída, $n$ indica o tamanho, $m$ indica o número de máquinas utilizadas.}

% \figureDvndSogMog{box}{time}{1}{Tempos}{100}{DVND}{dvnd}

Nas Figuras~\ref{fig:dvnd_sog_mog_time_in0} e \ref{fig:dvnd_sog_mog_time_in1} podemos ver uma sutil diferença favorecendo a versão sem múltiplas portas de saída (SOG), contudo a situação se inverte na instância 3.

% \figureDvndSogMog{box}{time}{2}{Tempos}{226}{DVND}{dvnd}

% \figureDvndSogMog{box}{time}{3}{Tempos}{318}{DVND}{dvnd}

Na Figura~\ref{fig:dvnd_sog_mog_time_in3} referente à instância 3 (de tamanho 318) se mostra o início da melhoria no tempo pelo uso de múltiplas saídas nos nós do grafo dataflow.

% \figureDvndSogMog{box}{time}{4}{Tempos}{501}{DVND}{dvnd}

O uso de múltiplas portas de saída, uma específica para cada nó de destino se mostra eficiente nas demais instâncias, podendo ser visto nas Figuras~\ref{fig:dvnd_sog_mog_time_in4}-\ref{fig:dvnd_sog_mog_time_in7}.
A melhoria no tempo inicia-se na instância 3 de tamanho 318 e permanece por todas as instâncias seguintes.

% \figureDvndSogMog{box}{time}{5}{Tempos}{657}{DVND}{dvnd}

% \figureDvndSogMog{box}{time}{6}{Tempos}{783}{DVND}{dvnd}

% \figureDvndSogMog{box}{time}{7}{Tempos}{1001}{DVND}{dvnd}

\section{RVND} \label{sec:resultadosRVND}

Como veremos nas seções seguintes, apesar da implementação do RVND em dataflow não conseguir melhorar os tempos da implementação clássica do RVND o DVND clássico e DVND dataflow conseguem melhorar os tempos relativo ao tempo do RVND para as maiores instâncias.

\subsection{Tempo}

A Tabela~\ref{tab:rvndDcDdTime} e as Figuras~\ref{fig:rvnd_dc_dd_time_in0}-\ref{fig:rvnd_dc_dd_time_in7} apresentam resultados para execuções utilizando apenas uma máquina ($m = 1$) pois pela construção naturalmente sequencial do RVND não utilizar paralelismo, logo o emprego de mais de uma máquina não traria ganhos em termos de desempenho tampouco no valor da solução.

\tabelaEstatisticas{Tempos comparativos do}{Time}{rvnd}{RVND}{
    \multirow{2}{*}{0} & \multirow{1}{*}{DD} & 1 & \multirow{2}{*}{52} & 0,4269 & 1,77 & 1,258 & 1,416 & 1,531 & 1,321 & 0,323 & 9,279e-13 \\
     & DC & 1 &  & 0,262 & 1,342 & 1,155 & 1,178 & 1,2 & 1,169 & 0,1 &  \\ \hline
    \multirow{2}{*}{1} & \multirow{1}{*}{DD} & 1 & \multirow{2}{*}{100} & 0,5236 & 2,773 & 0,8762 & 1,025 & 1,309 & 1,177 & 0,457 & 2,48e-07 \\
     & DC & 1 &  & 0,3116 & 1,771 & 1,377 & 1,473 & 1,523 & 1,378 & 0,297 &  \\ \hline
    \multirow{2}{*}{2} & \multirow{1}{*}{DD} & 1 & \multirow{2}{*}{226} & 1,483 & 9,029 & 2,194 & 2,556 & 3,308 & 3,103 & 1,49 & 0,02545 \\
     & DC & 1 &  & 1,413 & 6,555 & 2,565 & 2,989 & 3,616 & 3,106 & 0,926 &  \\ \hline
    \multirow{2}{*}{3} & \multirow{1}{*}{DD} & 1 & \multirow{2}{*}{318} & 1,983 & 7,007 & 3,133 & 3,541 & 4,125 & 3,824 & 1,07 & 2,559e-17 \\
     & DC & 1 &  & 1,931 & 3,949 & 2,445 & 2,67 & 3,038 & 2,75 & 0,439 &  \\ \hline
    \multirow{2}{*}{4} & \multirow{1}{*}{DD} & 1 & \multirow{2}{*}{501} & 3,614 & 13,5 & 5,374 & 6,104 & 7,05 & 6,597 & 1,98 & 4,07e-13 \\
     & DC & 1 &  & 3,63 & 7,018 & 4,697 & 4,987 & 5,52 & 5,095 & 0,59 &  \\ \hline
    \multirow{2}{*}{5} & \multirow{1}{*}{DD} & 1 & \multirow{2}{*}{657} & 6,878 & 22,9 & 9,351 & 10,12 & 11,74 & 11,43 & 3,82 & 1,202e-18 \\
     & DC & 1 &  & 6,325 & 11,35 & 7,855 & 8,356 & 8,821 & 8,345 & 0,752 &  \\ \hline
    \multirow{2}{*}{6} & \multirow{1}{*}{DD} & 1 & \multirow{2}{*}{783} & 9,997 & 35,13 & 13,26 & 14,9 & 17,5 & 16,91 & 5,93 & 2,107e-15 \\
     & DC & 1 &  & 10,38 & 15,38 & 11,7 & 12,49 & 13,04 & 12,5 & 1,05 &  \\ \hline
    \multirow{2}{*}{7} & \multirow{1}{*}{DD} & 1 & \multirow{2}{*}{1001} & 15,51 & 66,77 & 21,48 & 24,74 & 29,11 & 27,5 & 9,6 & \textbf{0,1547} \\
     & DC & 1 &  & 19,8 & 30,42 & 22,71 & 24,31 & 25,73 & 24,27 & 2,13 &  \\ \hline
}

Pela Tabela~\ref{tab:rvndDcDdTime} e as Figuras~\ref{fig:rvnd_dc_dd_time_in0}, \ref{fig:rvnd_dc_dd_time_in3}-\ref{fig:rvnd_dc_dd_time_in6} podemos ver que o RVND em sua implementação clássica (RC) apresentou melhores tempos que o RVND na versão dataflow (RD).

% \figureRvndDcDd{box}{time}{0}{Tempos}{52}

% Podemos ver na Figura~\ref{fig:time_dvndDcDd_in0} que o DVND clássico possui tempos bem menores que o DVND em dataflow e o uso de mais máquinas não consegue melhorar os tempos do procedimento.

% \figureRvndDcDd{box}{time}{1}{Tempos}{100}

% \figureRvndDcDd{box}{time}{2}{Tempos}{226}

% \figureRvndDcDd{box}{time}{3}{Tempos}{318}

% \figureRvndDcDd{box}{time}{4}{Tempos}{501}

% \figureRvndDcDd{box}{time}{5}{Tempos}{657}

% \figureRvndDcDd{box}{time}{6}{Tempos}{783}

\multiFigureInstancias{rvnd}{dc_dd}{time}{Tempos do RVND, $n$ representa o tamanho da instância, $m$ indica o número de máquinas, \textit{RC} refere-se ao RVND clássico e \textit{RD} ao RVND implementado em dataflow.}

Apenas no caso da instância 7, de tamanho 1001, representada pela Figura~\ref{fig:rvnd_dc_dd_time_in7}, não houve diferença significativa, segundo o teste de Wilcox, para afirmar a existência de diferença nos dados.

% \figureRvndDcDd{box}{time}{7}{Tempos}{1001}

\subsection{Melhoria no valor da solução}

Em termos de melhoria no valor da solução (dada pela Equação~\ref{eq:calculateImprovement}), podemos ver Tabela~\ref{tab:rvndDcDdImp} que não há grandes diferenças em termos de média ($\overline{x}$) nem de mediana ($2Q$) o que se comprova nos resultados do teste de Wilcox que não apresenta diferença significativa senão nos resultados da instância 4 de tamanho 501.

\tabelaEstatisticas{Comparativos de melhoria na solução para o}{Imp}{rvnd}{RVND}{
    \multirow{2}{*}{0} & \multirow{1}{*}{DD} & 1 & \multirow{2}{*}{52} & 3,595 & 6,305 & 4,693 & 5,126 & 5,473 & 5,048 & 0,547 & \textbf{0,05327} \\
     & DC & 1 &  & 3,501 & 6,265 & 4,82 & 5,332 & 5,614 & 5,173 & 0,624 &  \\ \hline
    \multirow{2}{*}{1} & \multirow{1}{*}{DD} & 1 & \multirow{2}{*}{100} & 6,163 & 9,343 & 7,686 & 8,125 & 8,317 & 8,049 & 0,58 & \textbf{0,5867} \\
     & DC & 1 &  & 6,535 & 9,369 & 7,751 & 8,136 & 8,46 & 8,093 & 0,53 &  \\ \hline
    \multirow{2}{*}{2} & \multirow{1}{*}{DD} & 1 & \multirow{2}{*}{226} & 21,46 & 31,22 & 25,24 & 26,51 & 27,78 & 26,46 & 1,93 & \textbf{0,3557} \\
     & DC & 1 &  & 22,88 & 31,07 & 25,64 & 26,55 & 27,88 & 26,77 & 1,73 &  \\ \hline
    \multirow{2}{*}{3} & \multirow{1}{*}{DD} & 1 & \multirow{2}{*}{318} & 13,8 & 17,03 & 14,79 & 15,24 & 15,61 & 15,21 & 0,587 & \textbf{0,8594} \\
     & DC & 1 &  & 13,62 & 16,04 & 14,85 & 15,2 & 15,55 & 15,16 & 0,515 &  \\ \hline
    \multirow{2}{*}{4} & \multirow{1}{*}{DD} & 1 & \multirow{2}{*}{501} & 15,58 & 17,42 & 16,2 & 16,49 & 16,74 & 16,47 & 0,411 & 0,02205 \\
     & DC & 1 &  & 15,11 & 17,17 & 16,07 & 16,37 & 16,61 & 16,33 & 0,412 &  \\ \hline
    \multirow{2}{*}{5} & \multirow{1}{*}{DD} & 1 & \multirow{2}{*}{657} & 18,1 & 20,84 & 19,07 & 19,4 & 19,79 & 19,42 & 0,52 & \textbf{0,4027} \\
     & DC & 1 &  & 18,24 & 20,86 & 19,08 & 19,38 & 19,63 & 19,36 & 0,489 &  \\ \hline
    \multirow{2}{*}{6} & \multirow{1}{*}{DD} & 1 & \multirow{2}{*}{783} & 19,53 & 21,7 & 20,25 & 20,52 & 20,92 & 20,56 & 0,457 & \textbf{0,2132} \\
     & DC & 1 &  & 19,37 & 21,83 & 20,16 & 20,42 & 20,75 & 20,48 & 0,485 &  \\ \hline
    \multirow{2}{*}{7} & \multirow{1}{*}{DD} & 1 & \multirow{2}{*}{1001} & 22,23 & 24,88 & 23,08 & 23,37 & 23,7 & 23,38 & 0,503 & \textbf{0,07059} \\
     & DC & 1 &  & 22,42 & 24,52 & 23,02 & 23,22 & 23,46 & 23,27 & 0,401 &  \\ \hline
}

Pelas Figuras~\ref{fig:rvnd_dc_dd_imp_in0}-\ref{fig:rvnd_dc_dd_imp_in7} se reforça a imagem de que o RVND clássico e o RVND implementado em dataflow apresentam resultados muito parecidos em termos de valor da solução encontrada.
Este comportamento é esperado visto que, salvo pela aleatoriedade inerente à implementação sugerida por \cite{souza2010}, ambas implementações cumprem a mesma tarefa.

% \figureRvndDcDd{box}{imp}{0}{Melhoria no valor da solução}{52}

% \figureRvndDcDd{box}{imp}{1}{Melhoria no valor da solução}{100}

% \figureRvndDcDd{box}{imp}{2}{Melhoria no valor da solução}{226}

% \figureRvndDcDd{box}{imp}{3}{Melhoria no valor da solução}{318}

% \figureRvndDcDd{box}{imp}{4}{Melhoria no valor da solução}{501}

% \figureRvndDcDd{box}{imp}{5}{Melhoria no valor da solução}{657}

% \figureRvndDcDd{box}{imp}{6}{Melhoria no valor da solução}{783}

% \figureRvndDcDd{box}{imp}{7}{Melhoria no valor da solução}{1001}

\multiFigureInstancias{rvnd}{dc_dd}{imp}{Melhoria no valor da solução para o RVND, $n$ representa o tamanho da instância, $m$ indica o número de máquinas, \textit{DC} refere-se ao RVND clássico e \textit{DD} ao RVND implementado em dataflow.}

\section{DVND} \label{sec:resultadosDVND}

Para avaliar os resultados do DVND foi comparada a sua implementação clássica (DC) apresentada na literatura, apresentada no Algoritmo~\ref{alg:dvnd}, com a implementação em dataflow (DD) apresentada na Figura~\ref{fig:dvndGraph}.
Os tempos de execução e melhoria na solução inicial são apresentados respectivamente na Tabela~\ref{tab:dvndDcDdTime} e Tabela~\ref{tab:dvndDcDdImp}, as colunas destas designam o número da instância (\#), tipo de implementação (Imp DC/DD), número de máquinas usado ($m$), tamanho da instância ($n$), valor mínimo ($min$), máximo ($max$), primeiro quartil (1Q), mediana (2Q), terceiro quartil (3Q), média ($\overline{x}$), desvio padrão ($\sigma$) e o p-valor para o teste de Wilcox entre o a versão dataflow e a clássica.

\subsection{Tempo}

Pode-se ver pela Tabela~\ref{tab:dvndDcDdTime} que o DVND clássico apresenta melhores tempos para as menores instâncias.
Até a instância 4, de tamanho 501, os tempos do DVND clássico são sensivelmente melhores que os tempos do DVND em dataflow, contudo a partir da instância 5, de tamanho 657, a implementação em dataflow alcança os tempos da implementação clássica quando usa mais de uma máquina.

\tabelaEstatisticas{Tempos comparativos do}{Time}{dvnd}{DVND}{
    \multirow{5}{*}{0} & \multirow{4}{*}{DD} & 1 & \multirow{5}{*}{52} & 1,487 & 1,82 & 1,566 & 1,603 & 1,648 & 1,61 & 0,067 & 1,238e-14 \\
     &  & 2 &  & 1,425 & 2,787 & 1,569 & 2,621 & 2,684 & 2,218 & 0,551 & 1,238e-14 \\
     &  & 3 &  & 1,504 & 2,95 & 2,495 & 2,643 & 2,727 & 2,497 & 0,386 & 1,238e-14 \\
     &  & 4 &  & 1,706 & 3,98 & 2,698 & 2,81 & 2,926 & 2,91 & 0,44 & 1,238e-14 \\
     & DC & 1 &  & 1,137 & 1,428 & 1,214 & 1,254 & 1,29 & 1,253 & 0,0554 &  \\ \hline
    \multirow{5}{*}{1} & \multirow{4}{*}{DD} & 1 & \multirow{5}{*}{100} & 1,711 & 2,195 & 1,839 & 1,903 & 1,981 & 1,918 & 0,104 & 1,238e-14 \\
     &  & 2 &  & 1,572 & 3,055 & 1,778 & 2,745 & 2,844 & 2,422 & 0,532 & 1,238e-14 \\
     &  & 3 &  & 1,702 & 3,018 & 2,601 & 2,716 & 2,85 & 2,606 & 0,369 & 1,238e-14 \\
     &  & 4 &  & 1,86 & 4,111 & 2,76 & 2,873 & 2,966 & 2,968 & 0,417 & 1,238e-14 \\
     & DC & 1 &  & 1,224 & 1,557 & 1,315 & 1,354 & 1,398 & 1,358 & 0,0583 &  \\ \hline
    \multirow{5}{*}{2} & \multirow{4}{*}{DD} & 1 & \multirow{5}{*}{226} & 2,253 & 3,011 & 2,517 & 2,605 & 2,74 & 2,621 & 0,155 & 1,238e-14 \\
     &  & 2 &  & 1,835 & 3,497 & 2,149 & 2,879 & 3,262 & 2,74 & 0,543 & 1,238e-14 \\
     &  & 3 &  & 1,969 & 3,465 & 2,738 & 2,921 & 3,17 & 2,881 & 0,389 & 1,238e-14 \\
     &  & 4 &  & 1,991 & 4,351 & 3,053 & 3,206 & 3,31 & 3,212 & 0,315 & 1,238e-14 \\
     & DC & 1 &  & 1,258 & 1,735 & 1,476 & 1,526 & 1,587 & 1,526 & 0,0937 &  \\ \hline
    \multirow{5}{*}{3} & \multirow{4}{*}{DD} & 1 & \multirow{5}{*}{318} & 2,929 & 3,909 & 3,247 & 3,345 & 3,487 & 3,352 & 0,2 & 1,238e-14 \\
     &  & 2 &  & 2,341 & 4,047 & 2,659 & 3,295 & 3,764 & 3,233 & 0,542 & 1,238e-14 \\
     &  & 3 &  & 2,474 & 4,032 & 3,181 & 3,434 & 3,691 & 3,39 & 0,37 & 1,238e-14 \\
     &  & 4 &  & 2,693 & 4,767 & 3,497 & 3,66 & 3,834 & 3,662 & 0,425 & 1,238e-14 \\
     & DC & 1 &  & 1,625 & 2,725 & 1,874 & 1,952 & 2,07 & 1,975 & 0,175 &  \\ \hline
    \multirow{5}{*}{4} & \multirow{4}{*}{DD} & 1 & \multirow{5}{*}{501} & 4,237 & 5,701 & 4,628 & 4,812 & 4,99 & 4,818 & 0,291 & 1,91e-14 \\
     &  & 2 &  & 3,115 & 5,038 & 3,597 & 3,958 & 4,626 & 4,066 & 0,553 & 1,91e-14 \\
     &  & 3 &  & 3,169 & 4,963 & 3,956 & 4,383 & 4,575 & 4,264 & 0,44 & 1,91e-14 \\
     &  & 4 &  & 3,386 & 5,563 & 4,351 & 4,534 & 4,766 & 4,539 & 0,433 & 1,91e-14 \\
     & DC & 1 &  & 1,76 & 3,549 & 2,632 & 2,811 & 3,014 & 2,842 & 0,315 &  \\ \hline
    \multirow{5}{*}{5} & \multirow{4}{*}{DD} & 1 & \multirow{5}{*}{657} & 5,512 & 7,375 & 6,173 & 6,41 & 6,677 & 6,425 & 0,369 & 0,03197 \\
     &  & 2 &  & 3,923 & 6,032 & 4,451 & 4,639 & 5,2 & 4,822 & 0,525 & 0,03197 \\
     &  & 3 &  & 4,038 & 6,053 & 4,77 & 5,235 & 5,482 & 5,12 & 0,482 & 0,03197 \\
     &  & 4 &  & 3,988 & 6,683 & 5,204 & 5,453 & 5,706 & 5,421 & 0,498 & 0,03197 \\
     & DC & 1 &  & 3,555 & 7,213 & 4,529 & 4,856 & 5,474 & 5,018 & 0,745 &  \\ \hline
    \multirow{5}{*}{6} & \multirow{4}{*}{DD} & 1 & \multirow{5}{*}{783} & 7,211 & 9,492 & 7,923 & 8,174 & 8,651 & 8,281 & 0,507 & 0,0001566 \\
     &  & 2 &  & 5,016 & 7,217 & 5,598 & 5,841 & 6,42 & 5,99 & 0,523 & 0,0001566 \\
     &  & 3 &  & 5,296 & 7,37 & 5,898 & 6,421 & 6,672 & 6,308 & 0,479 & 0,0001566 \\
     &  & 4 &  & 5,237 & 8,034 & 6,134 & 6,598 & 6,815 & 6,496 & 0,588 & 0,0001566 \\
     & DC & 1 &  & 3,031 & 8,138 & 5,537 & 6,066 & 6,543 & 6,105 & 0,848 &  \\ \hline
    \multirow{5}{*}{7} & \multirow{4}{*}{DD} & 1 & \multirow{5}{*}{1001} & 10,93 & 14,49 & 11,89 & 12,44 & 13,02 & 12,48 & 0,764 & 3,915e-11 \\
     &  & 2 &  & 7,773 & 10,46 & 8,416 & 8,933 & 9,448 & 8,964 & 0,688 & 3,915e-11 \\
     &  & 3 &  & 7,584 & 10,26 & 8,801 & 9,191 & 9,526 & 9,146 & 0,585 & 3,915e-11 \\
     &  & 4 &  & 7,516 & 10,46 & 8,647 & 9,008 & 9,373 & 8,99 & 0,578 & 3,915e-11 \\
     & DC & 1 &  & 3,024 & 17,28 & 12,19 & 13,34 & 14,39 & 13,28 & 1,92 &  \\ \hline
}

Na maior instância, de tamanho 1001, pode ser visto que o resultado da implementação dataflow para uma máquina é sutilmente melhor que da implementação clássica, o que se torna mais evidente ao utilizar mais de uma máquina, quando os tempos melhoram sensivelmente em relação à implementação clássica.
Conforme se vê na Tabela~\ref{tab:dvndDcDdTime} pela coluna $p-valor$ há significância estatística para se verificar a diferença entre as amostragens.

% \figureDvndDcDd{box}{time}{0}{Tempos}{52}

Podemos ver na Figura~\ref{fig:dvnd_dc_dd_time_in0} que o DVND clássico possui tempos bem menores que o DVND em dataflow e o uso de mais máquinas não consegue melhorar os tempos do procedimento.

% \figureDvndDcDd{box}{time}{1}{Tempos}{100}

A Figura~\ref{fig:dvnd_dc_dd_time_in1} é bem parecida com a anterior, inclusive com tempos bastante próximos, indicando que o aumento de 52 para 100 no tamanho da solução não é suficiente para causar um grande aumento no tempos de solução pelo método.

% \figureDvndDcDd{box}{time}{2}{Tempos}{226}

Para a Figura~\ref{fig:dvnd_dc_dd_time_in2} percebe-se que os tempos aumentam um pouco mas o comportamento é bastante semelhante, o DVND clássico é mais rápido para resolver o problema e aumentar o número de máquinas não melhora razoavelmente o desempenho.

% \figureDvndDcDd{box}{time}{3}{Tempos}{318}

% \figureDvndDcDd{box}{time}{4}{Tempos}{501}

Para a Figura~\ref{fig:dvnd_dc_dd_time_in4}, que representa a instância 4 de tamanho 501, percebe-se pela primeira vez uma melhoria razoável no tempo do DVND em dataflow pelo uso de mais de uma máquina, contudo ainda não sendo suficiente para melhorar os resultado do DVND clássico.

% \figureDvndDcDd{box}{time}{5}{Tempos}{657}

Na instância 5, de tamanho 657, ilustrada na Figura~\ref{fig:dvnd_dc_dd_time_in5}, o tempo do DVND em dataflow, quando usa mais de uma máquina, melhora alcançando ao DVND clássico.

% \figureDvndDcDd{box}{time}{6}{Tempos}{783}

Na instância 6, de tamanho 783, ilustrada na Figura~\ref{fig:dvnd_dc_dd_time_in6}, os resultados são bastante parecidos com a instância anterior.

% \figureDvndDcDd{box}{time}{7}{Tempos}{1001}

Na instância 7, de tamanho 1001, ilustrada na Figura~\ref{fig:dvnd_dc_dd_time_in7}, os tempos alcançados pelo DVND dataflow são menores que o DVND clássico para mais de uma máquina alcançando assim melhores tempos para a maior instância.

\multiFigureInstancias{dvnd}{dc_dd}{time}{Tempo do DVND, $n$ representa o tamanho da instância, $m$ indica o número de máquinas, \textit{DC} refere-se ao DVND clássico e \textit{DD} ao DVND implementado em dataflow.}

\subsection{Melhoria no valor da solução}

A Tabela~\ref{tab:dvndDcDdImp} e as Figuras~\ref{fig:dvnd_dc_dd_imp_in0}-\ref{fig:dvnd_dc_dd_imp_in7} apresentam resultados em termos da melhoria no valor da solução inicial.

\tabelaEstatisticas{Comparativos de melhoria na solução para o}{Imp}{dvnd}{DVND}{
    \multirow{5}{*}{0} & \multirow{4}{*}{DD} & 1 & \multirow{5}{*}{52} & 3,368 & 6,832 & 4,765 & 5,17 & 5,535 & 5,104 & 0,622 & 0,007459 \\
     &  & 2 &  & 4,013 & 6,65 & 4,926 & 5,163 & 5,572 & 5,201 & 0,513 & 0,007459 \\
     &  & 3 &  & 4,135 & 6,655 & 4,906 & 5,296 & 5,597 & 5,283 & 0,505 & 0,007459 \\
     &  & 4 &  & 4,293 & 6,329 & 5,026 & 5,332 & 5,611 & 5,316 & 0,455 & 0,007459 \\
     & DC & 1 &  & 4,436 & 7,611 & 5,176 & 5,497 & 6,106 & 5,657 & 0,694 &  \\ \hline
    \multirow{5}{*}{1} & \multirow{4}{*}{DD} & 1 & \multirow{5}{*}{100} & 6,22 & 9,514 & 7,7 & 8,089 & 8,428 & 8,05 & 0,583 & 3,84e-06 \\
     &  & 2 &  & 6,135 & 9,211 & 7,698 & 8,057 & 8,331 & 8,012 & 0,539 & 3,84e-06 \\
     &  & 3 &  & 6,277 & 9,14 & 7,763 & 8,034 & 8,398 & 8,038 & 0,507 & 3,84e-06 \\
     &  & 4 &  & 6,334 & 9,451 & 7,735 & 8,068 & 8,423 & 8,072 & 0,564 & 3,84e-06 \\
     & DC & 1 &  & 6,496 & 13,83 & 8,201 & 8,72 & 9,575 & 9,005 & 1,18 &  \\ \hline
    \multirow{5}{*}{2} & \multirow{4}{*}{DD} & 1 & \multirow{5}{*}{226} & 22,73 & 31,25 & 25,46 & 26,48 & 27,81 & 26,72 & 1,77 & 0,0144 \\
     &  & 2 &  & 21,87 & 31,28 & 25,4 & 26,74 & 28,2 & 26,78 & 2,03 & 0,0144 \\
     &  & 3 &  & 22,76 & 30,84 & 25,55 & 26,72 & 27,94 & 26,75 & 1,71 & 0,0144 \\
     &  & 4 &  & 22,99 & 30,9 & 25,64 & 26,6 & 27,65 & 26,65 & 1,66 & 0,0144 \\
     & DC & 1 &  & 7,518 & 33,16 & 26,24 & 27,66 & 29,12 & 27,11 & 3,81 &  \\ \hline
    \multirow{5}{*}{3} & \multirow{4}{*}{DD} & 1 & \multirow{5}{*}{318} & 13,61 & 16,52 & 14,88 & 15,24 & 15,59 & 15,21 & 0,555 & 1,525e-13 \\
     &  & 2 &  & 13,9 & 16,81 & 14,73 & 15,19 & 15,59 & 15,16 & 0,569 & 1,525e-13 \\
     &  & 3 &  & 13,86 & 16,77 & 14,75 & 15,12 & 15,51 & 15,15 & 0,559 & 1,525e-13 \\
     &  & 4 &  & 13,28 & 16,75 & 14,87 & 15,29 & 15,57 & 15,24 & 0,573 & 1,525e-13 \\
     & DC & 1 &  & 14,04 & 27,48 & 16,27 & 17,15 & 19,15 & 17,91 & 2,65 &  \\ \hline
    \multirow{5}{*}{4} & \multirow{4}{*}{DD} & 1 & \multirow{5}{*}{501} & 15,44 & 17,23 & 16,16 & 16,39 & 16,66 & 16,4 & 0,374 & 7,427e-10 \\
     &  & 2 &  & 15,39 & 17,45 & 16,19 & 16,44 & 16,76 & 16,44 & 0,404 & 7,427e-10 \\
     &  & 3 &  & 15,43 & 17,47 & 16,18 & 16,41 & 16,63 & 16,43 & 0,429 & 7,427e-10 \\
     &  & 4 &  & 15,44 & 17,49 & 16,09 & 16,35 & 16,71 & 16,39 & 0,419 & 7,427e-10 \\
     & DC & 1 &  & 13,66 & 27,01 & 17,14 & 18,09 & 19,74 & 18,64 & 2,27 &  \\ \hline
    \multirow{5}{*}{5} & \multirow{4}{*}{DD} & 1 & \multirow{5}{*}{657} & 18,37 & 20,86 & 19,16 & 19,39 & 19,74 & 19,45 & 0,461 & 1,784e-10 \\
     &  & 2 &  & 18,31 & 20,42 & 19,14 & 19,42 & 19,71 & 19,44 & 0,443 & 1,784e-10 \\
     &  & 3 &  & 18,22 & 20,59 & 19,07 & 19,46 & 19,75 & 19,41 & 0,486 & 1,784e-10 \\
     &  & 4 &  & 18,54 & 20,59 & 19,22 & 19,41 & 19,76 & 19,46 & 0,419 & 1,784e-10 \\
     & DC & 1 &  & 18,78 & 31,23 & 20,29 & 21,11 & 22,19 & 21,49 & 1,9 &  \\ \hline
    \multirow{5}{*}{6} & \multirow{4}{*}{DD} & 1 & \multirow{5}{*}{783} & 19,37 & 21,86 & 20,16 & 20,49 & 20,74 & 20,48 & 0,486 & 9,465e-13 \\
     &  & 2 &  & 19,7 & 21,51 & 20,18 & 20,44 & 20,79 & 20,5 & 0,439 & 9,465e-13 \\
     &  & 3 &  & 19,47 & 21,75 & 20,22 & 20,46 & 20,86 & 20,51 & 0,473 & 9,465e-13 \\
     &  & 4 &  & 19,34 & 21,58 & 20,18 & 20,52 & 20,8 & 20,5 & 0,453 & 9,465e-13 \\
     & DC & 1 &  & 20,12 & 32,86 & 21,6 & 22,66 & 24,37 & 23,27 & 2,26 &  \\ \hline
    \multirow{5}{*}{7} & \multirow{4}{*}{DD} & 1 & \multirow{5}{*}{1001} & 22,44 & 24,62 & 23,08 & 23,3 & 23,57 & 23,34 & 0,41 & 5,073e-12 \\
     &  & 2 &  & 22,54 & 24,81 & 23,01 & 23,32 & 23,62 & 23,33 & 0,441 & 5,073e-12 \\
     &  & 3 &  & 22,42 & 24,34 & 23,14 & 23,32 & 23,65 & 23,37 & 0,38 & 5,073e-12 \\
     &  & 4 &  & 22,39 & 24,8 & 22,97 & 23,27 & 23,52 & 23,27 & 0,44 & 5,073e-12 \\
     & DC & 1 &  & 11,57 & 38,99 & 23,94 & 24,44 & 25,83 & 25,05 & 2,54 &  \\ \hline
}

Como pode ser visto na Tabela~\ref{tab:dvndDcDdImp}, ficando mais evidente nas Figuras~\ref{fig:dvnd_dc_dd_imp_in0}-\ref{fig:dvnd_dc_dd_imp_in7}, em geral o DVND clássico (DC) consegue melhorar mais o valor da solução inicial quando comparado ao DVND dataflow (DD).

% \figureDvndDcDd{box}{imp}{0}{Melhoria no valor da solução}{52}

% \figureDvndDcDd{box}{imp}{1}{Melhoria no valor da solução}{100}

% \figureDvndDcDd{box}{imp}{2}{Melhoria no valor da solução}{226}

% \figureDvndDcDd{box}{imp}{3}{Melhoria no valor da solução}{318}

Ao aumentar o tamanho das instâncias o DVND clássico continua encontrando melhores resultados em termos de valor da solução mas também aumentando a variabilidade destes resultados, o que pode ser visto na Figura~\ref{fig:dvnd_dc_dd_imp_in3}, referente à instância 3 de tamanho 318, onde a amplitude interquartil do DVND clássico é de 2,88 número mais de 3 vezes o tamanho da maior amplitude interquartil para o DVND em dataflow para a mesma instância no valor de 0,86 para o DVND em dataflow com duas máquinas.

% \figureDvndDcDd{box}{imp}{4}{Melhoria no valor da solução}{501}

% \figureDvndDcDd{box}{imp}{5}{Melhoria no valor da solução}{657}

% \figureDvndDcDd{box}{imp}{6}{Melhoria no valor da solução}{783}

% \figureDvndDcDd{box}{imp}{7}{Melhoria no valor da solução}{1001}

\multiFigureInstancias{dvnd}{dc_dd}{imp}{Melhoria no valor da solução do DVND, $n$ representa o tamanho da instância, $m$ indica o número de máquinas, \textit{DC} refere-se ao DVND clássico e \textit{DD} ao DVND implementado em dataflow.}

\section{GDVND}

\subsection{Tempo}

Pode ser visto na Tabela~\ref{tab:gdvndDcDdTime} e nas Figuras~\ref{fig:gdvnd_full_time_time_in0}-\ref{fig:gdvnd_full_time_time_in7} que o DVND e GDVND apresentam comportamentos semelhantes em relação ao RVND, os tempos do RVND são melhores para as instâncias até o tamanho 318 e da instância 4 (tamanho 501) em diante o RVND começa a levar mais tempo para encontrar a resposta.

\tabelaEstatisticasGeral{Tempos comparativos do}{Time}{gdvnd}{GDVND}{com DVND e RVND.}{
    \multirow{3}{*}{0} & DVND & 4 & \multirow{3}{*}{52} & 2,43 & 4,028 & 2,698 & 2,811 & 2,928 & 2,953 & 0,429 & 7,588e-05 \\
     & RVND & 1 &  & 0,4269 & 1,77 & 1,258 & 1,416 & 1,531 & 1,321 & 0,323 & 3,564e-34 \\
     & GDVND & 4 &  & 1,6 & 4,955 & 2,856 & 3,069 & 3,57 & 3,182 & 0,572 &  \\ \hline
    \multirow{3}{*}{1} & DVND & 4 & \multirow{3}{*}{100} & 1,841 & 4,156 & 2,831 & 2,963 & 3,808 & 3,207 & 0,527 & 2,201e-06 \\
     & RVND & 1 &  & 0,5236 & 2,773 & 0,8762 & 1,025 & 1,309 & 1,177 & 0,457 & 3,673e-34 \\
     & GDVND & 4 &  & 2,346 & 5,178 & 3,078 & 3,367 & 4,061 & 3,495 & 0,597 &  \\ \hline
    \multirow{3}{*}{2} & DVND & 4 & \multirow{3}{*}{226} & 2,504 & 4,545 & 3,079 & 3,171 & 3,29 & 3,243 & 0,355 & 7,31e-21 \\
     & RVND & 1 &  & 1,483 & 9,029 & 2,194 & 2,556 & 3,308 & 3,103 & 1,49 & 2,196e-13 \\
     & GDVND & 4 &  & 2,347 & 5,639 & 3,579 & 3,858 & 4,532 & 4,011 & 0,657 &  \\ \hline
    \multirow{3}{*}{3} & DVND & 4 & \multirow{3}{*}{318} & 2,674 & 4,592 & 3,515 & 3,622 & 3,756 & 3,648 & 0,324 & 1,74e-18 \\
     & RVND & 1 &  & 1,983 & 7,007 & 3,133 & 3,541 & 4,125 & 3,824 & 1,07 & 8,397e-09 \\
     & GDVND & 4 &  & 2,265 & 6,533 & 4,091 & 4,401 & 4,942 & 4,46 & 0,763 &  \\ \hline
    \multirow{3}{*}{4} & DVND & 4 & \multirow{3}{*}{501} & 3,423 & 5,801 & 4,345 & 4,514 & 4,691 & 4,496 & 0,38 & 2,401e-26 \\
     & RVND & 1 &  & 3,614 & 13,5 & 5,374 & 6,104 & 7,05 & 6,597 & 1,98 & \textbf{0,8022} \\
     & GDVND & 4 &  & 2,487 & 8,283 & 5,697 & 6,173 & 6,7 & 6,092 & 0,933 &  \\ \hline
    \multirow{3}{*}{5} & DVND & 4 & \multirow{3}{*}{657} & 4,17 & 6,604 & 5,246 & 5,464 & 5,682 & 5,49 & 0,441 & 6,681e-34 \\
     & RVND & 1 &  & 6,878 & 22,9 & 9,351 & 10,12 & 11,74 & 11,43 & 3,82 & 1,967e-20 \\
     & GDVND & 4 &  & 5,796 & 10,71 & 7,5 & 8,074 & 8,605 & 8,097 & 0,876 &  \\ \hline
    \multirow{3}{*}{6} & DVND & 4 & \multirow{3}{*}{783} & 5,434 & 8,442 & 6,305 & 6,508 & 6,691 & 6,514 & 0,507 & 4,018e-34 \\
     & RVND & 1 &  & 9,997 & 35,13 & 13,26 & 14,9 & 17,5 & 16,91 & 5,93 & 1,396e-29 \\
     & GDVND & 4 &  & 7,187 & 13,09 & 9,821 & 10,61 & 11,15 & 10,49 & 1,1 &  \\ \hline
    \multirow{3}{*}{7} & DVND & 4 & \multirow{3}{*}{1001} & 7,871 & 10,88 & 8,836 & 9,09 & 9,493 & 9,164 & 0,566 & 8,482e-34 \\
     & RVND & 1 &  & 15,51 & 66,77 & 21,48 & 24,74 & 29,11 & 27,5 & 9,6 & 1,705e-30 \\
     & GDVND & 4 &  & 9,195 & 19,53 & 15,71 & 16,47 & 17,22 & 16,35 & 1,54 &   \\ \hline
}

% \figureGdvndDvndRvnd{box}{time}{0}{Tempo}{52}{GDVND}{gdvnd}{full_time}{gdvnd}

% \figureGdvndDvndRvnd{box}{time}{1}{Tempo}{100}{GDVND}{gdvnd}{full_time}{dvnd}

% \figureGdvndDvndRvnd{box}{time}{2}{Tempo}{226}{GDVND}{gdvnd}{full_time}{dvnd}

% \figureGdvndDvndRvnd{box}{time}{3}{Tempo}{318}{GDVND}{gdvnd}{full_time}{dvnd}

% \figureGdvndDvndRvnd{box}{time}{4}{Tempo}{501}{GDVND}{gdvnd}{full_time}{dvnd}

% \figureGdvndDvndRvnd{box}{time}{5}{Tempo}{657}{GDVND}{gdvnd}{full_time}{dvnd}

% \figureGdvndDvndRvnd{box}{time}{6}{Tempo}{783}{GDVND}{gdvnd}{full_time}{dvnd}

% \figureGdvndDvndRvnd{box}{time}{7}{Tempo}{1001}{GDVND}{gdvnd}{full_time}{dvnd}

\multiFigureInstancias{gdvnd}{full_time}{time}{Tempo dos algoritmos GDVND, DVND e RVND,  $n$ representa o tamanho da instância.}

\subsection{Melhoria no valor da solução}

A Tabela~\ref{tab:gdvndDcDdImp} mostra e as Figuras~\ref{fig:gdvnd_full_time_imp_in0}-\ref{fig:gdvnd_full_time_imp_in7} evidenciam que de forma geral não há grande diferença na qualidade da solução encontrada pelos métodos.
De fato, ao analisarmos o $p-valor$ encontrado pelo teste de Wilcox realizado da amostra do GDVND com as demais, do DVND e RVND podemos ver que para muitos casos não há significância estatística para afirmar a diferença entre as amostras, valores destacados em negrito na coluna $p-valor$ da Tabela~\ref{tab:gdvndDcDdImp}.

\tabelaEstatisticasGeral{Comparativos de melhoria na solução para o}{Imp}{gdvnd}{GDVND}{com DVND e RVND.}{
    \multirow{3}{*}{0} & DVND & 4 & \multirow{3}{*}{52} & 4,27 & 6,401 & 4,954 & 5,287 & 5,564 & 5,264 & 0,475 & 2,447e-08 \\
     & RVND & 1 &  & 3,595 & 6,305 & 4,693 & 5,126 & 5,473 & 5,048 & 0,547 & 0,0002838 \\
     & GDVND & 4 &  & 2,706 & 6,457 & 4,024 & 4,836 & 5,273 & 4,609 & 0,871 &  \\ \hline
    \multirow{3}{*}{1} & DVND & 4 & \multirow{3}{*}{100} & 6,884 & 9,227 & 7,743 & 8,025 & 8,411 & 8,05 & 0,526 & 0,00738 \\
     & RVND & 1 &  & 6,163 & 9,343 & 7,686 & 8,125 & 8,317 & 8,049 & 0,58 & 0,008289 \\
     & GDVND & 4 &  & 4,012 & 9,348 & 7,256 & 7,838 & 8,311 & 7,627 & 1,04 &  \\ \hline
    \multirow{3}{*}{2} & DVND & 4 & \multirow{3}{*}{226} & 22,47 & 31,19 & 25,48 & 26,69 & 27,78 & 26,7 & 1,77 & \textbf{0,06525} \\
     & RVND & 1 &  & 21,46 & 31,22 & 25,24 & 26,51 & 27,78 & 26,46 & 1,93 & \textbf{0,1826} \\
     & GDVND & 4 &  & 5,265 & 30,89 & 24,97 & 26,17 & 27,62 & 25,29 & 4,31 &  \\ \hline
    \multirow{3}{*}{3} & DVND & 4 & \multirow{3}{*}{318} & 13,39 & 16,69 & 14,86 & 15,16 & 15,63 & 15,19 & 0,585 & \textbf{0,1916} \\
     & RVND & 1 &  & 13,8 & 17,03 & 14,79 & 15,24 & 15,61 & 15,21 & 0,587 & \textbf{0,1484} \\
     & GDVND & 4 &  & 5,163 & 16,94 & 14,67 & 15,07 & 15,5 & 14,38 & 2,68 &  \\ \hline
    \multirow{3}{*}{4} & DVND & 4 & \multirow{3}{*}{501} & 15,23 & 17,47 & 16,14 & 16,4 & 16,62 & 16,37 & 0,399 & \textbf{0,185} \\
     & RVND & 1 &  & 15,58 & 17,42 & 16,2 & 16,49 & 16,74 & 16,47 & 0,411 & 0,005406 \\
     & GDVND & 4 &  & 6,112 & 17,56 & 16,02 & 16,32 & 16,59 & 15,9 & 2,03 &  \\ \hline
    \multirow{3}{*}{5} & DVND & 4 & \multirow{3}{*}{657} & 18,22 & 20,49 & 19,08 & 19,42 & 19,71 & 19,38 & 0,476 & \textbf{0,3123} \\
     & RVND & 1 &  & 18,1 & 20,84 & 19,07 & 19,4 & 19,79 & 19,42 & 0,52 & \textbf{0,1671} \\
     & GDVND & 4 &  & 17,63 & 20,49 & 18,96 & 19,26 & 19,71 & 19,3 & 0,515 &  \\ \hline
    \multirow{3}{*}{6} & DVND & 4 & \multirow{3}{*}{783} & 19,44 & 21,83 & 20,16 & 20,48 & 20,85 & 20,52 & 0,469 & \textbf{0,1648} \\
     & RVND & 1 &  & 19,53 & 21,7 & 20,25 & 20,52 & 20,92 & 20,56 & 0,457 & \textbf{0,05223} \\
     & GDVND & 4 &  & 18,73 & 21,45 & 20,11 & 20,37 & 20,74 & 20,4 & 0,513 &  \\ \hline
    \multirow{3}{*}{7} & DVND & 4 & \multirow{3}{*}{1001} & 22,29 & 24,33 & 23,04 & 23,35 & 23,62 & 23,35 & 0,441 & 0,02061 \\
     & RVND & 1 &  & 22,23 & 24,88 & 23,08 & 23,37 & 23,7 & 23,38 & 0,503 & 0,009164 \\ 
     & GDVND & 4 &  & 20,66 & 24,09 & 22,91 & 23,17 & 23,5 & 23,17 & 0,493 &  \\ \hline
}

% \figureGdvndDvndRvnd{box}{imp}{0}{Melhoria no valor da solução}{52}{GDVND}{gdvnd}{full_time}{gdvnd}

% \figureGdvndDvndRvnd{box}{imp}{1}{Melhoria no valor da solução}{100}{GDVND}{gdvnd}{full_time}{dvnd}

% \figureGdvndDvndRvnd{box}{imp}{2}{Melhoria no valor da solução}{226}{GDVND}{gdvnd}{full_time}{dvnd}

% \figureGdvndDvndRvnd{box}{imp}{3}{Melhoria no valor da solução}{318}{GDVND}{gdvnd}{full_time}{dvnd}

% \figureGdvndDvndRvnd{box}{imp}{4}{Melhoria no valor da solução}{501}{GDVND}{gdvnd}{full_time}{dvnd}

% \figureGdvndDvndRvnd{box}{imp}{5}{Melhoria no valor da solução}{657}{GDVND}{gdvnd}{full_time}{dvnd}

% \figureGdvndDvndRvnd{box}{imp}{6}{Melhoria no valor da solução}{783}{GDVND}{gdvnd}{full_time}{dvnd}

% \figureGdvndDvndRvnd{box}{imp}{7}{Melhoria no valor da solução}{1001}{GDVND}{gdvnd}{full_time}{dvnd}

\multiFigureInstancias{gdvnd}{full_time}{imp}{Melhoria no valor da solução para os algoritmos GDVND, DVND e RVND, $n$ representa o tamanho da instância.}

\subsection{Analisando o tempo para combinar movimentos} \label{sec:gdvndTimeMan}

Os tempos de execução do \textit{GDVND} comparados com o DVND são exibidos nas Figuras~\ref{fig:gdvnd_man_time_time_in0}-\ref{fig:gdvnd_man_time_time_in7}, onde podemos ver o tempo gasto pelo \textit{GDVND}, o tempo gasto pelo \textit{DVND} e o tempo gasto pelo \textit{GDVND} subtraído do tempo gasto para combinar os movimentos retornados pela busca local (\textit{GDVND-MAN}), logo este último representa o tempo efetivamente gasto na busca local.

% \figureDvndGdvndTime{scatter}{time}{0}{Tempo}{52}{DVND vs GDND}{gdvnd}

% \figureDvndGdvndTime{scatter}{time}{1}{Tempo}{100}{DVND vs GDND}{gdvnd}

De forma geral podemos ver que, exceto para as duas primeiras instâncias (Figura~\ref{fig:gdvnd_man_time_time_in0} e \ref{fig:gdvnd_man_time_time_in1}), o GDVND apresenta tempos de execução maiores que o DVND para a maioria das amostras.

% \figureDvndGdvndTime{scatter}{time}{2}{Tempo}{226}{DVND vs GDND}{gdvnd}

% \figureDvndGdvndTime{scatter}{time}{3}{Tempo}{318}{DVND vs GDND}{gdvnd}

Para as instâncias até o tamanho de 318 o tempo necessário para combinar os movimentos não representa grande diferença no tempo total de execução de forma que apenas a partir da instância 4 de tamanho 501 (Figura~\ref{fig:gdvnd_man_time_time_in4}) que o tempo sem as operações sobre os movimentos (\textit{GDVND-MAN}) consegue ser melhor em uma quantidade maior de amostras.

% \figureDvndGdvndTime{scatter}{time}{4}{Tempo}{501}{DVND vs GDND}{gdvnd}

% \figureDvndGdvndTime{scatter}{time}{5}{Tempo}{657}{DVND vs GDND}{gdvnd}

No caso da instância 5 (Figura~\ref{fig:gdvnd_man_time_time_in5}) a diferença representada pelo tempo de execução gasto ao combinar os movimentos é grande contudo o DVND ainda consegue alcançar melhores tempos em algumas das instâncias.

% \figureDvndGdvndTime{scatter}{time}{6}{Tempo}{783}{DVND vs GDND}{gdvnd}

% \figureDvndGdvndTime{scatter}{time}{7}{Tempo}{1001}{DVND vs GDND}{gdvnd}

Finalmente, nas instâncias 6 e 7 (Figuras~\ref{fig:gdvnd_man_time_time_in6} e \ref{fig:gdvnd_man_time_time_in7}) a diferença se mostra bastante significativa e o tempo do \textit{GDVND-MAN} é menor na maioria das amostras.

\multiFigureInstanciasGeral{gdvnd}{man_time}{time}{scatter}{Tempo do DVND vs GDND, \textit{DVND} refere-se ao tempo gasto pelo algoritmo de mesmo nome, para \textit{GDVND} é análogo ao anterior, no caso do \textit{GDVND-MAN} este se refere ao tempo do \textit{GDVND} subtraído do tempo para gerenciar os movimentos, $n$ representa o tamanho da instância, $m$ indica o número de máquinas.}
