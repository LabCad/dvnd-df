\subsection{Tempo}

A Tabela~\ref{tab:rvndDcDdImp} e as Figuras~\ref{fig:rvnd_dc_dd_time_in0}-\ref{fig:rvnd_dc_dd_time_in7} apresentam resultados para execuções utilizando apenas uma máquina ($m = 1$) pois pela construção naturalmente sequencial do RVND não utilizar paralelismo, logo o emprego de mais de uma máquina não traria ganhos em termos de desempenho tampouco no valor da solução.

\tabelaEstatisticas{Tempos comparativos do}{Time}{rvnd}{RVND}{
    \multirow{2}{*}{0} & \multirow{1}{*}{DD} & 1 & \multirow{2}{*}{52} & 0,4269 & 1,77 & 1,258 & 1,416 & 1,531 & 1,321 & 0,323 & 9,279e-13 \\
     & DC & 1 &  & 0,262 & 1,342 & 1,155 & 1,178 & 1,2 & 1,169 & 0,1 &  \\ \hline
    \multirow{2}{*}{1} & \multirow{1}{*}{DD} & 1 & \multirow{2}{*}{100} & 0,5236 & 2,773 & 0,8762 & 1,025 & 1,309 & 1,177 & 0,457 & 2,48e-07 \\
     & DC & 1 &  & 0,3116 & 1,771 & 1,377 & 1,473 & 1,523 & 1,378 & 0,297 &  \\ \hline
    \multirow{2}{*}{2} & \multirow{1}{*}{DD} & 1 & \multirow{2}{*}{226} & 1,483 & 9,029 & 2,194 & 2,556 & 3,308 & 3,103 & 1,49 & 0,02545 \\
     & DC & 1 &  & 1,413 & 6,555 & 2,565 & 2,989 & 3,616 & 3,106 & 0,926 &  \\ \hline
    \multirow{2}{*}{3} & \multirow{1}{*}{DD} & 1 & \multirow{2}{*}{318} & 1,983 & 7,007 & 3,133 & 3,541 & 4,125 & 3,824 & 1,07 & 2,559e-17 \\
     & DC & 1 &  & 1,931 & 3,949 & 2,445 & 2,67 & 3,038 & 2,75 & 0,439 &  \\ \hline
    \multirow{2}{*}{4} & \multirow{1}{*}{DD} & 1 & \multirow{2}{*}{501} & 3,614 & 13,5 & 5,374 & 6,104 & 7,05 & 6,597 & 1,98 & 4,07e-13 \\
     & DC & 1 &  & 3,63 & 7,018 & 4,697 & 4,987 & 5,52 & 5,095 & 0,59 &  \\ \hline
    \multirow{2}{*}{5} & \multirow{1}{*}{DD} & 1 & \multirow{2}{*}{657} & 6,878 & 22,9 & 9,351 & 10,12 & 11,74 & 11,43 & 3,82 & 1,202e-18 \\
     & DC & 1 &  & 6,325 & 11,35 & 7,855 & 8,356 & 8,821 & 8,345 & 0,752 &  \\ \hline
    \multirow{2}{*}{6} & \multirow{1}{*}{DD} & 1 & \multirow{2}{*}{783} & 9,997 & 35,13 & 13,26 & 14,9 & 17,5 & 16,91 & 5,93 & 2,107e-15 \\
     & DC & 1 &  & 10,38 & 15,38 & 11,7 & 12,49 & 13,04 & 12,5 & 1,05 &  \\ \hline
    \multirow{2}{*}{7} & \multirow{1}{*}{DD} & 1 & \multirow{2}{*}{1001} & 15,51 & 66,77 & 21,48 & 24,74 & 29,11 & 27,5 & 9,6 & \textbf{0,1547} \\
     & DC & 1 &  & 19,8 & 30,42 & 22,71 & 24,31 & 25,73 & 24,27 & 2,13 &  \\ \hline
}

Pela Tabela~\ref{tab:rvndDcDdImp} e as Figuras~\ref{fig:rvnd_dc_dd_time_in0}, \ref{fig:rvnd_dc_dd_time_in3}-\ref{fig:rvnd_dc_dd_time_in6} podemos ver que o RVND em sua implementação clássica (RC) apresentou melhores tempos que o RVND na versão dataflow (RD).

% \figureRvndDcDd{box}{time}{0}{Tempos}{52}

% Podemos ver na Figura~\ref{fig:time_dvndDcDd_in0} que o DVND clássico possui tempos bem menores que o DVND em dataflow e o uso de mais máquinas não consegue melhorar os tempos do procedimento.

% \figureRvndDcDd{box}{time}{1}{Tempos}{100}

% \figureRvndDcDd{box}{time}{2}{Tempos}{226}

% \figureRvndDcDd{box}{time}{3}{Tempos}{318}

% \figureRvndDcDd{box}{time}{4}{Tempos}{501}

% \figureRvndDcDd{box}{time}{5}{Tempos}{657}

% \figureRvndDcDd{box}{time}{6}{Tempos}{783}

\multiFigureInstancias{rvnd}{dc_dd}{time}{Tempos do RVND, $n$ representa o tamanho da instância, $m$ indica o número de máquinas, \textit{DC} refere-se ao RVND clássico e \textit{DD} ao RVND implementado em dataflow.}

Apenas no caso da instância 7, de tamanho 1001, representada pela Figura~\ref{fig:rvnd_dc_dd_time_in7}, não houve diferença significativa, segundo o teste de Wilcox, para afirmar a existência de diferença nos dados.

% \figureRvndDcDd{box}{time}{7}{Tempos}{1001}
