\subsection{Melhoria no valor da solução}

Em termos de melhoria no valor da solução (dada pela Equação~\ref{eq:calculateImprovement}), podemos ver Tabela~\ref{tab:rvndDcDdTime} que não há grandes diferenças em termos de média ($\overline{x}$) nem de mediana ($2Q$) o que se comprova nos resultados do teste de Wilcox que não apresenta diferença significativa senão nos resutlados da instância 4 de tamanho 501.

\tabelaEstatisticas{Comparativos de melhoria na solução para o}{Imp}{rvnd}{RVND}{
    \multirow{2}{*}{0} & \multirow{1}{*}{DD} & 1 & \multirow{2}{*}{52} & 3,595 & 6,305 & 4,693 & 5,126 & 5,473 & 5,048 & 0,547 & \textbf{0,05327} \\
     & DC & 1 &  & 3,501 & 6,265 & 4,82 & 5,332 & 5,614 & 5,173 & 0,624 &  \\ \hline
    \multirow{2}{*}{1} & \multirow{1}{*}{DD} & 1 & \multirow{2}{*}{100} & 6,163 & 9,343 & 7,686 & 8,125 & 8,317 & 8,049 & 0,58 & \textbf{0,5867} \\
     & DC & 1 &  & 6,535 & 9,369 & 7,751 & 8,136 & 8,46 & 8,093 & 0,53 &  \\ \hline
    \multirow{2}{*}{2} & \multirow{1}{*}{DD} & 1 & \multirow{2}{*}{226} & 21,46 & 31,22 & 25,24 & 26,51 & 27,78 & 26,46 & 1,93 & \textbf{0,3557} \\
     & DC & 1 &  & 22,88 & 31,07 & 25,64 & 26,55 & 27,88 & 26,77 & 1,73 &  \\ \hline
    \multirow{2}{*}{3} & \multirow{1}{*}{DD} & 1 & \multirow{2}{*}{318} & 13,8 & 17,03 & 14,79 & 15,24 & 15,61 & 15,21 & 0,587 & \textbf{0,8594} \\
     & DC & 1 &  & 13,62 & 16,04 & 14,85 & 15,2 & 15,55 & 15,16 & 0,515 &  \\ \hline
    \multirow{2}{*}{4} & \multirow{1}{*}{DD} & 1 & \multirow{2}{*}{501} & 15,58 & 17,42 & 16,2 & 16,49 & 16,74 & 16,47 & 0,411 & 0,02205 \\
     & DC & 1 &  & 15,11 & 17,17 & 16,07 & 16,37 & 16,61 & 16,33 & 0,412 &  \\ \hline
    \multirow{2}{*}{5} & \multirow{1}{*}{DD} & 1 & \multirow{2}{*}{657} & 18,1 & 20,84 & 19,07 & 19,4 & 19,79 & 19,42 & 0,52 & \textbf{0,4027} \\
     & DC & 1 &  & 18,24 & 20,86 & 19,08 & 19,38 & 19,63 & 19,36 & 0,489 &  \\ \hline
    \multirow{2}{*}{6} & \multirow{1}{*}{DD} & 1 & \multirow{2}{*}{783} & 19,53 & 21,7 & 20,25 & 20,52 & 20,92 & 20,56 & 0,457 & \textbf{0,2132} \\
     & DC & 1 &  & 19,37 & 21,83 & 20,16 & 20,42 & 20,75 & 20,48 & 0,485 &  \\ \hline
    \multirow{2}{*}{7} & \multirow{1}{*}{DD} & 1 & \multirow{2}{*}{1001} & 22,23 & 24,88 & 23,08 & 23,37 & 23,7 & 23,38 & 0,503 & \textbf{0,07059} \\
     & DC & 1 &  & 22,42 & 24,52 & 23,02 & 23,22 & 23,46 & 23,27 & 0,401 &  \\ \hline
}

Pelas Figuras~\ref{fig:rvnd_dc_dd_imp_in0}-\ref{fig:rvnd_dc_dd_imp_in7} se reforça a imagem de que o RVND clássico e o RVND implementado em dataflow apresentam resultados muito parecidos em termos de valor da solução encontrada.
Este comportamento é esperado visto que, salvo pela aleatoriedade inerente à implementação sugerida por \cite{souza2010}, ambas implementações cumprem a mesma tarefa.

% \figureRvndDcDd{box}{imp}{0}{Melhoria no valor da solução}{52}

% \figureRvndDcDd{box}{imp}{1}{Melhoria no valor da solução}{100}

% \figureRvndDcDd{box}{imp}{2}{Melhoria no valor da solução}{226}

% \figureRvndDcDd{box}{imp}{3}{Melhoria no valor da solução}{318}

% \figureRvndDcDd{box}{imp}{4}{Melhoria no valor da solução}{501}

% \figureRvndDcDd{box}{imp}{5}{Melhoria no valor da solução}{657}

% \figureRvndDcDd{box}{imp}{6}{Melhoria no valor da solução}{783}

% \figureRvndDcDd{box}{imp}{7}{Melhoria no valor da solução}{1001}

\multiFigureInstancias{rvnd}{dc_dd}{imp}{Melhoria no valor da solução para o RVND, $n$ representa o tamanho da instância, $m$ indica o número de máquinas, \textit{DC} refere-se ao RVND clássico e \textit{DD} ao RVND implementado em dataflow.}
