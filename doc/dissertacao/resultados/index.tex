\chapter{Resultados} \label{cap:resultados}

Este capítulo exibe os resultados computacionais dos algoritmos propostos no Capítulo~\ref{cap:metodologia} para o caso do PML, para cada instância foi gerado um conjunto com 100 soluções iniciais aleatórias que foram submetidas aos métodos para comparação dos resultados.

Quando há referência à melhoria na solução (\textit{Imp}), esta melhoria pode ser calculada pelo quociente do valor da solução inicial pela solução final, ou seja:
\begin{equation}\label{eq:calculateImprovement}
Imp = \frac{f(\textrm{solução inicial})}{f(\textrm{solução final})}
\end{equation}

Desta forma quanto maior for o valor da melhoria ($Imp$) mais o método melhorou o valor da solução inicial.

\section{Instâncias} \label{sec:instancias}

Todas as instâncias usadas nos testes computacionais e cujas configurações de lançamento foram descritas na Tabela~\ref{tab:neighborhoodsLaunchConfigurarion} são as mesmas usadas em~\cite{wamca2016}.
Para o RVND foi feita uma implementação do algoritmo clássico (Algoritmo~\ref{alg:rvnd}) e também a implementação dataflow mencionada na Figura~\ref{fig:rvndGraph} fazendo uso de uma máquina.
Para o caso do DVND foi utilizada a implementação clássica (Algoritmo~\ref{alg:dvnd}) e a implementação dataflow proposta (Figura~\ref{fig:dvndGraph}), os resultados foram obtidos com diferentes números de máquinas e os mesmos são indicados conforme o caso.

\section{Implementação e ambiente computacional}\label{sec:amb}

A implementação para cada algoritmo proposto no Capítulo~\ref{cap:metodologia} utiliza a linguagem de programação \textit{C++11} em conjunto com a API CUDA\texttrademark, para a implementação dos grafos e do ambiente dataflow foi utilizada a biblioteca Sucuri~\cite{sucuri-original}\footnote{Disponível em \url{https://github.com/tiagoaoa/Sucuri}} implementada em Python, para a integração entre o dataflow e o código CUDA foi utilizada a biblioteca SimplePyCuda~\cite{simple-pycuda}\footnote{Disponível em \url{https://github.com/igormcoelho/simple-pycuda}}. As implementações com múltiplas threads usaram a biblioteca OpenMP.
%As implementações foram compiladas através do \textit{GCC} \textit{(GNU Compiler Collection)}\footnote{O GCC está disponível no seguinte sítio eletrônico: \url{https://gcc.gnu.org/}.} com a \textit{flag} de otimização $-O3$.
O ambiente computacional utilizado em todos os testes neste trabalho consiste de 4 máquinas com a seguinte configuração:

\begin{itemize}
    \item Processador Intel\textregistered Core\texttrademark i7-4820K 3.7 GHz (4 núcleos);
    \item 8 GB de memória RAM;
    \item Sistema Operacional Ubuntu 14.10 (x64);
    \item NVIDIA GeForce GTX 780 com 2304 CUDA cores.
\end{itemize}

\newcommand{\figureDvndOrRvndDcDd}[7]{
% #1 {box, scatter}, #2 {count, imp, time}, #3 {instance number}, #4 {Tempo, Melhoria}, #5 tamanho instância, #6 {DVND, RVND}, #7 {dvnd, rvnd}
\begin{figure}%
    \centering
    \includegraphics[scale=0.9]{figuras/#7/dc_dd/#1/#7_#1100sol_#2_in#3.png}
    \caption{#4 do #6 para a instância #3 de tamanho #5. $m$ indica o número de máquinas, \textit{DC} refere-se ao #6 clássico e \textit{DD} ao #6 implementado em dataflow.}%
    \label{fig:#2_#7DcDd_in#3}%
\end{figure}
}

\newcommand{\figureDvndDcDd}[5]{
% #1 {box, scatter}, #2 {count, imp, time}, #3 {instance number}, #4 {Tempo, Melhoria}, #5 tamanho instância
    \figureDvndOrRvndDcDd{#1}{#2}{#3}{#4}{#5}{DVND}{dvnd}
}

\newcommand{\figureRvndDcDd}[5]{
% #1 {box, scatter}, #2 {count, imp, time}, #3 {instance number}, #4 {Tempo, Melhoria}, #5 tamanho instância
    \figureDvndOrRvndDcDd{#1}{#2}{#3}{#4}{#5}{RVND}{rvnd}
}

\newcommand{\tabelaEstatisticasGeral}[6]{
% #1 Descrição, #2 label, #3 {dvnd, rvnd}, #4 {DVND, RVND}, #6 DD/DC, #6 Conteúdo
\begin{table}[ht]
    \centering
    \begin{tabular}{c|ccc|cc|ccc|cc|c}
        \hline \hline
        \# & Tipo & $m$ & $n$ & $min$ & $max$ & 1Q & 2Q & 3Q & $\overline{x}$ & $\sigma$ & $p-valor$ \\ \hline
        #6
    \end{tabular}
    \caption{#1 #4 #5
        Instância (\#), tipo de implementação (Tipo), número de máquinas ($m$), tamanho da instância ($n$), valor mínimo ($min$), máximo ($max$), primeiro quartil (1Q), mediana (2Q), terceiro quartil (3Q), média ($\overline{x}$), desvio padrão ($\sigma$) e p-valor para o teste de Wilcox entre as versões (valores em negrito quando $p-valor > 0.05$).
    }
    \label{tab:#3DcDd#2}
\end{table}
}

\newcommand{\tabelaEstatisticas}[5]{
    \tabelaEstatisticasGeral{#1}{#2}{#3}{#4}{na implementação clássica (DC) e a proposta de implementação usando dataflow (DD).}{#5}
}

\newcommand{\figureDvndSogMog}[7]{
% #1 {box, scatter}, #2 {count, imp, time}, #3 {instance number}, #4 {Tempo, Melhoria}, #5 tamanho instância, #6 {DVND, RVND}, #7 {dvnd, rvnd}
\begin{figure}%
    \centering
    \includegraphics[scale=0.9]{figuras/#7/sog_mog/#1/#7_#1100sol_#2_in#3.png}
    \caption{#4 do #6 para a instância #3 de tamanho #5. \textit{SOG} refere-se a uma porta de saída e \textit{MOG} a múltiplas portas de saída.}%
    \label{fig:#2_#7SogMog_in#3}%
\end{figure}
}

\newcommand{\figureDvndGdvnd}[9]{
% #1 {box, scatter}, #2 {count, imp, time}, #3 {instance number}, #4 {Tempo, Melhoria}, #5 tamanho instância, #6 {DVND, RVND}, #7 {dvnd, rvnd}, #8 {man_time, full_time}, #9 {man, dvnd} #10 descricao
\begin{figure}%
    \centering
    \includegraphics[scale=0.9]{figuras/#7/#8/#1/#9_#1100sol_#2_in#3.png}
    \caption{#4 do #6 para a instância #3 de tamanho #5. \textit{DVND} refere-se ao tempo gasto pelo algoritmo de mesmo nome, para \textit{GDVND} é análogo ao anterior, no caso do \textit{GDVND-MAN} este se refere ao tempo do GDVND subtraido do tempo para gerenciar os movimentos.}%
    \label{fig:#2_#7_#8_in#3}%
\end{figure}
}

\newcommand{\figureDvndGdvndTime}[8]{
    \figureDvndGdvnd{#1}{#2}{#3}{#4}{#5}{#6}{#7}{man_time}{man}
}

\newcommand{\figureGdvndDvndRvnd}[9]{
% #1 {box, scatter}, #2 {count, imp, time}, #3 {instance number}, #4 {Tempo, Melhoria}, #5 tamanho instância, #6 {DVND, RVND}, #7 {dvnd, rvnd}, #8 {man_time, full_time}, #9 {man, dvnd} #10 descricao
\begin{figure}%
    \centering
    \includegraphics[scale=0.9]{figuras/#7/#8/#1/#9_#1100sol_#2_in#3.png}
    \caption{#4 do #6 para a instância #3 de tamanho #5. \textit{DVND}, \textit{GDVND} e \textit{RVND} referem-se ao tempo gasto pelos algoritmos de mesmo nome.}%
    \label{fig:#2_#7_#8_in#3}%
\end{figure}
}

% \subfloat[$m=#1$]{{ %scale=0.225
%         \includegraphics[scale=0.425]{figuras/dvnd/n#1/time#2.png}
%         \label{fig:timeDvndRvnd_n#1in#2}
%     }}%
% #1 {dvnd, rvnd, gdvnd}, #2 {sog_mog, dc_dd}, #3 {time, imp}, #4 in, #5 tamanho, #6 {box, scatter}
\newcommand{\subFig}[6]{
    \subfloat[][Instância #4, $n=#5$]{
        \includegraphics[scale=0.425]{figuras/#1/#2/#6/#1_#6100sol_#3_in#4.png}
		\label{fig:#1_#2_#3_in#4}
    }
% 	\begin{subfigure}{0.45\textwidth} % dvnd_box100sol_imp_in0
% 		\includegraphics[scale=0.425]{figuras/#1/#2/#6/#1_#6100sol_#3_in#4.png}
% 		\caption{Instância #4, $n=#5$}
        % \label{fig:#1_#2_#3_in#4}
    % \end{subfigure}
}

\newcommand{\subFigBox}[5]{
	\subFig{#1}{#2}{#3}{#4}{#5}{box}
}

\newcommand{\subFigScatter}[5]{
	\subFig{#1}{#2}{#3}{#4}{#5}{scatter}
}

% #1 {dvnd, rvnd, gdvnd}, #2 {sog_mog, dc_dd}, #3 {time, imp}, #4 {box, scatter}, #5 {Tempo do DVND...}
\newcommand{\multiFigureInstanciasGeral}[5]{
	\begin{figure}[ht]
		\centering
		\subFig{#1}{#2}{#3}{0}{52}{#4}
		~
		\subFig{#1}{#2}{#3}{1}{100}{#4}
		
		\subFig{#1}{#2}{#3}{2}{226}{#4}
		~
		\subFig{#1}{#2}{#3}{3}{318}{#4}
		\caption{#5 Instâncias 0 a 3.}
		\label{fig:#1_#2_#3_in0_4}
	\end{figure}
	
	\begin{figure}[ht]
		\centering
		\subFig{#1}{#2}{#3}{4}{501}{#4}
		~
		\subFig{#1}{#2}{#3}{5}{657}{#4}
		
		\subFig{#1}{#2}{#3}{6}{783}{#4}
		~
		\subFig{#1}{#2}{#3}{7}{1001}{#4}
		\caption{#5 Instâncias 5 a 7.}
		\label{fig:#1_#2_#3_in5_7}
	\end{figure}
}

% #1 {dvnd, rvnd, gdvnd}, #2 {sog_mog, dc_dd}, #3 {time, imp}, #4 {Tempo do DVND...}
\newcommand{\multiFigureInstancias}[4]{
    \multiFigureInstanciasGeral{#1}{#2}{#3}{box}{#4}
}


\chapter{Resultados} \label{cap:resultados}

Este capítulo exibe os resultados computacionais dos algoritmos propostos no Capítulo~\ref{cap:metodologia} para o caso do PML, para cada instância foi gerado um conjunto com 100 soluções iniciais aleatórias que foram submetidas aos métodos para comparação dos resultados.

Quando há referência à melhoria na solução (\textit{Imp}), esta melhoria pode ser calculada pelo quociente do valor da solução inicial pela solução final, ou seja:
\begin{equation}\label{eq:calculateImprovement}
Imp = \frac{f(\textrm{solução inicial})}{f(\textrm{solução final})}
\end{equation}

Desta forma quanto maior for o valor da melhoria ($Imp$) mais o método melhorou o valor da solução inicial.

\section{Instâncias} \label{sec:instancias}

Todas as instâncias usadas nos testes computacionais e cujas configurações de lançamento foram descritas na Tabela~\ref{tab:neighborhoodsLaunchConfigurarion} são as mesmas usadas em~\cite{wamca2016}.
Para o RVND foi feita uma implementação do algoritmo clássico (Algoritmo~\ref{alg:rvnd}) e também a implementação dataflow mencionada na Figura~\ref{fig:rvndGraph} fazendo uso de uma máquina.
Para o caso do DVND foi utilizada a implementação clássica (Algoritmo~\ref{alg:dvnd}) e a implementação dataflow proposta (Figura~\ref{fig:dvndGraph}), os resultados foram obtidos com diferentes números de máquinas e os mesmos são indicados conforme o caso.

\section{Implementação e ambiente computacional}\label{sec:amb}

A implementação para cada algoritmo proposto no Capítulo~\ref{cap:metodologia} utiliza a linguagem de programação \textit{C++11} em conjunto com a API CUDA\texttrademark, para a implementação dos grafos e do ambiente dataflow foi utilizada a biblioteca Sucuri~\cite{sucuri-original}\footnote{Disponível em \url{https://github.com/tiagoaoa/Sucuri}} implementada em Python, para a integração entre o dataflow e o código CUDA foi utilizada a biblioteca SimplePyCuda~\cite{simple-pycuda}\footnote{Disponível em \url{https://github.com/igormcoelho/simple-pycuda}}. As implementações com múltiplas threads usaram a biblioteca OpenMP.
%As implementações foram compiladas através do \textit{GCC} \textit{(GNU Compiler Collection)}\footnote{O GCC está disponível no seguinte sítio eletrônico: \url{https://gcc.gnu.org/}.} com a \textit{flag} de otimização $-O3$.
O ambiente computacional utilizado em todos os testes neste trabalho consiste de 4 máquinas com a seguinte configuração:

\begin{itemize}
    \item Processador Intel\textregistered Core\texttrademark i7-4820K 3.7 GHz (4 núcleos);
    \item 8 GB de memória RAM;
    \item Sistema Operacional Ubuntu 14.10 (x64);
    \item NVIDIA GeForce GTX 780 com 2304 CUDA cores.
\end{itemize}

\newcommand{\figureDvndOrRvndDcDd}[7]{
% #1 {box, scatter}, #2 {count, imp, time}, #3 {instance number}, #4 {Tempo, Melhoria}, #5 tamanho instância, #6 {DVND, RVND}, #7 {dvnd, rvnd}
\begin{figure}%
    \centering
    \includegraphics[scale=0.9]{figuras/#7/dc_dd/#1/#7_#1100sol_#2_in#3.png}
    \caption{#4 do #6 para a instância #3 de tamanho #5. $m$ indica o número de máquinas, \textit{DC} refere-se ao #6 clássico e \textit{DD} ao #6 implementado em dataflow.}%
    \label{fig:#2_#7DcDd_in#3}%
\end{figure}
}

\newcommand{\figureDvndDcDd}[5]{
% #1 {box, scatter}, #2 {count, imp, time}, #3 {instance number}, #4 {Tempo, Melhoria}, #5 tamanho instância
    \figureDvndOrRvndDcDd{#1}{#2}{#3}{#4}{#5}{DVND}{dvnd}
}

\newcommand{\figureRvndDcDd}[5]{
% #1 {box, scatter}, #2 {count, imp, time}, #3 {instance number}, #4 {Tempo, Melhoria}, #5 tamanho instância
    \figureDvndOrRvndDcDd{#1}{#2}{#3}{#4}{#5}{RVND}{rvnd}
}

\newcommand{\tabelaEstatisticasGeral}[6]{
% #1 Descrição, #2 label, #3 {dvnd, rvnd}, #4 {DVND, RVND}, #6 DD/DC, #6 Conteúdo
\begin{table}[ht]
    \centering
    \begin{tabular}{c|ccc|cc|ccc|cc|c}
        \hline \hline
        \# & Tipo & $m$ & $n$ & $min$ & $max$ & 1Q & 2Q & 3Q & $\overline{x}$ & $\sigma$ & $p-valor$ \\ \hline
        #6
    \end{tabular}
    \caption{#1 #4 #5
        Instância (\#), tipo de implementação (Tipo), número de máquinas ($m$), tamanho da instância ($n$), valor mínimo ($min$), máximo ($max$), primeiro quartil (1Q), mediana (2Q), terceiro quartil (3Q), média ($\overline{x}$), desvio padrão ($\sigma$) e p-valor para o teste de Wilcox entre as versões (valores em negrito quando $p-valor > 0.05$).
    }
    \label{tab:#3DcDd#2}
\end{table}
}

\newcommand{\tabelaEstatisticas}[5]{
    \tabelaEstatisticasGeral{#1}{#2}{#3}{#4}{na implementação clássica (DC) e a proposta de implementação usando dataflow (DD).}{#5}
}

\newcommand{\figureDvndSogMog}[7]{
% #1 {box, scatter}, #2 {count, imp, time}, #3 {instance number}, #4 {Tempo, Melhoria}, #5 tamanho instância, #6 {DVND, RVND}, #7 {dvnd, rvnd}
\begin{figure}%
    \centering
    \includegraphics[scale=0.9]{figuras/#7/sog_mog/#1/#7_#1100sol_#2_in#3.png}
    \caption{#4 do #6 para a instância #3 de tamanho #5. \textit{SOG} refere-se a uma porta de saída e \textit{MOG} a múltiplas portas de saída.}%
    \label{fig:#2_#7SogMog_in#3}%
\end{figure}
}

\newcommand{\figureDvndGdvnd}[9]{
% #1 {box, scatter}, #2 {count, imp, time}, #3 {instance number}, #4 {Tempo, Melhoria}, #5 tamanho instância, #6 {DVND, RVND}, #7 {dvnd, rvnd}, #8 {man_time, full_time}, #9 {man, dvnd} #10 descricao
\begin{figure}%
    \centering
    \includegraphics[scale=0.9]{figuras/#7/#8/#1/#9_#1100sol_#2_in#3.png}
    \caption{#4 do #6 para a instância #3 de tamanho #5. \textit{DVND} refere-se ao tempo gasto pelo algoritmo de mesmo nome, para \textit{GDVND} é análogo ao anterior, no caso do \textit{GDVND-MAN} este se refere ao tempo do GDVND subtraido do tempo para gerenciar os movimentos.}%
    \label{fig:#2_#7_#8_in#3}%
\end{figure}
}

\newcommand{\figureDvndGdvndTime}[8]{
    \figureDvndGdvnd{#1}{#2}{#3}{#4}{#5}{#6}{#7}{man_time}{man}
}

\newcommand{\figureGdvndDvndRvnd}[9]{
% #1 {box, scatter}, #2 {count, imp, time}, #3 {instance number}, #4 {Tempo, Melhoria}, #5 tamanho instância, #6 {DVND, RVND}, #7 {dvnd, rvnd}, #8 {man_time, full_time}, #9 {man, dvnd} #10 descricao
\begin{figure}%
    \centering
    \includegraphics[scale=0.9]{figuras/#7/#8/#1/#9_#1100sol_#2_in#3.png}
    \caption{#4 do #6 para a instância #3 de tamanho #5. \textit{DVND}, \textit{GDVND} e \textit{RVND} referem-se ao tempo gasto pelos algoritmos de mesmo nome.}%
    \label{fig:#2_#7_#8_in#3}%
\end{figure}
}

% \subfloat[$m=#1$]{{ %scale=0.225
%         \includegraphics[scale=0.425]{figuras/dvnd/n#1/time#2.png}
%         \label{fig:timeDvndRvnd_n#1in#2}
%     }}%
% #1 {dvnd, rvnd, gdvnd}, #2 {sog_mog, dc_dd}, #3 {time, imp}, #4 in, #5 tamanho, #6 {box, scatter}
\newcommand{\subFig}[6]{
    \subfloat[][Instância #4, $n=#5$]{
        \includegraphics[scale=0.425]{figuras/#1/#2/#6/#1_#6100sol_#3_in#4.png}
		\label{fig:#1_#2_#3_in#4}
    }
% 	\begin{subfigure}{0.45\textwidth} % dvnd_box100sol_imp_in0
% 		\includegraphics[scale=0.425]{figuras/#1/#2/#6/#1_#6100sol_#3_in#4.png}
% 		\caption{Instância #4, $n=#5$}
        % \label{fig:#1_#2_#3_in#4}
    % \end{subfigure}
}

\newcommand{\subFigBox}[5]{
	\subFig{#1}{#2}{#3}{#4}{#5}{box}
}

\newcommand{\subFigScatter}[5]{
	\subFig{#1}{#2}{#3}{#4}{#5}{scatter}
}

% #1 {dvnd, rvnd, gdvnd}, #2 {sog_mog, dc_dd}, #3 {time, imp}, #4 {box, scatter}, #5 {Tempo do DVND...}
\newcommand{\multiFigureInstanciasGeral}[5]{
	\begin{figure}[ht]
		\centering
		\subFig{#1}{#2}{#3}{0}{52}{#4}
		~
		\subFig{#1}{#2}{#3}{1}{100}{#4}
		
		\subFig{#1}{#2}{#3}{2}{226}{#4}
		~
		\subFig{#1}{#2}{#3}{3}{318}{#4}
		\caption{#5 Instâncias 0 a 3.}
		\label{fig:#1_#2_#3_in0_4}
	\end{figure}
	
	\begin{figure}[ht]
		\centering
		\subFig{#1}{#2}{#3}{4}{501}{#4}
		~
		\subFig{#1}{#2}{#3}{5}{657}{#4}
		
		\subFig{#1}{#2}{#3}{6}{783}{#4}
		~
		\subFig{#1}{#2}{#3}{7}{1001}{#4}
		\caption{#5 Instâncias 5 a 7.}
		\label{fig:#1_#2_#3_in5_7}
	\end{figure}
}

% #1 {dvnd, rvnd, gdvnd}, #2 {sog_mog, dc_dd}, #3 {time, imp}, #4 {Tempo do DVND...}
\newcommand{\multiFigureInstancias}[4]{
    \multiFigureInstanciasGeral{#1}{#2}{#3}{box}{#4}
}


\chapter{Resultados} \label{cap:resultados}

Este capítulo exibe os resultados computacionais dos algoritmos propostos no Capítulo~\ref{cap:metodologia} para o caso do PML, para cada instância foi gerado um conjunto com 100 soluções iniciais aleatórias que foram submetidas aos métodos para comparação dos resultados.

Quando há referência à melhoria na solução (\textit{Imp}), esta melhoria pode ser calculada pelo quociente do valor da solução inicial pela solução final, ou seja:
\begin{equation}\label{eq:calculateImprovement}
Imp = \frac{f(\textrm{solução inicial})}{f(\textrm{solução final})}
\end{equation}

Desta forma quanto maior for o valor da melhoria ($Imp$) mais o método melhorou o valor da solução inicial.

\section{Instâncias} \label{sec:instancias}

Todas as instâncias usadas nos testes computacionais e cujas configurações de lançamento foram descritas na Tabela~\ref{tab:neighborhoodsLaunchConfigurarion} são as mesmas usadas em~\cite{wamca2016}.
Para o RVND foi feita uma implementação do algoritmo clássico (Algoritmo~\ref{alg:rvnd}) e também a implementação dataflow mencionada na Figura~\ref{fig:rvndGraph} fazendo uso de uma máquina.
Para o caso do DVND foi utilizada a implementação clássica (Algoritmo~\ref{alg:dvnd}) e a implementação dataflow proposta (Figura~\ref{fig:dvndGraph}), os resultados foram obtidos com diferentes números de máquinas e os mesmos são indicados conforme o caso.

\section{Implementação e ambiente computacional}\label{sec:amb}

A implementação para cada algoritmo proposto no Capítulo~\ref{cap:metodologia} utiliza a linguagem de programação \textit{C++11} em conjunto com a API CUDA\texttrademark, para a implementação dos grafos e do ambiente dataflow foi utilizada a biblioteca Sucuri~\cite{sucuri-original}\footnote{Disponível em \url{https://github.com/tiagoaoa/Sucuri}} implementada em Python, para a integração entre o dataflow e o código CUDA foi utilizada a biblioteca SimplePyCuda~\cite{simple-pycuda}\footnote{Disponível em \url{https://github.com/igormcoelho/simple-pycuda}}. As implementações com múltiplas threads usaram a biblioteca OpenMP.
%As implementações foram compiladas através do \textit{GCC} \textit{(GNU Compiler Collection)}\footnote{O GCC está disponível no seguinte sítio eletrônico: \url{https://gcc.gnu.org/}.} com a \textit{flag} de otimização $-O3$.
O ambiente computacional utilizado em todos os testes neste trabalho consiste de 4 máquinas com a seguinte configuração:

\begin{itemize}
    \item Processador Intel\textregistered Core\texttrademark i7-4820K 3.7 GHz (4 núcleos);
    \item 8 GB de memória RAM;
    \item Sistema Operacional Ubuntu 14.10 (x64);
    \item NVIDIA GeForce GTX 780 com 2304 CUDA cores.
\end{itemize}

\newcommand{\figureDvndOrRvndDcDd}[7]{
% #1 {box, scatter}, #2 {count, imp, time}, #3 {instance number}, #4 {Tempo, Melhoria}, #5 tamanho instância, #6 {DVND, RVND}, #7 {dvnd, rvnd}
\begin{figure}%
    \centering
    \includegraphics[scale=0.9]{figuras/#7/dc_dd/#1/#7_#1100sol_#2_in#3.png}
    \caption{#4 do #6 para a instância #3 de tamanho #5. $m$ indica o número de máquinas, \textit{DC} refere-se ao #6 clássico e \textit{DD} ao #6 implementado em dataflow.}%
    \label{fig:#2_#7DcDd_in#3}%
\end{figure}
}

\newcommand{\figureDvndDcDd}[5]{
% #1 {box, scatter}, #2 {count, imp, time}, #3 {instance number}, #4 {Tempo, Melhoria}, #5 tamanho instância
    \figureDvndOrRvndDcDd{#1}{#2}{#3}{#4}{#5}{DVND}{dvnd}
}

\newcommand{\figureRvndDcDd}[5]{
% #1 {box, scatter}, #2 {count, imp, time}, #3 {instance number}, #4 {Tempo, Melhoria}, #5 tamanho instância
    \figureDvndOrRvndDcDd{#1}{#2}{#3}{#4}{#5}{RVND}{rvnd}
}

\newcommand{\tabelaEstatisticasGeral}[6]{
% #1 Descrição, #2 label, #3 {dvnd, rvnd}, #4 {DVND, RVND}, #6 DD/DC, #6 Conteúdo
\begin{table}[ht]
    \centering
    \begin{tabular}{c|ccc|cc|ccc|cc|c}
        \hline \hline
        \# & Tipo & $m$ & $n$ & $min$ & $max$ & 1Q & 2Q & 3Q & $\overline{x}$ & $\sigma$ & $p-valor$ \\ \hline
        #6
    \end{tabular}
    \caption{#1 #4 #5
        Instância (\#), tipo de implementação (Tipo), número de máquinas ($m$), tamanho da instância ($n$), valor mínimo ($min$), máximo ($max$), primeiro quartil (1Q), mediana (2Q), terceiro quartil (3Q), média ($\overline{x}$), desvio padrão ($\sigma$) e p-valor para o teste de Wilcox entre as versões (valores em negrito quando $p-valor > 0.05$).
    }
    \label{tab:#3DcDd#2}
\end{table}
}

\newcommand{\tabelaEstatisticas}[5]{
    \tabelaEstatisticasGeral{#1}{#2}{#3}{#4}{na implementação clássica (DC) e a proposta de implementação usando dataflow (DD).}{#5}
}

\newcommand{\figureDvndSogMog}[7]{
% #1 {box, scatter}, #2 {count, imp, time}, #3 {instance number}, #4 {Tempo, Melhoria}, #5 tamanho instância, #6 {DVND, RVND}, #7 {dvnd, rvnd}
\begin{figure}%
    \centering
    \includegraphics[scale=0.9]{figuras/#7/sog_mog/#1/#7_#1100sol_#2_in#3.png}
    \caption{#4 do #6 para a instância #3 de tamanho #5. \textit{SOG} refere-se a uma porta de saída e \textit{MOG} a múltiplas portas de saída.}%
    \label{fig:#2_#7SogMog_in#3}%
\end{figure}
}

\newcommand{\figureDvndGdvnd}[9]{
% #1 {box, scatter}, #2 {count, imp, time}, #3 {instance number}, #4 {Tempo, Melhoria}, #5 tamanho instância, #6 {DVND, RVND}, #7 {dvnd, rvnd}, #8 {man_time, full_time}, #9 {man, dvnd} #10 descricao
\begin{figure}%
    \centering
    \includegraphics[scale=0.9]{figuras/#7/#8/#1/#9_#1100sol_#2_in#3.png}
    \caption{#4 do #6 para a instância #3 de tamanho #5. \textit{DVND} refere-se ao tempo gasto pelo algoritmo de mesmo nome, para \textit{GDVND} é análogo ao anterior, no caso do \textit{GDVND-MAN} este se refere ao tempo do GDVND subtraido do tempo para gerenciar os movimentos.}%
    \label{fig:#2_#7_#8_in#3}%
\end{figure}
}

\newcommand{\figureDvndGdvndTime}[8]{
    \figureDvndGdvnd{#1}{#2}{#3}{#4}{#5}{#6}{#7}{man_time}{man}
}

\newcommand{\figureGdvndDvndRvnd}[9]{
% #1 {box, scatter}, #2 {count, imp, time}, #3 {instance number}, #4 {Tempo, Melhoria}, #5 tamanho instância, #6 {DVND, RVND}, #7 {dvnd, rvnd}, #8 {man_time, full_time}, #9 {man, dvnd} #10 descricao
\begin{figure}%
    \centering
    \includegraphics[scale=0.9]{figuras/#7/#8/#1/#9_#1100sol_#2_in#3.png}
    \caption{#4 do #6 para a instância #3 de tamanho #5. \textit{DVND}, \textit{GDVND} e \textit{RVND} referem-se ao tempo gasto pelos algoritmos de mesmo nome.}%
    \label{fig:#2_#7_#8_in#3}%
\end{figure}
}

% \subfloat[$m=#1$]{{ %scale=0.225
%         \includegraphics[scale=0.425]{figuras/dvnd/n#1/time#2.png}
%         \label{fig:timeDvndRvnd_n#1in#2}
%     }}%
% #1 {dvnd, rvnd, gdvnd}, #2 {sog_mog, dc_dd}, #3 {time, imp}, #4 in, #5 tamanho, #6 {box, scatter}
\newcommand{\subFig}[6]{
    \subfloat[][Instância #4, $n=#5$]{
        \includegraphics[scale=0.425]{figuras/#1/#2/#6/#1_#6100sol_#3_in#4.png}
		\label{fig:#1_#2_#3_in#4}
    }
% 	\begin{subfigure}{0.45\textwidth} % dvnd_box100sol_imp_in0
% 		\includegraphics[scale=0.425]{figuras/#1/#2/#6/#1_#6100sol_#3_in#4.png}
% 		\caption{Instância #4, $n=#5$}
        % \label{fig:#1_#2_#3_in#4}
    % \end{subfigure}
}

\newcommand{\subFigBox}[5]{
	\subFig{#1}{#2}{#3}{#4}{#5}{box}
}

\newcommand{\subFigScatter}[5]{
	\subFig{#1}{#2}{#3}{#4}{#5}{scatter}
}

% #1 {dvnd, rvnd, gdvnd}, #2 {sog_mog, dc_dd}, #3 {time, imp}, #4 {box, scatter}, #5 {Tempo do DVND...}
\newcommand{\multiFigureInstanciasGeral}[5]{
	\begin{figure}[ht]
		\centering
		\subFig{#1}{#2}{#3}{0}{52}{#4}
		~
		\subFig{#1}{#2}{#3}{1}{100}{#4}
		
		\subFig{#1}{#2}{#3}{2}{226}{#4}
		~
		\subFig{#1}{#2}{#3}{3}{318}{#4}
		\caption{#5 Instâncias 0 a 3.}
		\label{fig:#1_#2_#3_in0_4}
	\end{figure}
	
	\begin{figure}[ht]
		\centering
		\subFig{#1}{#2}{#3}{4}{501}{#4}
		~
		\subFig{#1}{#2}{#3}{5}{657}{#4}
		
		\subFig{#1}{#2}{#3}{6}{783}{#4}
		~
		\subFig{#1}{#2}{#3}{7}{1001}{#4}
		\caption{#5 Instâncias 5 a 7.}
		\label{fig:#1_#2_#3_in5_7}
	\end{figure}
}

% #1 {dvnd, rvnd, gdvnd}, #2 {sog_mog, dc_dd}, #3 {time, imp}, #4 {Tempo do DVND...}
\newcommand{\multiFigureInstancias}[4]{
    \multiFigureInstanciasGeral{#1}{#2}{#3}{box}{#4}
}


\chapter{Resultados} \label{cap:resultados}

Este capítulo exibe os resultados computacionais dos algoritmos propostos no Capítulo~\ref{cap:metodologia} para o caso do PML, para cada instância foi gerado um conjunto com 100 soluções iniciais aleatórias que foram submetidas aos métodos para comparação dos resultados.

Quando há referência à melhoria na solução (\textit{Imp}), esta melhoria pode ser calculada pelo quociente do valor da solução inicial pela solução final, ou seja:
\begin{equation}\label{eq:calculateImprovement}
Imp = \frac{f(\textrm{solução inicial})}{f(\textrm{solução final})}
\end{equation}

Desta forma quanto maior for o valor da melhoria ($Imp$) mais o método melhorou o valor da solução inicial.

\section{Instâncias} \label{sec:instancias}

Todas as instâncias usadas nos testes computacionais e cujas configurações de lançamento foram descritas na Tabela~\ref{tab:neighborhoodsLaunchConfigurarion} são as mesmas usadas em~\cite{wamca2016}.
Para o RVND foi feita uma implementação do algoritmo clássico (Algoritmo~\ref{alg:rvnd}) e também a implementação dataflow mencionada na Figura~\ref{fig:rvndGraph} fazendo uso de uma máquina.
Para o caso do DVND foi utilizada a implementação clássica (Algoritmo~\ref{alg:dvnd}) e a implementação dataflow proposta (Figura~\ref{fig:dvndGraph}), os resultados foram obtidos com diferentes números de máquinas e os mesmos são indicados conforme o caso.

\section{Implementação e ambiente computacional}\label{sec:amb}

A implementação para cada algoritmo proposto no Capítulo~\ref{cap:metodologia} utiliza a linguagem de programação \textit{C++11} em conjunto com a API CUDA\texttrademark, para a implementação dos grafos e do ambiente dataflow foi utilizada a biblioteca Sucuri~\cite{sucuri-original}\footnote{Disponível em \url{https://github.com/tiagoaoa/Sucuri}} implementada em Python, para a integração entre o dataflow e o código CUDA foi utilizada a biblioteca SimplePyCuda~\cite{simple-pycuda}\footnote{Disponível em \url{https://github.com/igormcoelho/simple-pycuda}}. As implementações com múltiplas threads usaram a biblioteca OpenMP.
%As implementações foram compiladas através do \textit{GCC} \textit{(GNU Compiler Collection)}\footnote{O GCC está disponível no seguinte sítio eletrônico: \url{https://gcc.gnu.org/}.} com a \textit{flag} de otimização $-O3$.
O ambiente computacional utilizado em todos os testes neste trabalho consiste de 4 máquinas com a seguinte configuração:

\begin{itemize}
    \item Processador Intel\textregistered Core\texttrademark i7-4820K 3.7 GHz (4 núcleos);
    \item 8 GB de memória RAM;
    \item Sistema Operacional Ubuntu 14.10 (x64);
    \item NVIDIA GeForce GTX 780 com 2304 CUDA cores.
\end{itemize}

\input{resultados/comandosFiguras.tex}

\input{resultados/sog_mog/index.tex}

\input{resultados/rvnd/index.tex}

\input{resultados/dvnd/index.tex}

\input{resultados/gdvnd/index.tex}


\chapter{Resultados} \label{cap:resultados}

Este capítulo exibe os resultados computacionais dos algoritmos propostos no Capítulo~\ref{cap:metodologia} para o caso do PML, para cada instância foi gerado um conjunto com 100 soluções iniciais aleatórias que foram submetidas aos métodos para comparação dos resultados.

Quando há referência à melhoria na solução (\textit{Imp}), esta melhoria pode ser calculada pelo quociente do valor da solução inicial pela solução final, ou seja:
\begin{equation}\label{eq:calculateImprovement}
Imp = \frac{f(\textrm{solução inicial})}{f(\textrm{solução final})}
\end{equation}

Desta forma quanto maior for o valor da melhoria ($Imp$) mais o método melhorou o valor da solução inicial.

\section{Instâncias} \label{sec:instancias}

Todas as instâncias usadas nos testes computacionais e cujas configurações de lançamento foram descritas na Tabela~\ref{tab:neighborhoodsLaunchConfigurarion} são as mesmas usadas em~\cite{wamca2016}.
Para o RVND foi feita uma implementação do algoritmo clássico (Algoritmo~\ref{alg:rvnd}) e também a implementação dataflow mencionada na Figura~\ref{fig:rvndGraph} fazendo uso de uma máquina.
Para o caso do DVND foi utilizada a implementação clássica (Algoritmo~\ref{alg:dvnd}) e a implementação dataflow proposta (Figura~\ref{fig:dvndGraph}), os resultados foram obtidos com diferentes números de máquinas e os mesmos são indicados conforme o caso.

\section{Implementação e ambiente computacional}\label{sec:amb}

A implementação para cada algoritmo proposto no Capítulo~\ref{cap:metodologia} utiliza a linguagem de programação \textit{C++11} em conjunto com a API CUDA\texttrademark, para a implementação dos grafos e do ambiente dataflow foi utilizada a biblioteca Sucuri~\cite{sucuri-original}\footnote{Disponível em \url{https://github.com/tiagoaoa/Sucuri}} implementada em Python, para a integração entre o dataflow e o código CUDA foi utilizada a biblioteca SimplePyCuda~\cite{simple-pycuda}\footnote{Disponível em \url{https://github.com/igormcoelho/simple-pycuda}}. As implementações com múltiplas threads usaram a biblioteca OpenMP.
%As implementações foram compiladas através do \textit{GCC} \textit{(GNU Compiler Collection)}\footnote{O GCC está disponível no seguinte sítio eletrônico: \url{https://gcc.gnu.org/}.} com a \textit{flag} de otimização $-O3$.
O ambiente computacional utilizado em todos os testes neste trabalho consiste de 4 máquinas com a seguinte configuração:

\begin{itemize}
    \item Processador Intel\textregistered Core\texttrademark i7-4820K 3.7 GHz (4 núcleos);
    \item 8 GB de memória RAM;
    \item Sistema Operacional Ubuntu 14.10 (x64);
    \item NVIDIA GeForce GTX 780 com 2304 CUDA cores.
\end{itemize}

\input{resultados/comandosFiguras.tex}

\input{resultados/sog_mog/index.tex}

\input{resultados/rvnd/index.tex}

\input{resultados/dvnd/index.tex}

\input{resultados/gdvnd/index.tex}


\chapter{Resultados} \label{cap:resultados}

Este capítulo exibe os resultados computacionais dos algoritmos propostos no Capítulo~\ref{cap:metodologia} para o caso do PML, para cada instância foi gerado um conjunto com 100 soluções iniciais aleatórias que foram submetidas aos métodos para comparação dos resultados.

Quando há referência à melhoria na solução (\textit{Imp}), esta melhoria pode ser calculada pelo quociente do valor da solução inicial pela solução final, ou seja:
\begin{equation}\label{eq:calculateImprovement}
Imp = \frac{f(\textrm{solução inicial})}{f(\textrm{solução final})}
\end{equation}

Desta forma quanto maior for o valor da melhoria ($Imp$) mais o método melhorou o valor da solução inicial.

\section{Instâncias} \label{sec:instancias}

Todas as instâncias usadas nos testes computacionais e cujas configurações de lançamento foram descritas na Tabela~\ref{tab:neighborhoodsLaunchConfigurarion} são as mesmas usadas em~\cite{wamca2016}.
Para o RVND foi feita uma implementação do algoritmo clássico (Algoritmo~\ref{alg:rvnd}) e também a implementação dataflow mencionada na Figura~\ref{fig:rvndGraph} fazendo uso de uma máquina.
Para o caso do DVND foi utilizada a implementação clássica (Algoritmo~\ref{alg:dvnd}) e a implementação dataflow proposta (Figura~\ref{fig:dvndGraph}), os resultados foram obtidos com diferentes números de máquinas e os mesmos são indicados conforme o caso.

\section{Implementação e ambiente computacional}\label{sec:amb}

A implementação para cada algoritmo proposto no Capítulo~\ref{cap:metodologia} utiliza a linguagem de programação \textit{C++11} em conjunto com a API CUDA\texttrademark, para a implementação dos grafos e do ambiente dataflow foi utilizada a biblioteca Sucuri~\cite{sucuri-original}\footnote{Disponível em \url{https://github.com/tiagoaoa/Sucuri}} implementada em Python, para a integração entre o dataflow e o código CUDA foi utilizada a biblioteca SimplePyCuda~\cite{simple-pycuda}\footnote{Disponível em \url{https://github.com/igormcoelho/simple-pycuda}}. As implementações com múltiplas threads usaram a biblioteca OpenMP.
%As implementações foram compiladas através do \textit{GCC} \textit{(GNU Compiler Collection)}\footnote{O GCC está disponível no seguinte sítio eletrônico: \url{https://gcc.gnu.org/}.} com a \textit{flag} de otimização $-O3$.
O ambiente computacional utilizado em todos os testes neste trabalho consiste de 4 máquinas com a seguinte configuração:

\begin{itemize}
    \item Processador Intel\textregistered Core\texttrademark i7-4820K 3.7 GHz (4 núcleos);
    \item 8 GB de memória RAM;
    \item Sistema Operacional Ubuntu 14.10 (x64);
    \item NVIDIA GeForce GTX 780 com 2304 CUDA cores.
\end{itemize}

\input{resultados/comandosFiguras.tex}

\input{resultados/sog_mog/index.tex}

\input{resultados/rvnd/index.tex}

\input{resultados/dvnd/index.tex}

\input{resultados/gdvnd/index.tex}


\chapter{Resultados} \label{cap:resultados}

Este capítulo exibe os resultados computacionais dos algoritmos propostos no Capítulo~\ref{cap:metodologia} para o caso do PML, para cada instância foi gerado um conjunto com 100 soluções iniciais aleatórias que foram submetidas aos métodos para comparação dos resultados.

Quando há referência à melhoria na solução (\textit{Imp}), esta melhoria pode ser calculada pelo quociente do valor da solução inicial pela solução final, ou seja:
\begin{equation}\label{eq:calculateImprovement}
Imp = \frac{f(\textrm{solução inicial})}{f(\textrm{solução final})}
\end{equation}

Desta forma quanto maior for o valor da melhoria ($Imp$) mais o método melhorou o valor da solução inicial.

\section{Instâncias} \label{sec:instancias}

Todas as instâncias usadas nos testes computacionais e cujas configurações de lançamento foram descritas na Tabela~\ref{tab:neighborhoodsLaunchConfigurarion} são as mesmas usadas em~\cite{wamca2016}.
Para o RVND foi feita uma implementação do algoritmo clássico (Algoritmo~\ref{alg:rvnd}) e também a implementação dataflow mencionada na Figura~\ref{fig:rvndGraph} fazendo uso de uma máquina.
Para o caso do DVND foi utilizada a implementação clássica (Algoritmo~\ref{alg:dvnd}) e a implementação dataflow proposta (Figura~\ref{fig:dvndGraph}), os resultados foram obtidos com diferentes números de máquinas e os mesmos são indicados conforme o caso.

\section{Implementação e ambiente computacional}\label{sec:amb}

A implementação para cada algoritmo proposto no Capítulo~\ref{cap:metodologia} utiliza a linguagem de programação \textit{C++11} em conjunto com a API CUDA\texttrademark, para a implementação dos grafos e do ambiente dataflow foi utilizada a biblioteca Sucuri~\cite{sucuri-original}\footnote{Disponível em \url{https://github.com/tiagoaoa/Sucuri}} implementada em Python, para a integração entre o dataflow e o código CUDA foi utilizada a biblioteca SimplePyCuda~\cite{simple-pycuda}\footnote{Disponível em \url{https://github.com/igormcoelho/simple-pycuda}}. As implementações com múltiplas threads usaram a biblioteca OpenMP.
%As implementações foram compiladas através do \textit{GCC} \textit{(GNU Compiler Collection)}\footnote{O GCC está disponível no seguinte sítio eletrônico: \url{https://gcc.gnu.org/}.} com a \textit{flag} de otimização $-O3$.
O ambiente computacional utilizado em todos os testes neste trabalho consiste de 4 máquinas com a seguinte configuração:

\begin{itemize}
    \item Processador Intel\textregistered Core\texttrademark i7-4820K 3.7 GHz (4 núcleos);
    \item 8 GB de memória RAM;
    \item Sistema Operacional Ubuntu 14.10 (x64);
    \item NVIDIA GeForce GTX 780 com 2304 CUDA cores.
\end{itemize}

\input{resultados/comandosFiguras.tex}

\input{resultados/sog_mog/index.tex}

\input{resultados/rvnd/index.tex}

\input{resultados/dvnd/index.tex}

\input{resultados/gdvnd/index.tex}



\chapter{Resultados} \label{cap:resultados}

Este capítulo exibe os resultados computacionais dos algoritmos propostos no Capítulo~\ref{cap:metodologia} para o caso do PML, para cada instância foi gerado um conjunto com 100 soluções iniciais aleatórias que foram submetidas aos métodos para comparação dos resultados.

Quando há referência à melhoria na solução (\textit{Imp}), esta melhoria pode ser calculada pelo quociente do valor da solução inicial pela solução final, ou seja:
\begin{equation}\label{eq:calculateImprovement}
Imp = \frac{f(\textrm{solução inicial})}{f(\textrm{solução final})}
\end{equation}

Desta forma quanto maior for o valor da melhoria ($Imp$) mais o método melhorou o valor da solução inicial.

\section{Instâncias} \label{sec:instancias}

Todas as instâncias usadas nos testes computacionais e cujas configurações de lançamento foram descritas na Tabela~\ref{tab:neighborhoodsLaunchConfigurarion} são as mesmas usadas em~\cite{wamca2016}.
Para o RVND foi feita uma implementação do algoritmo clássico (Algoritmo~\ref{alg:rvnd}) e também a implementação dataflow mencionada na Figura~\ref{fig:rvndGraph} fazendo uso de uma máquina.
Para o caso do DVND foi utilizada a implementação clássica (Algoritmo~\ref{alg:dvnd}) e a implementação dataflow proposta (Figura~\ref{fig:dvndGraph}), os resultados foram obtidos com diferentes números de máquinas e os mesmos são indicados conforme o caso.

\section{Implementação e ambiente computacional}\label{sec:amb}

A implementação para cada algoritmo proposto no Capítulo~\ref{cap:metodologia} utiliza a linguagem de programação \textit{C++11} em conjunto com a API CUDA\texttrademark, para a implementação dos grafos e do ambiente dataflow foi utilizada a biblioteca Sucuri~\cite{sucuri-original}\footnote{Disponível em \url{https://github.com/tiagoaoa/Sucuri}} implementada em Python, para a integração entre o dataflow e o código CUDA foi utilizada a biblioteca SimplePyCuda~\cite{simple-pycuda}\footnote{Disponível em \url{https://github.com/igormcoelho/simple-pycuda}}. As implementações com múltiplas threads usaram a biblioteca OpenMP.
%As implementações foram compiladas através do \textit{GCC} \textit{(GNU Compiler Collection)}\footnote{O GCC está disponível no seguinte sítio eletrônico: \url{https://gcc.gnu.org/}.} com a \textit{flag} de otimização $-O3$.
O ambiente computacional utilizado em todos os testes neste trabalho consiste de 4 máquinas com a seguinte configuração:

\begin{itemize}
    \item Processador Intel\textregistered Core\texttrademark i7-4820K 3.7 GHz (4 núcleos);
    \item 8 GB de memória RAM;
    \item Sistema Operacional Ubuntu 14.10 (x64);
    \item NVIDIA GeForce GTX 780 com 2304 CUDA cores.
\end{itemize}

\newcommand{\figureDvndOrRvndDcDd}[7]{
% #1 {box, scatter}, #2 {count, imp, time}, #3 {instance number}, #4 {Tempo, Melhoria}, #5 tamanho instância, #6 {DVND, RVND}, #7 {dvnd, rvnd}
\begin{figure}%
    \centering
    \includegraphics[scale=0.9]{figuras/#7/dc_dd/#1/#7_#1100sol_#2_in#3.png}
    \caption{#4 do #6 para a instância #3 de tamanho #5. $m$ indica o número de máquinas, \textit{DC} refere-se ao #6 clássico e \textit{DD} ao #6 implementado em dataflow.}%
    \label{fig:#2_#7DcDd_in#3}%
\end{figure}
}

\newcommand{\figureDvndDcDd}[5]{
% #1 {box, scatter}, #2 {count, imp, time}, #3 {instance number}, #4 {Tempo, Melhoria}, #5 tamanho instância
    \figureDvndOrRvndDcDd{#1}{#2}{#3}{#4}{#5}{DVND}{dvnd}
}

\newcommand{\figureRvndDcDd}[5]{
% #1 {box, scatter}, #2 {count, imp, time}, #3 {instance number}, #4 {Tempo, Melhoria}, #5 tamanho instância
    \figureDvndOrRvndDcDd{#1}{#2}{#3}{#4}{#5}{RVND}{rvnd}
}

\newcommand{\tabelaEstatisticasGeral}[6]{
% #1 Descrição, #2 label, #3 {dvnd, rvnd}, #4 {DVND, RVND}, #6 DD/DC, #6 Conteúdo
\begin{table}[ht]
    \centering
    \begin{tabular}{c|ccc|cc|ccc|cc|c}
        \hline \hline
        \# & Tipo & $m$ & $n$ & $min$ & $max$ & 1Q & 2Q & 3Q & $\overline{x}$ & $\sigma$ & $p-valor$ \\ \hline
        #6
    \end{tabular}
    \caption{#1 #4 #5
        Instância (\#), tipo de implementação (Tipo), número de máquinas ($m$), tamanho da instância ($n$), valor mínimo ($min$), máximo ($max$), primeiro quartil (1Q), mediana (2Q), terceiro quartil (3Q), média ($\overline{x}$), desvio padrão ($\sigma$) e p-valor para o teste de Wilcox entre as versões (valores em negrito quando $p-valor > 0.05$).
    }
    \label{tab:#3DcDd#2}
\end{table}
}

\newcommand{\tabelaEstatisticas}[5]{
    \tabelaEstatisticasGeral{#1}{#2}{#3}{#4}{na implementação clássica (DC) e a proposta de implementação usando dataflow (DD).}{#5}
}

\newcommand{\figureDvndSogMog}[7]{
% #1 {box, scatter}, #2 {count, imp, time}, #3 {instance number}, #4 {Tempo, Melhoria}, #5 tamanho instância, #6 {DVND, RVND}, #7 {dvnd, rvnd}
\begin{figure}%
    \centering
    \includegraphics[scale=0.9]{figuras/#7/sog_mog/#1/#7_#1100sol_#2_in#3.png}
    \caption{#4 do #6 para a instância #3 de tamanho #5. \textit{SOG} refere-se a uma porta de saída e \textit{MOG} a múltiplas portas de saída.}%
    \label{fig:#2_#7SogMog_in#3}%
\end{figure}
}

\newcommand{\figureDvndGdvnd}[9]{
% #1 {box, scatter}, #2 {count, imp, time}, #3 {instance number}, #4 {Tempo, Melhoria}, #5 tamanho instância, #6 {DVND, RVND}, #7 {dvnd, rvnd}, #8 {man_time, full_time}, #9 {man, dvnd} #10 descricao
\begin{figure}%
    \centering
    \includegraphics[scale=0.9]{figuras/#7/#8/#1/#9_#1100sol_#2_in#3.png}
    \caption{#4 do #6 para a instância #3 de tamanho #5. \textit{DVND} refere-se ao tempo gasto pelo algoritmo de mesmo nome, para \textit{GDVND} é análogo ao anterior, no caso do \textit{GDVND-MAN} este se refere ao tempo do GDVND subtraido do tempo para gerenciar os movimentos.}%
    \label{fig:#2_#7_#8_in#3}%
\end{figure}
}

\newcommand{\figureDvndGdvndTime}[8]{
    \figureDvndGdvnd{#1}{#2}{#3}{#4}{#5}{#6}{#7}{man_time}{man}
}

\newcommand{\figureGdvndDvndRvnd}[9]{
% #1 {box, scatter}, #2 {count, imp, time}, #3 {instance number}, #4 {Tempo, Melhoria}, #5 tamanho instância, #6 {DVND, RVND}, #7 {dvnd, rvnd}, #8 {man_time, full_time}, #9 {man, dvnd} #10 descricao
\begin{figure}%
    \centering
    \includegraphics[scale=0.9]{figuras/#7/#8/#1/#9_#1100sol_#2_in#3.png}
    \caption{#4 do #6 para a instância #3 de tamanho #5. \textit{DVND}, \textit{GDVND} e \textit{RVND} referem-se ao tempo gasto pelos algoritmos de mesmo nome.}%
    \label{fig:#2_#7_#8_in#3}%
\end{figure}
}

% \subfloat[$m=#1$]{{ %scale=0.225
%         \includegraphics[scale=0.425]{figuras/dvnd/n#1/time#2.png}
%         \label{fig:timeDvndRvnd_n#1in#2}
%     }}%
% #1 {dvnd, rvnd, gdvnd}, #2 {sog_mog, dc_dd}, #3 {time, imp}, #4 in, #5 tamanho, #6 {box, scatter}
\newcommand{\subFig}[6]{
    \subfloat[][Instância #4, $n=#5$]{
        \includegraphics[scale=0.425]{figuras/#1/#2/#6/#1_#6100sol_#3_in#4.png}
		\label{fig:#1_#2_#3_in#4}
    }
% 	\begin{subfigure}{0.45\textwidth} % dvnd_box100sol_imp_in0
% 		\includegraphics[scale=0.425]{figuras/#1/#2/#6/#1_#6100sol_#3_in#4.png}
% 		\caption{Instância #4, $n=#5$}
        % \label{fig:#1_#2_#3_in#4}
    % \end{subfigure}
}

\newcommand{\subFigBox}[5]{
	\subFig{#1}{#2}{#3}{#4}{#5}{box}
}

\newcommand{\subFigScatter}[5]{
	\subFig{#1}{#2}{#3}{#4}{#5}{scatter}
}

% #1 {dvnd, rvnd, gdvnd}, #2 {sog_mog, dc_dd}, #3 {time, imp}, #4 {box, scatter}, #5 {Tempo do DVND...}
\newcommand{\multiFigureInstanciasGeral}[5]{
	\begin{figure}[ht]
		\centering
		\subFig{#1}{#2}{#3}{0}{52}{#4}
		~
		\subFig{#1}{#2}{#3}{1}{100}{#4}
		
		\subFig{#1}{#2}{#3}{2}{226}{#4}
		~
		\subFig{#1}{#2}{#3}{3}{318}{#4}
		\caption{#5 Instâncias 0 a 3.}
		\label{fig:#1_#2_#3_in0_4}
	\end{figure}
	
	\begin{figure}[ht]
		\centering
		\subFig{#1}{#2}{#3}{4}{501}{#4}
		~
		\subFig{#1}{#2}{#3}{5}{657}{#4}
		
		\subFig{#1}{#2}{#3}{6}{783}{#4}
		~
		\subFig{#1}{#2}{#3}{7}{1001}{#4}
		\caption{#5 Instâncias 5 a 7.}
		\label{fig:#1_#2_#3_in5_7}
	\end{figure}
}

% #1 {dvnd, rvnd, gdvnd}, #2 {sog_mog, dc_dd}, #3 {time, imp}, #4 {Tempo do DVND...}
\newcommand{\multiFigureInstancias}[4]{
    \multiFigureInstanciasGeral{#1}{#2}{#3}{box}{#4}
}


\chapter{Resultados} \label{cap:resultados}

Este capítulo exibe os resultados computacionais dos algoritmos propostos no Capítulo~\ref{cap:metodologia} para o caso do PML, para cada instância foi gerado um conjunto com 100 soluções iniciais aleatórias que foram submetidas aos métodos para comparação dos resultados.

Quando há referência à melhoria na solução (\textit{Imp}), esta melhoria pode ser calculada pelo quociente do valor da solução inicial pela solução final, ou seja:
\begin{equation}\label{eq:calculateImprovement}
Imp = \frac{f(\textrm{solução inicial})}{f(\textrm{solução final})}
\end{equation}

Desta forma quanto maior for o valor da melhoria ($Imp$) mais o método melhorou o valor da solução inicial.

\section{Instâncias} \label{sec:instancias}

Todas as instâncias usadas nos testes computacionais e cujas configurações de lançamento foram descritas na Tabela~\ref{tab:neighborhoodsLaunchConfigurarion} são as mesmas usadas em~\cite{wamca2016}.
Para o RVND foi feita uma implementação do algoritmo clássico (Algoritmo~\ref{alg:rvnd}) e também a implementação dataflow mencionada na Figura~\ref{fig:rvndGraph} fazendo uso de uma máquina.
Para o caso do DVND foi utilizada a implementação clássica (Algoritmo~\ref{alg:dvnd}) e a implementação dataflow proposta (Figura~\ref{fig:dvndGraph}), os resultados foram obtidos com diferentes números de máquinas e os mesmos são indicados conforme o caso.

\section{Implementação e ambiente computacional}\label{sec:amb}

A implementação para cada algoritmo proposto no Capítulo~\ref{cap:metodologia} utiliza a linguagem de programação \textit{C++11} em conjunto com a API CUDA\texttrademark, para a implementação dos grafos e do ambiente dataflow foi utilizada a biblioteca Sucuri~\cite{sucuri-original}\footnote{Disponível em \url{https://github.com/tiagoaoa/Sucuri}} implementada em Python, para a integração entre o dataflow e o código CUDA foi utilizada a biblioteca SimplePyCuda~\cite{simple-pycuda}\footnote{Disponível em \url{https://github.com/igormcoelho/simple-pycuda}}. As implementações com múltiplas threads usaram a biblioteca OpenMP.
%As implementações foram compiladas através do \textit{GCC} \textit{(GNU Compiler Collection)}\footnote{O GCC está disponível no seguinte sítio eletrônico: \url{https://gcc.gnu.org/}.} com a \textit{flag} de otimização $-O3$.
O ambiente computacional utilizado em todos os testes neste trabalho consiste de 4 máquinas com a seguinte configuração:

\begin{itemize}
    \item Processador Intel\textregistered Core\texttrademark i7-4820K 3.7 GHz (4 núcleos);
    \item 8 GB de memória RAM;
    \item Sistema Operacional Ubuntu 14.10 (x64);
    \item NVIDIA GeForce GTX 780 com 2304 CUDA cores.
\end{itemize}

\input{resultados/comandosFiguras.tex}

\input{resultados/sog_mog/index.tex}

\input{resultados/rvnd/index.tex}

\input{resultados/dvnd/index.tex}

\input{resultados/gdvnd/index.tex}


\chapter{Resultados} \label{cap:resultados}

Este capítulo exibe os resultados computacionais dos algoritmos propostos no Capítulo~\ref{cap:metodologia} para o caso do PML, para cada instância foi gerado um conjunto com 100 soluções iniciais aleatórias que foram submetidas aos métodos para comparação dos resultados.

Quando há referência à melhoria na solução (\textit{Imp}), esta melhoria pode ser calculada pelo quociente do valor da solução inicial pela solução final, ou seja:
\begin{equation}\label{eq:calculateImprovement}
Imp = \frac{f(\textrm{solução inicial})}{f(\textrm{solução final})}
\end{equation}

Desta forma quanto maior for o valor da melhoria ($Imp$) mais o método melhorou o valor da solução inicial.

\section{Instâncias} \label{sec:instancias}

Todas as instâncias usadas nos testes computacionais e cujas configurações de lançamento foram descritas na Tabela~\ref{tab:neighborhoodsLaunchConfigurarion} são as mesmas usadas em~\cite{wamca2016}.
Para o RVND foi feita uma implementação do algoritmo clássico (Algoritmo~\ref{alg:rvnd}) e também a implementação dataflow mencionada na Figura~\ref{fig:rvndGraph} fazendo uso de uma máquina.
Para o caso do DVND foi utilizada a implementação clássica (Algoritmo~\ref{alg:dvnd}) e a implementação dataflow proposta (Figura~\ref{fig:dvndGraph}), os resultados foram obtidos com diferentes números de máquinas e os mesmos são indicados conforme o caso.

\section{Implementação e ambiente computacional}\label{sec:amb}

A implementação para cada algoritmo proposto no Capítulo~\ref{cap:metodologia} utiliza a linguagem de programação \textit{C++11} em conjunto com a API CUDA\texttrademark, para a implementação dos grafos e do ambiente dataflow foi utilizada a biblioteca Sucuri~\cite{sucuri-original}\footnote{Disponível em \url{https://github.com/tiagoaoa/Sucuri}} implementada em Python, para a integração entre o dataflow e o código CUDA foi utilizada a biblioteca SimplePyCuda~\cite{simple-pycuda}\footnote{Disponível em \url{https://github.com/igormcoelho/simple-pycuda}}. As implementações com múltiplas threads usaram a biblioteca OpenMP.
%As implementações foram compiladas através do \textit{GCC} \textit{(GNU Compiler Collection)}\footnote{O GCC está disponível no seguinte sítio eletrônico: \url{https://gcc.gnu.org/}.} com a \textit{flag} de otimização $-O3$.
O ambiente computacional utilizado em todos os testes neste trabalho consiste de 4 máquinas com a seguinte configuração:

\begin{itemize}
    \item Processador Intel\textregistered Core\texttrademark i7-4820K 3.7 GHz (4 núcleos);
    \item 8 GB de memória RAM;
    \item Sistema Operacional Ubuntu 14.10 (x64);
    \item NVIDIA GeForce GTX 780 com 2304 CUDA cores.
\end{itemize}

\input{resultados/comandosFiguras.tex}

\input{resultados/sog_mog/index.tex}

\input{resultados/rvnd/index.tex}

\input{resultados/dvnd/index.tex}

\input{resultados/gdvnd/index.tex}


\chapter{Resultados} \label{cap:resultados}

Este capítulo exibe os resultados computacionais dos algoritmos propostos no Capítulo~\ref{cap:metodologia} para o caso do PML, para cada instância foi gerado um conjunto com 100 soluções iniciais aleatórias que foram submetidas aos métodos para comparação dos resultados.

Quando há referência à melhoria na solução (\textit{Imp}), esta melhoria pode ser calculada pelo quociente do valor da solução inicial pela solução final, ou seja:
\begin{equation}\label{eq:calculateImprovement}
Imp = \frac{f(\textrm{solução inicial})}{f(\textrm{solução final})}
\end{equation}

Desta forma quanto maior for o valor da melhoria ($Imp$) mais o método melhorou o valor da solução inicial.

\section{Instâncias} \label{sec:instancias}

Todas as instâncias usadas nos testes computacionais e cujas configurações de lançamento foram descritas na Tabela~\ref{tab:neighborhoodsLaunchConfigurarion} são as mesmas usadas em~\cite{wamca2016}.
Para o RVND foi feita uma implementação do algoritmo clássico (Algoritmo~\ref{alg:rvnd}) e também a implementação dataflow mencionada na Figura~\ref{fig:rvndGraph} fazendo uso de uma máquina.
Para o caso do DVND foi utilizada a implementação clássica (Algoritmo~\ref{alg:dvnd}) e a implementação dataflow proposta (Figura~\ref{fig:dvndGraph}), os resultados foram obtidos com diferentes números de máquinas e os mesmos são indicados conforme o caso.

\section{Implementação e ambiente computacional}\label{sec:amb}

A implementação para cada algoritmo proposto no Capítulo~\ref{cap:metodologia} utiliza a linguagem de programação \textit{C++11} em conjunto com a API CUDA\texttrademark, para a implementação dos grafos e do ambiente dataflow foi utilizada a biblioteca Sucuri~\cite{sucuri-original}\footnote{Disponível em \url{https://github.com/tiagoaoa/Sucuri}} implementada em Python, para a integração entre o dataflow e o código CUDA foi utilizada a biblioteca SimplePyCuda~\cite{simple-pycuda}\footnote{Disponível em \url{https://github.com/igormcoelho/simple-pycuda}}. As implementações com múltiplas threads usaram a biblioteca OpenMP.
%As implementações foram compiladas através do \textit{GCC} \textit{(GNU Compiler Collection)}\footnote{O GCC está disponível no seguinte sítio eletrônico: \url{https://gcc.gnu.org/}.} com a \textit{flag} de otimização $-O3$.
O ambiente computacional utilizado em todos os testes neste trabalho consiste de 4 máquinas com a seguinte configuração:

\begin{itemize}
    \item Processador Intel\textregistered Core\texttrademark i7-4820K 3.7 GHz (4 núcleos);
    \item 8 GB de memória RAM;
    \item Sistema Operacional Ubuntu 14.10 (x64);
    \item NVIDIA GeForce GTX 780 com 2304 CUDA cores.
\end{itemize}

\input{resultados/comandosFiguras.tex}

\input{resultados/sog_mog/index.tex}

\input{resultados/rvnd/index.tex}

\input{resultados/dvnd/index.tex}

\input{resultados/gdvnd/index.tex}


\chapter{Resultados} \label{cap:resultados}

Este capítulo exibe os resultados computacionais dos algoritmos propostos no Capítulo~\ref{cap:metodologia} para o caso do PML, para cada instância foi gerado um conjunto com 100 soluções iniciais aleatórias que foram submetidas aos métodos para comparação dos resultados.

Quando há referência à melhoria na solução (\textit{Imp}), esta melhoria pode ser calculada pelo quociente do valor da solução inicial pela solução final, ou seja:
\begin{equation}\label{eq:calculateImprovement}
Imp = \frac{f(\textrm{solução inicial})}{f(\textrm{solução final})}
\end{equation}

Desta forma quanto maior for o valor da melhoria ($Imp$) mais o método melhorou o valor da solução inicial.

\section{Instâncias} \label{sec:instancias}

Todas as instâncias usadas nos testes computacionais e cujas configurações de lançamento foram descritas na Tabela~\ref{tab:neighborhoodsLaunchConfigurarion} são as mesmas usadas em~\cite{wamca2016}.
Para o RVND foi feita uma implementação do algoritmo clássico (Algoritmo~\ref{alg:rvnd}) e também a implementação dataflow mencionada na Figura~\ref{fig:rvndGraph} fazendo uso de uma máquina.
Para o caso do DVND foi utilizada a implementação clássica (Algoritmo~\ref{alg:dvnd}) e a implementação dataflow proposta (Figura~\ref{fig:dvndGraph}), os resultados foram obtidos com diferentes números de máquinas e os mesmos são indicados conforme o caso.

\section{Implementação e ambiente computacional}\label{sec:amb}

A implementação para cada algoritmo proposto no Capítulo~\ref{cap:metodologia} utiliza a linguagem de programação \textit{C++11} em conjunto com a API CUDA\texttrademark, para a implementação dos grafos e do ambiente dataflow foi utilizada a biblioteca Sucuri~\cite{sucuri-original}\footnote{Disponível em \url{https://github.com/tiagoaoa/Sucuri}} implementada em Python, para a integração entre o dataflow e o código CUDA foi utilizada a biblioteca SimplePyCuda~\cite{simple-pycuda}\footnote{Disponível em \url{https://github.com/igormcoelho/simple-pycuda}}. As implementações com múltiplas threads usaram a biblioteca OpenMP.
%As implementações foram compiladas através do \textit{GCC} \textit{(GNU Compiler Collection)}\footnote{O GCC está disponível no seguinte sítio eletrônico: \url{https://gcc.gnu.org/}.} com a \textit{flag} de otimização $-O3$.
O ambiente computacional utilizado em todos os testes neste trabalho consiste de 4 máquinas com a seguinte configuração:

\begin{itemize}
    \item Processador Intel\textregistered Core\texttrademark i7-4820K 3.7 GHz (4 núcleos);
    \item 8 GB de memória RAM;
    \item Sistema Operacional Ubuntu 14.10 (x64);
    \item NVIDIA GeForce GTX 780 com 2304 CUDA cores.
\end{itemize}

\input{resultados/comandosFiguras.tex}

\input{resultados/sog_mog/index.tex}

\input{resultados/rvnd/index.tex}

\input{resultados/dvnd/index.tex}

\input{resultados/gdvnd/index.tex}



\chapter{Resultados} \label{cap:resultados}

Este capítulo exibe os resultados computacionais dos algoritmos propostos no Capítulo~\ref{cap:metodologia} para o caso do PML, para cada instância foi gerado um conjunto com 100 soluções iniciais aleatórias que foram submetidas aos métodos para comparação dos resultados.

Quando há referência à melhoria na solução (\textit{Imp}), esta melhoria pode ser calculada pelo quociente do valor da solução inicial pela solução final, ou seja:
\begin{equation}\label{eq:calculateImprovement}
Imp = \frac{f(\textrm{solução inicial})}{f(\textrm{solução final})}
\end{equation}

Desta forma quanto maior for o valor da melhoria ($Imp$) mais o método melhorou o valor da solução inicial.

\section{Instâncias} \label{sec:instancias}

Todas as instâncias usadas nos testes computacionais e cujas configurações de lançamento foram descritas na Tabela~\ref{tab:neighborhoodsLaunchConfigurarion} são as mesmas usadas em~\cite{wamca2016}.
Para o RVND foi feita uma implementação do algoritmo clássico (Algoritmo~\ref{alg:rvnd}) e também a implementação dataflow mencionada na Figura~\ref{fig:rvndGraph} fazendo uso de uma máquina.
Para o caso do DVND foi utilizada a implementação clássica (Algoritmo~\ref{alg:dvnd}) e a implementação dataflow proposta (Figura~\ref{fig:dvndGraph}), os resultados foram obtidos com diferentes números de máquinas e os mesmos são indicados conforme o caso.

\section{Implementação e ambiente computacional}\label{sec:amb}

A implementação para cada algoritmo proposto no Capítulo~\ref{cap:metodologia} utiliza a linguagem de programação \textit{C++11} em conjunto com a API CUDA\texttrademark, para a implementação dos grafos e do ambiente dataflow foi utilizada a biblioteca Sucuri~\cite{sucuri-original}\footnote{Disponível em \url{https://github.com/tiagoaoa/Sucuri}} implementada em Python, para a integração entre o dataflow e o código CUDA foi utilizada a biblioteca SimplePyCuda~\cite{simple-pycuda}\footnote{Disponível em \url{https://github.com/igormcoelho/simple-pycuda}}. As implementações com múltiplas threads usaram a biblioteca OpenMP.
%As implementações foram compiladas através do \textit{GCC} \textit{(GNU Compiler Collection)}\footnote{O GCC está disponível no seguinte sítio eletrônico: \url{https://gcc.gnu.org/}.} com a \textit{flag} de otimização $-O3$.
O ambiente computacional utilizado em todos os testes neste trabalho consiste de 4 máquinas com a seguinte configuração:

\begin{itemize}
    \item Processador Intel\textregistered Core\texttrademark i7-4820K 3.7 GHz (4 núcleos);
    \item 8 GB de memória RAM;
    \item Sistema Operacional Ubuntu 14.10 (x64);
    \item NVIDIA GeForce GTX 780 com 2304 CUDA cores.
\end{itemize}

\newcommand{\figureDvndOrRvndDcDd}[7]{
% #1 {box, scatter}, #2 {count, imp, time}, #3 {instance number}, #4 {Tempo, Melhoria}, #5 tamanho instância, #6 {DVND, RVND}, #7 {dvnd, rvnd}
\begin{figure}%
    \centering
    \includegraphics[scale=0.9]{figuras/#7/dc_dd/#1/#7_#1100sol_#2_in#3.png}
    \caption{#4 do #6 para a instância #3 de tamanho #5. $m$ indica o número de máquinas, \textit{DC} refere-se ao #6 clássico e \textit{DD} ao #6 implementado em dataflow.}%
    \label{fig:#2_#7DcDd_in#3}%
\end{figure}
}

\newcommand{\figureDvndDcDd}[5]{
% #1 {box, scatter}, #2 {count, imp, time}, #3 {instance number}, #4 {Tempo, Melhoria}, #5 tamanho instância
    \figureDvndOrRvndDcDd{#1}{#2}{#3}{#4}{#5}{DVND}{dvnd}
}

\newcommand{\figureRvndDcDd}[5]{
% #1 {box, scatter}, #2 {count, imp, time}, #3 {instance number}, #4 {Tempo, Melhoria}, #5 tamanho instância
    \figureDvndOrRvndDcDd{#1}{#2}{#3}{#4}{#5}{RVND}{rvnd}
}

\newcommand{\tabelaEstatisticasGeral}[6]{
% #1 Descrição, #2 label, #3 {dvnd, rvnd}, #4 {DVND, RVND}, #6 DD/DC, #6 Conteúdo
\begin{table}[ht]
    \centering
    \begin{tabular}{c|ccc|cc|ccc|cc|c}
        \hline \hline
        \# & Tipo & $m$ & $n$ & $min$ & $max$ & 1Q & 2Q & 3Q & $\overline{x}$ & $\sigma$ & $p-valor$ \\ \hline
        #6
    \end{tabular}
    \caption{#1 #4 #5
        Instância (\#), tipo de implementação (Tipo), número de máquinas ($m$), tamanho da instância ($n$), valor mínimo ($min$), máximo ($max$), primeiro quartil (1Q), mediana (2Q), terceiro quartil (3Q), média ($\overline{x}$), desvio padrão ($\sigma$) e p-valor para o teste de Wilcox entre as versões (valores em negrito quando $p-valor > 0.05$).
    }
    \label{tab:#3DcDd#2}
\end{table}
}

\newcommand{\tabelaEstatisticas}[5]{
    \tabelaEstatisticasGeral{#1}{#2}{#3}{#4}{na implementação clássica (DC) e a proposta de implementação usando dataflow (DD).}{#5}
}

\newcommand{\figureDvndSogMog}[7]{
% #1 {box, scatter}, #2 {count, imp, time}, #3 {instance number}, #4 {Tempo, Melhoria}, #5 tamanho instância, #6 {DVND, RVND}, #7 {dvnd, rvnd}
\begin{figure}%
    \centering
    \includegraphics[scale=0.9]{figuras/#7/sog_mog/#1/#7_#1100sol_#2_in#3.png}
    \caption{#4 do #6 para a instância #3 de tamanho #5. \textit{SOG} refere-se a uma porta de saída e \textit{MOG} a múltiplas portas de saída.}%
    \label{fig:#2_#7SogMog_in#3}%
\end{figure}
}

\newcommand{\figureDvndGdvnd}[9]{
% #1 {box, scatter}, #2 {count, imp, time}, #3 {instance number}, #4 {Tempo, Melhoria}, #5 tamanho instância, #6 {DVND, RVND}, #7 {dvnd, rvnd}, #8 {man_time, full_time}, #9 {man, dvnd} #10 descricao
\begin{figure}%
    \centering
    \includegraphics[scale=0.9]{figuras/#7/#8/#1/#9_#1100sol_#2_in#3.png}
    \caption{#4 do #6 para a instância #3 de tamanho #5. \textit{DVND} refere-se ao tempo gasto pelo algoritmo de mesmo nome, para \textit{GDVND} é análogo ao anterior, no caso do \textit{GDVND-MAN} este se refere ao tempo do GDVND subtraido do tempo para gerenciar os movimentos.}%
    \label{fig:#2_#7_#8_in#3}%
\end{figure}
}

\newcommand{\figureDvndGdvndTime}[8]{
    \figureDvndGdvnd{#1}{#2}{#3}{#4}{#5}{#6}{#7}{man_time}{man}
}

\newcommand{\figureGdvndDvndRvnd}[9]{
% #1 {box, scatter}, #2 {count, imp, time}, #3 {instance number}, #4 {Tempo, Melhoria}, #5 tamanho instância, #6 {DVND, RVND}, #7 {dvnd, rvnd}, #8 {man_time, full_time}, #9 {man, dvnd} #10 descricao
\begin{figure}%
    \centering
    \includegraphics[scale=0.9]{figuras/#7/#8/#1/#9_#1100sol_#2_in#3.png}
    \caption{#4 do #6 para a instância #3 de tamanho #5. \textit{DVND}, \textit{GDVND} e \textit{RVND} referem-se ao tempo gasto pelos algoritmos de mesmo nome.}%
    \label{fig:#2_#7_#8_in#3}%
\end{figure}
}

% \subfloat[$m=#1$]{{ %scale=0.225
%         \includegraphics[scale=0.425]{figuras/dvnd/n#1/time#2.png}
%         \label{fig:timeDvndRvnd_n#1in#2}
%     }}%
% #1 {dvnd, rvnd, gdvnd}, #2 {sog_mog, dc_dd}, #3 {time, imp}, #4 in, #5 tamanho, #6 {box, scatter}
\newcommand{\subFig}[6]{
    \subfloat[][Instância #4, $n=#5$]{
        \includegraphics[scale=0.425]{figuras/#1/#2/#6/#1_#6100sol_#3_in#4.png}
		\label{fig:#1_#2_#3_in#4}
    }
% 	\begin{subfigure}{0.45\textwidth} % dvnd_box100sol_imp_in0
% 		\includegraphics[scale=0.425]{figuras/#1/#2/#6/#1_#6100sol_#3_in#4.png}
% 		\caption{Instância #4, $n=#5$}
        % \label{fig:#1_#2_#3_in#4}
    % \end{subfigure}
}

\newcommand{\subFigBox}[5]{
	\subFig{#1}{#2}{#3}{#4}{#5}{box}
}

\newcommand{\subFigScatter}[5]{
	\subFig{#1}{#2}{#3}{#4}{#5}{scatter}
}

% #1 {dvnd, rvnd, gdvnd}, #2 {sog_mog, dc_dd}, #3 {time, imp}, #4 {box, scatter}, #5 {Tempo do DVND...}
\newcommand{\multiFigureInstanciasGeral}[5]{
	\begin{figure}[ht]
		\centering
		\subFig{#1}{#2}{#3}{0}{52}{#4}
		~
		\subFig{#1}{#2}{#3}{1}{100}{#4}
		
		\subFig{#1}{#2}{#3}{2}{226}{#4}
		~
		\subFig{#1}{#2}{#3}{3}{318}{#4}
		\caption{#5 Instâncias 0 a 3.}
		\label{fig:#1_#2_#3_in0_4}
	\end{figure}
	
	\begin{figure}[ht]
		\centering
		\subFig{#1}{#2}{#3}{4}{501}{#4}
		~
		\subFig{#1}{#2}{#3}{5}{657}{#4}
		
		\subFig{#1}{#2}{#3}{6}{783}{#4}
		~
		\subFig{#1}{#2}{#3}{7}{1001}{#4}
		\caption{#5 Instâncias 5 a 7.}
		\label{fig:#1_#2_#3_in5_7}
	\end{figure}
}

% #1 {dvnd, rvnd, gdvnd}, #2 {sog_mog, dc_dd}, #3 {time, imp}, #4 {Tempo do DVND...}
\newcommand{\multiFigureInstancias}[4]{
    \multiFigureInstanciasGeral{#1}{#2}{#3}{box}{#4}
}


\chapter{Resultados} \label{cap:resultados}

Este capítulo exibe os resultados computacionais dos algoritmos propostos no Capítulo~\ref{cap:metodologia} para o caso do PML, para cada instância foi gerado um conjunto com 100 soluções iniciais aleatórias que foram submetidas aos métodos para comparação dos resultados.

Quando há referência à melhoria na solução (\textit{Imp}), esta melhoria pode ser calculada pelo quociente do valor da solução inicial pela solução final, ou seja:
\begin{equation}\label{eq:calculateImprovement}
Imp = \frac{f(\textrm{solução inicial})}{f(\textrm{solução final})}
\end{equation}

Desta forma quanto maior for o valor da melhoria ($Imp$) mais o método melhorou o valor da solução inicial.

\section{Instâncias} \label{sec:instancias}

Todas as instâncias usadas nos testes computacionais e cujas configurações de lançamento foram descritas na Tabela~\ref{tab:neighborhoodsLaunchConfigurarion} são as mesmas usadas em~\cite{wamca2016}.
Para o RVND foi feita uma implementação do algoritmo clássico (Algoritmo~\ref{alg:rvnd}) e também a implementação dataflow mencionada na Figura~\ref{fig:rvndGraph} fazendo uso de uma máquina.
Para o caso do DVND foi utilizada a implementação clássica (Algoritmo~\ref{alg:dvnd}) e a implementação dataflow proposta (Figura~\ref{fig:dvndGraph}), os resultados foram obtidos com diferentes números de máquinas e os mesmos são indicados conforme o caso.

\section{Implementação e ambiente computacional}\label{sec:amb}

A implementação para cada algoritmo proposto no Capítulo~\ref{cap:metodologia} utiliza a linguagem de programação \textit{C++11} em conjunto com a API CUDA\texttrademark, para a implementação dos grafos e do ambiente dataflow foi utilizada a biblioteca Sucuri~\cite{sucuri-original}\footnote{Disponível em \url{https://github.com/tiagoaoa/Sucuri}} implementada em Python, para a integração entre o dataflow e o código CUDA foi utilizada a biblioteca SimplePyCuda~\cite{simple-pycuda}\footnote{Disponível em \url{https://github.com/igormcoelho/simple-pycuda}}. As implementações com múltiplas threads usaram a biblioteca OpenMP.
%As implementações foram compiladas através do \textit{GCC} \textit{(GNU Compiler Collection)}\footnote{O GCC está disponível no seguinte sítio eletrônico: \url{https://gcc.gnu.org/}.} com a \textit{flag} de otimização $-O3$.
O ambiente computacional utilizado em todos os testes neste trabalho consiste de 4 máquinas com a seguinte configuração:

\begin{itemize}
    \item Processador Intel\textregistered Core\texttrademark i7-4820K 3.7 GHz (4 núcleos);
    \item 8 GB de memória RAM;
    \item Sistema Operacional Ubuntu 14.10 (x64);
    \item NVIDIA GeForce GTX 780 com 2304 CUDA cores.
\end{itemize}

\input{resultados/comandosFiguras.tex}

\input{resultados/sog_mog/index.tex}

\input{resultados/rvnd/index.tex}

\input{resultados/dvnd/index.tex}

\input{resultados/gdvnd/index.tex}


\chapter{Resultados} \label{cap:resultados}

Este capítulo exibe os resultados computacionais dos algoritmos propostos no Capítulo~\ref{cap:metodologia} para o caso do PML, para cada instância foi gerado um conjunto com 100 soluções iniciais aleatórias que foram submetidas aos métodos para comparação dos resultados.

Quando há referência à melhoria na solução (\textit{Imp}), esta melhoria pode ser calculada pelo quociente do valor da solução inicial pela solução final, ou seja:
\begin{equation}\label{eq:calculateImprovement}
Imp = \frac{f(\textrm{solução inicial})}{f(\textrm{solução final})}
\end{equation}

Desta forma quanto maior for o valor da melhoria ($Imp$) mais o método melhorou o valor da solução inicial.

\section{Instâncias} \label{sec:instancias}

Todas as instâncias usadas nos testes computacionais e cujas configurações de lançamento foram descritas na Tabela~\ref{tab:neighborhoodsLaunchConfigurarion} são as mesmas usadas em~\cite{wamca2016}.
Para o RVND foi feita uma implementação do algoritmo clássico (Algoritmo~\ref{alg:rvnd}) e também a implementação dataflow mencionada na Figura~\ref{fig:rvndGraph} fazendo uso de uma máquina.
Para o caso do DVND foi utilizada a implementação clássica (Algoritmo~\ref{alg:dvnd}) e a implementação dataflow proposta (Figura~\ref{fig:dvndGraph}), os resultados foram obtidos com diferentes números de máquinas e os mesmos são indicados conforme o caso.

\section{Implementação e ambiente computacional}\label{sec:amb}

A implementação para cada algoritmo proposto no Capítulo~\ref{cap:metodologia} utiliza a linguagem de programação \textit{C++11} em conjunto com a API CUDA\texttrademark, para a implementação dos grafos e do ambiente dataflow foi utilizada a biblioteca Sucuri~\cite{sucuri-original}\footnote{Disponível em \url{https://github.com/tiagoaoa/Sucuri}} implementada em Python, para a integração entre o dataflow e o código CUDA foi utilizada a biblioteca SimplePyCuda~\cite{simple-pycuda}\footnote{Disponível em \url{https://github.com/igormcoelho/simple-pycuda}}. As implementações com múltiplas threads usaram a biblioteca OpenMP.
%As implementações foram compiladas através do \textit{GCC} \textit{(GNU Compiler Collection)}\footnote{O GCC está disponível no seguinte sítio eletrônico: \url{https://gcc.gnu.org/}.} com a \textit{flag} de otimização $-O3$.
O ambiente computacional utilizado em todos os testes neste trabalho consiste de 4 máquinas com a seguinte configuração:

\begin{itemize}
    \item Processador Intel\textregistered Core\texttrademark i7-4820K 3.7 GHz (4 núcleos);
    \item 8 GB de memória RAM;
    \item Sistema Operacional Ubuntu 14.10 (x64);
    \item NVIDIA GeForce GTX 780 com 2304 CUDA cores.
\end{itemize}

\input{resultados/comandosFiguras.tex}

\input{resultados/sog_mog/index.tex}

\input{resultados/rvnd/index.tex}

\input{resultados/dvnd/index.tex}

\input{resultados/gdvnd/index.tex}


\chapter{Resultados} \label{cap:resultados}

Este capítulo exibe os resultados computacionais dos algoritmos propostos no Capítulo~\ref{cap:metodologia} para o caso do PML, para cada instância foi gerado um conjunto com 100 soluções iniciais aleatórias que foram submetidas aos métodos para comparação dos resultados.

Quando há referência à melhoria na solução (\textit{Imp}), esta melhoria pode ser calculada pelo quociente do valor da solução inicial pela solução final, ou seja:
\begin{equation}\label{eq:calculateImprovement}
Imp = \frac{f(\textrm{solução inicial})}{f(\textrm{solução final})}
\end{equation}

Desta forma quanto maior for o valor da melhoria ($Imp$) mais o método melhorou o valor da solução inicial.

\section{Instâncias} \label{sec:instancias}

Todas as instâncias usadas nos testes computacionais e cujas configurações de lançamento foram descritas na Tabela~\ref{tab:neighborhoodsLaunchConfigurarion} são as mesmas usadas em~\cite{wamca2016}.
Para o RVND foi feita uma implementação do algoritmo clássico (Algoritmo~\ref{alg:rvnd}) e também a implementação dataflow mencionada na Figura~\ref{fig:rvndGraph} fazendo uso de uma máquina.
Para o caso do DVND foi utilizada a implementação clássica (Algoritmo~\ref{alg:dvnd}) e a implementação dataflow proposta (Figura~\ref{fig:dvndGraph}), os resultados foram obtidos com diferentes números de máquinas e os mesmos são indicados conforme o caso.

\section{Implementação e ambiente computacional}\label{sec:amb}

A implementação para cada algoritmo proposto no Capítulo~\ref{cap:metodologia} utiliza a linguagem de programação \textit{C++11} em conjunto com a API CUDA\texttrademark, para a implementação dos grafos e do ambiente dataflow foi utilizada a biblioteca Sucuri~\cite{sucuri-original}\footnote{Disponível em \url{https://github.com/tiagoaoa/Sucuri}} implementada em Python, para a integração entre o dataflow e o código CUDA foi utilizada a biblioteca SimplePyCuda~\cite{simple-pycuda}\footnote{Disponível em \url{https://github.com/igormcoelho/simple-pycuda}}. As implementações com múltiplas threads usaram a biblioteca OpenMP.
%As implementações foram compiladas através do \textit{GCC} \textit{(GNU Compiler Collection)}\footnote{O GCC está disponível no seguinte sítio eletrônico: \url{https://gcc.gnu.org/}.} com a \textit{flag} de otimização $-O3$.
O ambiente computacional utilizado em todos os testes neste trabalho consiste de 4 máquinas com a seguinte configuração:

\begin{itemize}
    \item Processador Intel\textregistered Core\texttrademark i7-4820K 3.7 GHz (4 núcleos);
    \item 8 GB de memória RAM;
    \item Sistema Operacional Ubuntu 14.10 (x64);
    \item NVIDIA GeForce GTX 780 com 2304 CUDA cores.
\end{itemize}

\input{resultados/comandosFiguras.tex}

\input{resultados/sog_mog/index.tex}

\input{resultados/rvnd/index.tex}

\input{resultados/dvnd/index.tex}

\input{resultados/gdvnd/index.tex}


\chapter{Resultados} \label{cap:resultados}

Este capítulo exibe os resultados computacionais dos algoritmos propostos no Capítulo~\ref{cap:metodologia} para o caso do PML, para cada instância foi gerado um conjunto com 100 soluções iniciais aleatórias que foram submetidas aos métodos para comparação dos resultados.

Quando há referência à melhoria na solução (\textit{Imp}), esta melhoria pode ser calculada pelo quociente do valor da solução inicial pela solução final, ou seja:
\begin{equation}\label{eq:calculateImprovement}
Imp = \frac{f(\textrm{solução inicial})}{f(\textrm{solução final})}
\end{equation}

Desta forma quanto maior for o valor da melhoria ($Imp$) mais o método melhorou o valor da solução inicial.

\section{Instâncias} \label{sec:instancias}

Todas as instâncias usadas nos testes computacionais e cujas configurações de lançamento foram descritas na Tabela~\ref{tab:neighborhoodsLaunchConfigurarion} são as mesmas usadas em~\cite{wamca2016}.
Para o RVND foi feita uma implementação do algoritmo clássico (Algoritmo~\ref{alg:rvnd}) e também a implementação dataflow mencionada na Figura~\ref{fig:rvndGraph} fazendo uso de uma máquina.
Para o caso do DVND foi utilizada a implementação clássica (Algoritmo~\ref{alg:dvnd}) e a implementação dataflow proposta (Figura~\ref{fig:dvndGraph}), os resultados foram obtidos com diferentes números de máquinas e os mesmos são indicados conforme o caso.

\section{Implementação e ambiente computacional}\label{sec:amb}

A implementação para cada algoritmo proposto no Capítulo~\ref{cap:metodologia} utiliza a linguagem de programação \textit{C++11} em conjunto com a API CUDA\texttrademark, para a implementação dos grafos e do ambiente dataflow foi utilizada a biblioteca Sucuri~\cite{sucuri-original}\footnote{Disponível em \url{https://github.com/tiagoaoa/Sucuri}} implementada em Python, para a integração entre o dataflow e o código CUDA foi utilizada a biblioteca SimplePyCuda~\cite{simple-pycuda}\footnote{Disponível em \url{https://github.com/igormcoelho/simple-pycuda}}. As implementações com múltiplas threads usaram a biblioteca OpenMP.
%As implementações foram compiladas através do \textit{GCC} \textit{(GNU Compiler Collection)}\footnote{O GCC está disponível no seguinte sítio eletrônico: \url{https://gcc.gnu.org/}.} com a \textit{flag} de otimização $-O3$.
O ambiente computacional utilizado em todos os testes neste trabalho consiste de 4 máquinas com a seguinte configuração:

\begin{itemize}
    \item Processador Intel\textregistered Core\texttrademark i7-4820K 3.7 GHz (4 núcleos);
    \item 8 GB de memória RAM;
    \item Sistema Operacional Ubuntu 14.10 (x64);
    \item NVIDIA GeForce GTX 780 com 2304 CUDA cores.
\end{itemize}

\input{resultados/comandosFiguras.tex}

\input{resultados/sog_mog/index.tex}

\input{resultados/rvnd/index.tex}

\input{resultados/dvnd/index.tex}

\input{resultados/gdvnd/index.tex}



\chapter{Resultados} \label{cap:resultados}

Este capítulo exibe os resultados computacionais dos algoritmos propostos no Capítulo~\ref{cap:metodologia} para o caso do PML, para cada instância foi gerado um conjunto com 100 soluções iniciais aleatórias que foram submetidas aos métodos para comparação dos resultados.

Quando há referência à melhoria na solução (\textit{Imp}), esta melhoria pode ser calculada pelo quociente do valor da solução inicial pela solução final, ou seja:
\begin{equation}\label{eq:calculateImprovement}
Imp = \frac{f(\textrm{solução inicial})}{f(\textrm{solução final})}
\end{equation}

Desta forma quanto maior for o valor da melhoria ($Imp$) mais o método melhorou o valor da solução inicial.

\section{Instâncias} \label{sec:instancias}

Todas as instâncias usadas nos testes computacionais e cujas configurações de lançamento foram descritas na Tabela~\ref{tab:neighborhoodsLaunchConfigurarion} são as mesmas usadas em~\cite{wamca2016}.
Para o RVND foi feita uma implementação do algoritmo clássico (Algoritmo~\ref{alg:rvnd}) e também a implementação dataflow mencionada na Figura~\ref{fig:rvndGraph} fazendo uso de uma máquina.
Para o caso do DVND foi utilizada a implementação clássica (Algoritmo~\ref{alg:dvnd}) e a implementação dataflow proposta (Figura~\ref{fig:dvndGraph}), os resultados foram obtidos com diferentes números de máquinas e os mesmos são indicados conforme o caso.

\section{Implementação e ambiente computacional}\label{sec:amb}

A implementação para cada algoritmo proposto no Capítulo~\ref{cap:metodologia} utiliza a linguagem de programação \textit{C++11} em conjunto com a API CUDA\texttrademark, para a implementação dos grafos e do ambiente dataflow foi utilizada a biblioteca Sucuri~\cite{sucuri-original}\footnote{Disponível em \url{https://github.com/tiagoaoa/Sucuri}} implementada em Python, para a integração entre o dataflow e o código CUDA foi utilizada a biblioteca SimplePyCuda~\cite{simple-pycuda}\footnote{Disponível em \url{https://github.com/igormcoelho/simple-pycuda}}. As implementações com múltiplas threads usaram a biblioteca OpenMP.
%As implementações foram compiladas através do \textit{GCC} \textit{(GNU Compiler Collection)}\footnote{O GCC está disponível no seguinte sítio eletrônico: \url{https://gcc.gnu.org/}.} com a \textit{flag} de otimização $-O3$.
O ambiente computacional utilizado em todos os testes neste trabalho consiste de 4 máquinas com a seguinte configuração:

\begin{itemize}
    \item Processador Intel\textregistered Core\texttrademark i7-4820K 3.7 GHz (4 núcleos);
    \item 8 GB de memória RAM;
    \item Sistema Operacional Ubuntu 14.10 (x64);
    \item NVIDIA GeForce GTX 780 com 2304 CUDA cores.
\end{itemize}

\newcommand{\figureDvndOrRvndDcDd}[7]{
% #1 {box, scatter}, #2 {count, imp, time}, #3 {instance number}, #4 {Tempo, Melhoria}, #5 tamanho instância, #6 {DVND, RVND}, #7 {dvnd, rvnd}
\begin{figure}%
    \centering
    \includegraphics[scale=0.9]{figuras/#7/dc_dd/#1/#7_#1100sol_#2_in#3.png}
    \caption{#4 do #6 para a instância #3 de tamanho #5. $m$ indica o número de máquinas, \textit{DC} refere-se ao #6 clássico e \textit{DD} ao #6 implementado em dataflow.}%
    \label{fig:#2_#7DcDd_in#3}%
\end{figure}
}

\newcommand{\figureDvndDcDd}[5]{
% #1 {box, scatter}, #2 {count, imp, time}, #3 {instance number}, #4 {Tempo, Melhoria}, #5 tamanho instância
    \figureDvndOrRvndDcDd{#1}{#2}{#3}{#4}{#5}{DVND}{dvnd}
}

\newcommand{\figureRvndDcDd}[5]{
% #1 {box, scatter}, #2 {count, imp, time}, #3 {instance number}, #4 {Tempo, Melhoria}, #5 tamanho instância
    \figureDvndOrRvndDcDd{#1}{#2}{#3}{#4}{#5}{RVND}{rvnd}
}

\newcommand{\tabelaEstatisticasGeral}[6]{
% #1 Descrição, #2 label, #3 {dvnd, rvnd}, #4 {DVND, RVND}, #6 DD/DC, #6 Conteúdo
\begin{table}[ht]
    \centering
    \begin{tabular}{c|ccc|cc|ccc|cc|c}
        \hline \hline
        \# & Tipo & $m$ & $n$ & $min$ & $max$ & 1Q & 2Q & 3Q & $\overline{x}$ & $\sigma$ & $p-valor$ \\ \hline
        #6
    \end{tabular}
    \caption{#1 #4 #5
        Instância (\#), tipo de implementação (Tipo), número de máquinas ($m$), tamanho da instância ($n$), valor mínimo ($min$), máximo ($max$), primeiro quartil (1Q), mediana (2Q), terceiro quartil (3Q), média ($\overline{x}$), desvio padrão ($\sigma$) e p-valor para o teste de Wilcox entre as versões (valores em negrito quando $p-valor > 0.05$).
    }
    \label{tab:#3DcDd#2}
\end{table}
}

\newcommand{\tabelaEstatisticas}[5]{
    \tabelaEstatisticasGeral{#1}{#2}{#3}{#4}{na implementação clássica (DC) e a proposta de implementação usando dataflow (DD).}{#5}
}

\newcommand{\figureDvndSogMog}[7]{
% #1 {box, scatter}, #2 {count, imp, time}, #3 {instance number}, #4 {Tempo, Melhoria}, #5 tamanho instância, #6 {DVND, RVND}, #7 {dvnd, rvnd}
\begin{figure}%
    \centering
    \includegraphics[scale=0.9]{figuras/#7/sog_mog/#1/#7_#1100sol_#2_in#3.png}
    \caption{#4 do #6 para a instância #3 de tamanho #5. \textit{SOG} refere-se a uma porta de saída e \textit{MOG} a múltiplas portas de saída.}%
    \label{fig:#2_#7SogMog_in#3}%
\end{figure}
}

\newcommand{\figureDvndGdvnd}[9]{
% #1 {box, scatter}, #2 {count, imp, time}, #3 {instance number}, #4 {Tempo, Melhoria}, #5 tamanho instância, #6 {DVND, RVND}, #7 {dvnd, rvnd}, #8 {man_time, full_time}, #9 {man, dvnd} #10 descricao
\begin{figure}%
    \centering
    \includegraphics[scale=0.9]{figuras/#7/#8/#1/#9_#1100sol_#2_in#3.png}
    \caption{#4 do #6 para a instância #3 de tamanho #5. \textit{DVND} refere-se ao tempo gasto pelo algoritmo de mesmo nome, para \textit{GDVND} é análogo ao anterior, no caso do \textit{GDVND-MAN} este se refere ao tempo do GDVND subtraido do tempo para gerenciar os movimentos.}%
    \label{fig:#2_#7_#8_in#3}%
\end{figure}
}

\newcommand{\figureDvndGdvndTime}[8]{
    \figureDvndGdvnd{#1}{#2}{#3}{#4}{#5}{#6}{#7}{man_time}{man}
}

\newcommand{\figureGdvndDvndRvnd}[9]{
% #1 {box, scatter}, #2 {count, imp, time}, #3 {instance number}, #4 {Tempo, Melhoria}, #5 tamanho instância, #6 {DVND, RVND}, #7 {dvnd, rvnd}, #8 {man_time, full_time}, #9 {man, dvnd} #10 descricao
\begin{figure}%
    \centering
    \includegraphics[scale=0.9]{figuras/#7/#8/#1/#9_#1100sol_#2_in#3.png}
    \caption{#4 do #6 para a instância #3 de tamanho #5. \textit{DVND}, \textit{GDVND} e \textit{RVND} referem-se ao tempo gasto pelos algoritmos de mesmo nome.}%
    \label{fig:#2_#7_#8_in#3}%
\end{figure}
}

% \subfloat[$m=#1$]{{ %scale=0.225
%         \includegraphics[scale=0.425]{figuras/dvnd/n#1/time#2.png}
%         \label{fig:timeDvndRvnd_n#1in#2}
%     }}%
% #1 {dvnd, rvnd, gdvnd}, #2 {sog_mog, dc_dd}, #3 {time, imp}, #4 in, #5 tamanho, #6 {box, scatter}
\newcommand{\subFig}[6]{
    \subfloat[][Instância #4, $n=#5$]{
        \includegraphics[scale=0.425]{figuras/#1/#2/#6/#1_#6100sol_#3_in#4.png}
		\label{fig:#1_#2_#3_in#4}
    }
% 	\begin{subfigure}{0.45\textwidth} % dvnd_box100sol_imp_in0
% 		\includegraphics[scale=0.425]{figuras/#1/#2/#6/#1_#6100sol_#3_in#4.png}
% 		\caption{Instância #4, $n=#5$}
        % \label{fig:#1_#2_#3_in#4}
    % \end{subfigure}
}

\newcommand{\subFigBox}[5]{
	\subFig{#1}{#2}{#3}{#4}{#5}{box}
}

\newcommand{\subFigScatter}[5]{
	\subFig{#1}{#2}{#3}{#4}{#5}{scatter}
}

% #1 {dvnd, rvnd, gdvnd}, #2 {sog_mog, dc_dd}, #3 {time, imp}, #4 {box, scatter}, #5 {Tempo do DVND...}
\newcommand{\multiFigureInstanciasGeral}[5]{
	\begin{figure}[ht]
		\centering
		\subFig{#1}{#2}{#3}{0}{52}{#4}
		~
		\subFig{#1}{#2}{#3}{1}{100}{#4}
		
		\subFig{#1}{#2}{#3}{2}{226}{#4}
		~
		\subFig{#1}{#2}{#3}{3}{318}{#4}
		\caption{#5 Instâncias 0 a 3.}
		\label{fig:#1_#2_#3_in0_4}
	\end{figure}
	
	\begin{figure}[ht]
		\centering
		\subFig{#1}{#2}{#3}{4}{501}{#4}
		~
		\subFig{#1}{#2}{#3}{5}{657}{#4}
		
		\subFig{#1}{#2}{#3}{6}{783}{#4}
		~
		\subFig{#1}{#2}{#3}{7}{1001}{#4}
		\caption{#5 Instâncias 5 a 7.}
		\label{fig:#1_#2_#3_in5_7}
	\end{figure}
}

% #1 {dvnd, rvnd, gdvnd}, #2 {sog_mog, dc_dd}, #3 {time, imp}, #4 {Tempo do DVND...}
\newcommand{\multiFigureInstancias}[4]{
    \multiFigureInstanciasGeral{#1}{#2}{#3}{box}{#4}
}


\chapter{Resultados} \label{cap:resultados}

Este capítulo exibe os resultados computacionais dos algoritmos propostos no Capítulo~\ref{cap:metodologia} para o caso do PML, para cada instância foi gerado um conjunto com 100 soluções iniciais aleatórias que foram submetidas aos métodos para comparação dos resultados.

Quando há referência à melhoria na solução (\textit{Imp}), esta melhoria pode ser calculada pelo quociente do valor da solução inicial pela solução final, ou seja:
\begin{equation}\label{eq:calculateImprovement}
Imp = \frac{f(\textrm{solução inicial})}{f(\textrm{solução final})}
\end{equation}

Desta forma quanto maior for o valor da melhoria ($Imp$) mais o método melhorou o valor da solução inicial.

\section{Instâncias} \label{sec:instancias}

Todas as instâncias usadas nos testes computacionais e cujas configurações de lançamento foram descritas na Tabela~\ref{tab:neighborhoodsLaunchConfigurarion} são as mesmas usadas em~\cite{wamca2016}.
Para o RVND foi feita uma implementação do algoritmo clássico (Algoritmo~\ref{alg:rvnd}) e também a implementação dataflow mencionada na Figura~\ref{fig:rvndGraph} fazendo uso de uma máquina.
Para o caso do DVND foi utilizada a implementação clássica (Algoritmo~\ref{alg:dvnd}) e a implementação dataflow proposta (Figura~\ref{fig:dvndGraph}), os resultados foram obtidos com diferentes números de máquinas e os mesmos são indicados conforme o caso.

\section{Implementação e ambiente computacional}\label{sec:amb}

A implementação para cada algoritmo proposto no Capítulo~\ref{cap:metodologia} utiliza a linguagem de programação \textit{C++11} em conjunto com a API CUDA\texttrademark, para a implementação dos grafos e do ambiente dataflow foi utilizada a biblioteca Sucuri~\cite{sucuri-original}\footnote{Disponível em \url{https://github.com/tiagoaoa/Sucuri}} implementada em Python, para a integração entre o dataflow e o código CUDA foi utilizada a biblioteca SimplePyCuda~\cite{simple-pycuda}\footnote{Disponível em \url{https://github.com/igormcoelho/simple-pycuda}}. As implementações com múltiplas threads usaram a biblioteca OpenMP.
%As implementações foram compiladas através do \textit{GCC} \textit{(GNU Compiler Collection)}\footnote{O GCC está disponível no seguinte sítio eletrônico: \url{https://gcc.gnu.org/}.} com a \textit{flag} de otimização $-O3$.
O ambiente computacional utilizado em todos os testes neste trabalho consiste de 4 máquinas com a seguinte configuração:

\begin{itemize}
    \item Processador Intel\textregistered Core\texttrademark i7-4820K 3.7 GHz (4 núcleos);
    \item 8 GB de memória RAM;
    \item Sistema Operacional Ubuntu 14.10 (x64);
    \item NVIDIA GeForce GTX 780 com 2304 CUDA cores.
\end{itemize}

\input{resultados/comandosFiguras.tex}

\input{resultados/sog_mog/index.tex}

\input{resultados/rvnd/index.tex}

\input{resultados/dvnd/index.tex}

\input{resultados/gdvnd/index.tex}


\chapter{Resultados} \label{cap:resultados}

Este capítulo exibe os resultados computacionais dos algoritmos propostos no Capítulo~\ref{cap:metodologia} para o caso do PML, para cada instância foi gerado um conjunto com 100 soluções iniciais aleatórias que foram submetidas aos métodos para comparação dos resultados.

Quando há referência à melhoria na solução (\textit{Imp}), esta melhoria pode ser calculada pelo quociente do valor da solução inicial pela solução final, ou seja:
\begin{equation}\label{eq:calculateImprovement}
Imp = \frac{f(\textrm{solução inicial})}{f(\textrm{solução final})}
\end{equation}

Desta forma quanto maior for o valor da melhoria ($Imp$) mais o método melhorou o valor da solução inicial.

\section{Instâncias} \label{sec:instancias}

Todas as instâncias usadas nos testes computacionais e cujas configurações de lançamento foram descritas na Tabela~\ref{tab:neighborhoodsLaunchConfigurarion} são as mesmas usadas em~\cite{wamca2016}.
Para o RVND foi feita uma implementação do algoritmo clássico (Algoritmo~\ref{alg:rvnd}) e também a implementação dataflow mencionada na Figura~\ref{fig:rvndGraph} fazendo uso de uma máquina.
Para o caso do DVND foi utilizada a implementação clássica (Algoritmo~\ref{alg:dvnd}) e a implementação dataflow proposta (Figura~\ref{fig:dvndGraph}), os resultados foram obtidos com diferentes números de máquinas e os mesmos são indicados conforme o caso.

\section{Implementação e ambiente computacional}\label{sec:amb}

A implementação para cada algoritmo proposto no Capítulo~\ref{cap:metodologia} utiliza a linguagem de programação \textit{C++11} em conjunto com a API CUDA\texttrademark, para a implementação dos grafos e do ambiente dataflow foi utilizada a biblioteca Sucuri~\cite{sucuri-original}\footnote{Disponível em \url{https://github.com/tiagoaoa/Sucuri}} implementada em Python, para a integração entre o dataflow e o código CUDA foi utilizada a biblioteca SimplePyCuda~\cite{simple-pycuda}\footnote{Disponível em \url{https://github.com/igormcoelho/simple-pycuda}}. As implementações com múltiplas threads usaram a biblioteca OpenMP.
%As implementações foram compiladas através do \textit{GCC} \textit{(GNU Compiler Collection)}\footnote{O GCC está disponível no seguinte sítio eletrônico: \url{https://gcc.gnu.org/}.} com a \textit{flag} de otimização $-O3$.
O ambiente computacional utilizado em todos os testes neste trabalho consiste de 4 máquinas com a seguinte configuração:

\begin{itemize}
    \item Processador Intel\textregistered Core\texttrademark i7-4820K 3.7 GHz (4 núcleos);
    \item 8 GB de memória RAM;
    \item Sistema Operacional Ubuntu 14.10 (x64);
    \item NVIDIA GeForce GTX 780 com 2304 CUDA cores.
\end{itemize}

\input{resultados/comandosFiguras.tex}

\input{resultados/sog_mog/index.tex}

\input{resultados/rvnd/index.tex}

\input{resultados/dvnd/index.tex}

\input{resultados/gdvnd/index.tex}


\chapter{Resultados} \label{cap:resultados}

Este capítulo exibe os resultados computacionais dos algoritmos propostos no Capítulo~\ref{cap:metodologia} para o caso do PML, para cada instância foi gerado um conjunto com 100 soluções iniciais aleatórias que foram submetidas aos métodos para comparação dos resultados.

Quando há referência à melhoria na solução (\textit{Imp}), esta melhoria pode ser calculada pelo quociente do valor da solução inicial pela solução final, ou seja:
\begin{equation}\label{eq:calculateImprovement}
Imp = \frac{f(\textrm{solução inicial})}{f(\textrm{solução final})}
\end{equation}

Desta forma quanto maior for o valor da melhoria ($Imp$) mais o método melhorou o valor da solução inicial.

\section{Instâncias} \label{sec:instancias}

Todas as instâncias usadas nos testes computacionais e cujas configurações de lançamento foram descritas na Tabela~\ref{tab:neighborhoodsLaunchConfigurarion} são as mesmas usadas em~\cite{wamca2016}.
Para o RVND foi feita uma implementação do algoritmo clássico (Algoritmo~\ref{alg:rvnd}) e também a implementação dataflow mencionada na Figura~\ref{fig:rvndGraph} fazendo uso de uma máquina.
Para o caso do DVND foi utilizada a implementação clássica (Algoritmo~\ref{alg:dvnd}) e a implementação dataflow proposta (Figura~\ref{fig:dvndGraph}), os resultados foram obtidos com diferentes números de máquinas e os mesmos são indicados conforme o caso.

\section{Implementação e ambiente computacional}\label{sec:amb}

A implementação para cada algoritmo proposto no Capítulo~\ref{cap:metodologia} utiliza a linguagem de programação \textit{C++11} em conjunto com a API CUDA\texttrademark, para a implementação dos grafos e do ambiente dataflow foi utilizada a biblioteca Sucuri~\cite{sucuri-original}\footnote{Disponível em \url{https://github.com/tiagoaoa/Sucuri}} implementada em Python, para a integração entre o dataflow e o código CUDA foi utilizada a biblioteca SimplePyCuda~\cite{simple-pycuda}\footnote{Disponível em \url{https://github.com/igormcoelho/simple-pycuda}}. As implementações com múltiplas threads usaram a biblioteca OpenMP.
%As implementações foram compiladas através do \textit{GCC} \textit{(GNU Compiler Collection)}\footnote{O GCC está disponível no seguinte sítio eletrônico: \url{https://gcc.gnu.org/}.} com a \textit{flag} de otimização $-O3$.
O ambiente computacional utilizado em todos os testes neste trabalho consiste de 4 máquinas com a seguinte configuração:

\begin{itemize}
    \item Processador Intel\textregistered Core\texttrademark i7-4820K 3.7 GHz (4 núcleos);
    \item 8 GB de memória RAM;
    \item Sistema Operacional Ubuntu 14.10 (x64);
    \item NVIDIA GeForce GTX 780 com 2304 CUDA cores.
\end{itemize}

\input{resultados/comandosFiguras.tex}

\input{resultados/sog_mog/index.tex}

\input{resultados/rvnd/index.tex}

\input{resultados/dvnd/index.tex}

\input{resultados/gdvnd/index.tex}


\chapter{Resultados} \label{cap:resultados}

Este capítulo exibe os resultados computacionais dos algoritmos propostos no Capítulo~\ref{cap:metodologia} para o caso do PML, para cada instância foi gerado um conjunto com 100 soluções iniciais aleatórias que foram submetidas aos métodos para comparação dos resultados.

Quando há referência à melhoria na solução (\textit{Imp}), esta melhoria pode ser calculada pelo quociente do valor da solução inicial pela solução final, ou seja:
\begin{equation}\label{eq:calculateImprovement}
Imp = \frac{f(\textrm{solução inicial})}{f(\textrm{solução final})}
\end{equation}

Desta forma quanto maior for o valor da melhoria ($Imp$) mais o método melhorou o valor da solução inicial.

\section{Instâncias} \label{sec:instancias}

Todas as instâncias usadas nos testes computacionais e cujas configurações de lançamento foram descritas na Tabela~\ref{tab:neighborhoodsLaunchConfigurarion} são as mesmas usadas em~\cite{wamca2016}.
Para o RVND foi feita uma implementação do algoritmo clássico (Algoritmo~\ref{alg:rvnd}) e também a implementação dataflow mencionada na Figura~\ref{fig:rvndGraph} fazendo uso de uma máquina.
Para o caso do DVND foi utilizada a implementação clássica (Algoritmo~\ref{alg:dvnd}) e a implementação dataflow proposta (Figura~\ref{fig:dvndGraph}), os resultados foram obtidos com diferentes números de máquinas e os mesmos são indicados conforme o caso.

\section{Implementação e ambiente computacional}\label{sec:amb}

A implementação para cada algoritmo proposto no Capítulo~\ref{cap:metodologia} utiliza a linguagem de programação \textit{C++11} em conjunto com a API CUDA\texttrademark, para a implementação dos grafos e do ambiente dataflow foi utilizada a biblioteca Sucuri~\cite{sucuri-original}\footnote{Disponível em \url{https://github.com/tiagoaoa/Sucuri}} implementada em Python, para a integração entre o dataflow e o código CUDA foi utilizada a biblioteca SimplePyCuda~\cite{simple-pycuda}\footnote{Disponível em \url{https://github.com/igormcoelho/simple-pycuda}}. As implementações com múltiplas threads usaram a biblioteca OpenMP.
%As implementações foram compiladas através do \textit{GCC} \textit{(GNU Compiler Collection)}\footnote{O GCC está disponível no seguinte sítio eletrônico: \url{https://gcc.gnu.org/}.} com a \textit{flag} de otimização $-O3$.
O ambiente computacional utilizado em todos os testes neste trabalho consiste de 4 máquinas com a seguinte configuração:

\begin{itemize}
    \item Processador Intel\textregistered Core\texttrademark i7-4820K 3.7 GHz (4 núcleos);
    \item 8 GB de memória RAM;
    \item Sistema Operacional Ubuntu 14.10 (x64);
    \item NVIDIA GeForce GTX 780 com 2304 CUDA cores.
\end{itemize}

\input{resultados/comandosFiguras.tex}

\input{resultados/sog_mog/index.tex}

\input{resultados/rvnd/index.tex}

\input{resultados/dvnd/index.tex}

\input{resultados/gdvnd/index.tex}




\chapter{Resultados} \label{cap:resultados}

Este capítulo exibe os resultados computacionais dos algoritmos propostos no Capítulo~\ref{cap:metodologia} para o caso do PML, para cada instância foi gerado um conjunto com 100 soluções iniciais aleatórias que foram submetidas aos métodos para comparação dos resultados.

Quando há referência à melhoria na solução (\textit{Imp}), esta melhoria pode ser calculada pelo quociente do valor da solução inicial pela solução final, ou seja:
\begin{equation}\label{eq:calculateImprovement}
Imp = \frac{f(\textrm{solução inicial})}{f(\textrm{solução final})}
\end{equation}

Desta forma quanto maior for o valor da melhoria ($Imp$) mais o método melhorou o valor da solução inicial.

\section{Instâncias} \label{sec:instancias}

Todas as instâncias usadas nos testes computacionais e cujas configurações de lançamento foram descritas na Tabela~\ref{tab:neighborhoodsLaunchConfigurarion} são as mesmas usadas em~\cite{wamca2016}.
Para o RVND foi feita uma implementação do algoritmo clássico (Algoritmo~\ref{alg:rvnd}) e também a implementação dataflow mencionada na Figura~\ref{fig:rvndGraph} fazendo uso de uma máquina.
Para o caso do DVND foi utilizada a implementação clássica (Algoritmo~\ref{alg:dvnd}) e a implementação dataflow proposta (Figura~\ref{fig:dvndGraph}), os resultados foram obtidos com diferentes números de máquinas e os mesmos são indicados conforme o caso.

\section{Implementação e ambiente computacional}\label{sec:amb}

A implementação para cada algoritmo proposto no Capítulo~\ref{cap:metodologia} utiliza a linguagem de programação \textit{C++11} em conjunto com a API CUDA\texttrademark, para a implementação dos grafos e do ambiente dataflow foi utilizada a biblioteca Sucuri~\cite{sucuri-original}\footnote{Disponível em \url{https://github.com/tiagoaoa/Sucuri}} implementada em Python, para a integração entre o dataflow e o código CUDA foi utilizada a biblioteca SimplePyCuda~\cite{simple-pycuda}\footnote{Disponível em \url{https://github.com/igormcoelho/simple-pycuda}}. As implementações com múltiplas threads usaram a biblioteca OpenMP.
%As implementações foram compiladas através do \textit{GCC} \textit{(GNU Compiler Collection)}\footnote{O GCC está disponível no seguinte sítio eletrônico: \url{https://gcc.gnu.org/}.} com a \textit{flag} de otimização $-O3$.
O ambiente computacional utilizado em todos os testes neste trabalho consiste de 4 máquinas com a seguinte configuração:

\begin{itemize}
    \item Processador Intel\textregistered Core\texttrademark i7-4820K 3.7 GHz (4 núcleos);
    \item 8 GB de memória RAM;
    \item Sistema Operacional Ubuntu 14.10 (x64);
    \item NVIDIA GeForce GTX 780 com 2304 CUDA cores.
\end{itemize}

\newcommand{\figureDvndOrRvndDcDd}[7]{
% #1 {box, scatter}, #2 {count, imp, time}, #3 {instance number}, #4 {Tempo, Melhoria}, #5 tamanho instância, #6 {DVND, RVND}, #7 {dvnd, rvnd}
\begin{figure}%
    \centering
    \includegraphics[scale=0.9]{figuras/#7/dc_dd/#1/#7_#1100sol_#2_in#3.png}
    \caption{#4 do #6 para a instância #3 de tamanho #5. $m$ indica o número de máquinas, \textit{DC} refere-se ao #6 clássico e \textit{DD} ao #6 implementado em dataflow.}%
    \label{fig:#2_#7DcDd_in#3}%
\end{figure}
}

\newcommand{\figureDvndDcDd}[5]{
% #1 {box, scatter}, #2 {count, imp, time}, #3 {instance number}, #4 {Tempo, Melhoria}, #5 tamanho instância
    \figureDvndOrRvndDcDd{#1}{#2}{#3}{#4}{#5}{DVND}{dvnd}
}

\newcommand{\figureRvndDcDd}[5]{
% #1 {box, scatter}, #2 {count, imp, time}, #3 {instance number}, #4 {Tempo, Melhoria}, #5 tamanho instância
    \figureDvndOrRvndDcDd{#1}{#2}{#3}{#4}{#5}{RVND}{rvnd}
}

\newcommand{\tabelaEstatisticasGeral}[6]{
% #1 Descrição, #2 label, #3 {dvnd, rvnd}, #4 {DVND, RVND}, #6 DD/DC, #6 Conteúdo
\begin{table}[ht]
    \centering
    \begin{tabular}{c|ccc|cc|ccc|cc|c}
        \hline \hline
        \# & Tipo & $m$ & $n$ & $min$ & $max$ & 1Q & 2Q & 3Q & $\overline{x}$ & $\sigma$ & $p-valor$ \\ \hline
        #6
    \end{tabular}
    \caption{#1 #4 #5
        Instância (\#), tipo de implementação (Tipo), número de máquinas ($m$), tamanho da instância ($n$), valor mínimo ($min$), máximo ($max$), primeiro quartil (1Q), mediana (2Q), terceiro quartil (3Q), média ($\overline{x}$), desvio padrão ($\sigma$) e p-valor para o teste de Wilcox entre as versões (valores em negrito quando $p-valor > 0.05$).
    }
    \label{tab:#3DcDd#2}
\end{table}
}

\newcommand{\tabelaEstatisticas}[5]{
    \tabelaEstatisticasGeral{#1}{#2}{#3}{#4}{na implementação clássica (DC) e a proposta de implementação usando dataflow (DD).}{#5}
}

\newcommand{\figureDvndSogMog}[7]{
% #1 {box, scatter}, #2 {count, imp, time}, #3 {instance number}, #4 {Tempo, Melhoria}, #5 tamanho instância, #6 {DVND, RVND}, #7 {dvnd, rvnd}
\begin{figure}%
    \centering
    \includegraphics[scale=0.9]{figuras/#7/sog_mog/#1/#7_#1100sol_#2_in#3.png}
    \caption{#4 do #6 para a instância #3 de tamanho #5. \textit{SOG} refere-se a uma porta de saída e \textit{MOG} a múltiplas portas de saída.}%
    \label{fig:#2_#7SogMog_in#3}%
\end{figure}
}

\newcommand{\figureDvndGdvnd}[9]{
% #1 {box, scatter}, #2 {count, imp, time}, #3 {instance number}, #4 {Tempo, Melhoria}, #5 tamanho instância, #6 {DVND, RVND}, #7 {dvnd, rvnd}, #8 {man_time, full_time}, #9 {man, dvnd} #10 descricao
\begin{figure}%
    \centering
    \includegraphics[scale=0.9]{figuras/#7/#8/#1/#9_#1100sol_#2_in#3.png}
    \caption{#4 do #6 para a instância #3 de tamanho #5. \textit{DVND} refere-se ao tempo gasto pelo algoritmo de mesmo nome, para \textit{GDVND} é análogo ao anterior, no caso do \textit{GDVND-MAN} este se refere ao tempo do GDVND subtraido do tempo para gerenciar os movimentos.}%
    \label{fig:#2_#7_#8_in#3}%
\end{figure}
}

\newcommand{\figureDvndGdvndTime}[8]{
    \figureDvndGdvnd{#1}{#2}{#3}{#4}{#5}{#6}{#7}{man_time}{man}
}

\newcommand{\figureGdvndDvndRvnd}[9]{
% #1 {box, scatter}, #2 {count, imp, time}, #3 {instance number}, #4 {Tempo, Melhoria}, #5 tamanho instância, #6 {DVND, RVND}, #7 {dvnd, rvnd}, #8 {man_time, full_time}, #9 {man, dvnd} #10 descricao
\begin{figure}%
    \centering
    \includegraphics[scale=0.9]{figuras/#7/#8/#1/#9_#1100sol_#2_in#3.png}
    \caption{#4 do #6 para a instância #3 de tamanho #5. \textit{DVND}, \textit{GDVND} e \textit{RVND} referem-se ao tempo gasto pelos algoritmos de mesmo nome.}%
    \label{fig:#2_#7_#8_in#3}%
\end{figure}
}

% \subfloat[$m=#1$]{{ %scale=0.225
%         \includegraphics[scale=0.425]{figuras/dvnd/n#1/time#2.png}
%         \label{fig:timeDvndRvnd_n#1in#2}
%     }}%
% #1 {dvnd, rvnd, gdvnd}, #2 {sog_mog, dc_dd}, #3 {time, imp}, #4 in, #5 tamanho, #6 {box, scatter}
\newcommand{\subFig}[6]{
    \subfloat[][Instância #4, $n=#5$]{
        \includegraphics[scale=0.425]{figuras/#1/#2/#6/#1_#6100sol_#3_in#4.png}
		\label{fig:#1_#2_#3_in#4}
    }
% 	\begin{subfigure}{0.45\textwidth} % dvnd_box100sol_imp_in0
% 		\includegraphics[scale=0.425]{figuras/#1/#2/#6/#1_#6100sol_#3_in#4.png}
% 		\caption{Instância #4, $n=#5$}
        % \label{fig:#1_#2_#3_in#4}
    % \end{subfigure}
}

\newcommand{\subFigBox}[5]{
	\subFig{#1}{#2}{#3}{#4}{#5}{box}
}

\newcommand{\subFigScatter}[5]{
	\subFig{#1}{#2}{#3}{#4}{#5}{scatter}
}

% #1 {dvnd, rvnd, gdvnd}, #2 {sog_mog, dc_dd}, #3 {time, imp}, #4 {box, scatter}, #5 {Tempo do DVND...}
\newcommand{\multiFigureInstanciasGeral}[5]{
	\begin{figure}[ht]
		\centering
		\subFig{#1}{#2}{#3}{0}{52}{#4}
		~
		\subFig{#1}{#2}{#3}{1}{100}{#4}
		
		\subFig{#1}{#2}{#3}{2}{226}{#4}
		~
		\subFig{#1}{#2}{#3}{3}{318}{#4}
		\caption{#5 Instâncias 0 a 3.}
		\label{fig:#1_#2_#3_in0_4}
	\end{figure}
	
	\begin{figure}[ht]
		\centering
		\subFig{#1}{#2}{#3}{4}{501}{#4}
		~
		\subFig{#1}{#2}{#3}{5}{657}{#4}
		
		\subFig{#1}{#2}{#3}{6}{783}{#4}
		~
		\subFig{#1}{#2}{#3}{7}{1001}{#4}
		\caption{#5 Instâncias 5 a 7.}
		\label{fig:#1_#2_#3_in5_7}
	\end{figure}
}

% #1 {dvnd, rvnd, gdvnd}, #2 {sog_mog, dc_dd}, #3 {time, imp}, #4 {Tempo do DVND...}
\newcommand{\multiFigureInstancias}[4]{
    \multiFigureInstanciasGeral{#1}{#2}{#3}{box}{#4}
}


\chapter{Resultados} \label{cap:resultados}

Este capítulo exibe os resultados computacionais dos algoritmos propostos no Capítulo~\ref{cap:metodologia} para o caso do PML, para cada instância foi gerado um conjunto com 100 soluções iniciais aleatórias que foram submetidas aos métodos para comparação dos resultados.

Quando há referência à melhoria na solução (\textit{Imp}), esta melhoria pode ser calculada pelo quociente do valor da solução inicial pela solução final, ou seja:
\begin{equation}\label{eq:calculateImprovement}
Imp = \frac{f(\textrm{solução inicial})}{f(\textrm{solução final})}
\end{equation}

Desta forma quanto maior for o valor da melhoria ($Imp$) mais o método melhorou o valor da solução inicial.

\section{Instâncias} \label{sec:instancias}

Todas as instâncias usadas nos testes computacionais e cujas configurações de lançamento foram descritas na Tabela~\ref{tab:neighborhoodsLaunchConfigurarion} são as mesmas usadas em~\cite{wamca2016}.
Para o RVND foi feita uma implementação do algoritmo clássico (Algoritmo~\ref{alg:rvnd}) e também a implementação dataflow mencionada na Figura~\ref{fig:rvndGraph} fazendo uso de uma máquina.
Para o caso do DVND foi utilizada a implementação clássica (Algoritmo~\ref{alg:dvnd}) e a implementação dataflow proposta (Figura~\ref{fig:dvndGraph}), os resultados foram obtidos com diferentes números de máquinas e os mesmos são indicados conforme o caso.

\section{Implementação e ambiente computacional}\label{sec:amb}

A implementação para cada algoritmo proposto no Capítulo~\ref{cap:metodologia} utiliza a linguagem de programação \textit{C++11} em conjunto com a API CUDA\texttrademark, para a implementação dos grafos e do ambiente dataflow foi utilizada a biblioteca Sucuri~\cite{sucuri-original}\footnote{Disponível em \url{https://github.com/tiagoaoa/Sucuri}} implementada em Python, para a integração entre o dataflow e o código CUDA foi utilizada a biblioteca SimplePyCuda~\cite{simple-pycuda}\footnote{Disponível em \url{https://github.com/igormcoelho/simple-pycuda}}. As implementações com múltiplas threads usaram a biblioteca OpenMP.
%As implementações foram compiladas através do \textit{GCC} \textit{(GNU Compiler Collection)}\footnote{O GCC está disponível no seguinte sítio eletrônico: \url{https://gcc.gnu.org/}.} com a \textit{flag} de otimização $-O3$.
O ambiente computacional utilizado em todos os testes neste trabalho consiste de 4 máquinas com a seguinte configuração:

\begin{itemize}
    \item Processador Intel\textregistered Core\texttrademark i7-4820K 3.7 GHz (4 núcleos);
    \item 8 GB de memória RAM;
    \item Sistema Operacional Ubuntu 14.10 (x64);
    \item NVIDIA GeForce GTX 780 com 2304 CUDA cores.
\end{itemize}

\newcommand{\figureDvndOrRvndDcDd}[7]{
% #1 {box, scatter}, #2 {count, imp, time}, #3 {instance number}, #4 {Tempo, Melhoria}, #5 tamanho instância, #6 {DVND, RVND}, #7 {dvnd, rvnd}
\begin{figure}%
    \centering
    \includegraphics[scale=0.9]{figuras/#7/dc_dd/#1/#7_#1100sol_#2_in#3.png}
    \caption{#4 do #6 para a instância #3 de tamanho #5. $m$ indica o número de máquinas, \textit{DC} refere-se ao #6 clássico e \textit{DD} ao #6 implementado em dataflow.}%
    \label{fig:#2_#7DcDd_in#3}%
\end{figure}
}

\newcommand{\figureDvndDcDd}[5]{
% #1 {box, scatter}, #2 {count, imp, time}, #3 {instance number}, #4 {Tempo, Melhoria}, #5 tamanho instância
    \figureDvndOrRvndDcDd{#1}{#2}{#3}{#4}{#5}{DVND}{dvnd}
}

\newcommand{\figureRvndDcDd}[5]{
% #1 {box, scatter}, #2 {count, imp, time}, #3 {instance number}, #4 {Tempo, Melhoria}, #5 tamanho instância
    \figureDvndOrRvndDcDd{#1}{#2}{#3}{#4}{#5}{RVND}{rvnd}
}

\newcommand{\tabelaEstatisticasGeral}[6]{
% #1 Descrição, #2 label, #3 {dvnd, rvnd}, #4 {DVND, RVND}, #6 DD/DC, #6 Conteúdo
\begin{table}[ht]
    \centering
    \begin{tabular}{c|ccc|cc|ccc|cc|c}
        \hline \hline
        \# & Tipo & $m$ & $n$ & $min$ & $max$ & 1Q & 2Q & 3Q & $\overline{x}$ & $\sigma$ & $p-valor$ \\ \hline
        #6
    \end{tabular}
    \caption{#1 #4 #5
        Instância (\#), tipo de implementação (Tipo), número de máquinas ($m$), tamanho da instância ($n$), valor mínimo ($min$), máximo ($max$), primeiro quartil (1Q), mediana (2Q), terceiro quartil (3Q), média ($\overline{x}$), desvio padrão ($\sigma$) e p-valor para o teste de Wilcox entre as versões (valores em negrito quando $p-valor > 0.05$).
    }
    \label{tab:#3DcDd#2}
\end{table}
}

\newcommand{\tabelaEstatisticas}[5]{
    \tabelaEstatisticasGeral{#1}{#2}{#3}{#4}{na implementação clássica (DC) e a proposta de implementação usando dataflow (DD).}{#5}
}

\newcommand{\figureDvndSogMog}[7]{
% #1 {box, scatter}, #2 {count, imp, time}, #3 {instance number}, #4 {Tempo, Melhoria}, #5 tamanho instância, #6 {DVND, RVND}, #7 {dvnd, rvnd}
\begin{figure}%
    \centering
    \includegraphics[scale=0.9]{figuras/#7/sog_mog/#1/#7_#1100sol_#2_in#3.png}
    \caption{#4 do #6 para a instância #3 de tamanho #5. \textit{SOG} refere-se a uma porta de saída e \textit{MOG} a múltiplas portas de saída.}%
    \label{fig:#2_#7SogMog_in#3}%
\end{figure}
}

\newcommand{\figureDvndGdvnd}[9]{
% #1 {box, scatter}, #2 {count, imp, time}, #3 {instance number}, #4 {Tempo, Melhoria}, #5 tamanho instância, #6 {DVND, RVND}, #7 {dvnd, rvnd}, #8 {man_time, full_time}, #9 {man, dvnd} #10 descricao
\begin{figure}%
    \centering
    \includegraphics[scale=0.9]{figuras/#7/#8/#1/#9_#1100sol_#2_in#3.png}
    \caption{#4 do #6 para a instância #3 de tamanho #5. \textit{DVND} refere-se ao tempo gasto pelo algoritmo de mesmo nome, para \textit{GDVND} é análogo ao anterior, no caso do \textit{GDVND-MAN} este se refere ao tempo do GDVND subtraido do tempo para gerenciar os movimentos.}%
    \label{fig:#2_#7_#8_in#3}%
\end{figure}
}

\newcommand{\figureDvndGdvndTime}[8]{
    \figureDvndGdvnd{#1}{#2}{#3}{#4}{#5}{#6}{#7}{man_time}{man}
}

\newcommand{\figureGdvndDvndRvnd}[9]{
% #1 {box, scatter}, #2 {count, imp, time}, #3 {instance number}, #4 {Tempo, Melhoria}, #5 tamanho instância, #6 {DVND, RVND}, #7 {dvnd, rvnd}, #8 {man_time, full_time}, #9 {man, dvnd} #10 descricao
\begin{figure}%
    \centering
    \includegraphics[scale=0.9]{figuras/#7/#8/#1/#9_#1100sol_#2_in#3.png}
    \caption{#4 do #6 para a instância #3 de tamanho #5. \textit{DVND}, \textit{GDVND} e \textit{RVND} referem-se ao tempo gasto pelos algoritmos de mesmo nome.}%
    \label{fig:#2_#7_#8_in#3}%
\end{figure}
}

% \subfloat[$m=#1$]{{ %scale=0.225
%         \includegraphics[scale=0.425]{figuras/dvnd/n#1/time#2.png}
%         \label{fig:timeDvndRvnd_n#1in#2}
%     }}%
% #1 {dvnd, rvnd, gdvnd}, #2 {sog_mog, dc_dd}, #3 {time, imp}, #4 in, #5 tamanho, #6 {box, scatter}
\newcommand{\subFig}[6]{
    \subfloat[][Instância #4, $n=#5$]{
        \includegraphics[scale=0.425]{figuras/#1/#2/#6/#1_#6100sol_#3_in#4.png}
		\label{fig:#1_#2_#3_in#4}
    }
% 	\begin{subfigure}{0.45\textwidth} % dvnd_box100sol_imp_in0
% 		\includegraphics[scale=0.425]{figuras/#1/#2/#6/#1_#6100sol_#3_in#4.png}
% 		\caption{Instância #4, $n=#5$}
        % \label{fig:#1_#2_#3_in#4}
    % \end{subfigure}
}

\newcommand{\subFigBox}[5]{
	\subFig{#1}{#2}{#3}{#4}{#5}{box}
}

\newcommand{\subFigScatter}[5]{
	\subFig{#1}{#2}{#3}{#4}{#5}{scatter}
}

% #1 {dvnd, rvnd, gdvnd}, #2 {sog_mog, dc_dd}, #3 {time, imp}, #4 {box, scatter}, #5 {Tempo do DVND...}
\newcommand{\multiFigureInstanciasGeral}[5]{
	\begin{figure}[ht]
		\centering
		\subFig{#1}{#2}{#3}{0}{52}{#4}
		~
		\subFig{#1}{#2}{#3}{1}{100}{#4}
		
		\subFig{#1}{#2}{#3}{2}{226}{#4}
		~
		\subFig{#1}{#2}{#3}{3}{318}{#4}
		\caption{#5 Instâncias 0 a 3.}
		\label{fig:#1_#2_#3_in0_4}
	\end{figure}
	
	\begin{figure}[ht]
		\centering
		\subFig{#1}{#2}{#3}{4}{501}{#4}
		~
		\subFig{#1}{#2}{#3}{5}{657}{#4}
		
		\subFig{#1}{#2}{#3}{6}{783}{#4}
		~
		\subFig{#1}{#2}{#3}{7}{1001}{#4}
		\caption{#5 Instâncias 5 a 7.}
		\label{fig:#1_#2_#3_in5_7}
	\end{figure}
}

% #1 {dvnd, rvnd, gdvnd}, #2 {sog_mog, dc_dd}, #3 {time, imp}, #4 {Tempo do DVND...}
\newcommand{\multiFigureInstancias}[4]{
    \multiFigureInstanciasGeral{#1}{#2}{#3}{box}{#4}
}


\chapter{Resultados} \label{cap:resultados}

Este capítulo exibe os resultados computacionais dos algoritmos propostos no Capítulo~\ref{cap:metodologia} para o caso do PML, para cada instância foi gerado um conjunto com 100 soluções iniciais aleatórias que foram submetidas aos métodos para comparação dos resultados.

Quando há referência à melhoria na solução (\textit{Imp}), esta melhoria pode ser calculada pelo quociente do valor da solução inicial pela solução final, ou seja:
\begin{equation}\label{eq:calculateImprovement}
Imp = \frac{f(\textrm{solução inicial})}{f(\textrm{solução final})}
\end{equation}

Desta forma quanto maior for o valor da melhoria ($Imp$) mais o método melhorou o valor da solução inicial.

\section{Instâncias} \label{sec:instancias}

Todas as instâncias usadas nos testes computacionais e cujas configurações de lançamento foram descritas na Tabela~\ref{tab:neighborhoodsLaunchConfigurarion} são as mesmas usadas em~\cite{wamca2016}.
Para o RVND foi feita uma implementação do algoritmo clássico (Algoritmo~\ref{alg:rvnd}) e também a implementação dataflow mencionada na Figura~\ref{fig:rvndGraph} fazendo uso de uma máquina.
Para o caso do DVND foi utilizada a implementação clássica (Algoritmo~\ref{alg:dvnd}) e a implementação dataflow proposta (Figura~\ref{fig:dvndGraph}), os resultados foram obtidos com diferentes números de máquinas e os mesmos são indicados conforme o caso.

\section{Implementação e ambiente computacional}\label{sec:amb}

A implementação para cada algoritmo proposto no Capítulo~\ref{cap:metodologia} utiliza a linguagem de programação \textit{C++11} em conjunto com a API CUDA\texttrademark, para a implementação dos grafos e do ambiente dataflow foi utilizada a biblioteca Sucuri~\cite{sucuri-original}\footnote{Disponível em \url{https://github.com/tiagoaoa/Sucuri}} implementada em Python, para a integração entre o dataflow e o código CUDA foi utilizada a biblioteca SimplePyCuda~\cite{simple-pycuda}\footnote{Disponível em \url{https://github.com/igormcoelho/simple-pycuda}}. As implementações com múltiplas threads usaram a biblioteca OpenMP.
%As implementações foram compiladas através do \textit{GCC} \textit{(GNU Compiler Collection)}\footnote{O GCC está disponível no seguinte sítio eletrônico: \url{https://gcc.gnu.org/}.} com a \textit{flag} de otimização $-O3$.
O ambiente computacional utilizado em todos os testes neste trabalho consiste de 4 máquinas com a seguinte configuração:

\begin{itemize}
    \item Processador Intel\textregistered Core\texttrademark i7-4820K 3.7 GHz (4 núcleos);
    \item 8 GB de memória RAM;
    \item Sistema Operacional Ubuntu 14.10 (x64);
    \item NVIDIA GeForce GTX 780 com 2304 CUDA cores.
\end{itemize}

\input{resultados/comandosFiguras.tex}

\input{resultados/sog_mog/index.tex}

\input{resultados/rvnd/index.tex}

\input{resultados/dvnd/index.tex}

\input{resultados/gdvnd/index.tex}


\chapter{Resultados} \label{cap:resultados}

Este capítulo exibe os resultados computacionais dos algoritmos propostos no Capítulo~\ref{cap:metodologia} para o caso do PML, para cada instância foi gerado um conjunto com 100 soluções iniciais aleatórias que foram submetidas aos métodos para comparação dos resultados.

Quando há referência à melhoria na solução (\textit{Imp}), esta melhoria pode ser calculada pelo quociente do valor da solução inicial pela solução final, ou seja:
\begin{equation}\label{eq:calculateImprovement}
Imp = \frac{f(\textrm{solução inicial})}{f(\textrm{solução final})}
\end{equation}

Desta forma quanto maior for o valor da melhoria ($Imp$) mais o método melhorou o valor da solução inicial.

\section{Instâncias} \label{sec:instancias}

Todas as instâncias usadas nos testes computacionais e cujas configurações de lançamento foram descritas na Tabela~\ref{tab:neighborhoodsLaunchConfigurarion} são as mesmas usadas em~\cite{wamca2016}.
Para o RVND foi feita uma implementação do algoritmo clássico (Algoritmo~\ref{alg:rvnd}) e também a implementação dataflow mencionada na Figura~\ref{fig:rvndGraph} fazendo uso de uma máquina.
Para o caso do DVND foi utilizada a implementação clássica (Algoritmo~\ref{alg:dvnd}) e a implementação dataflow proposta (Figura~\ref{fig:dvndGraph}), os resultados foram obtidos com diferentes números de máquinas e os mesmos são indicados conforme o caso.

\section{Implementação e ambiente computacional}\label{sec:amb}

A implementação para cada algoritmo proposto no Capítulo~\ref{cap:metodologia} utiliza a linguagem de programação \textit{C++11} em conjunto com a API CUDA\texttrademark, para a implementação dos grafos e do ambiente dataflow foi utilizada a biblioteca Sucuri~\cite{sucuri-original}\footnote{Disponível em \url{https://github.com/tiagoaoa/Sucuri}} implementada em Python, para a integração entre o dataflow e o código CUDA foi utilizada a biblioteca SimplePyCuda~\cite{simple-pycuda}\footnote{Disponível em \url{https://github.com/igormcoelho/simple-pycuda}}. As implementações com múltiplas threads usaram a biblioteca OpenMP.
%As implementações foram compiladas através do \textit{GCC} \textit{(GNU Compiler Collection)}\footnote{O GCC está disponível no seguinte sítio eletrônico: \url{https://gcc.gnu.org/}.} com a \textit{flag} de otimização $-O3$.
O ambiente computacional utilizado em todos os testes neste trabalho consiste de 4 máquinas com a seguinte configuração:

\begin{itemize}
    \item Processador Intel\textregistered Core\texttrademark i7-4820K 3.7 GHz (4 núcleos);
    \item 8 GB de memória RAM;
    \item Sistema Operacional Ubuntu 14.10 (x64);
    \item NVIDIA GeForce GTX 780 com 2304 CUDA cores.
\end{itemize}

\input{resultados/comandosFiguras.tex}

\input{resultados/sog_mog/index.tex}

\input{resultados/rvnd/index.tex}

\input{resultados/dvnd/index.tex}

\input{resultados/gdvnd/index.tex}


\chapter{Resultados} \label{cap:resultados}

Este capítulo exibe os resultados computacionais dos algoritmos propostos no Capítulo~\ref{cap:metodologia} para o caso do PML, para cada instância foi gerado um conjunto com 100 soluções iniciais aleatórias que foram submetidas aos métodos para comparação dos resultados.

Quando há referência à melhoria na solução (\textit{Imp}), esta melhoria pode ser calculada pelo quociente do valor da solução inicial pela solução final, ou seja:
\begin{equation}\label{eq:calculateImprovement}
Imp = \frac{f(\textrm{solução inicial})}{f(\textrm{solução final})}
\end{equation}

Desta forma quanto maior for o valor da melhoria ($Imp$) mais o método melhorou o valor da solução inicial.

\section{Instâncias} \label{sec:instancias}

Todas as instâncias usadas nos testes computacionais e cujas configurações de lançamento foram descritas na Tabela~\ref{tab:neighborhoodsLaunchConfigurarion} são as mesmas usadas em~\cite{wamca2016}.
Para o RVND foi feita uma implementação do algoritmo clássico (Algoritmo~\ref{alg:rvnd}) e também a implementação dataflow mencionada na Figura~\ref{fig:rvndGraph} fazendo uso de uma máquina.
Para o caso do DVND foi utilizada a implementação clássica (Algoritmo~\ref{alg:dvnd}) e a implementação dataflow proposta (Figura~\ref{fig:dvndGraph}), os resultados foram obtidos com diferentes números de máquinas e os mesmos são indicados conforme o caso.

\section{Implementação e ambiente computacional}\label{sec:amb}

A implementação para cada algoritmo proposto no Capítulo~\ref{cap:metodologia} utiliza a linguagem de programação \textit{C++11} em conjunto com a API CUDA\texttrademark, para a implementação dos grafos e do ambiente dataflow foi utilizada a biblioteca Sucuri~\cite{sucuri-original}\footnote{Disponível em \url{https://github.com/tiagoaoa/Sucuri}} implementada em Python, para a integração entre o dataflow e o código CUDA foi utilizada a biblioteca SimplePyCuda~\cite{simple-pycuda}\footnote{Disponível em \url{https://github.com/igormcoelho/simple-pycuda}}. As implementações com múltiplas threads usaram a biblioteca OpenMP.
%As implementações foram compiladas através do \textit{GCC} \textit{(GNU Compiler Collection)}\footnote{O GCC está disponível no seguinte sítio eletrônico: \url{https://gcc.gnu.org/}.} com a \textit{flag} de otimização $-O3$.
O ambiente computacional utilizado em todos os testes neste trabalho consiste de 4 máquinas com a seguinte configuração:

\begin{itemize}
    \item Processador Intel\textregistered Core\texttrademark i7-4820K 3.7 GHz (4 núcleos);
    \item 8 GB de memória RAM;
    \item Sistema Operacional Ubuntu 14.10 (x64);
    \item NVIDIA GeForce GTX 780 com 2304 CUDA cores.
\end{itemize}

\input{resultados/comandosFiguras.tex}

\input{resultados/sog_mog/index.tex}

\input{resultados/rvnd/index.tex}

\input{resultados/dvnd/index.tex}

\input{resultados/gdvnd/index.tex}


\chapter{Resultados} \label{cap:resultados}

Este capítulo exibe os resultados computacionais dos algoritmos propostos no Capítulo~\ref{cap:metodologia} para o caso do PML, para cada instância foi gerado um conjunto com 100 soluções iniciais aleatórias que foram submetidas aos métodos para comparação dos resultados.

Quando há referência à melhoria na solução (\textit{Imp}), esta melhoria pode ser calculada pelo quociente do valor da solução inicial pela solução final, ou seja:
\begin{equation}\label{eq:calculateImprovement}
Imp = \frac{f(\textrm{solução inicial})}{f(\textrm{solução final})}
\end{equation}

Desta forma quanto maior for o valor da melhoria ($Imp$) mais o método melhorou o valor da solução inicial.

\section{Instâncias} \label{sec:instancias}

Todas as instâncias usadas nos testes computacionais e cujas configurações de lançamento foram descritas na Tabela~\ref{tab:neighborhoodsLaunchConfigurarion} são as mesmas usadas em~\cite{wamca2016}.
Para o RVND foi feita uma implementação do algoritmo clássico (Algoritmo~\ref{alg:rvnd}) e também a implementação dataflow mencionada na Figura~\ref{fig:rvndGraph} fazendo uso de uma máquina.
Para o caso do DVND foi utilizada a implementação clássica (Algoritmo~\ref{alg:dvnd}) e a implementação dataflow proposta (Figura~\ref{fig:dvndGraph}), os resultados foram obtidos com diferentes números de máquinas e os mesmos são indicados conforme o caso.

\section{Implementação e ambiente computacional}\label{sec:amb}

A implementação para cada algoritmo proposto no Capítulo~\ref{cap:metodologia} utiliza a linguagem de programação \textit{C++11} em conjunto com a API CUDA\texttrademark, para a implementação dos grafos e do ambiente dataflow foi utilizada a biblioteca Sucuri~\cite{sucuri-original}\footnote{Disponível em \url{https://github.com/tiagoaoa/Sucuri}} implementada em Python, para a integração entre o dataflow e o código CUDA foi utilizada a biblioteca SimplePyCuda~\cite{simple-pycuda}\footnote{Disponível em \url{https://github.com/igormcoelho/simple-pycuda}}. As implementações com múltiplas threads usaram a biblioteca OpenMP.
%As implementações foram compiladas através do \textit{GCC} \textit{(GNU Compiler Collection)}\footnote{O GCC está disponível no seguinte sítio eletrônico: \url{https://gcc.gnu.org/}.} com a \textit{flag} de otimização $-O3$.
O ambiente computacional utilizado em todos os testes neste trabalho consiste de 4 máquinas com a seguinte configuração:

\begin{itemize}
    \item Processador Intel\textregistered Core\texttrademark i7-4820K 3.7 GHz (4 núcleos);
    \item 8 GB de memória RAM;
    \item Sistema Operacional Ubuntu 14.10 (x64);
    \item NVIDIA GeForce GTX 780 com 2304 CUDA cores.
\end{itemize}

\input{resultados/comandosFiguras.tex}

\input{resultados/sog_mog/index.tex}

\input{resultados/rvnd/index.tex}

\input{resultados/dvnd/index.tex}

\input{resultados/gdvnd/index.tex}



\chapter{Resultados} \label{cap:resultados}

Este capítulo exibe os resultados computacionais dos algoritmos propostos no Capítulo~\ref{cap:metodologia} para o caso do PML, para cada instância foi gerado um conjunto com 100 soluções iniciais aleatórias que foram submetidas aos métodos para comparação dos resultados.

Quando há referência à melhoria na solução (\textit{Imp}), esta melhoria pode ser calculada pelo quociente do valor da solução inicial pela solução final, ou seja:
\begin{equation}\label{eq:calculateImprovement}
Imp = \frac{f(\textrm{solução inicial})}{f(\textrm{solução final})}
\end{equation}

Desta forma quanto maior for o valor da melhoria ($Imp$) mais o método melhorou o valor da solução inicial.

\section{Instâncias} \label{sec:instancias}

Todas as instâncias usadas nos testes computacionais e cujas configurações de lançamento foram descritas na Tabela~\ref{tab:neighborhoodsLaunchConfigurarion} são as mesmas usadas em~\cite{wamca2016}.
Para o RVND foi feita uma implementação do algoritmo clássico (Algoritmo~\ref{alg:rvnd}) e também a implementação dataflow mencionada na Figura~\ref{fig:rvndGraph} fazendo uso de uma máquina.
Para o caso do DVND foi utilizada a implementação clássica (Algoritmo~\ref{alg:dvnd}) e a implementação dataflow proposta (Figura~\ref{fig:dvndGraph}), os resultados foram obtidos com diferentes números de máquinas e os mesmos são indicados conforme o caso.

\section{Implementação e ambiente computacional}\label{sec:amb}

A implementação para cada algoritmo proposto no Capítulo~\ref{cap:metodologia} utiliza a linguagem de programação \textit{C++11} em conjunto com a API CUDA\texttrademark, para a implementação dos grafos e do ambiente dataflow foi utilizada a biblioteca Sucuri~\cite{sucuri-original}\footnote{Disponível em \url{https://github.com/tiagoaoa/Sucuri}} implementada em Python, para a integração entre o dataflow e o código CUDA foi utilizada a biblioteca SimplePyCuda~\cite{simple-pycuda}\footnote{Disponível em \url{https://github.com/igormcoelho/simple-pycuda}}. As implementações com múltiplas threads usaram a biblioteca OpenMP.
%As implementações foram compiladas através do \textit{GCC} \textit{(GNU Compiler Collection)}\footnote{O GCC está disponível no seguinte sítio eletrônico: \url{https://gcc.gnu.org/}.} com a \textit{flag} de otimização $-O3$.
O ambiente computacional utilizado em todos os testes neste trabalho consiste de 4 máquinas com a seguinte configuração:

\begin{itemize}
    \item Processador Intel\textregistered Core\texttrademark i7-4820K 3.7 GHz (4 núcleos);
    \item 8 GB de memória RAM;
    \item Sistema Operacional Ubuntu 14.10 (x64);
    \item NVIDIA GeForce GTX 780 com 2304 CUDA cores.
\end{itemize}

\newcommand{\figureDvndOrRvndDcDd}[7]{
% #1 {box, scatter}, #2 {count, imp, time}, #3 {instance number}, #4 {Tempo, Melhoria}, #5 tamanho instância, #6 {DVND, RVND}, #7 {dvnd, rvnd}
\begin{figure}%
    \centering
    \includegraphics[scale=0.9]{figuras/#7/dc_dd/#1/#7_#1100sol_#2_in#3.png}
    \caption{#4 do #6 para a instância #3 de tamanho #5. $m$ indica o número de máquinas, \textit{DC} refere-se ao #6 clássico e \textit{DD} ao #6 implementado em dataflow.}%
    \label{fig:#2_#7DcDd_in#3}%
\end{figure}
}

\newcommand{\figureDvndDcDd}[5]{
% #1 {box, scatter}, #2 {count, imp, time}, #3 {instance number}, #4 {Tempo, Melhoria}, #5 tamanho instância
    \figureDvndOrRvndDcDd{#1}{#2}{#3}{#4}{#5}{DVND}{dvnd}
}

\newcommand{\figureRvndDcDd}[5]{
% #1 {box, scatter}, #2 {count, imp, time}, #3 {instance number}, #4 {Tempo, Melhoria}, #5 tamanho instância
    \figureDvndOrRvndDcDd{#1}{#2}{#3}{#4}{#5}{RVND}{rvnd}
}

\newcommand{\tabelaEstatisticasGeral}[6]{
% #1 Descrição, #2 label, #3 {dvnd, rvnd}, #4 {DVND, RVND}, #6 DD/DC, #6 Conteúdo
\begin{table}[ht]
    \centering
    \begin{tabular}{c|ccc|cc|ccc|cc|c}
        \hline \hline
        \# & Tipo & $m$ & $n$ & $min$ & $max$ & 1Q & 2Q & 3Q & $\overline{x}$ & $\sigma$ & $p-valor$ \\ \hline
        #6
    \end{tabular}
    \caption{#1 #4 #5
        Instância (\#), tipo de implementação (Tipo), número de máquinas ($m$), tamanho da instância ($n$), valor mínimo ($min$), máximo ($max$), primeiro quartil (1Q), mediana (2Q), terceiro quartil (3Q), média ($\overline{x}$), desvio padrão ($\sigma$) e p-valor para o teste de Wilcox entre as versões (valores em negrito quando $p-valor > 0.05$).
    }
    \label{tab:#3DcDd#2}
\end{table}
}

\newcommand{\tabelaEstatisticas}[5]{
    \tabelaEstatisticasGeral{#1}{#2}{#3}{#4}{na implementação clássica (DC) e a proposta de implementação usando dataflow (DD).}{#5}
}

\newcommand{\figureDvndSogMog}[7]{
% #1 {box, scatter}, #2 {count, imp, time}, #3 {instance number}, #4 {Tempo, Melhoria}, #5 tamanho instância, #6 {DVND, RVND}, #7 {dvnd, rvnd}
\begin{figure}%
    \centering
    \includegraphics[scale=0.9]{figuras/#7/sog_mog/#1/#7_#1100sol_#2_in#3.png}
    \caption{#4 do #6 para a instância #3 de tamanho #5. \textit{SOG} refere-se a uma porta de saída e \textit{MOG} a múltiplas portas de saída.}%
    \label{fig:#2_#7SogMog_in#3}%
\end{figure}
}

\newcommand{\figureDvndGdvnd}[9]{
% #1 {box, scatter}, #2 {count, imp, time}, #3 {instance number}, #4 {Tempo, Melhoria}, #5 tamanho instância, #6 {DVND, RVND}, #7 {dvnd, rvnd}, #8 {man_time, full_time}, #9 {man, dvnd} #10 descricao
\begin{figure}%
    \centering
    \includegraphics[scale=0.9]{figuras/#7/#8/#1/#9_#1100sol_#2_in#3.png}
    \caption{#4 do #6 para a instância #3 de tamanho #5. \textit{DVND} refere-se ao tempo gasto pelo algoritmo de mesmo nome, para \textit{GDVND} é análogo ao anterior, no caso do \textit{GDVND-MAN} este se refere ao tempo do GDVND subtraido do tempo para gerenciar os movimentos.}%
    \label{fig:#2_#7_#8_in#3}%
\end{figure}
}

\newcommand{\figureDvndGdvndTime}[8]{
    \figureDvndGdvnd{#1}{#2}{#3}{#4}{#5}{#6}{#7}{man_time}{man}
}

\newcommand{\figureGdvndDvndRvnd}[9]{
% #1 {box, scatter}, #2 {count, imp, time}, #3 {instance number}, #4 {Tempo, Melhoria}, #5 tamanho instância, #6 {DVND, RVND}, #7 {dvnd, rvnd}, #8 {man_time, full_time}, #9 {man, dvnd} #10 descricao
\begin{figure}%
    \centering
    \includegraphics[scale=0.9]{figuras/#7/#8/#1/#9_#1100sol_#2_in#3.png}
    \caption{#4 do #6 para a instância #3 de tamanho #5. \textit{DVND}, \textit{GDVND} e \textit{RVND} referem-se ao tempo gasto pelos algoritmos de mesmo nome.}%
    \label{fig:#2_#7_#8_in#3}%
\end{figure}
}

% \subfloat[$m=#1$]{{ %scale=0.225
%         \includegraphics[scale=0.425]{figuras/dvnd/n#1/time#2.png}
%         \label{fig:timeDvndRvnd_n#1in#2}
%     }}%
% #1 {dvnd, rvnd, gdvnd}, #2 {sog_mog, dc_dd}, #3 {time, imp}, #4 in, #5 tamanho, #6 {box, scatter}
\newcommand{\subFig}[6]{
    \subfloat[][Instância #4, $n=#5$]{
        \includegraphics[scale=0.425]{figuras/#1/#2/#6/#1_#6100sol_#3_in#4.png}
		\label{fig:#1_#2_#3_in#4}
    }
% 	\begin{subfigure}{0.45\textwidth} % dvnd_box100sol_imp_in0
% 		\includegraphics[scale=0.425]{figuras/#1/#2/#6/#1_#6100sol_#3_in#4.png}
% 		\caption{Instância #4, $n=#5$}
        % \label{fig:#1_#2_#3_in#4}
    % \end{subfigure}
}

\newcommand{\subFigBox}[5]{
	\subFig{#1}{#2}{#3}{#4}{#5}{box}
}

\newcommand{\subFigScatter}[5]{
	\subFig{#1}{#2}{#3}{#4}{#5}{scatter}
}

% #1 {dvnd, rvnd, gdvnd}, #2 {sog_mog, dc_dd}, #3 {time, imp}, #4 {box, scatter}, #5 {Tempo do DVND...}
\newcommand{\multiFigureInstanciasGeral}[5]{
	\begin{figure}[ht]
		\centering
		\subFig{#1}{#2}{#3}{0}{52}{#4}
		~
		\subFig{#1}{#2}{#3}{1}{100}{#4}
		
		\subFig{#1}{#2}{#3}{2}{226}{#4}
		~
		\subFig{#1}{#2}{#3}{3}{318}{#4}
		\caption{#5 Instâncias 0 a 3.}
		\label{fig:#1_#2_#3_in0_4}
	\end{figure}
	
	\begin{figure}[ht]
		\centering
		\subFig{#1}{#2}{#3}{4}{501}{#4}
		~
		\subFig{#1}{#2}{#3}{5}{657}{#4}
		
		\subFig{#1}{#2}{#3}{6}{783}{#4}
		~
		\subFig{#1}{#2}{#3}{7}{1001}{#4}
		\caption{#5 Instâncias 5 a 7.}
		\label{fig:#1_#2_#3_in5_7}
	\end{figure}
}

% #1 {dvnd, rvnd, gdvnd}, #2 {sog_mog, dc_dd}, #3 {time, imp}, #4 {Tempo do DVND...}
\newcommand{\multiFigureInstancias}[4]{
    \multiFigureInstanciasGeral{#1}{#2}{#3}{box}{#4}
}


\chapter{Resultados} \label{cap:resultados}

Este capítulo exibe os resultados computacionais dos algoritmos propostos no Capítulo~\ref{cap:metodologia} para o caso do PML, para cada instância foi gerado um conjunto com 100 soluções iniciais aleatórias que foram submetidas aos métodos para comparação dos resultados.

Quando há referência à melhoria na solução (\textit{Imp}), esta melhoria pode ser calculada pelo quociente do valor da solução inicial pela solução final, ou seja:
\begin{equation}\label{eq:calculateImprovement}
Imp = \frac{f(\textrm{solução inicial})}{f(\textrm{solução final})}
\end{equation}

Desta forma quanto maior for o valor da melhoria ($Imp$) mais o método melhorou o valor da solução inicial.

\section{Instâncias} \label{sec:instancias}

Todas as instâncias usadas nos testes computacionais e cujas configurações de lançamento foram descritas na Tabela~\ref{tab:neighborhoodsLaunchConfigurarion} são as mesmas usadas em~\cite{wamca2016}.
Para o RVND foi feita uma implementação do algoritmo clássico (Algoritmo~\ref{alg:rvnd}) e também a implementação dataflow mencionada na Figura~\ref{fig:rvndGraph} fazendo uso de uma máquina.
Para o caso do DVND foi utilizada a implementação clássica (Algoritmo~\ref{alg:dvnd}) e a implementação dataflow proposta (Figura~\ref{fig:dvndGraph}), os resultados foram obtidos com diferentes números de máquinas e os mesmos são indicados conforme o caso.

\section{Implementação e ambiente computacional}\label{sec:amb}

A implementação para cada algoritmo proposto no Capítulo~\ref{cap:metodologia} utiliza a linguagem de programação \textit{C++11} em conjunto com a API CUDA\texttrademark, para a implementação dos grafos e do ambiente dataflow foi utilizada a biblioteca Sucuri~\cite{sucuri-original}\footnote{Disponível em \url{https://github.com/tiagoaoa/Sucuri}} implementada em Python, para a integração entre o dataflow e o código CUDA foi utilizada a biblioteca SimplePyCuda~\cite{simple-pycuda}\footnote{Disponível em \url{https://github.com/igormcoelho/simple-pycuda}}. As implementações com múltiplas threads usaram a biblioteca OpenMP.
%As implementações foram compiladas através do \textit{GCC} \textit{(GNU Compiler Collection)}\footnote{O GCC está disponível no seguinte sítio eletrônico: \url{https://gcc.gnu.org/}.} com a \textit{flag} de otimização $-O3$.
O ambiente computacional utilizado em todos os testes neste trabalho consiste de 4 máquinas com a seguinte configuração:

\begin{itemize}
    \item Processador Intel\textregistered Core\texttrademark i7-4820K 3.7 GHz (4 núcleos);
    \item 8 GB de memória RAM;
    \item Sistema Operacional Ubuntu 14.10 (x64);
    \item NVIDIA GeForce GTX 780 com 2304 CUDA cores.
\end{itemize}

\input{resultados/comandosFiguras.tex}

\input{resultados/sog_mog/index.tex}

\input{resultados/rvnd/index.tex}

\input{resultados/dvnd/index.tex}

\input{resultados/gdvnd/index.tex}


\chapter{Resultados} \label{cap:resultados}

Este capítulo exibe os resultados computacionais dos algoritmos propostos no Capítulo~\ref{cap:metodologia} para o caso do PML, para cada instância foi gerado um conjunto com 100 soluções iniciais aleatórias que foram submetidas aos métodos para comparação dos resultados.

Quando há referência à melhoria na solução (\textit{Imp}), esta melhoria pode ser calculada pelo quociente do valor da solução inicial pela solução final, ou seja:
\begin{equation}\label{eq:calculateImprovement}
Imp = \frac{f(\textrm{solução inicial})}{f(\textrm{solução final})}
\end{equation}

Desta forma quanto maior for o valor da melhoria ($Imp$) mais o método melhorou o valor da solução inicial.

\section{Instâncias} \label{sec:instancias}

Todas as instâncias usadas nos testes computacionais e cujas configurações de lançamento foram descritas na Tabela~\ref{tab:neighborhoodsLaunchConfigurarion} são as mesmas usadas em~\cite{wamca2016}.
Para o RVND foi feita uma implementação do algoritmo clássico (Algoritmo~\ref{alg:rvnd}) e também a implementação dataflow mencionada na Figura~\ref{fig:rvndGraph} fazendo uso de uma máquina.
Para o caso do DVND foi utilizada a implementação clássica (Algoritmo~\ref{alg:dvnd}) e a implementação dataflow proposta (Figura~\ref{fig:dvndGraph}), os resultados foram obtidos com diferentes números de máquinas e os mesmos são indicados conforme o caso.

\section{Implementação e ambiente computacional}\label{sec:amb}

A implementação para cada algoritmo proposto no Capítulo~\ref{cap:metodologia} utiliza a linguagem de programação \textit{C++11} em conjunto com a API CUDA\texttrademark, para a implementação dos grafos e do ambiente dataflow foi utilizada a biblioteca Sucuri~\cite{sucuri-original}\footnote{Disponível em \url{https://github.com/tiagoaoa/Sucuri}} implementada em Python, para a integração entre o dataflow e o código CUDA foi utilizada a biblioteca SimplePyCuda~\cite{simple-pycuda}\footnote{Disponível em \url{https://github.com/igormcoelho/simple-pycuda}}. As implementações com múltiplas threads usaram a biblioteca OpenMP.
%As implementações foram compiladas através do \textit{GCC} \textit{(GNU Compiler Collection)}\footnote{O GCC está disponível no seguinte sítio eletrônico: \url{https://gcc.gnu.org/}.} com a \textit{flag} de otimização $-O3$.
O ambiente computacional utilizado em todos os testes neste trabalho consiste de 4 máquinas com a seguinte configuração:

\begin{itemize}
    \item Processador Intel\textregistered Core\texttrademark i7-4820K 3.7 GHz (4 núcleos);
    \item 8 GB de memória RAM;
    \item Sistema Operacional Ubuntu 14.10 (x64);
    \item NVIDIA GeForce GTX 780 com 2304 CUDA cores.
\end{itemize}

\input{resultados/comandosFiguras.tex}

\input{resultados/sog_mog/index.tex}

\input{resultados/rvnd/index.tex}

\input{resultados/dvnd/index.tex}

\input{resultados/gdvnd/index.tex}


\chapter{Resultados} \label{cap:resultados}

Este capítulo exibe os resultados computacionais dos algoritmos propostos no Capítulo~\ref{cap:metodologia} para o caso do PML, para cada instância foi gerado um conjunto com 100 soluções iniciais aleatórias que foram submetidas aos métodos para comparação dos resultados.

Quando há referência à melhoria na solução (\textit{Imp}), esta melhoria pode ser calculada pelo quociente do valor da solução inicial pela solução final, ou seja:
\begin{equation}\label{eq:calculateImprovement}
Imp = \frac{f(\textrm{solução inicial})}{f(\textrm{solução final})}
\end{equation}

Desta forma quanto maior for o valor da melhoria ($Imp$) mais o método melhorou o valor da solução inicial.

\section{Instâncias} \label{sec:instancias}

Todas as instâncias usadas nos testes computacionais e cujas configurações de lançamento foram descritas na Tabela~\ref{tab:neighborhoodsLaunchConfigurarion} são as mesmas usadas em~\cite{wamca2016}.
Para o RVND foi feita uma implementação do algoritmo clássico (Algoritmo~\ref{alg:rvnd}) e também a implementação dataflow mencionada na Figura~\ref{fig:rvndGraph} fazendo uso de uma máquina.
Para o caso do DVND foi utilizada a implementação clássica (Algoritmo~\ref{alg:dvnd}) e a implementação dataflow proposta (Figura~\ref{fig:dvndGraph}), os resultados foram obtidos com diferentes números de máquinas e os mesmos são indicados conforme o caso.

\section{Implementação e ambiente computacional}\label{sec:amb}

A implementação para cada algoritmo proposto no Capítulo~\ref{cap:metodologia} utiliza a linguagem de programação \textit{C++11} em conjunto com a API CUDA\texttrademark, para a implementação dos grafos e do ambiente dataflow foi utilizada a biblioteca Sucuri~\cite{sucuri-original}\footnote{Disponível em \url{https://github.com/tiagoaoa/Sucuri}} implementada em Python, para a integração entre o dataflow e o código CUDA foi utilizada a biblioteca SimplePyCuda~\cite{simple-pycuda}\footnote{Disponível em \url{https://github.com/igormcoelho/simple-pycuda}}. As implementações com múltiplas threads usaram a biblioteca OpenMP.
%As implementações foram compiladas através do \textit{GCC} \textit{(GNU Compiler Collection)}\footnote{O GCC está disponível no seguinte sítio eletrônico: \url{https://gcc.gnu.org/}.} com a \textit{flag} de otimização $-O3$.
O ambiente computacional utilizado em todos os testes neste trabalho consiste de 4 máquinas com a seguinte configuração:

\begin{itemize}
    \item Processador Intel\textregistered Core\texttrademark i7-4820K 3.7 GHz (4 núcleos);
    \item 8 GB de memória RAM;
    \item Sistema Operacional Ubuntu 14.10 (x64);
    \item NVIDIA GeForce GTX 780 com 2304 CUDA cores.
\end{itemize}

\input{resultados/comandosFiguras.tex}

\input{resultados/sog_mog/index.tex}

\input{resultados/rvnd/index.tex}

\input{resultados/dvnd/index.tex}

\input{resultados/gdvnd/index.tex}


\chapter{Resultados} \label{cap:resultados}

Este capítulo exibe os resultados computacionais dos algoritmos propostos no Capítulo~\ref{cap:metodologia} para o caso do PML, para cada instância foi gerado um conjunto com 100 soluções iniciais aleatórias que foram submetidas aos métodos para comparação dos resultados.

Quando há referência à melhoria na solução (\textit{Imp}), esta melhoria pode ser calculada pelo quociente do valor da solução inicial pela solução final, ou seja:
\begin{equation}\label{eq:calculateImprovement}
Imp = \frac{f(\textrm{solução inicial})}{f(\textrm{solução final})}
\end{equation}

Desta forma quanto maior for o valor da melhoria ($Imp$) mais o método melhorou o valor da solução inicial.

\section{Instâncias} \label{sec:instancias}

Todas as instâncias usadas nos testes computacionais e cujas configurações de lançamento foram descritas na Tabela~\ref{tab:neighborhoodsLaunchConfigurarion} são as mesmas usadas em~\cite{wamca2016}.
Para o RVND foi feita uma implementação do algoritmo clássico (Algoritmo~\ref{alg:rvnd}) e também a implementação dataflow mencionada na Figura~\ref{fig:rvndGraph} fazendo uso de uma máquina.
Para o caso do DVND foi utilizada a implementação clássica (Algoritmo~\ref{alg:dvnd}) e a implementação dataflow proposta (Figura~\ref{fig:dvndGraph}), os resultados foram obtidos com diferentes números de máquinas e os mesmos são indicados conforme o caso.

\section{Implementação e ambiente computacional}\label{sec:amb}

A implementação para cada algoritmo proposto no Capítulo~\ref{cap:metodologia} utiliza a linguagem de programação \textit{C++11} em conjunto com a API CUDA\texttrademark, para a implementação dos grafos e do ambiente dataflow foi utilizada a biblioteca Sucuri~\cite{sucuri-original}\footnote{Disponível em \url{https://github.com/tiagoaoa/Sucuri}} implementada em Python, para a integração entre o dataflow e o código CUDA foi utilizada a biblioteca SimplePyCuda~\cite{simple-pycuda}\footnote{Disponível em \url{https://github.com/igormcoelho/simple-pycuda}}. As implementações com múltiplas threads usaram a biblioteca OpenMP.
%As implementações foram compiladas através do \textit{GCC} \textit{(GNU Compiler Collection)}\footnote{O GCC está disponível no seguinte sítio eletrônico: \url{https://gcc.gnu.org/}.} com a \textit{flag} de otimização $-O3$.
O ambiente computacional utilizado em todos os testes neste trabalho consiste de 4 máquinas com a seguinte configuração:

\begin{itemize}
    \item Processador Intel\textregistered Core\texttrademark i7-4820K 3.7 GHz (4 núcleos);
    \item 8 GB de memória RAM;
    \item Sistema Operacional Ubuntu 14.10 (x64);
    \item NVIDIA GeForce GTX 780 com 2304 CUDA cores.
\end{itemize}

\input{resultados/comandosFiguras.tex}

\input{resultados/sog_mog/index.tex}

\input{resultados/rvnd/index.tex}

\input{resultados/dvnd/index.tex}

\input{resultados/gdvnd/index.tex}



\chapter{Resultados} \label{cap:resultados}

Este capítulo exibe os resultados computacionais dos algoritmos propostos no Capítulo~\ref{cap:metodologia} para o caso do PML, para cada instância foi gerado um conjunto com 100 soluções iniciais aleatórias que foram submetidas aos métodos para comparação dos resultados.

Quando há referência à melhoria na solução (\textit{Imp}), esta melhoria pode ser calculada pelo quociente do valor da solução inicial pela solução final, ou seja:
\begin{equation}\label{eq:calculateImprovement}
Imp = \frac{f(\textrm{solução inicial})}{f(\textrm{solução final})}
\end{equation}

Desta forma quanto maior for o valor da melhoria ($Imp$) mais o método melhorou o valor da solução inicial.

\section{Instâncias} \label{sec:instancias}

Todas as instâncias usadas nos testes computacionais e cujas configurações de lançamento foram descritas na Tabela~\ref{tab:neighborhoodsLaunchConfigurarion} são as mesmas usadas em~\cite{wamca2016}.
Para o RVND foi feita uma implementação do algoritmo clássico (Algoritmo~\ref{alg:rvnd}) e também a implementação dataflow mencionada na Figura~\ref{fig:rvndGraph} fazendo uso de uma máquina.
Para o caso do DVND foi utilizada a implementação clássica (Algoritmo~\ref{alg:dvnd}) e a implementação dataflow proposta (Figura~\ref{fig:dvndGraph}), os resultados foram obtidos com diferentes números de máquinas e os mesmos são indicados conforme o caso.

\section{Implementação e ambiente computacional}\label{sec:amb}

A implementação para cada algoritmo proposto no Capítulo~\ref{cap:metodologia} utiliza a linguagem de programação \textit{C++11} em conjunto com a API CUDA\texttrademark, para a implementação dos grafos e do ambiente dataflow foi utilizada a biblioteca Sucuri~\cite{sucuri-original}\footnote{Disponível em \url{https://github.com/tiagoaoa/Sucuri}} implementada em Python, para a integração entre o dataflow e o código CUDA foi utilizada a biblioteca SimplePyCuda~\cite{simple-pycuda}\footnote{Disponível em \url{https://github.com/igormcoelho/simple-pycuda}}. As implementações com múltiplas threads usaram a biblioteca OpenMP.
%As implementações foram compiladas através do \textit{GCC} \textit{(GNU Compiler Collection)}\footnote{O GCC está disponível no seguinte sítio eletrônico: \url{https://gcc.gnu.org/}.} com a \textit{flag} de otimização $-O3$.
O ambiente computacional utilizado em todos os testes neste trabalho consiste de 4 máquinas com a seguinte configuração:

\begin{itemize}
    \item Processador Intel\textregistered Core\texttrademark i7-4820K 3.7 GHz (4 núcleos);
    \item 8 GB de memória RAM;
    \item Sistema Operacional Ubuntu 14.10 (x64);
    \item NVIDIA GeForce GTX 780 com 2304 CUDA cores.
\end{itemize}

\newcommand{\figureDvndOrRvndDcDd}[7]{
% #1 {box, scatter}, #2 {count, imp, time}, #3 {instance number}, #4 {Tempo, Melhoria}, #5 tamanho instância, #6 {DVND, RVND}, #7 {dvnd, rvnd}
\begin{figure}%
    \centering
    \includegraphics[scale=0.9]{figuras/#7/dc_dd/#1/#7_#1100sol_#2_in#3.png}
    \caption{#4 do #6 para a instância #3 de tamanho #5. $m$ indica o número de máquinas, \textit{DC} refere-se ao #6 clássico e \textit{DD} ao #6 implementado em dataflow.}%
    \label{fig:#2_#7DcDd_in#3}%
\end{figure}
}

\newcommand{\figureDvndDcDd}[5]{
% #1 {box, scatter}, #2 {count, imp, time}, #3 {instance number}, #4 {Tempo, Melhoria}, #5 tamanho instância
    \figureDvndOrRvndDcDd{#1}{#2}{#3}{#4}{#5}{DVND}{dvnd}
}

\newcommand{\figureRvndDcDd}[5]{
% #1 {box, scatter}, #2 {count, imp, time}, #3 {instance number}, #4 {Tempo, Melhoria}, #5 tamanho instância
    \figureDvndOrRvndDcDd{#1}{#2}{#3}{#4}{#5}{RVND}{rvnd}
}

\newcommand{\tabelaEstatisticasGeral}[6]{
% #1 Descrição, #2 label, #3 {dvnd, rvnd}, #4 {DVND, RVND}, #6 DD/DC, #6 Conteúdo
\begin{table}[ht]
    \centering
    \begin{tabular}{c|ccc|cc|ccc|cc|c}
        \hline \hline
        \# & Tipo & $m$ & $n$ & $min$ & $max$ & 1Q & 2Q & 3Q & $\overline{x}$ & $\sigma$ & $p-valor$ \\ \hline
        #6
    \end{tabular}
    \caption{#1 #4 #5
        Instância (\#), tipo de implementação (Tipo), número de máquinas ($m$), tamanho da instância ($n$), valor mínimo ($min$), máximo ($max$), primeiro quartil (1Q), mediana (2Q), terceiro quartil (3Q), média ($\overline{x}$), desvio padrão ($\sigma$) e p-valor para o teste de Wilcox entre as versões (valores em negrito quando $p-valor > 0.05$).
    }
    \label{tab:#3DcDd#2}
\end{table}
}

\newcommand{\tabelaEstatisticas}[5]{
    \tabelaEstatisticasGeral{#1}{#2}{#3}{#4}{na implementação clássica (DC) e a proposta de implementação usando dataflow (DD).}{#5}
}

\newcommand{\figureDvndSogMog}[7]{
% #1 {box, scatter}, #2 {count, imp, time}, #3 {instance number}, #4 {Tempo, Melhoria}, #5 tamanho instância, #6 {DVND, RVND}, #7 {dvnd, rvnd}
\begin{figure}%
    \centering
    \includegraphics[scale=0.9]{figuras/#7/sog_mog/#1/#7_#1100sol_#2_in#3.png}
    \caption{#4 do #6 para a instância #3 de tamanho #5. \textit{SOG} refere-se a uma porta de saída e \textit{MOG} a múltiplas portas de saída.}%
    \label{fig:#2_#7SogMog_in#3}%
\end{figure}
}

\newcommand{\figureDvndGdvnd}[9]{
% #1 {box, scatter}, #2 {count, imp, time}, #3 {instance number}, #4 {Tempo, Melhoria}, #5 tamanho instância, #6 {DVND, RVND}, #7 {dvnd, rvnd}, #8 {man_time, full_time}, #9 {man, dvnd} #10 descricao
\begin{figure}%
    \centering
    \includegraphics[scale=0.9]{figuras/#7/#8/#1/#9_#1100sol_#2_in#3.png}
    \caption{#4 do #6 para a instância #3 de tamanho #5. \textit{DVND} refere-se ao tempo gasto pelo algoritmo de mesmo nome, para \textit{GDVND} é análogo ao anterior, no caso do \textit{GDVND-MAN} este se refere ao tempo do GDVND subtraido do tempo para gerenciar os movimentos.}%
    \label{fig:#2_#7_#8_in#3}%
\end{figure}
}

\newcommand{\figureDvndGdvndTime}[8]{
    \figureDvndGdvnd{#1}{#2}{#3}{#4}{#5}{#6}{#7}{man_time}{man}
}

\newcommand{\figureGdvndDvndRvnd}[9]{
% #1 {box, scatter}, #2 {count, imp, time}, #3 {instance number}, #4 {Tempo, Melhoria}, #5 tamanho instância, #6 {DVND, RVND}, #7 {dvnd, rvnd}, #8 {man_time, full_time}, #9 {man, dvnd} #10 descricao
\begin{figure}%
    \centering
    \includegraphics[scale=0.9]{figuras/#7/#8/#1/#9_#1100sol_#2_in#3.png}
    \caption{#4 do #6 para a instância #3 de tamanho #5. \textit{DVND}, \textit{GDVND} e \textit{RVND} referem-se ao tempo gasto pelos algoritmos de mesmo nome.}%
    \label{fig:#2_#7_#8_in#3}%
\end{figure}
}

% \subfloat[$m=#1$]{{ %scale=0.225
%         \includegraphics[scale=0.425]{figuras/dvnd/n#1/time#2.png}
%         \label{fig:timeDvndRvnd_n#1in#2}
%     }}%
% #1 {dvnd, rvnd, gdvnd}, #2 {sog_mog, dc_dd}, #3 {time, imp}, #4 in, #5 tamanho, #6 {box, scatter}
\newcommand{\subFig}[6]{
    \subfloat[][Instância #4, $n=#5$]{
        \includegraphics[scale=0.425]{figuras/#1/#2/#6/#1_#6100sol_#3_in#4.png}
		\label{fig:#1_#2_#3_in#4}
    }
% 	\begin{subfigure}{0.45\textwidth} % dvnd_box100sol_imp_in0
% 		\includegraphics[scale=0.425]{figuras/#1/#2/#6/#1_#6100sol_#3_in#4.png}
% 		\caption{Instância #4, $n=#5$}
        % \label{fig:#1_#2_#3_in#4}
    % \end{subfigure}
}

\newcommand{\subFigBox}[5]{
	\subFig{#1}{#2}{#3}{#4}{#5}{box}
}

\newcommand{\subFigScatter}[5]{
	\subFig{#1}{#2}{#3}{#4}{#5}{scatter}
}

% #1 {dvnd, rvnd, gdvnd}, #2 {sog_mog, dc_dd}, #3 {time, imp}, #4 {box, scatter}, #5 {Tempo do DVND...}
\newcommand{\multiFigureInstanciasGeral}[5]{
	\begin{figure}[ht]
		\centering
		\subFig{#1}{#2}{#3}{0}{52}{#4}
		~
		\subFig{#1}{#2}{#3}{1}{100}{#4}
		
		\subFig{#1}{#2}{#3}{2}{226}{#4}
		~
		\subFig{#1}{#2}{#3}{3}{318}{#4}
		\caption{#5 Instâncias 0 a 3.}
		\label{fig:#1_#2_#3_in0_4}
	\end{figure}
	
	\begin{figure}[ht]
		\centering
		\subFig{#1}{#2}{#3}{4}{501}{#4}
		~
		\subFig{#1}{#2}{#3}{5}{657}{#4}
		
		\subFig{#1}{#2}{#3}{6}{783}{#4}
		~
		\subFig{#1}{#2}{#3}{7}{1001}{#4}
		\caption{#5 Instâncias 5 a 7.}
		\label{fig:#1_#2_#3_in5_7}
	\end{figure}
}

% #1 {dvnd, rvnd, gdvnd}, #2 {sog_mog, dc_dd}, #3 {time, imp}, #4 {Tempo do DVND...}
\newcommand{\multiFigureInstancias}[4]{
    \multiFigureInstanciasGeral{#1}{#2}{#3}{box}{#4}
}


\chapter{Resultados} \label{cap:resultados}

Este capítulo exibe os resultados computacionais dos algoritmos propostos no Capítulo~\ref{cap:metodologia} para o caso do PML, para cada instância foi gerado um conjunto com 100 soluções iniciais aleatórias que foram submetidas aos métodos para comparação dos resultados.

Quando há referência à melhoria na solução (\textit{Imp}), esta melhoria pode ser calculada pelo quociente do valor da solução inicial pela solução final, ou seja:
\begin{equation}\label{eq:calculateImprovement}
Imp = \frac{f(\textrm{solução inicial})}{f(\textrm{solução final})}
\end{equation}

Desta forma quanto maior for o valor da melhoria ($Imp$) mais o método melhorou o valor da solução inicial.

\section{Instâncias} \label{sec:instancias}

Todas as instâncias usadas nos testes computacionais e cujas configurações de lançamento foram descritas na Tabela~\ref{tab:neighborhoodsLaunchConfigurarion} são as mesmas usadas em~\cite{wamca2016}.
Para o RVND foi feita uma implementação do algoritmo clássico (Algoritmo~\ref{alg:rvnd}) e também a implementação dataflow mencionada na Figura~\ref{fig:rvndGraph} fazendo uso de uma máquina.
Para o caso do DVND foi utilizada a implementação clássica (Algoritmo~\ref{alg:dvnd}) e a implementação dataflow proposta (Figura~\ref{fig:dvndGraph}), os resultados foram obtidos com diferentes números de máquinas e os mesmos são indicados conforme o caso.

\section{Implementação e ambiente computacional}\label{sec:amb}

A implementação para cada algoritmo proposto no Capítulo~\ref{cap:metodologia} utiliza a linguagem de programação \textit{C++11} em conjunto com a API CUDA\texttrademark, para a implementação dos grafos e do ambiente dataflow foi utilizada a biblioteca Sucuri~\cite{sucuri-original}\footnote{Disponível em \url{https://github.com/tiagoaoa/Sucuri}} implementada em Python, para a integração entre o dataflow e o código CUDA foi utilizada a biblioteca SimplePyCuda~\cite{simple-pycuda}\footnote{Disponível em \url{https://github.com/igormcoelho/simple-pycuda}}. As implementações com múltiplas threads usaram a biblioteca OpenMP.
%As implementações foram compiladas através do \textit{GCC} \textit{(GNU Compiler Collection)}\footnote{O GCC está disponível no seguinte sítio eletrônico: \url{https://gcc.gnu.org/}.} com a \textit{flag} de otimização $-O3$.
O ambiente computacional utilizado em todos os testes neste trabalho consiste de 4 máquinas com a seguinte configuração:

\begin{itemize}
    \item Processador Intel\textregistered Core\texttrademark i7-4820K 3.7 GHz (4 núcleos);
    \item 8 GB de memória RAM;
    \item Sistema Operacional Ubuntu 14.10 (x64);
    \item NVIDIA GeForce GTX 780 com 2304 CUDA cores.
\end{itemize}

\input{resultados/comandosFiguras.tex}

\input{resultados/sog_mog/index.tex}

\input{resultados/rvnd/index.tex}

\input{resultados/dvnd/index.tex}

\input{resultados/gdvnd/index.tex}


\chapter{Resultados} \label{cap:resultados}

Este capítulo exibe os resultados computacionais dos algoritmos propostos no Capítulo~\ref{cap:metodologia} para o caso do PML, para cada instância foi gerado um conjunto com 100 soluções iniciais aleatórias que foram submetidas aos métodos para comparação dos resultados.

Quando há referência à melhoria na solução (\textit{Imp}), esta melhoria pode ser calculada pelo quociente do valor da solução inicial pela solução final, ou seja:
\begin{equation}\label{eq:calculateImprovement}
Imp = \frac{f(\textrm{solução inicial})}{f(\textrm{solução final})}
\end{equation}

Desta forma quanto maior for o valor da melhoria ($Imp$) mais o método melhorou o valor da solução inicial.

\section{Instâncias} \label{sec:instancias}

Todas as instâncias usadas nos testes computacionais e cujas configurações de lançamento foram descritas na Tabela~\ref{tab:neighborhoodsLaunchConfigurarion} são as mesmas usadas em~\cite{wamca2016}.
Para o RVND foi feita uma implementação do algoritmo clássico (Algoritmo~\ref{alg:rvnd}) e também a implementação dataflow mencionada na Figura~\ref{fig:rvndGraph} fazendo uso de uma máquina.
Para o caso do DVND foi utilizada a implementação clássica (Algoritmo~\ref{alg:dvnd}) e a implementação dataflow proposta (Figura~\ref{fig:dvndGraph}), os resultados foram obtidos com diferentes números de máquinas e os mesmos são indicados conforme o caso.

\section{Implementação e ambiente computacional}\label{sec:amb}

A implementação para cada algoritmo proposto no Capítulo~\ref{cap:metodologia} utiliza a linguagem de programação \textit{C++11} em conjunto com a API CUDA\texttrademark, para a implementação dos grafos e do ambiente dataflow foi utilizada a biblioteca Sucuri~\cite{sucuri-original}\footnote{Disponível em \url{https://github.com/tiagoaoa/Sucuri}} implementada em Python, para a integração entre o dataflow e o código CUDA foi utilizada a biblioteca SimplePyCuda~\cite{simple-pycuda}\footnote{Disponível em \url{https://github.com/igormcoelho/simple-pycuda}}. As implementações com múltiplas threads usaram a biblioteca OpenMP.
%As implementações foram compiladas através do \textit{GCC} \textit{(GNU Compiler Collection)}\footnote{O GCC está disponível no seguinte sítio eletrônico: \url{https://gcc.gnu.org/}.} com a \textit{flag} de otimização $-O3$.
O ambiente computacional utilizado em todos os testes neste trabalho consiste de 4 máquinas com a seguinte configuração:

\begin{itemize}
    \item Processador Intel\textregistered Core\texttrademark i7-4820K 3.7 GHz (4 núcleos);
    \item 8 GB de memória RAM;
    \item Sistema Operacional Ubuntu 14.10 (x64);
    \item NVIDIA GeForce GTX 780 com 2304 CUDA cores.
\end{itemize}

\input{resultados/comandosFiguras.tex}

\input{resultados/sog_mog/index.tex}

\input{resultados/rvnd/index.tex}

\input{resultados/dvnd/index.tex}

\input{resultados/gdvnd/index.tex}


\chapter{Resultados} \label{cap:resultados}

Este capítulo exibe os resultados computacionais dos algoritmos propostos no Capítulo~\ref{cap:metodologia} para o caso do PML, para cada instância foi gerado um conjunto com 100 soluções iniciais aleatórias que foram submetidas aos métodos para comparação dos resultados.

Quando há referência à melhoria na solução (\textit{Imp}), esta melhoria pode ser calculada pelo quociente do valor da solução inicial pela solução final, ou seja:
\begin{equation}\label{eq:calculateImprovement}
Imp = \frac{f(\textrm{solução inicial})}{f(\textrm{solução final})}
\end{equation}

Desta forma quanto maior for o valor da melhoria ($Imp$) mais o método melhorou o valor da solução inicial.

\section{Instâncias} \label{sec:instancias}

Todas as instâncias usadas nos testes computacionais e cujas configurações de lançamento foram descritas na Tabela~\ref{tab:neighborhoodsLaunchConfigurarion} são as mesmas usadas em~\cite{wamca2016}.
Para o RVND foi feita uma implementação do algoritmo clássico (Algoritmo~\ref{alg:rvnd}) e também a implementação dataflow mencionada na Figura~\ref{fig:rvndGraph} fazendo uso de uma máquina.
Para o caso do DVND foi utilizada a implementação clássica (Algoritmo~\ref{alg:dvnd}) e a implementação dataflow proposta (Figura~\ref{fig:dvndGraph}), os resultados foram obtidos com diferentes números de máquinas e os mesmos são indicados conforme o caso.

\section{Implementação e ambiente computacional}\label{sec:amb}

A implementação para cada algoritmo proposto no Capítulo~\ref{cap:metodologia} utiliza a linguagem de programação \textit{C++11} em conjunto com a API CUDA\texttrademark, para a implementação dos grafos e do ambiente dataflow foi utilizada a biblioteca Sucuri~\cite{sucuri-original}\footnote{Disponível em \url{https://github.com/tiagoaoa/Sucuri}} implementada em Python, para a integração entre o dataflow e o código CUDA foi utilizada a biblioteca SimplePyCuda~\cite{simple-pycuda}\footnote{Disponível em \url{https://github.com/igormcoelho/simple-pycuda}}. As implementações com múltiplas threads usaram a biblioteca OpenMP.
%As implementações foram compiladas através do \textit{GCC} \textit{(GNU Compiler Collection)}\footnote{O GCC está disponível no seguinte sítio eletrônico: \url{https://gcc.gnu.org/}.} com a \textit{flag} de otimização $-O3$.
O ambiente computacional utilizado em todos os testes neste trabalho consiste de 4 máquinas com a seguinte configuração:

\begin{itemize}
    \item Processador Intel\textregistered Core\texttrademark i7-4820K 3.7 GHz (4 núcleos);
    \item 8 GB de memória RAM;
    \item Sistema Operacional Ubuntu 14.10 (x64);
    \item NVIDIA GeForce GTX 780 com 2304 CUDA cores.
\end{itemize}

\input{resultados/comandosFiguras.tex}

\input{resultados/sog_mog/index.tex}

\input{resultados/rvnd/index.tex}

\input{resultados/dvnd/index.tex}

\input{resultados/gdvnd/index.tex}


\chapter{Resultados} \label{cap:resultados}

Este capítulo exibe os resultados computacionais dos algoritmos propostos no Capítulo~\ref{cap:metodologia} para o caso do PML, para cada instância foi gerado um conjunto com 100 soluções iniciais aleatórias que foram submetidas aos métodos para comparação dos resultados.

Quando há referência à melhoria na solução (\textit{Imp}), esta melhoria pode ser calculada pelo quociente do valor da solução inicial pela solução final, ou seja:
\begin{equation}\label{eq:calculateImprovement}
Imp = \frac{f(\textrm{solução inicial})}{f(\textrm{solução final})}
\end{equation}

Desta forma quanto maior for o valor da melhoria ($Imp$) mais o método melhorou o valor da solução inicial.

\section{Instâncias} \label{sec:instancias}

Todas as instâncias usadas nos testes computacionais e cujas configurações de lançamento foram descritas na Tabela~\ref{tab:neighborhoodsLaunchConfigurarion} são as mesmas usadas em~\cite{wamca2016}.
Para o RVND foi feita uma implementação do algoritmo clássico (Algoritmo~\ref{alg:rvnd}) e também a implementação dataflow mencionada na Figura~\ref{fig:rvndGraph} fazendo uso de uma máquina.
Para o caso do DVND foi utilizada a implementação clássica (Algoritmo~\ref{alg:dvnd}) e a implementação dataflow proposta (Figura~\ref{fig:dvndGraph}), os resultados foram obtidos com diferentes números de máquinas e os mesmos são indicados conforme o caso.

\section{Implementação e ambiente computacional}\label{sec:amb}

A implementação para cada algoritmo proposto no Capítulo~\ref{cap:metodologia} utiliza a linguagem de programação \textit{C++11} em conjunto com a API CUDA\texttrademark, para a implementação dos grafos e do ambiente dataflow foi utilizada a biblioteca Sucuri~\cite{sucuri-original}\footnote{Disponível em \url{https://github.com/tiagoaoa/Sucuri}} implementada em Python, para a integração entre o dataflow e o código CUDA foi utilizada a biblioteca SimplePyCuda~\cite{simple-pycuda}\footnote{Disponível em \url{https://github.com/igormcoelho/simple-pycuda}}. As implementações com múltiplas threads usaram a biblioteca OpenMP.
%As implementações foram compiladas através do \textit{GCC} \textit{(GNU Compiler Collection)}\footnote{O GCC está disponível no seguinte sítio eletrônico: \url{https://gcc.gnu.org/}.} com a \textit{flag} de otimização $-O3$.
O ambiente computacional utilizado em todos os testes neste trabalho consiste de 4 máquinas com a seguinte configuração:

\begin{itemize}
    \item Processador Intel\textregistered Core\texttrademark i7-4820K 3.7 GHz (4 núcleos);
    \item 8 GB de memória RAM;
    \item Sistema Operacional Ubuntu 14.10 (x64);
    \item NVIDIA GeForce GTX 780 com 2304 CUDA cores.
\end{itemize}

\input{resultados/comandosFiguras.tex}

\input{resultados/sog_mog/index.tex}

\input{resultados/rvnd/index.tex}

\input{resultados/dvnd/index.tex}

\input{resultados/gdvnd/index.tex}



\chapter{Resultados} \label{cap:resultados}

Este capítulo exibe os resultados computacionais dos algoritmos propostos no Capítulo~\ref{cap:metodologia} para o caso do PML, para cada instância foi gerado um conjunto com 100 soluções iniciais aleatórias que foram submetidas aos métodos para comparação dos resultados.

Quando há referência à melhoria na solução (\textit{Imp}), esta melhoria pode ser calculada pelo quociente do valor da solução inicial pela solução final, ou seja:
\begin{equation}\label{eq:calculateImprovement}
Imp = \frac{f(\textrm{solução inicial})}{f(\textrm{solução final})}
\end{equation}

Desta forma quanto maior for o valor da melhoria ($Imp$) mais o método melhorou o valor da solução inicial.

\section{Instâncias} \label{sec:instancias}

Todas as instâncias usadas nos testes computacionais e cujas configurações de lançamento foram descritas na Tabela~\ref{tab:neighborhoodsLaunchConfigurarion} são as mesmas usadas em~\cite{wamca2016}.
Para o RVND foi feita uma implementação do algoritmo clássico (Algoritmo~\ref{alg:rvnd}) e também a implementação dataflow mencionada na Figura~\ref{fig:rvndGraph} fazendo uso de uma máquina.
Para o caso do DVND foi utilizada a implementação clássica (Algoritmo~\ref{alg:dvnd}) e a implementação dataflow proposta (Figura~\ref{fig:dvndGraph}), os resultados foram obtidos com diferentes números de máquinas e os mesmos são indicados conforme o caso.

\section{Implementação e ambiente computacional}\label{sec:amb}

A implementação para cada algoritmo proposto no Capítulo~\ref{cap:metodologia} utiliza a linguagem de programação \textit{C++11} em conjunto com a API CUDA\texttrademark, para a implementação dos grafos e do ambiente dataflow foi utilizada a biblioteca Sucuri~\cite{sucuri-original}\footnote{Disponível em \url{https://github.com/tiagoaoa/Sucuri}} implementada em Python, para a integração entre o dataflow e o código CUDA foi utilizada a biblioteca SimplePyCuda~\cite{simple-pycuda}\footnote{Disponível em \url{https://github.com/igormcoelho/simple-pycuda}}. As implementações com múltiplas threads usaram a biblioteca OpenMP.
%As implementações foram compiladas através do \textit{GCC} \textit{(GNU Compiler Collection)}\footnote{O GCC está disponível no seguinte sítio eletrônico: \url{https://gcc.gnu.org/}.} com a \textit{flag} de otimização $-O3$.
O ambiente computacional utilizado em todos os testes neste trabalho consiste de 4 máquinas com a seguinte configuração:

\begin{itemize}
    \item Processador Intel\textregistered Core\texttrademark i7-4820K 3.7 GHz (4 núcleos);
    \item 8 GB de memória RAM;
    \item Sistema Operacional Ubuntu 14.10 (x64);
    \item NVIDIA GeForce GTX 780 com 2304 CUDA cores.
\end{itemize}

\newcommand{\figureDvndOrRvndDcDd}[7]{
% #1 {box, scatter}, #2 {count, imp, time}, #3 {instance number}, #4 {Tempo, Melhoria}, #5 tamanho instância, #6 {DVND, RVND}, #7 {dvnd, rvnd}
\begin{figure}%
    \centering
    \includegraphics[scale=0.9]{figuras/#7/dc_dd/#1/#7_#1100sol_#2_in#3.png}
    \caption{#4 do #6 para a instância #3 de tamanho #5. $m$ indica o número de máquinas, \textit{DC} refere-se ao #6 clássico e \textit{DD} ao #6 implementado em dataflow.}%
    \label{fig:#2_#7DcDd_in#3}%
\end{figure}
}

\newcommand{\figureDvndDcDd}[5]{
% #1 {box, scatter}, #2 {count, imp, time}, #3 {instance number}, #4 {Tempo, Melhoria}, #5 tamanho instância
    \figureDvndOrRvndDcDd{#1}{#2}{#3}{#4}{#5}{DVND}{dvnd}
}

\newcommand{\figureRvndDcDd}[5]{
% #1 {box, scatter}, #2 {count, imp, time}, #3 {instance number}, #4 {Tempo, Melhoria}, #5 tamanho instância
    \figureDvndOrRvndDcDd{#1}{#2}{#3}{#4}{#5}{RVND}{rvnd}
}

\newcommand{\tabelaEstatisticasGeral}[6]{
% #1 Descrição, #2 label, #3 {dvnd, rvnd}, #4 {DVND, RVND}, #6 DD/DC, #6 Conteúdo
\begin{table}[ht]
    \centering
    \begin{tabular}{c|ccc|cc|ccc|cc|c}
        \hline \hline
        \# & Tipo & $m$ & $n$ & $min$ & $max$ & 1Q & 2Q & 3Q & $\overline{x}$ & $\sigma$ & $p-valor$ \\ \hline
        #6
    \end{tabular}
    \caption{#1 #4 #5
        Instância (\#), tipo de implementação (Tipo), número de máquinas ($m$), tamanho da instância ($n$), valor mínimo ($min$), máximo ($max$), primeiro quartil (1Q), mediana (2Q), terceiro quartil (3Q), média ($\overline{x}$), desvio padrão ($\sigma$) e p-valor para o teste de Wilcox entre as versões (valores em negrito quando $p-valor > 0.05$).
    }
    \label{tab:#3DcDd#2}
\end{table}
}

\newcommand{\tabelaEstatisticas}[5]{
    \tabelaEstatisticasGeral{#1}{#2}{#3}{#4}{na implementação clássica (DC) e a proposta de implementação usando dataflow (DD).}{#5}
}

\newcommand{\figureDvndSogMog}[7]{
% #1 {box, scatter}, #2 {count, imp, time}, #3 {instance number}, #4 {Tempo, Melhoria}, #5 tamanho instância, #6 {DVND, RVND}, #7 {dvnd, rvnd}
\begin{figure}%
    \centering
    \includegraphics[scale=0.9]{figuras/#7/sog_mog/#1/#7_#1100sol_#2_in#3.png}
    \caption{#4 do #6 para a instância #3 de tamanho #5. \textit{SOG} refere-se a uma porta de saída e \textit{MOG} a múltiplas portas de saída.}%
    \label{fig:#2_#7SogMog_in#3}%
\end{figure}
}

\newcommand{\figureDvndGdvnd}[9]{
% #1 {box, scatter}, #2 {count, imp, time}, #3 {instance number}, #4 {Tempo, Melhoria}, #5 tamanho instância, #6 {DVND, RVND}, #7 {dvnd, rvnd}, #8 {man_time, full_time}, #9 {man, dvnd} #10 descricao
\begin{figure}%
    \centering
    \includegraphics[scale=0.9]{figuras/#7/#8/#1/#9_#1100sol_#2_in#3.png}
    \caption{#4 do #6 para a instância #3 de tamanho #5. \textit{DVND} refere-se ao tempo gasto pelo algoritmo de mesmo nome, para \textit{GDVND} é análogo ao anterior, no caso do \textit{GDVND-MAN} este se refere ao tempo do GDVND subtraido do tempo para gerenciar os movimentos.}%
    \label{fig:#2_#7_#8_in#3}%
\end{figure}
}

\newcommand{\figureDvndGdvndTime}[8]{
    \figureDvndGdvnd{#1}{#2}{#3}{#4}{#5}{#6}{#7}{man_time}{man}
}

\newcommand{\figureGdvndDvndRvnd}[9]{
% #1 {box, scatter}, #2 {count, imp, time}, #3 {instance number}, #4 {Tempo, Melhoria}, #5 tamanho instância, #6 {DVND, RVND}, #7 {dvnd, rvnd}, #8 {man_time, full_time}, #9 {man, dvnd} #10 descricao
\begin{figure}%
    \centering
    \includegraphics[scale=0.9]{figuras/#7/#8/#1/#9_#1100sol_#2_in#3.png}
    \caption{#4 do #6 para a instância #3 de tamanho #5. \textit{DVND}, \textit{GDVND} e \textit{RVND} referem-se ao tempo gasto pelos algoritmos de mesmo nome.}%
    \label{fig:#2_#7_#8_in#3}%
\end{figure}
}

% \subfloat[$m=#1$]{{ %scale=0.225
%         \includegraphics[scale=0.425]{figuras/dvnd/n#1/time#2.png}
%         \label{fig:timeDvndRvnd_n#1in#2}
%     }}%
% #1 {dvnd, rvnd, gdvnd}, #2 {sog_mog, dc_dd}, #3 {time, imp}, #4 in, #5 tamanho, #6 {box, scatter}
\newcommand{\subFig}[6]{
    \subfloat[][Instância #4, $n=#5$]{
        \includegraphics[scale=0.425]{figuras/#1/#2/#6/#1_#6100sol_#3_in#4.png}
		\label{fig:#1_#2_#3_in#4}
    }
% 	\begin{subfigure}{0.45\textwidth} % dvnd_box100sol_imp_in0
% 		\includegraphics[scale=0.425]{figuras/#1/#2/#6/#1_#6100sol_#3_in#4.png}
% 		\caption{Instância #4, $n=#5$}
        % \label{fig:#1_#2_#3_in#4}
    % \end{subfigure}
}

\newcommand{\subFigBox}[5]{
	\subFig{#1}{#2}{#3}{#4}{#5}{box}
}

\newcommand{\subFigScatter}[5]{
	\subFig{#1}{#2}{#3}{#4}{#5}{scatter}
}

% #1 {dvnd, rvnd, gdvnd}, #2 {sog_mog, dc_dd}, #3 {time, imp}, #4 {box, scatter}, #5 {Tempo do DVND...}
\newcommand{\multiFigureInstanciasGeral}[5]{
	\begin{figure}[ht]
		\centering
		\subFig{#1}{#2}{#3}{0}{52}{#4}
		~
		\subFig{#1}{#2}{#3}{1}{100}{#4}
		
		\subFig{#1}{#2}{#3}{2}{226}{#4}
		~
		\subFig{#1}{#2}{#3}{3}{318}{#4}
		\caption{#5 Instâncias 0 a 3.}
		\label{fig:#1_#2_#3_in0_4}
	\end{figure}
	
	\begin{figure}[ht]
		\centering
		\subFig{#1}{#2}{#3}{4}{501}{#4}
		~
		\subFig{#1}{#2}{#3}{5}{657}{#4}
		
		\subFig{#1}{#2}{#3}{6}{783}{#4}
		~
		\subFig{#1}{#2}{#3}{7}{1001}{#4}
		\caption{#5 Instâncias 5 a 7.}
		\label{fig:#1_#2_#3_in5_7}
	\end{figure}
}

% #1 {dvnd, rvnd, gdvnd}, #2 {sog_mog, dc_dd}, #3 {time, imp}, #4 {Tempo do DVND...}
\newcommand{\multiFigureInstancias}[4]{
    \multiFigureInstanciasGeral{#1}{#2}{#3}{box}{#4}
}


\chapter{Resultados} \label{cap:resultados}

Este capítulo exibe os resultados computacionais dos algoritmos propostos no Capítulo~\ref{cap:metodologia} para o caso do PML, para cada instância foi gerado um conjunto com 100 soluções iniciais aleatórias que foram submetidas aos métodos para comparação dos resultados.

Quando há referência à melhoria na solução (\textit{Imp}), esta melhoria pode ser calculada pelo quociente do valor da solução inicial pela solução final, ou seja:
\begin{equation}\label{eq:calculateImprovement}
Imp = \frac{f(\textrm{solução inicial})}{f(\textrm{solução final})}
\end{equation}

Desta forma quanto maior for o valor da melhoria ($Imp$) mais o método melhorou o valor da solução inicial.

\section{Instâncias} \label{sec:instancias}

Todas as instâncias usadas nos testes computacionais e cujas configurações de lançamento foram descritas na Tabela~\ref{tab:neighborhoodsLaunchConfigurarion} são as mesmas usadas em~\cite{wamca2016}.
Para o RVND foi feita uma implementação do algoritmo clássico (Algoritmo~\ref{alg:rvnd}) e também a implementação dataflow mencionada na Figura~\ref{fig:rvndGraph} fazendo uso de uma máquina.
Para o caso do DVND foi utilizada a implementação clássica (Algoritmo~\ref{alg:dvnd}) e a implementação dataflow proposta (Figura~\ref{fig:dvndGraph}), os resultados foram obtidos com diferentes números de máquinas e os mesmos são indicados conforme o caso.

\section{Implementação e ambiente computacional}\label{sec:amb}

A implementação para cada algoritmo proposto no Capítulo~\ref{cap:metodologia} utiliza a linguagem de programação \textit{C++11} em conjunto com a API CUDA\texttrademark, para a implementação dos grafos e do ambiente dataflow foi utilizada a biblioteca Sucuri~\cite{sucuri-original}\footnote{Disponível em \url{https://github.com/tiagoaoa/Sucuri}} implementada em Python, para a integração entre o dataflow e o código CUDA foi utilizada a biblioteca SimplePyCuda~\cite{simple-pycuda}\footnote{Disponível em \url{https://github.com/igormcoelho/simple-pycuda}}. As implementações com múltiplas threads usaram a biblioteca OpenMP.
%As implementações foram compiladas através do \textit{GCC} \textit{(GNU Compiler Collection)}\footnote{O GCC está disponível no seguinte sítio eletrônico: \url{https://gcc.gnu.org/}.} com a \textit{flag} de otimização $-O3$.
O ambiente computacional utilizado em todos os testes neste trabalho consiste de 4 máquinas com a seguinte configuração:

\begin{itemize}
    \item Processador Intel\textregistered Core\texttrademark i7-4820K 3.7 GHz (4 núcleos);
    \item 8 GB de memória RAM;
    \item Sistema Operacional Ubuntu 14.10 (x64);
    \item NVIDIA GeForce GTX 780 com 2304 CUDA cores.
\end{itemize}

\input{resultados/comandosFiguras.tex}

\input{resultados/sog_mog/index.tex}

\input{resultados/rvnd/index.tex}

\input{resultados/dvnd/index.tex}

\input{resultados/gdvnd/index.tex}


\chapter{Resultados} \label{cap:resultados}

Este capítulo exibe os resultados computacionais dos algoritmos propostos no Capítulo~\ref{cap:metodologia} para o caso do PML, para cada instância foi gerado um conjunto com 100 soluções iniciais aleatórias que foram submetidas aos métodos para comparação dos resultados.

Quando há referência à melhoria na solução (\textit{Imp}), esta melhoria pode ser calculada pelo quociente do valor da solução inicial pela solução final, ou seja:
\begin{equation}\label{eq:calculateImprovement}
Imp = \frac{f(\textrm{solução inicial})}{f(\textrm{solução final})}
\end{equation}

Desta forma quanto maior for o valor da melhoria ($Imp$) mais o método melhorou o valor da solução inicial.

\section{Instâncias} \label{sec:instancias}

Todas as instâncias usadas nos testes computacionais e cujas configurações de lançamento foram descritas na Tabela~\ref{tab:neighborhoodsLaunchConfigurarion} são as mesmas usadas em~\cite{wamca2016}.
Para o RVND foi feita uma implementação do algoritmo clássico (Algoritmo~\ref{alg:rvnd}) e também a implementação dataflow mencionada na Figura~\ref{fig:rvndGraph} fazendo uso de uma máquina.
Para o caso do DVND foi utilizada a implementação clássica (Algoritmo~\ref{alg:dvnd}) e a implementação dataflow proposta (Figura~\ref{fig:dvndGraph}), os resultados foram obtidos com diferentes números de máquinas e os mesmos são indicados conforme o caso.

\section{Implementação e ambiente computacional}\label{sec:amb}

A implementação para cada algoritmo proposto no Capítulo~\ref{cap:metodologia} utiliza a linguagem de programação \textit{C++11} em conjunto com a API CUDA\texttrademark, para a implementação dos grafos e do ambiente dataflow foi utilizada a biblioteca Sucuri~\cite{sucuri-original}\footnote{Disponível em \url{https://github.com/tiagoaoa/Sucuri}} implementada em Python, para a integração entre o dataflow e o código CUDA foi utilizada a biblioteca SimplePyCuda~\cite{simple-pycuda}\footnote{Disponível em \url{https://github.com/igormcoelho/simple-pycuda}}. As implementações com múltiplas threads usaram a biblioteca OpenMP.
%As implementações foram compiladas através do \textit{GCC} \textit{(GNU Compiler Collection)}\footnote{O GCC está disponível no seguinte sítio eletrônico: \url{https://gcc.gnu.org/}.} com a \textit{flag} de otimização $-O3$.
O ambiente computacional utilizado em todos os testes neste trabalho consiste de 4 máquinas com a seguinte configuração:

\begin{itemize}
    \item Processador Intel\textregistered Core\texttrademark i7-4820K 3.7 GHz (4 núcleos);
    \item 8 GB de memória RAM;
    \item Sistema Operacional Ubuntu 14.10 (x64);
    \item NVIDIA GeForce GTX 780 com 2304 CUDA cores.
\end{itemize}

\input{resultados/comandosFiguras.tex}

\input{resultados/sog_mog/index.tex}

\input{resultados/rvnd/index.tex}

\input{resultados/dvnd/index.tex}

\input{resultados/gdvnd/index.tex}


\chapter{Resultados} \label{cap:resultados}

Este capítulo exibe os resultados computacionais dos algoritmos propostos no Capítulo~\ref{cap:metodologia} para o caso do PML, para cada instância foi gerado um conjunto com 100 soluções iniciais aleatórias que foram submetidas aos métodos para comparação dos resultados.

Quando há referência à melhoria na solução (\textit{Imp}), esta melhoria pode ser calculada pelo quociente do valor da solução inicial pela solução final, ou seja:
\begin{equation}\label{eq:calculateImprovement}
Imp = \frac{f(\textrm{solução inicial})}{f(\textrm{solução final})}
\end{equation}

Desta forma quanto maior for o valor da melhoria ($Imp$) mais o método melhorou o valor da solução inicial.

\section{Instâncias} \label{sec:instancias}

Todas as instâncias usadas nos testes computacionais e cujas configurações de lançamento foram descritas na Tabela~\ref{tab:neighborhoodsLaunchConfigurarion} são as mesmas usadas em~\cite{wamca2016}.
Para o RVND foi feita uma implementação do algoritmo clássico (Algoritmo~\ref{alg:rvnd}) e também a implementação dataflow mencionada na Figura~\ref{fig:rvndGraph} fazendo uso de uma máquina.
Para o caso do DVND foi utilizada a implementação clássica (Algoritmo~\ref{alg:dvnd}) e a implementação dataflow proposta (Figura~\ref{fig:dvndGraph}), os resultados foram obtidos com diferentes números de máquinas e os mesmos são indicados conforme o caso.

\section{Implementação e ambiente computacional}\label{sec:amb}

A implementação para cada algoritmo proposto no Capítulo~\ref{cap:metodologia} utiliza a linguagem de programação \textit{C++11} em conjunto com a API CUDA\texttrademark, para a implementação dos grafos e do ambiente dataflow foi utilizada a biblioteca Sucuri~\cite{sucuri-original}\footnote{Disponível em \url{https://github.com/tiagoaoa/Sucuri}} implementada em Python, para a integração entre o dataflow e o código CUDA foi utilizada a biblioteca SimplePyCuda~\cite{simple-pycuda}\footnote{Disponível em \url{https://github.com/igormcoelho/simple-pycuda}}. As implementações com múltiplas threads usaram a biblioteca OpenMP.
%As implementações foram compiladas através do \textit{GCC} \textit{(GNU Compiler Collection)}\footnote{O GCC está disponível no seguinte sítio eletrônico: \url{https://gcc.gnu.org/}.} com a \textit{flag} de otimização $-O3$.
O ambiente computacional utilizado em todos os testes neste trabalho consiste de 4 máquinas com a seguinte configuração:

\begin{itemize}
    \item Processador Intel\textregistered Core\texttrademark i7-4820K 3.7 GHz (4 núcleos);
    \item 8 GB de memória RAM;
    \item Sistema Operacional Ubuntu 14.10 (x64);
    \item NVIDIA GeForce GTX 780 com 2304 CUDA cores.
\end{itemize}

\input{resultados/comandosFiguras.tex}

\input{resultados/sog_mog/index.tex}

\input{resultados/rvnd/index.tex}

\input{resultados/dvnd/index.tex}

\input{resultados/gdvnd/index.tex}


\chapter{Resultados} \label{cap:resultados}

Este capítulo exibe os resultados computacionais dos algoritmos propostos no Capítulo~\ref{cap:metodologia} para o caso do PML, para cada instância foi gerado um conjunto com 100 soluções iniciais aleatórias que foram submetidas aos métodos para comparação dos resultados.

Quando há referência à melhoria na solução (\textit{Imp}), esta melhoria pode ser calculada pelo quociente do valor da solução inicial pela solução final, ou seja:
\begin{equation}\label{eq:calculateImprovement}
Imp = \frac{f(\textrm{solução inicial})}{f(\textrm{solução final})}
\end{equation}

Desta forma quanto maior for o valor da melhoria ($Imp$) mais o método melhorou o valor da solução inicial.

\section{Instâncias} \label{sec:instancias}

Todas as instâncias usadas nos testes computacionais e cujas configurações de lançamento foram descritas na Tabela~\ref{tab:neighborhoodsLaunchConfigurarion} são as mesmas usadas em~\cite{wamca2016}.
Para o RVND foi feita uma implementação do algoritmo clássico (Algoritmo~\ref{alg:rvnd}) e também a implementação dataflow mencionada na Figura~\ref{fig:rvndGraph} fazendo uso de uma máquina.
Para o caso do DVND foi utilizada a implementação clássica (Algoritmo~\ref{alg:dvnd}) e a implementação dataflow proposta (Figura~\ref{fig:dvndGraph}), os resultados foram obtidos com diferentes números de máquinas e os mesmos são indicados conforme o caso.

\section{Implementação e ambiente computacional}\label{sec:amb}

A implementação para cada algoritmo proposto no Capítulo~\ref{cap:metodologia} utiliza a linguagem de programação \textit{C++11} em conjunto com a API CUDA\texttrademark, para a implementação dos grafos e do ambiente dataflow foi utilizada a biblioteca Sucuri~\cite{sucuri-original}\footnote{Disponível em \url{https://github.com/tiagoaoa/Sucuri}} implementada em Python, para a integração entre o dataflow e o código CUDA foi utilizada a biblioteca SimplePyCuda~\cite{simple-pycuda}\footnote{Disponível em \url{https://github.com/igormcoelho/simple-pycuda}}. As implementações com múltiplas threads usaram a biblioteca OpenMP.
%As implementações foram compiladas através do \textit{GCC} \textit{(GNU Compiler Collection)}\footnote{O GCC está disponível no seguinte sítio eletrônico: \url{https://gcc.gnu.org/}.} com a \textit{flag} de otimização $-O3$.
O ambiente computacional utilizado em todos os testes neste trabalho consiste de 4 máquinas com a seguinte configuração:

\begin{itemize}
    \item Processador Intel\textregistered Core\texttrademark i7-4820K 3.7 GHz (4 núcleos);
    \item 8 GB de memória RAM;
    \item Sistema Operacional Ubuntu 14.10 (x64);
    \item NVIDIA GeForce GTX 780 com 2304 CUDA cores.
\end{itemize}

\input{resultados/comandosFiguras.tex}

\input{resultados/sog_mog/index.tex}

\input{resultados/rvnd/index.tex}

\input{resultados/dvnd/index.tex}

\input{resultados/gdvnd/index.tex}




\chapter{Resultados} \label{cap:resultados}

Este capítulo exibe os resultados computacionais dos algoritmos propostos no Capítulo~\ref{cap:metodologia} para o caso do PML, para cada instância foi gerado um conjunto com 100 soluções iniciais aleatórias que foram submetidas aos métodos para comparação dos resultados.

Quando há referência à melhoria na solução (\textit{Imp}), esta melhoria pode ser calculada pelo quociente do valor da solução inicial pela solução final, ou seja:
\begin{equation}\label{eq:calculateImprovement}
Imp = \frac{f(\textrm{solução inicial})}{f(\textrm{solução final})}
\end{equation}

Desta forma quanto maior for o valor da melhoria ($Imp$) mais o método melhorou o valor da solução inicial.

\section{Instâncias} \label{sec:instancias}

Todas as instâncias usadas nos testes computacionais e cujas configurações de lançamento foram descritas na Tabela~\ref{tab:neighborhoodsLaunchConfigurarion} são as mesmas usadas em~\cite{wamca2016}.
Para o RVND foi feita uma implementação do algoritmo clássico (Algoritmo~\ref{alg:rvnd}) e também a implementação dataflow mencionada na Figura~\ref{fig:rvndGraph} fazendo uso de uma máquina.
Para o caso do DVND foi utilizada a implementação clássica (Algoritmo~\ref{alg:dvnd}) e a implementação dataflow proposta (Figura~\ref{fig:dvndGraph}), os resultados foram obtidos com diferentes números de máquinas e os mesmos são indicados conforme o caso.

\section{Implementação e ambiente computacional}\label{sec:amb}

A implementação para cada algoritmo proposto no Capítulo~\ref{cap:metodologia} utiliza a linguagem de programação \textit{C++11} em conjunto com a API CUDA\texttrademark, para a implementação dos grafos e do ambiente dataflow foi utilizada a biblioteca Sucuri~\cite{sucuri-original}\footnote{Disponível em \url{https://github.com/tiagoaoa/Sucuri}} implementada em Python, para a integração entre o dataflow e o código CUDA foi utilizada a biblioteca SimplePyCuda~\cite{simple-pycuda}\footnote{Disponível em \url{https://github.com/igormcoelho/simple-pycuda}}. As implementações com múltiplas threads usaram a biblioteca OpenMP.
%As implementações foram compiladas através do \textit{GCC} \textit{(GNU Compiler Collection)}\footnote{O GCC está disponível no seguinte sítio eletrônico: \url{https://gcc.gnu.org/}.} com a \textit{flag} de otimização $-O3$.
O ambiente computacional utilizado em todos os testes neste trabalho consiste de 4 máquinas com a seguinte configuração:

\begin{itemize}
    \item Processador Intel\textregistered Core\texttrademark i7-4820K 3.7 GHz (4 núcleos);
    \item 8 GB de memória RAM;
    \item Sistema Operacional Ubuntu 14.10 (x64);
    \item NVIDIA GeForce GTX 780 com 2304 CUDA cores.
\end{itemize}

\newcommand{\figureDvndOrRvndDcDd}[7]{
% #1 {box, scatter}, #2 {count, imp, time}, #3 {instance number}, #4 {Tempo, Melhoria}, #5 tamanho instância, #6 {DVND, RVND}, #7 {dvnd, rvnd}
\begin{figure}%
    \centering
    \includegraphics[scale=0.9]{figuras/#7/dc_dd/#1/#7_#1100sol_#2_in#3.png}
    \caption{#4 do #6 para a instância #3 de tamanho #5. $m$ indica o número de máquinas, \textit{DC} refere-se ao #6 clássico e \textit{DD} ao #6 implementado em dataflow.}%
    \label{fig:#2_#7DcDd_in#3}%
\end{figure}
}

\newcommand{\figureDvndDcDd}[5]{
% #1 {box, scatter}, #2 {count, imp, time}, #3 {instance number}, #4 {Tempo, Melhoria}, #5 tamanho instância
    \figureDvndOrRvndDcDd{#1}{#2}{#3}{#4}{#5}{DVND}{dvnd}
}

\newcommand{\figureRvndDcDd}[5]{
% #1 {box, scatter}, #2 {count, imp, time}, #3 {instance number}, #4 {Tempo, Melhoria}, #5 tamanho instância
    \figureDvndOrRvndDcDd{#1}{#2}{#3}{#4}{#5}{RVND}{rvnd}
}

\newcommand{\tabelaEstatisticasGeral}[6]{
% #1 Descrição, #2 label, #3 {dvnd, rvnd}, #4 {DVND, RVND}, #6 DD/DC, #6 Conteúdo
\begin{table}[ht]
    \centering
    \begin{tabular}{c|ccc|cc|ccc|cc|c}
        \hline \hline
        \# & Tipo & $m$ & $n$ & $min$ & $max$ & 1Q & 2Q & 3Q & $\overline{x}$ & $\sigma$ & $p-valor$ \\ \hline
        #6
    \end{tabular}
    \caption{#1 #4 #5
        Instância (\#), tipo de implementação (Tipo), número de máquinas ($m$), tamanho da instância ($n$), valor mínimo ($min$), máximo ($max$), primeiro quartil (1Q), mediana (2Q), terceiro quartil (3Q), média ($\overline{x}$), desvio padrão ($\sigma$) e p-valor para o teste de Wilcox entre as versões (valores em negrito quando $p-valor > 0.05$).
    }
    \label{tab:#3DcDd#2}
\end{table}
}

\newcommand{\tabelaEstatisticas}[5]{
    \tabelaEstatisticasGeral{#1}{#2}{#3}{#4}{na implementação clássica (DC) e a proposta de implementação usando dataflow (DD).}{#5}
}

\newcommand{\figureDvndSogMog}[7]{
% #1 {box, scatter}, #2 {count, imp, time}, #3 {instance number}, #4 {Tempo, Melhoria}, #5 tamanho instância, #6 {DVND, RVND}, #7 {dvnd, rvnd}
\begin{figure}%
    \centering
    \includegraphics[scale=0.9]{figuras/#7/sog_mog/#1/#7_#1100sol_#2_in#3.png}
    \caption{#4 do #6 para a instância #3 de tamanho #5. \textit{SOG} refere-se a uma porta de saída e \textit{MOG} a múltiplas portas de saída.}%
    \label{fig:#2_#7SogMog_in#3}%
\end{figure}
}

\newcommand{\figureDvndGdvnd}[9]{
% #1 {box, scatter}, #2 {count, imp, time}, #3 {instance number}, #4 {Tempo, Melhoria}, #5 tamanho instância, #6 {DVND, RVND}, #7 {dvnd, rvnd}, #8 {man_time, full_time}, #9 {man, dvnd} #10 descricao
\begin{figure}%
    \centering
    \includegraphics[scale=0.9]{figuras/#7/#8/#1/#9_#1100sol_#2_in#3.png}
    \caption{#4 do #6 para a instância #3 de tamanho #5. \textit{DVND} refere-se ao tempo gasto pelo algoritmo de mesmo nome, para \textit{GDVND} é análogo ao anterior, no caso do \textit{GDVND-MAN} este se refere ao tempo do GDVND subtraido do tempo para gerenciar os movimentos.}%
    \label{fig:#2_#7_#8_in#3}%
\end{figure}
}

\newcommand{\figureDvndGdvndTime}[8]{
    \figureDvndGdvnd{#1}{#2}{#3}{#4}{#5}{#6}{#7}{man_time}{man}
}

\newcommand{\figureGdvndDvndRvnd}[9]{
% #1 {box, scatter}, #2 {count, imp, time}, #3 {instance number}, #4 {Tempo, Melhoria}, #5 tamanho instância, #6 {DVND, RVND}, #7 {dvnd, rvnd}, #8 {man_time, full_time}, #9 {man, dvnd} #10 descricao
\begin{figure}%
    \centering
    \includegraphics[scale=0.9]{figuras/#7/#8/#1/#9_#1100sol_#2_in#3.png}
    \caption{#4 do #6 para a instância #3 de tamanho #5. \textit{DVND}, \textit{GDVND} e \textit{RVND} referem-se ao tempo gasto pelos algoritmos de mesmo nome.}%
    \label{fig:#2_#7_#8_in#3}%
\end{figure}
}

% \subfloat[$m=#1$]{{ %scale=0.225
%         \includegraphics[scale=0.425]{figuras/dvnd/n#1/time#2.png}
%         \label{fig:timeDvndRvnd_n#1in#2}
%     }}%
% #1 {dvnd, rvnd, gdvnd}, #2 {sog_mog, dc_dd}, #3 {time, imp}, #4 in, #5 tamanho, #6 {box, scatter}
\newcommand{\subFig}[6]{
    \subfloat[][Instância #4, $n=#5$]{
        \includegraphics[scale=0.425]{figuras/#1/#2/#6/#1_#6100sol_#3_in#4.png}
		\label{fig:#1_#2_#3_in#4}
    }
% 	\begin{subfigure}{0.45\textwidth} % dvnd_box100sol_imp_in0
% 		\includegraphics[scale=0.425]{figuras/#1/#2/#6/#1_#6100sol_#3_in#4.png}
% 		\caption{Instância #4, $n=#5$}
        % \label{fig:#1_#2_#3_in#4}
    % \end{subfigure}
}

\newcommand{\subFigBox}[5]{
	\subFig{#1}{#2}{#3}{#4}{#5}{box}
}

\newcommand{\subFigScatter}[5]{
	\subFig{#1}{#2}{#3}{#4}{#5}{scatter}
}

% #1 {dvnd, rvnd, gdvnd}, #2 {sog_mog, dc_dd}, #3 {time, imp}, #4 {box, scatter}, #5 {Tempo do DVND...}
\newcommand{\multiFigureInstanciasGeral}[5]{
	\begin{figure}[ht]
		\centering
		\subFig{#1}{#2}{#3}{0}{52}{#4}
		~
		\subFig{#1}{#2}{#3}{1}{100}{#4}
		
		\subFig{#1}{#2}{#3}{2}{226}{#4}
		~
		\subFig{#1}{#2}{#3}{3}{318}{#4}
		\caption{#5 Instâncias 0 a 3.}
		\label{fig:#1_#2_#3_in0_4}
	\end{figure}
	
	\begin{figure}[ht]
		\centering
		\subFig{#1}{#2}{#3}{4}{501}{#4}
		~
		\subFig{#1}{#2}{#3}{5}{657}{#4}
		
		\subFig{#1}{#2}{#3}{6}{783}{#4}
		~
		\subFig{#1}{#2}{#3}{7}{1001}{#4}
		\caption{#5 Instâncias 5 a 7.}
		\label{fig:#1_#2_#3_in5_7}
	\end{figure}
}

% #1 {dvnd, rvnd, gdvnd}, #2 {sog_mog, dc_dd}, #3 {time, imp}, #4 {Tempo do DVND...}
\newcommand{\multiFigureInstancias}[4]{
    \multiFigureInstanciasGeral{#1}{#2}{#3}{box}{#4}
}


\chapter{Resultados} \label{cap:resultados}

Este capítulo exibe os resultados computacionais dos algoritmos propostos no Capítulo~\ref{cap:metodologia} para o caso do PML, para cada instância foi gerado um conjunto com 100 soluções iniciais aleatórias que foram submetidas aos métodos para comparação dos resultados.

Quando há referência à melhoria na solução (\textit{Imp}), esta melhoria pode ser calculada pelo quociente do valor da solução inicial pela solução final, ou seja:
\begin{equation}\label{eq:calculateImprovement}
Imp = \frac{f(\textrm{solução inicial})}{f(\textrm{solução final})}
\end{equation}

Desta forma quanto maior for o valor da melhoria ($Imp$) mais o método melhorou o valor da solução inicial.

\section{Instâncias} \label{sec:instancias}

Todas as instâncias usadas nos testes computacionais e cujas configurações de lançamento foram descritas na Tabela~\ref{tab:neighborhoodsLaunchConfigurarion} são as mesmas usadas em~\cite{wamca2016}.
Para o RVND foi feita uma implementação do algoritmo clássico (Algoritmo~\ref{alg:rvnd}) e também a implementação dataflow mencionada na Figura~\ref{fig:rvndGraph} fazendo uso de uma máquina.
Para o caso do DVND foi utilizada a implementação clássica (Algoritmo~\ref{alg:dvnd}) e a implementação dataflow proposta (Figura~\ref{fig:dvndGraph}), os resultados foram obtidos com diferentes números de máquinas e os mesmos são indicados conforme o caso.

\section{Implementação e ambiente computacional}\label{sec:amb}

A implementação para cada algoritmo proposto no Capítulo~\ref{cap:metodologia} utiliza a linguagem de programação \textit{C++11} em conjunto com a API CUDA\texttrademark, para a implementação dos grafos e do ambiente dataflow foi utilizada a biblioteca Sucuri~\cite{sucuri-original}\footnote{Disponível em \url{https://github.com/tiagoaoa/Sucuri}} implementada em Python, para a integração entre o dataflow e o código CUDA foi utilizada a biblioteca SimplePyCuda~\cite{simple-pycuda}\footnote{Disponível em \url{https://github.com/igormcoelho/simple-pycuda}}. As implementações com múltiplas threads usaram a biblioteca OpenMP.
%As implementações foram compiladas através do \textit{GCC} \textit{(GNU Compiler Collection)}\footnote{O GCC está disponível no seguinte sítio eletrônico: \url{https://gcc.gnu.org/}.} com a \textit{flag} de otimização $-O3$.
O ambiente computacional utilizado em todos os testes neste trabalho consiste de 4 máquinas com a seguinte configuração:

\begin{itemize}
    \item Processador Intel\textregistered Core\texttrademark i7-4820K 3.7 GHz (4 núcleos);
    \item 8 GB de memória RAM;
    \item Sistema Operacional Ubuntu 14.10 (x64);
    \item NVIDIA GeForce GTX 780 com 2304 CUDA cores.
\end{itemize}

\newcommand{\figureDvndOrRvndDcDd}[7]{
% #1 {box, scatter}, #2 {count, imp, time}, #3 {instance number}, #4 {Tempo, Melhoria}, #5 tamanho instância, #6 {DVND, RVND}, #7 {dvnd, rvnd}
\begin{figure}%
    \centering
    \includegraphics[scale=0.9]{figuras/#7/dc_dd/#1/#7_#1100sol_#2_in#3.png}
    \caption{#4 do #6 para a instância #3 de tamanho #5. $m$ indica o número de máquinas, \textit{DC} refere-se ao #6 clássico e \textit{DD} ao #6 implementado em dataflow.}%
    \label{fig:#2_#7DcDd_in#3}%
\end{figure}
}

\newcommand{\figureDvndDcDd}[5]{
% #1 {box, scatter}, #2 {count, imp, time}, #3 {instance number}, #4 {Tempo, Melhoria}, #5 tamanho instância
    \figureDvndOrRvndDcDd{#1}{#2}{#3}{#4}{#5}{DVND}{dvnd}
}

\newcommand{\figureRvndDcDd}[5]{
% #1 {box, scatter}, #2 {count, imp, time}, #3 {instance number}, #4 {Tempo, Melhoria}, #5 tamanho instância
    \figureDvndOrRvndDcDd{#1}{#2}{#3}{#4}{#5}{RVND}{rvnd}
}

\newcommand{\tabelaEstatisticasGeral}[6]{
% #1 Descrição, #2 label, #3 {dvnd, rvnd}, #4 {DVND, RVND}, #6 DD/DC, #6 Conteúdo
\begin{table}[ht]
    \centering
    \begin{tabular}{c|ccc|cc|ccc|cc|c}
        \hline \hline
        \# & Tipo & $m$ & $n$ & $min$ & $max$ & 1Q & 2Q & 3Q & $\overline{x}$ & $\sigma$ & $p-valor$ \\ \hline
        #6
    \end{tabular}
    \caption{#1 #4 #5
        Instância (\#), tipo de implementação (Tipo), número de máquinas ($m$), tamanho da instância ($n$), valor mínimo ($min$), máximo ($max$), primeiro quartil (1Q), mediana (2Q), terceiro quartil (3Q), média ($\overline{x}$), desvio padrão ($\sigma$) e p-valor para o teste de Wilcox entre as versões (valores em negrito quando $p-valor > 0.05$).
    }
    \label{tab:#3DcDd#2}
\end{table}
}

\newcommand{\tabelaEstatisticas}[5]{
    \tabelaEstatisticasGeral{#1}{#2}{#3}{#4}{na implementação clássica (DC) e a proposta de implementação usando dataflow (DD).}{#5}
}

\newcommand{\figureDvndSogMog}[7]{
% #1 {box, scatter}, #2 {count, imp, time}, #3 {instance number}, #4 {Tempo, Melhoria}, #5 tamanho instância, #6 {DVND, RVND}, #7 {dvnd, rvnd}
\begin{figure}%
    \centering
    \includegraphics[scale=0.9]{figuras/#7/sog_mog/#1/#7_#1100sol_#2_in#3.png}
    \caption{#4 do #6 para a instância #3 de tamanho #5. \textit{SOG} refere-se a uma porta de saída e \textit{MOG} a múltiplas portas de saída.}%
    \label{fig:#2_#7SogMog_in#3}%
\end{figure}
}

\newcommand{\figureDvndGdvnd}[9]{
% #1 {box, scatter}, #2 {count, imp, time}, #3 {instance number}, #4 {Tempo, Melhoria}, #5 tamanho instância, #6 {DVND, RVND}, #7 {dvnd, rvnd}, #8 {man_time, full_time}, #9 {man, dvnd} #10 descricao
\begin{figure}%
    \centering
    \includegraphics[scale=0.9]{figuras/#7/#8/#1/#9_#1100sol_#2_in#3.png}
    \caption{#4 do #6 para a instância #3 de tamanho #5. \textit{DVND} refere-se ao tempo gasto pelo algoritmo de mesmo nome, para \textit{GDVND} é análogo ao anterior, no caso do \textit{GDVND-MAN} este se refere ao tempo do GDVND subtraido do tempo para gerenciar os movimentos.}%
    \label{fig:#2_#7_#8_in#3}%
\end{figure}
}

\newcommand{\figureDvndGdvndTime}[8]{
    \figureDvndGdvnd{#1}{#2}{#3}{#4}{#5}{#6}{#7}{man_time}{man}
}

\newcommand{\figureGdvndDvndRvnd}[9]{
% #1 {box, scatter}, #2 {count, imp, time}, #3 {instance number}, #4 {Tempo, Melhoria}, #5 tamanho instância, #6 {DVND, RVND}, #7 {dvnd, rvnd}, #8 {man_time, full_time}, #9 {man, dvnd} #10 descricao
\begin{figure}%
    \centering
    \includegraphics[scale=0.9]{figuras/#7/#8/#1/#9_#1100sol_#2_in#3.png}
    \caption{#4 do #6 para a instância #3 de tamanho #5. \textit{DVND}, \textit{GDVND} e \textit{RVND} referem-se ao tempo gasto pelos algoritmos de mesmo nome.}%
    \label{fig:#2_#7_#8_in#3}%
\end{figure}
}

% \subfloat[$m=#1$]{{ %scale=0.225
%         \includegraphics[scale=0.425]{figuras/dvnd/n#1/time#2.png}
%         \label{fig:timeDvndRvnd_n#1in#2}
%     }}%
% #1 {dvnd, rvnd, gdvnd}, #2 {sog_mog, dc_dd}, #3 {time, imp}, #4 in, #5 tamanho, #6 {box, scatter}
\newcommand{\subFig}[6]{
    \subfloat[][Instância #4, $n=#5$]{
        \includegraphics[scale=0.425]{figuras/#1/#2/#6/#1_#6100sol_#3_in#4.png}
		\label{fig:#1_#2_#3_in#4}
    }
% 	\begin{subfigure}{0.45\textwidth} % dvnd_box100sol_imp_in0
% 		\includegraphics[scale=0.425]{figuras/#1/#2/#6/#1_#6100sol_#3_in#4.png}
% 		\caption{Instância #4, $n=#5$}
        % \label{fig:#1_#2_#3_in#4}
    % \end{subfigure}
}

\newcommand{\subFigBox}[5]{
	\subFig{#1}{#2}{#3}{#4}{#5}{box}
}

\newcommand{\subFigScatter}[5]{
	\subFig{#1}{#2}{#3}{#4}{#5}{scatter}
}

% #1 {dvnd, rvnd, gdvnd}, #2 {sog_mog, dc_dd}, #3 {time, imp}, #4 {box, scatter}, #5 {Tempo do DVND...}
\newcommand{\multiFigureInstanciasGeral}[5]{
	\begin{figure}[ht]
		\centering
		\subFig{#1}{#2}{#3}{0}{52}{#4}
		~
		\subFig{#1}{#2}{#3}{1}{100}{#4}
		
		\subFig{#1}{#2}{#3}{2}{226}{#4}
		~
		\subFig{#1}{#2}{#3}{3}{318}{#4}
		\caption{#5 Instâncias 0 a 3.}
		\label{fig:#1_#2_#3_in0_4}
	\end{figure}
	
	\begin{figure}[ht]
		\centering
		\subFig{#1}{#2}{#3}{4}{501}{#4}
		~
		\subFig{#1}{#2}{#3}{5}{657}{#4}
		
		\subFig{#1}{#2}{#3}{6}{783}{#4}
		~
		\subFig{#1}{#2}{#3}{7}{1001}{#4}
		\caption{#5 Instâncias 5 a 7.}
		\label{fig:#1_#2_#3_in5_7}
	\end{figure}
}

% #1 {dvnd, rvnd, gdvnd}, #2 {sog_mog, dc_dd}, #3 {time, imp}, #4 {Tempo do DVND...}
\newcommand{\multiFigureInstancias}[4]{
    \multiFigureInstanciasGeral{#1}{#2}{#3}{box}{#4}
}


\chapter{Resultados} \label{cap:resultados}

Este capítulo exibe os resultados computacionais dos algoritmos propostos no Capítulo~\ref{cap:metodologia} para o caso do PML, para cada instância foi gerado um conjunto com 100 soluções iniciais aleatórias que foram submetidas aos métodos para comparação dos resultados.

Quando há referência à melhoria na solução (\textit{Imp}), esta melhoria pode ser calculada pelo quociente do valor da solução inicial pela solução final, ou seja:
\begin{equation}\label{eq:calculateImprovement}
Imp = \frac{f(\textrm{solução inicial})}{f(\textrm{solução final})}
\end{equation}

Desta forma quanto maior for o valor da melhoria ($Imp$) mais o método melhorou o valor da solução inicial.

\section{Instâncias} \label{sec:instancias}

Todas as instâncias usadas nos testes computacionais e cujas configurações de lançamento foram descritas na Tabela~\ref{tab:neighborhoodsLaunchConfigurarion} são as mesmas usadas em~\cite{wamca2016}.
Para o RVND foi feita uma implementação do algoritmo clássico (Algoritmo~\ref{alg:rvnd}) e também a implementação dataflow mencionada na Figura~\ref{fig:rvndGraph} fazendo uso de uma máquina.
Para o caso do DVND foi utilizada a implementação clássica (Algoritmo~\ref{alg:dvnd}) e a implementação dataflow proposta (Figura~\ref{fig:dvndGraph}), os resultados foram obtidos com diferentes números de máquinas e os mesmos são indicados conforme o caso.

\section{Implementação e ambiente computacional}\label{sec:amb}

A implementação para cada algoritmo proposto no Capítulo~\ref{cap:metodologia} utiliza a linguagem de programação \textit{C++11} em conjunto com a API CUDA\texttrademark, para a implementação dos grafos e do ambiente dataflow foi utilizada a biblioteca Sucuri~\cite{sucuri-original}\footnote{Disponível em \url{https://github.com/tiagoaoa/Sucuri}} implementada em Python, para a integração entre o dataflow e o código CUDA foi utilizada a biblioteca SimplePyCuda~\cite{simple-pycuda}\footnote{Disponível em \url{https://github.com/igormcoelho/simple-pycuda}}. As implementações com múltiplas threads usaram a biblioteca OpenMP.
%As implementações foram compiladas através do \textit{GCC} \textit{(GNU Compiler Collection)}\footnote{O GCC está disponível no seguinte sítio eletrônico: \url{https://gcc.gnu.org/}.} com a \textit{flag} de otimização $-O3$.
O ambiente computacional utilizado em todos os testes neste trabalho consiste de 4 máquinas com a seguinte configuração:

\begin{itemize}
    \item Processador Intel\textregistered Core\texttrademark i7-4820K 3.7 GHz (4 núcleos);
    \item 8 GB de memória RAM;
    \item Sistema Operacional Ubuntu 14.10 (x64);
    \item NVIDIA GeForce GTX 780 com 2304 CUDA cores.
\end{itemize}

\input{resultados/comandosFiguras.tex}

\input{resultados/sog_mog/index.tex}

\input{resultados/rvnd/index.tex}

\input{resultados/dvnd/index.tex}

\input{resultados/gdvnd/index.tex}


\chapter{Resultados} \label{cap:resultados}

Este capítulo exibe os resultados computacionais dos algoritmos propostos no Capítulo~\ref{cap:metodologia} para o caso do PML, para cada instância foi gerado um conjunto com 100 soluções iniciais aleatórias que foram submetidas aos métodos para comparação dos resultados.

Quando há referência à melhoria na solução (\textit{Imp}), esta melhoria pode ser calculada pelo quociente do valor da solução inicial pela solução final, ou seja:
\begin{equation}\label{eq:calculateImprovement}
Imp = \frac{f(\textrm{solução inicial})}{f(\textrm{solução final})}
\end{equation}

Desta forma quanto maior for o valor da melhoria ($Imp$) mais o método melhorou o valor da solução inicial.

\section{Instâncias} \label{sec:instancias}

Todas as instâncias usadas nos testes computacionais e cujas configurações de lançamento foram descritas na Tabela~\ref{tab:neighborhoodsLaunchConfigurarion} são as mesmas usadas em~\cite{wamca2016}.
Para o RVND foi feita uma implementação do algoritmo clássico (Algoritmo~\ref{alg:rvnd}) e também a implementação dataflow mencionada na Figura~\ref{fig:rvndGraph} fazendo uso de uma máquina.
Para o caso do DVND foi utilizada a implementação clássica (Algoritmo~\ref{alg:dvnd}) e a implementação dataflow proposta (Figura~\ref{fig:dvndGraph}), os resultados foram obtidos com diferentes números de máquinas e os mesmos são indicados conforme o caso.

\section{Implementação e ambiente computacional}\label{sec:amb}

A implementação para cada algoritmo proposto no Capítulo~\ref{cap:metodologia} utiliza a linguagem de programação \textit{C++11} em conjunto com a API CUDA\texttrademark, para a implementação dos grafos e do ambiente dataflow foi utilizada a biblioteca Sucuri~\cite{sucuri-original}\footnote{Disponível em \url{https://github.com/tiagoaoa/Sucuri}} implementada em Python, para a integração entre o dataflow e o código CUDA foi utilizada a biblioteca SimplePyCuda~\cite{simple-pycuda}\footnote{Disponível em \url{https://github.com/igormcoelho/simple-pycuda}}. As implementações com múltiplas threads usaram a biblioteca OpenMP.
%As implementações foram compiladas através do \textit{GCC} \textit{(GNU Compiler Collection)}\footnote{O GCC está disponível no seguinte sítio eletrônico: \url{https://gcc.gnu.org/}.} com a \textit{flag} de otimização $-O3$.
O ambiente computacional utilizado em todos os testes neste trabalho consiste de 4 máquinas com a seguinte configuração:

\begin{itemize}
    \item Processador Intel\textregistered Core\texttrademark i7-4820K 3.7 GHz (4 núcleos);
    \item 8 GB de memória RAM;
    \item Sistema Operacional Ubuntu 14.10 (x64);
    \item NVIDIA GeForce GTX 780 com 2304 CUDA cores.
\end{itemize}

\input{resultados/comandosFiguras.tex}

\input{resultados/sog_mog/index.tex}

\input{resultados/rvnd/index.tex}

\input{resultados/dvnd/index.tex}

\input{resultados/gdvnd/index.tex}


\chapter{Resultados} \label{cap:resultados}

Este capítulo exibe os resultados computacionais dos algoritmos propostos no Capítulo~\ref{cap:metodologia} para o caso do PML, para cada instância foi gerado um conjunto com 100 soluções iniciais aleatórias que foram submetidas aos métodos para comparação dos resultados.

Quando há referência à melhoria na solução (\textit{Imp}), esta melhoria pode ser calculada pelo quociente do valor da solução inicial pela solução final, ou seja:
\begin{equation}\label{eq:calculateImprovement}
Imp = \frac{f(\textrm{solução inicial})}{f(\textrm{solução final})}
\end{equation}

Desta forma quanto maior for o valor da melhoria ($Imp$) mais o método melhorou o valor da solução inicial.

\section{Instâncias} \label{sec:instancias}

Todas as instâncias usadas nos testes computacionais e cujas configurações de lançamento foram descritas na Tabela~\ref{tab:neighborhoodsLaunchConfigurarion} são as mesmas usadas em~\cite{wamca2016}.
Para o RVND foi feita uma implementação do algoritmo clássico (Algoritmo~\ref{alg:rvnd}) e também a implementação dataflow mencionada na Figura~\ref{fig:rvndGraph} fazendo uso de uma máquina.
Para o caso do DVND foi utilizada a implementação clássica (Algoritmo~\ref{alg:dvnd}) e a implementação dataflow proposta (Figura~\ref{fig:dvndGraph}), os resultados foram obtidos com diferentes números de máquinas e os mesmos são indicados conforme o caso.

\section{Implementação e ambiente computacional}\label{sec:amb}

A implementação para cada algoritmo proposto no Capítulo~\ref{cap:metodologia} utiliza a linguagem de programação \textit{C++11} em conjunto com a API CUDA\texttrademark, para a implementação dos grafos e do ambiente dataflow foi utilizada a biblioteca Sucuri~\cite{sucuri-original}\footnote{Disponível em \url{https://github.com/tiagoaoa/Sucuri}} implementada em Python, para a integração entre o dataflow e o código CUDA foi utilizada a biblioteca SimplePyCuda~\cite{simple-pycuda}\footnote{Disponível em \url{https://github.com/igormcoelho/simple-pycuda}}. As implementações com múltiplas threads usaram a biblioteca OpenMP.
%As implementações foram compiladas através do \textit{GCC} \textit{(GNU Compiler Collection)}\footnote{O GCC está disponível no seguinte sítio eletrônico: \url{https://gcc.gnu.org/}.} com a \textit{flag} de otimização $-O3$.
O ambiente computacional utilizado em todos os testes neste trabalho consiste de 4 máquinas com a seguinte configuração:

\begin{itemize}
    \item Processador Intel\textregistered Core\texttrademark i7-4820K 3.7 GHz (4 núcleos);
    \item 8 GB de memória RAM;
    \item Sistema Operacional Ubuntu 14.10 (x64);
    \item NVIDIA GeForce GTX 780 com 2304 CUDA cores.
\end{itemize}

\input{resultados/comandosFiguras.tex}

\input{resultados/sog_mog/index.tex}

\input{resultados/rvnd/index.tex}

\input{resultados/dvnd/index.tex}

\input{resultados/gdvnd/index.tex}


\chapter{Resultados} \label{cap:resultados}

Este capítulo exibe os resultados computacionais dos algoritmos propostos no Capítulo~\ref{cap:metodologia} para o caso do PML, para cada instância foi gerado um conjunto com 100 soluções iniciais aleatórias que foram submetidas aos métodos para comparação dos resultados.

Quando há referência à melhoria na solução (\textit{Imp}), esta melhoria pode ser calculada pelo quociente do valor da solução inicial pela solução final, ou seja:
\begin{equation}\label{eq:calculateImprovement}
Imp = \frac{f(\textrm{solução inicial})}{f(\textrm{solução final})}
\end{equation}

Desta forma quanto maior for o valor da melhoria ($Imp$) mais o método melhorou o valor da solução inicial.

\section{Instâncias} \label{sec:instancias}

Todas as instâncias usadas nos testes computacionais e cujas configurações de lançamento foram descritas na Tabela~\ref{tab:neighborhoodsLaunchConfigurarion} são as mesmas usadas em~\cite{wamca2016}.
Para o RVND foi feita uma implementação do algoritmo clássico (Algoritmo~\ref{alg:rvnd}) e também a implementação dataflow mencionada na Figura~\ref{fig:rvndGraph} fazendo uso de uma máquina.
Para o caso do DVND foi utilizada a implementação clássica (Algoritmo~\ref{alg:dvnd}) e a implementação dataflow proposta (Figura~\ref{fig:dvndGraph}), os resultados foram obtidos com diferentes números de máquinas e os mesmos são indicados conforme o caso.

\section{Implementação e ambiente computacional}\label{sec:amb}

A implementação para cada algoritmo proposto no Capítulo~\ref{cap:metodologia} utiliza a linguagem de programação \textit{C++11} em conjunto com a API CUDA\texttrademark, para a implementação dos grafos e do ambiente dataflow foi utilizada a biblioteca Sucuri~\cite{sucuri-original}\footnote{Disponível em \url{https://github.com/tiagoaoa/Sucuri}} implementada em Python, para a integração entre o dataflow e o código CUDA foi utilizada a biblioteca SimplePyCuda~\cite{simple-pycuda}\footnote{Disponível em \url{https://github.com/igormcoelho/simple-pycuda}}. As implementações com múltiplas threads usaram a biblioteca OpenMP.
%As implementações foram compiladas através do \textit{GCC} \textit{(GNU Compiler Collection)}\footnote{O GCC está disponível no seguinte sítio eletrônico: \url{https://gcc.gnu.org/}.} com a \textit{flag} de otimização $-O3$.
O ambiente computacional utilizado em todos os testes neste trabalho consiste de 4 máquinas com a seguinte configuração:

\begin{itemize}
    \item Processador Intel\textregistered Core\texttrademark i7-4820K 3.7 GHz (4 núcleos);
    \item 8 GB de memória RAM;
    \item Sistema Operacional Ubuntu 14.10 (x64);
    \item NVIDIA GeForce GTX 780 com 2304 CUDA cores.
\end{itemize}

\input{resultados/comandosFiguras.tex}

\input{resultados/sog_mog/index.tex}

\input{resultados/rvnd/index.tex}

\input{resultados/dvnd/index.tex}

\input{resultados/gdvnd/index.tex}



\chapter{Resultados} \label{cap:resultados}

Este capítulo exibe os resultados computacionais dos algoritmos propostos no Capítulo~\ref{cap:metodologia} para o caso do PML, para cada instância foi gerado um conjunto com 100 soluções iniciais aleatórias que foram submetidas aos métodos para comparação dos resultados.

Quando há referência à melhoria na solução (\textit{Imp}), esta melhoria pode ser calculada pelo quociente do valor da solução inicial pela solução final, ou seja:
\begin{equation}\label{eq:calculateImprovement}
Imp = \frac{f(\textrm{solução inicial})}{f(\textrm{solução final})}
\end{equation}

Desta forma quanto maior for o valor da melhoria ($Imp$) mais o método melhorou o valor da solução inicial.

\section{Instâncias} \label{sec:instancias}

Todas as instâncias usadas nos testes computacionais e cujas configurações de lançamento foram descritas na Tabela~\ref{tab:neighborhoodsLaunchConfigurarion} são as mesmas usadas em~\cite{wamca2016}.
Para o RVND foi feita uma implementação do algoritmo clássico (Algoritmo~\ref{alg:rvnd}) e também a implementação dataflow mencionada na Figura~\ref{fig:rvndGraph} fazendo uso de uma máquina.
Para o caso do DVND foi utilizada a implementação clássica (Algoritmo~\ref{alg:dvnd}) e a implementação dataflow proposta (Figura~\ref{fig:dvndGraph}), os resultados foram obtidos com diferentes números de máquinas e os mesmos são indicados conforme o caso.

\section{Implementação e ambiente computacional}\label{sec:amb}

A implementação para cada algoritmo proposto no Capítulo~\ref{cap:metodologia} utiliza a linguagem de programação \textit{C++11} em conjunto com a API CUDA\texttrademark, para a implementação dos grafos e do ambiente dataflow foi utilizada a biblioteca Sucuri~\cite{sucuri-original}\footnote{Disponível em \url{https://github.com/tiagoaoa/Sucuri}} implementada em Python, para a integração entre o dataflow e o código CUDA foi utilizada a biblioteca SimplePyCuda~\cite{simple-pycuda}\footnote{Disponível em \url{https://github.com/igormcoelho/simple-pycuda}}. As implementações com múltiplas threads usaram a biblioteca OpenMP.
%As implementações foram compiladas através do \textit{GCC} \textit{(GNU Compiler Collection)}\footnote{O GCC está disponível no seguinte sítio eletrônico: \url{https://gcc.gnu.org/}.} com a \textit{flag} de otimização $-O3$.
O ambiente computacional utilizado em todos os testes neste trabalho consiste de 4 máquinas com a seguinte configuração:

\begin{itemize}
    \item Processador Intel\textregistered Core\texttrademark i7-4820K 3.7 GHz (4 núcleos);
    \item 8 GB de memória RAM;
    \item Sistema Operacional Ubuntu 14.10 (x64);
    \item NVIDIA GeForce GTX 780 com 2304 CUDA cores.
\end{itemize}

\newcommand{\figureDvndOrRvndDcDd}[7]{
% #1 {box, scatter}, #2 {count, imp, time}, #3 {instance number}, #4 {Tempo, Melhoria}, #5 tamanho instância, #6 {DVND, RVND}, #7 {dvnd, rvnd}
\begin{figure}%
    \centering
    \includegraphics[scale=0.9]{figuras/#7/dc_dd/#1/#7_#1100sol_#2_in#3.png}
    \caption{#4 do #6 para a instância #3 de tamanho #5. $m$ indica o número de máquinas, \textit{DC} refere-se ao #6 clássico e \textit{DD} ao #6 implementado em dataflow.}%
    \label{fig:#2_#7DcDd_in#3}%
\end{figure}
}

\newcommand{\figureDvndDcDd}[5]{
% #1 {box, scatter}, #2 {count, imp, time}, #3 {instance number}, #4 {Tempo, Melhoria}, #5 tamanho instância
    \figureDvndOrRvndDcDd{#1}{#2}{#3}{#4}{#5}{DVND}{dvnd}
}

\newcommand{\figureRvndDcDd}[5]{
% #1 {box, scatter}, #2 {count, imp, time}, #3 {instance number}, #4 {Tempo, Melhoria}, #5 tamanho instância
    \figureDvndOrRvndDcDd{#1}{#2}{#3}{#4}{#5}{RVND}{rvnd}
}

\newcommand{\tabelaEstatisticasGeral}[6]{
% #1 Descrição, #2 label, #3 {dvnd, rvnd}, #4 {DVND, RVND}, #6 DD/DC, #6 Conteúdo
\begin{table}[ht]
    \centering
    \begin{tabular}{c|ccc|cc|ccc|cc|c}
        \hline \hline
        \# & Tipo & $m$ & $n$ & $min$ & $max$ & 1Q & 2Q & 3Q & $\overline{x}$ & $\sigma$ & $p-valor$ \\ \hline
        #6
    \end{tabular}
    \caption{#1 #4 #5
        Instância (\#), tipo de implementação (Tipo), número de máquinas ($m$), tamanho da instância ($n$), valor mínimo ($min$), máximo ($max$), primeiro quartil (1Q), mediana (2Q), terceiro quartil (3Q), média ($\overline{x}$), desvio padrão ($\sigma$) e p-valor para o teste de Wilcox entre as versões (valores em negrito quando $p-valor > 0.05$).
    }
    \label{tab:#3DcDd#2}
\end{table}
}

\newcommand{\tabelaEstatisticas}[5]{
    \tabelaEstatisticasGeral{#1}{#2}{#3}{#4}{na implementação clássica (DC) e a proposta de implementação usando dataflow (DD).}{#5}
}

\newcommand{\figureDvndSogMog}[7]{
% #1 {box, scatter}, #2 {count, imp, time}, #3 {instance number}, #4 {Tempo, Melhoria}, #5 tamanho instância, #6 {DVND, RVND}, #7 {dvnd, rvnd}
\begin{figure}%
    \centering
    \includegraphics[scale=0.9]{figuras/#7/sog_mog/#1/#7_#1100sol_#2_in#3.png}
    \caption{#4 do #6 para a instância #3 de tamanho #5. \textit{SOG} refere-se a uma porta de saída e \textit{MOG} a múltiplas portas de saída.}%
    \label{fig:#2_#7SogMog_in#3}%
\end{figure}
}

\newcommand{\figureDvndGdvnd}[9]{
% #1 {box, scatter}, #2 {count, imp, time}, #3 {instance number}, #4 {Tempo, Melhoria}, #5 tamanho instância, #6 {DVND, RVND}, #7 {dvnd, rvnd}, #8 {man_time, full_time}, #9 {man, dvnd} #10 descricao
\begin{figure}%
    \centering
    \includegraphics[scale=0.9]{figuras/#7/#8/#1/#9_#1100sol_#2_in#3.png}
    \caption{#4 do #6 para a instância #3 de tamanho #5. \textit{DVND} refere-se ao tempo gasto pelo algoritmo de mesmo nome, para \textit{GDVND} é análogo ao anterior, no caso do \textit{GDVND-MAN} este se refere ao tempo do GDVND subtraido do tempo para gerenciar os movimentos.}%
    \label{fig:#2_#7_#8_in#3}%
\end{figure}
}

\newcommand{\figureDvndGdvndTime}[8]{
    \figureDvndGdvnd{#1}{#2}{#3}{#4}{#5}{#6}{#7}{man_time}{man}
}

\newcommand{\figureGdvndDvndRvnd}[9]{
% #1 {box, scatter}, #2 {count, imp, time}, #3 {instance number}, #4 {Tempo, Melhoria}, #5 tamanho instância, #6 {DVND, RVND}, #7 {dvnd, rvnd}, #8 {man_time, full_time}, #9 {man, dvnd} #10 descricao
\begin{figure}%
    \centering
    \includegraphics[scale=0.9]{figuras/#7/#8/#1/#9_#1100sol_#2_in#3.png}
    \caption{#4 do #6 para a instância #3 de tamanho #5. \textit{DVND}, \textit{GDVND} e \textit{RVND} referem-se ao tempo gasto pelos algoritmos de mesmo nome.}%
    \label{fig:#2_#7_#8_in#3}%
\end{figure}
}

% \subfloat[$m=#1$]{{ %scale=0.225
%         \includegraphics[scale=0.425]{figuras/dvnd/n#1/time#2.png}
%         \label{fig:timeDvndRvnd_n#1in#2}
%     }}%
% #1 {dvnd, rvnd, gdvnd}, #2 {sog_mog, dc_dd}, #3 {time, imp}, #4 in, #5 tamanho, #6 {box, scatter}
\newcommand{\subFig}[6]{
    \subfloat[][Instância #4, $n=#5$]{
        \includegraphics[scale=0.425]{figuras/#1/#2/#6/#1_#6100sol_#3_in#4.png}
		\label{fig:#1_#2_#3_in#4}
    }
% 	\begin{subfigure}{0.45\textwidth} % dvnd_box100sol_imp_in0
% 		\includegraphics[scale=0.425]{figuras/#1/#2/#6/#1_#6100sol_#3_in#4.png}
% 		\caption{Instância #4, $n=#5$}
        % \label{fig:#1_#2_#3_in#4}
    % \end{subfigure}
}

\newcommand{\subFigBox}[5]{
	\subFig{#1}{#2}{#3}{#4}{#5}{box}
}

\newcommand{\subFigScatter}[5]{
	\subFig{#1}{#2}{#3}{#4}{#5}{scatter}
}

% #1 {dvnd, rvnd, gdvnd}, #2 {sog_mog, dc_dd}, #3 {time, imp}, #4 {box, scatter}, #5 {Tempo do DVND...}
\newcommand{\multiFigureInstanciasGeral}[5]{
	\begin{figure}[ht]
		\centering
		\subFig{#1}{#2}{#3}{0}{52}{#4}
		~
		\subFig{#1}{#2}{#3}{1}{100}{#4}
		
		\subFig{#1}{#2}{#3}{2}{226}{#4}
		~
		\subFig{#1}{#2}{#3}{3}{318}{#4}
		\caption{#5 Instâncias 0 a 3.}
		\label{fig:#1_#2_#3_in0_4}
	\end{figure}
	
	\begin{figure}[ht]
		\centering
		\subFig{#1}{#2}{#3}{4}{501}{#4}
		~
		\subFig{#1}{#2}{#3}{5}{657}{#4}
		
		\subFig{#1}{#2}{#3}{6}{783}{#4}
		~
		\subFig{#1}{#2}{#3}{7}{1001}{#4}
		\caption{#5 Instâncias 5 a 7.}
		\label{fig:#1_#2_#3_in5_7}
	\end{figure}
}

% #1 {dvnd, rvnd, gdvnd}, #2 {sog_mog, dc_dd}, #3 {time, imp}, #4 {Tempo do DVND...}
\newcommand{\multiFigureInstancias}[4]{
    \multiFigureInstanciasGeral{#1}{#2}{#3}{box}{#4}
}


\chapter{Resultados} \label{cap:resultados}

Este capítulo exibe os resultados computacionais dos algoritmos propostos no Capítulo~\ref{cap:metodologia} para o caso do PML, para cada instância foi gerado um conjunto com 100 soluções iniciais aleatórias que foram submetidas aos métodos para comparação dos resultados.

Quando há referência à melhoria na solução (\textit{Imp}), esta melhoria pode ser calculada pelo quociente do valor da solução inicial pela solução final, ou seja:
\begin{equation}\label{eq:calculateImprovement}
Imp = \frac{f(\textrm{solução inicial})}{f(\textrm{solução final})}
\end{equation}

Desta forma quanto maior for o valor da melhoria ($Imp$) mais o método melhorou o valor da solução inicial.

\section{Instâncias} \label{sec:instancias}

Todas as instâncias usadas nos testes computacionais e cujas configurações de lançamento foram descritas na Tabela~\ref{tab:neighborhoodsLaunchConfigurarion} são as mesmas usadas em~\cite{wamca2016}.
Para o RVND foi feita uma implementação do algoritmo clássico (Algoritmo~\ref{alg:rvnd}) e também a implementação dataflow mencionada na Figura~\ref{fig:rvndGraph} fazendo uso de uma máquina.
Para o caso do DVND foi utilizada a implementação clássica (Algoritmo~\ref{alg:dvnd}) e a implementação dataflow proposta (Figura~\ref{fig:dvndGraph}), os resultados foram obtidos com diferentes números de máquinas e os mesmos são indicados conforme o caso.

\section{Implementação e ambiente computacional}\label{sec:amb}

A implementação para cada algoritmo proposto no Capítulo~\ref{cap:metodologia} utiliza a linguagem de programação \textit{C++11} em conjunto com a API CUDA\texttrademark, para a implementação dos grafos e do ambiente dataflow foi utilizada a biblioteca Sucuri~\cite{sucuri-original}\footnote{Disponível em \url{https://github.com/tiagoaoa/Sucuri}} implementada em Python, para a integração entre o dataflow e o código CUDA foi utilizada a biblioteca SimplePyCuda~\cite{simple-pycuda}\footnote{Disponível em \url{https://github.com/igormcoelho/simple-pycuda}}. As implementações com múltiplas threads usaram a biblioteca OpenMP.
%As implementações foram compiladas através do \textit{GCC} \textit{(GNU Compiler Collection)}\footnote{O GCC está disponível no seguinte sítio eletrônico: \url{https://gcc.gnu.org/}.} com a \textit{flag} de otimização $-O3$.
O ambiente computacional utilizado em todos os testes neste trabalho consiste de 4 máquinas com a seguinte configuração:

\begin{itemize}
    \item Processador Intel\textregistered Core\texttrademark i7-4820K 3.7 GHz (4 núcleos);
    \item 8 GB de memória RAM;
    \item Sistema Operacional Ubuntu 14.10 (x64);
    \item NVIDIA GeForce GTX 780 com 2304 CUDA cores.
\end{itemize}

\input{resultados/comandosFiguras.tex}

\input{resultados/sog_mog/index.tex}

\input{resultados/rvnd/index.tex}

\input{resultados/dvnd/index.tex}

\input{resultados/gdvnd/index.tex}


\chapter{Resultados} \label{cap:resultados}

Este capítulo exibe os resultados computacionais dos algoritmos propostos no Capítulo~\ref{cap:metodologia} para o caso do PML, para cada instância foi gerado um conjunto com 100 soluções iniciais aleatórias que foram submetidas aos métodos para comparação dos resultados.

Quando há referência à melhoria na solução (\textit{Imp}), esta melhoria pode ser calculada pelo quociente do valor da solução inicial pela solução final, ou seja:
\begin{equation}\label{eq:calculateImprovement}
Imp = \frac{f(\textrm{solução inicial})}{f(\textrm{solução final})}
\end{equation}

Desta forma quanto maior for o valor da melhoria ($Imp$) mais o método melhorou o valor da solução inicial.

\section{Instâncias} \label{sec:instancias}

Todas as instâncias usadas nos testes computacionais e cujas configurações de lançamento foram descritas na Tabela~\ref{tab:neighborhoodsLaunchConfigurarion} são as mesmas usadas em~\cite{wamca2016}.
Para o RVND foi feita uma implementação do algoritmo clássico (Algoritmo~\ref{alg:rvnd}) e também a implementação dataflow mencionada na Figura~\ref{fig:rvndGraph} fazendo uso de uma máquina.
Para o caso do DVND foi utilizada a implementação clássica (Algoritmo~\ref{alg:dvnd}) e a implementação dataflow proposta (Figura~\ref{fig:dvndGraph}), os resultados foram obtidos com diferentes números de máquinas e os mesmos são indicados conforme o caso.

\section{Implementação e ambiente computacional}\label{sec:amb}

A implementação para cada algoritmo proposto no Capítulo~\ref{cap:metodologia} utiliza a linguagem de programação \textit{C++11} em conjunto com a API CUDA\texttrademark, para a implementação dos grafos e do ambiente dataflow foi utilizada a biblioteca Sucuri~\cite{sucuri-original}\footnote{Disponível em \url{https://github.com/tiagoaoa/Sucuri}} implementada em Python, para a integração entre o dataflow e o código CUDA foi utilizada a biblioteca SimplePyCuda~\cite{simple-pycuda}\footnote{Disponível em \url{https://github.com/igormcoelho/simple-pycuda}}. As implementações com múltiplas threads usaram a biblioteca OpenMP.
%As implementações foram compiladas através do \textit{GCC} \textit{(GNU Compiler Collection)}\footnote{O GCC está disponível no seguinte sítio eletrônico: \url{https://gcc.gnu.org/}.} com a \textit{flag} de otimização $-O3$.
O ambiente computacional utilizado em todos os testes neste trabalho consiste de 4 máquinas com a seguinte configuração:

\begin{itemize}
    \item Processador Intel\textregistered Core\texttrademark i7-4820K 3.7 GHz (4 núcleos);
    \item 8 GB de memória RAM;
    \item Sistema Operacional Ubuntu 14.10 (x64);
    \item NVIDIA GeForce GTX 780 com 2304 CUDA cores.
\end{itemize}

\input{resultados/comandosFiguras.tex}

\input{resultados/sog_mog/index.tex}

\input{resultados/rvnd/index.tex}

\input{resultados/dvnd/index.tex}

\input{resultados/gdvnd/index.tex}


\chapter{Resultados} \label{cap:resultados}

Este capítulo exibe os resultados computacionais dos algoritmos propostos no Capítulo~\ref{cap:metodologia} para o caso do PML, para cada instância foi gerado um conjunto com 100 soluções iniciais aleatórias que foram submetidas aos métodos para comparação dos resultados.

Quando há referência à melhoria na solução (\textit{Imp}), esta melhoria pode ser calculada pelo quociente do valor da solução inicial pela solução final, ou seja:
\begin{equation}\label{eq:calculateImprovement}
Imp = \frac{f(\textrm{solução inicial})}{f(\textrm{solução final})}
\end{equation}

Desta forma quanto maior for o valor da melhoria ($Imp$) mais o método melhorou o valor da solução inicial.

\section{Instâncias} \label{sec:instancias}

Todas as instâncias usadas nos testes computacionais e cujas configurações de lançamento foram descritas na Tabela~\ref{tab:neighborhoodsLaunchConfigurarion} são as mesmas usadas em~\cite{wamca2016}.
Para o RVND foi feita uma implementação do algoritmo clássico (Algoritmo~\ref{alg:rvnd}) e também a implementação dataflow mencionada na Figura~\ref{fig:rvndGraph} fazendo uso de uma máquina.
Para o caso do DVND foi utilizada a implementação clássica (Algoritmo~\ref{alg:dvnd}) e a implementação dataflow proposta (Figura~\ref{fig:dvndGraph}), os resultados foram obtidos com diferentes números de máquinas e os mesmos são indicados conforme o caso.

\section{Implementação e ambiente computacional}\label{sec:amb}

A implementação para cada algoritmo proposto no Capítulo~\ref{cap:metodologia} utiliza a linguagem de programação \textit{C++11} em conjunto com a API CUDA\texttrademark, para a implementação dos grafos e do ambiente dataflow foi utilizada a biblioteca Sucuri~\cite{sucuri-original}\footnote{Disponível em \url{https://github.com/tiagoaoa/Sucuri}} implementada em Python, para a integração entre o dataflow e o código CUDA foi utilizada a biblioteca SimplePyCuda~\cite{simple-pycuda}\footnote{Disponível em \url{https://github.com/igormcoelho/simple-pycuda}}. As implementações com múltiplas threads usaram a biblioteca OpenMP.
%As implementações foram compiladas através do \textit{GCC} \textit{(GNU Compiler Collection)}\footnote{O GCC está disponível no seguinte sítio eletrônico: \url{https://gcc.gnu.org/}.} com a \textit{flag} de otimização $-O3$.
O ambiente computacional utilizado em todos os testes neste trabalho consiste de 4 máquinas com a seguinte configuração:

\begin{itemize}
    \item Processador Intel\textregistered Core\texttrademark i7-4820K 3.7 GHz (4 núcleos);
    \item 8 GB de memória RAM;
    \item Sistema Operacional Ubuntu 14.10 (x64);
    \item NVIDIA GeForce GTX 780 com 2304 CUDA cores.
\end{itemize}

\input{resultados/comandosFiguras.tex}

\input{resultados/sog_mog/index.tex}

\input{resultados/rvnd/index.tex}

\input{resultados/dvnd/index.tex}

\input{resultados/gdvnd/index.tex}


\chapter{Resultados} \label{cap:resultados}

Este capítulo exibe os resultados computacionais dos algoritmos propostos no Capítulo~\ref{cap:metodologia} para o caso do PML, para cada instância foi gerado um conjunto com 100 soluções iniciais aleatórias que foram submetidas aos métodos para comparação dos resultados.

Quando há referência à melhoria na solução (\textit{Imp}), esta melhoria pode ser calculada pelo quociente do valor da solução inicial pela solução final, ou seja:
\begin{equation}\label{eq:calculateImprovement}
Imp = \frac{f(\textrm{solução inicial})}{f(\textrm{solução final})}
\end{equation}

Desta forma quanto maior for o valor da melhoria ($Imp$) mais o método melhorou o valor da solução inicial.

\section{Instâncias} \label{sec:instancias}

Todas as instâncias usadas nos testes computacionais e cujas configurações de lançamento foram descritas na Tabela~\ref{tab:neighborhoodsLaunchConfigurarion} são as mesmas usadas em~\cite{wamca2016}.
Para o RVND foi feita uma implementação do algoritmo clássico (Algoritmo~\ref{alg:rvnd}) e também a implementação dataflow mencionada na Figura~\ref{fig:rvndGraph} fazendo uso de uma máquina.
Para o caso do DVND foi utilizada a implementação clássica (Algoritmo~\ref{alg:dvnd}) e a implementação dataflow proposta (Figura~\ref{fig:dvndGraph}), os resultados foram obtidos com diferentes números de máquinas e os mesmos são indicados conforme o caso.

\section{Implementação e ambiente computacional}\label{sec:amb}

A implementação para cada algoritmo proposto no Capítulo~\ref{cap:metodologia} utiliza a linguagem de programação \textit{C++11} em conjunto com a API CUDA\texttrademark, para a implementação dos grafos e do ambiente dataflow foi utilizada a biblioteca Sucuri~\cite{sucuri-original}\footnote{Disponível em \url{https://github.com/tiagoaoa/Sucuri}} implementada em Python, para a integração entre o dataflow e o código CUDA foi utilizada a biblioteca SimplePyCuda~\cite{simple-pycuda}\footnote{Disponível em \url{https://github.com/igormcoelho/simple-pycuda}}. As implementações com múltiplas threads usaram a biblioteca OpenMP.
%As implementações foram compiladas através do \textit{GCC} \textit{(GNU Compiler Collection)}\footnote{O GCC está disponível no seguinte sítio eletrônico: \url{https://gcc.gnu.org/}.} com a \textit{flag} de otimização $-O3$.
O ambiente computacional utilizado em todos os testes neste trabalho consiste de 4 máquinas com a seguinte configuração:

\begin{itemize}
    \item Processador Intel\textregistered Core\texttrademark i7-4820K 3.7 GHz (4 núcleos);
    \item 8 GB de memória RAM;
    \item Sistema Operacional Ubuntu 14.10 (x64);
    \item NVIDIA GeForce GTX 780 com 2304 CUDA cores.
\end{itemize}

\input{resultados/comandosFiguras.tex}

\input{resultados/sog_mog/index.tex}

\input{resultados/rvnd/index.tex}

\input{resultados/dvnd/index.tex}

\input{resultados/gdvnd/index.tex}



\chapter{Resultados} \label{cap:resultados}

Este capítulo exibe os resultados computacionais dos algoritmos propostos no Capítulo~\ref{cap:metodologia} para o caso do PML, para cada instância foi gerado um conjunto com 100 soluções iniciais aleatórias que foram submetidas aos métodos para comparação dos resultados.

Quando há referência à melhoria na solução (\textit{Imp}), esta melhoria pode ser calculada pelo quociente do valor da solução inicial pela solução final, ou seja:
\begin{equation}\label{eq:calculateImprovement}
Imp = \frac{f(\textrm{solução inicial})}{f(\textrm{solução final})}
\end{equation}

Desta forma quanto maior for o valor da melhoria ($Imp$) mais o método melhorou o valor da solução inicial.

\section{Instâncias} \label{sec:instancias}

Todas as instâncias usadas nos testes computacionais e cujas configurações de lançamento foram descritas na Tabela~\ref{tab:neighborhoodsLaunchConfigurarion} são as mesmas usadas em~\cite{wamca2016}.
Para o RVND foi feita uma implementação do algoritmo clássico (Algoritmo~\ref{alg:rvnd}) e também a implementação dataflow mencionada na Figura~\ref{fig:rvndGraph} fazendo uso de uma máquina.
Para o caso do DVND foi utilizada a implementação clássica (Algoritmo~\ref{alg:dvnd}) e a implementação dataflow proposta (Figura~\ref{fig:dvndGraph}), os resultados foram obtidos com diferentes números de máquinas e os mesmos são indicados conforme o caso.

\section{Implementação e ambiente computacional}\label{sec:amb}

A implementação para cada algoritmo proposto no Capítulo~\ref{cap:metodologia} utiliza a linguagem de programação \textit{C++11} em conjunto com a API CUDA\texttrademark, para a implementação dos grafos e do ambiente dataflow foi utilizada a biblioteca Sucuri~\cite{sucuri-original}\footnote{Disponível em \url{https://github.com/tiagoaoa/Sucuri}} implementada em Python, para a integração entre o dataflow e o código CUDA foi utilizada a biblioteca SimplePyCuda~\cite{simple-pycuda}\footnote{Disponível em \url{https://github.com/igormcoelho/simple-pycuda}}. As implementações com múltiplas threads usaram a biblioteca OpenMP.
%As implementações foram compiladas através do \textit{GCC} \textit{(GNU Compiler Collection)}\footnote{O GCC está disponível no seguinte sítio eletrônico: \url{https://gcc.gnu.org/}.} com a \textit{flag} de otimização $-O3$.
O ambiente computacional utilizado em todos os testes neste trabalho consiste de 4 máquinas com a seguinte configuração:

\begin{itemize}
    \item Processador Intel\textregistered Core\texttrademark i7-4820K 3.7 GHz (4 núcleos);
    \item 8 GB de memória RAM;
    \item Sistema Operacional Ubuntu 14.10 (x64);
    \item NVIDIA GeForce GTX 780 com 2304 CUDA cores.
\end{itemize}

\newcommand{\figureDvndOrRvndDcDd}[7]{
% #1 {box, scatter}, #2 {count, imp, time}, #3 {instance number}, #4 {Tempo, Melhoria}, #5 tamanho instância, #6 {DVND, RVND}, #7 {dvnd, rvnd}
\begin{figure}%
    \centering
    \includegraphics[scale=0.9]{figuras/#7/dc_dd/#1/#7_#1100sol_#2_in#3.png}
    \caption{#4 do #6 para a instância #3 de tamanho #5. $m$ indica o número de máquinas, \textit{DC} refere-se ao #6 clássico e \textit{DD} ao #6 implementado em dataflow.}%
    \label{fig:#2_#7DcDd_in#3}%
\end{figure}
}

\newcommand{\figureDvndDcDd}[5]{
% #1 {box, scatter}, #2 {count, imp, time}, #3 {instance number}, #4 {Tempo, Melhoria}, #5 tamanho instância
    \figureDvndOrRvndDcDd{#1}{#2}{#3}{#4}{#5}{DVND}{dvnd}
}

\newcommand{\figureRvndDcDd}[5]{
% #1 {box, scatter}, #2 {count, imp, time}, #3 {instance number}, #4 {Tempo, Melhoria}, #5 tamanho instância
    \figureDvndOrRvndDcDd{#1}{#2}{#3}{#4}{#5}{RVND}{rvnd}
}

\newcommand{\tabelaEstatisticasGeral}[6]{
% #1 Descrição, #2 label, #3 {dvnd, rvnd}, #4 {DVND, RVND}, #6 DD/DC, #6 Conteúdo
\begin{table}[ht]
    \centering
    \begin{tabular}{c|ccc|cc|ccc|cc|c}
        \hline \hline
        \# & Tipo & $m$ & $n$ & $min$ & $max$ & 1Q & 2Q & 3Q & $\overline{x}$ & $\sigma$ & $p-valor$ \\ \hline
        #6
    \end{tabular}
    \caption{#1 #4 #5
        Instância (\#), tipo de implementação (Tipo), número de máquinas ($m$), tamanho da instância ($n$), valor mínimo ($min$), máximo ($max$), primeiro quartil (1Q), mediana (2Q), terceiro quartil (3Q), média ($\overline{x}$), desvio padrão ($\sigma$) e p-valor para o teste de Wilcox entre as versões (valores em negrito quando $p-valor > 0.05$).
    }
    \label{tab:#3DcDd#2}
\end{table}
}

\newcommand{\tabelaEstatisticas}[5]{
    \tabelaEstatisticasGeral{#1}{#2}{#3}{#4}{na implementação clássica (DC) e a proposta de implementação usando dataflow (DD).}{#5}
}

\newcommand{\figureDvndSogMog}[7]{
% #1 {box, scatter}, #2 {count, imp, time}, #3 {instance number}, #4 {Tempo, Melhoria}, #5 tamanho instância, #6 {DVND, RVND}, #7 {dvnd, rvnd}
\begin{figure}%
    \centering
    \includegraphics[scale=0.9]{figuras/#7/sog_mog/#1/#7_#1100sol_#2_in#3.png}
    \caption{#4 do #6 para a instância #3 de tamanho #5. \textit{SOG} refere-se a uma porta de saída e \textit{MOG} a múltiplas portas de saída.}%
    \label{fig:#2_#7SogMog_in#3}%
\end{figure}
}

\newcommand{\figureDvndGdvnd}[9]{
% #1 {box, scatter}, #2 {count, imp, time}, #3 {instance number}, #4 {Tempo, Melhoria}, #5 tamanho instância, #6 {DVND, RVND}, #7 {dvnd, rvnd}, #8 {man_time, full_time}, #9 {man, dvnd} #10 descricao
\begin{figure}%
    \centering
    \includegraphics[scale=0.9]{figuras/#7/#8/#1/#9_#1100sol_#2_in#3.png}
    \caption{#4 do #6 para a instância #3 de tamanho #5. \textit{DVND} refere-se ao tempo gasto pelo algoritmo de mesmo nome, para \textit{GDVND} é análogo ao anterior, no caso do \textit{GDVND-MAN} este se refere ao tempo do GDVND subtraido do tempo para gerenciar os movimentos.}%
    \label{fig:#2_#7_#8_in#3}%
\end{figure}
}

\newcommand{\figureDvndGdvndTime}[8]{
    \figureDvndGdvnd{#1}{#2}{#3}{#4}{#5}{#6}{#7}{man_time}{man}
}

\newcommand{\figureGdvndDvndRvnd}[9]{
% #1 {box, scatter}, #2 {count, imp, time}, #3 {instance number}, #4 {Tempo, Melhoria}, #5 tamanho instância, #6 {DVND, RVND}, #7 {dvnd, rvnd}, #8 {man_time, full_time}, #9 {man, dvnd} #10 descricao
\begin{figure}%
    \centering
    \includegraphics[scale=0.9]{figuras/#7/#8/#1/#9_#1100sol_#2_in#3.png}
    \caption{#4 do #6 para a instância #3 de tamanho #5. \textit{DVND}, \textit{GDVND} e \textit{RVND} referem-se ao tempo gasto pelos algoritmos de mesmo nome.}%
    \label{fig:#2_#7_#8_in#3}%
\end{figure}
}

% \subfloat[$m=#1$]{{ %scale=0.225
%         \includegraphics[scale=0.425]{figuras/dvnd/n#1/time#2.png}
%         \label{fig:timeDvndRvnd_n#1in#2}
%     }}%
% #1 {dvnd, rvnd, gdvnd}, #2 {sog_mog, dc_dd}, #3 {time, imp}, #4 in, #5 tamanho, #6 {box, scatter}
\newcommand{\subFig}[6]{
    \subfloat[][Instância #4, $n=#5$]{
        \includegraphics[scale=0.425]{figuras/#1/#2/#6/#1_#6100sol_#3_in#4.png}
		\label{fig:#1_#2_#3_in#4}
    }
% 	\begin{subfigure}{0.45\textwidth} % dvnd_box100sol_imp_in0
% 		\includegraphics[scale=0.425]{figuras/#1/#2/#6/#1_#6100sol_#3_in#4.png}
% 		\caption{Instância #4, $n=#5$}
        % \label{fig:#1_#2_#3_in#4}
    % \end{subfigure}
}

\newcommand{\subFigBox}[5]{
	\subFig{#1}{#2}{#3}{#4}{#5}{box}
}

\newcommand{\subFigScatter}[5]{
	\subFig{#1}{#2}{#3}{#4}{#5}{scatter}
}

% #1 {dvnd, rvnd, gdvnd}, #2 {sog_mog, dc_dd}, #3 {time, imp}, #4 {box, scatter}, #5 {Tempo do DVND...}
\newcommand{\multiFigureInstanciasGeral}[5]{
	\begin{figure}[ht]
		\centering
		\subFig{#1}{#2}{#3}{0}{52}{#4}
		~
		\subFig{#1}{#2}{#3}{1}{100}{#4}
		
		\subFig{#1}{#2}{#3}{2}{226}{#4}
		~
		\subFig{#1}{#2}{#3}{3}{318}{#4}
		\caption{#5 Instâncias 0 a 3.}
		\label{fig:#1_#2_#3_in0_4}
	\end{figure}
	
	\begin{figure}[ht]
		\centering
		\subFig{#1}{#2}{#3}{4}{501}{#4}
		~
		\subFig{#1}{#2}{#3}{5}{657}{#4}
		
		\subFig{#1}{#2}{#3}{6}{783}{#4}
		~
		\subFig{#1}{#2}{#3}{7}{1001}{#4}
		\caption{#5 Instâncias 5 a 7.}
		\label{fig:#1_#2_#3_in5_7}
	\end{figure}
}

% #1 {dvnd, rvnd, gdvnd}, #2 {sog_mog, dc_dd}, #3 {time, imp}, #4 {Tempo do DVND...}
\newcommand{\multiFigureInstancias}[4]{
    \multiFigureInstanciasGeral{#1}{#2}{#3}{box}{#4}
}


\chapter{Resultados} \label{cap:resultados}

Este capítulo exibe os resultados computacionais dos algoritmos propostos no Capítulo~\ref{cap:metodologia} para o caso do PML, para cada instância foi gerado um conjunto com 100 soluções iniciais aleatórias que foram submetidas aos métodos para comparação dos resultados.

Quando há referência à melhoria na solução (\textit{Imp}), esta melhoria pode ser calculada pelo quociente do valor da solução inicial pela solução final, ou seja:
\begin{equation}\label{eq:calculateImprovement}
Imp = \frac{f(\textrm{solução inicial})}{f(\textrm{solução final})}
\end{equation}

Desta forma quanto maior for o valor da melhoria ($Imp$) mais o método melhorou o valor da solução inicial.

\section{Instâncias} \label{sec:instancias}

Todas as instâncias usadas nos testes computacionais e cujas configurações de lançamento foram descritas na Tabela~\ref{tab:neighborhoodsLaunchConfigurarion} são as mesmas usadas em~\cite{wamca2016}.
Para o RVND foi feita uma implementação do algoritmo clássico (Algoritmo~\ref{alg:rvnd}) e também a implementação dataflow mencionada na Figura~\ref{fig:rvndGraph} fazendo uso de uma máquina.
Para o caso do DVND foi utilizada a implementação clássica (Algoritmo~\ref{alg:dvnd}) e a implementação dataflow proposta (Figura~\ref{fig:dvndGraph}), os resultados foram obtidos com diferentes números de máquinas e os mesmos são indicados conforme o caso.

\section{Implementação e ambiente computacional}\label{sec:amb}

A implementação para cada algoritmo proposto no Capítulo~\ref{cap:metodologia} utiliza a linguagem de programação \textit{C++11} em conjunto com a API CUDA\texttrademark, para a implementação dos grafos e do ambiente dataflow foi utilizada a biblioteca Sucuri~\cite{sucuri-original}\footnote{Disponível em \url{https://github.com/tiagoaoa/Sucuri}} implementada em Python, para a integração entre o dataflow e o código CUDA foi utilizada a biblioteca SimplePyCuda~\cite{simple-pycuda}\footnote{Disponível em \url{https://github.com/igormcoelho/simple-pycuda}}. As implementações com múltiplas threads usaram a biblioteca OpenMP.
%As implementações foram compiladas através do \textit{GCC} \textit{(GNU Compiler Collection)}\footnote{O GCC está disponível no seguinte sítio eletrônico: \url{https://gcc.gnu.org/}.} com a \textit{flag} de otimização $-O3$.
O ambiente computacional utilizado em todos os testes neste trabalho consiste de 4 máquinas com a seguinte configuração:

\begin{itemize}
    \item Processador Intel\textregistered Core\texttrademark i7-4820K 3.7 GHz (4 núcleos);
    \item 8 GB de memória RAM;
    \item Sistema Operacional Ubuntu 14.10 (x64);
    \item NVIDIA GeForce GTX 780 com 2304 CUDA cores.
\end{itemize}

\input{resultados/comandosFiguras.tex}

\input{resultados/sog_mog/index.tex}

\input{resultados/rvnd/index.tex}

\input{resultados/dvnd/index.tex}

\input{resultados/gdvnd/index.tex}


\chapter{Resultados} \label{cap:resultados}

Este capítulo exibe os resultados computacionais dos algoritmos propostos no Capítulo~\ref{cap:metodologia} para o caso do PML, para cada instância foi gerado um conjunto com 100 soluções iniciais aleatórias que foram submetidas aos métodos para comparação dos resultados.

Quando há referência à melhoria na solução (\textit{Imp}), esta melhoria pode ser calculada pelo quociente do valor da solução inicial pela solução final, ou seja:
\begin{equation}\label{eq:calculateImprovement}
Imp = \frac{f(\textrm{solução inicial})}{f(\textrm{solução final})}
\end{equation}

Desta forma quanto maior for o valor da melhoria ($Imp$) mais o método melhorou o valor da solução inicial.

\section{Instâncias} \label{sec:instancias}

Todas as instâncias usadas nos testes computacionais e cujas configurações de lançamento foram descritas na Tabela~\ref{tab:neighborhoodsLaunchConfigurarion} são as mesmas usadas em~\cite{wamca2016}.
Para o RVND foi feita uma implementação do algoritmo clássico (Algoritmo~\ref{alg:rvnd}) e também a implementação dataflow mencionada na Figura~\ref{fig:rvndGraph} fazendo uso de uma máquina.
Para o caso do DVND foi utilizada a implementação clássica (Algoritmo~\ref{alg:dvnd}) e a implementação dataflow proposta (Figura~\ref{fig:dvndGraph}), os resultados foram obtidos com diferentes números de máquinas e os mesmos são indicados conforme o caso.

\section{Implementação e ambiente computacional}\label{sec:amb}

A implementação para cada algoritmo proposto no Capítulo~\ref{cap:metodologia} utiliza a linguagem de programação \textit{C++11} em conjunto com a API CUDA\texttrademark, para a implementação dos grafos e do ambiente dataflow foi utilizada a biblioteca Sucuri~\cite{sucuri-original}\footnote{Disponível em \url{https://github.com/tiagoaoa/Sucuri}} implementada em Python, para a integração entre o dataflow e o código CUDA foi utilizada a biblioteca SimplePyCuda~\cite{simple-pycuda}\footnote{Disponível em \url{https://github.com/igormcoelho/simple-pycuda}}. As implementações com múltiplas threads usaram a biblioteca OpenMP.
%As implementações foram compiladas através do \textit{GCC} \textit{(GNU Compiler Collection)}\footnote{O GCC está disponível no seguinte sítio eletrônico: \url{https://gcc.gnu.org/}.} com a \textit{flag} de otimização $-O3$.
O ambiente computacional utilizado em todos os testes neste trabalho consiste de 4 máquinas com a seguinte configuração:

\begin{itemize}
    \item Processador Intel\textregistered Core\texttrademark i7-4820K 3.7 GHz (4 núcleos);
    \item 8 GB de memória RAM;
    \item Sistema Operacional Ubuntu 14.10 (x64);
    \item NVIDIA GeForce GTX 780 com 2304 CUDA cores.
\end{itemize}

\input{resultados/comandosFiguras.tex}

\input{resultados/sog_mog/index.tex}

\input{resultados/rvnd/index.tex}

\input{resultados/dvnd/index.tex}

\input{resultados/gdvnd/index.tex}


\chapter{Resultados} \label{cap:resultados}

Este capítulo exibe os resultados computacionais dos algoritmos propostos no Capítulo~\ref{cap:metodologia} para o caso do PML, para cada instância foi gerado um conjunto com 100 soluções iniciais aleatórias que foram submetidas aos métodos para comparação dos resultados.

Quando há referência à melhoria na solução (\textit{Imp}), esta melhoria pode ser calculada pelo quociente do valor da solução inicial pela solução final, ou seja:
\begin{equation}\label{eq:calculateImprovement}
Imp = \frac{f(\textrm{solução inicial})}{f(\textrm{solução final})}
\end{equation}

Desta forma quanto maior for o valor da melhoria ($Imp$) mais o método melhorou o valor da solução inicial.

\section{Instâncias} \label{sec:instancias}

Todas as instâncias usadas nos testes computacionais e cujas configurações de lançamento foram descritas na Tabela~\ref{tab:neighborhoodsLaunchConfigurarion} são as mesmas usadas em~\cite{wamca2016}.
Para o RVND foi feita uma implementação do algoritmo clássico (Algoritmo~\ref{alg:rvnd}) e também a implementação dataflow mencionada na Figura~\ref{fig:rvndGraph} fazendo uso de uma máquina.
Para o caso do DVND foi utilizada a implementação clássica (Algoritmo~\ref{alg:dvnd}) e a implementação dataflow proposta (Figura~\ref{fig:dvndGraph}), os resultados foram obtidos com diferentes números de máquinas e os mesmos são indicados conforme o caso.

\section{Implementação e ambiente computacional}\label{sec:amb}

A implementação para cada algoritmo proposto no Capítulo~\ref{cap:metodologia} utiliza a linguagem de programação \textit{C++11} em conjunto com a API CUDA\texttrademark, para a implementação dos grafos e do ambiente dataflow foi utilizada a biblioteca Sucuri~\cite{sucuri-original}\footnote{Disponível em \url{https://github.com/tiagoaoa/Sucuri}} implementada em Python, para a integração entre o dataflow e o código CUDA foi utilizada a biblioteca SimplePyCuda~\cite{simple-pycuda}\footnote{Disponível em \url{https://github.com/igormcoelho/simple-pycuda}}. As implementações com múltiplas threads usaram a biblioteca OpenMP.
%As implementações foram compiladas através do \textit{GCC} \textit{(GNU Compiler Collection)}\footnote{O GCC está disponível no seguinte sítio eletrônico: \url{https://gcc.gnu.org/}.} com a \textit{flag} de otimização $-O3$.
O ambiente computacional utilizado em todos os testes neste trabalho consiste de 4 máquinas com a seguinte configuração:

\begin{itemize}
    \item Processador Intel\textregistered Core\texttrademark i7-4820K 3.7 GHz (4 núcleos);
    \item 8 GB de memória RAM;
    \item Sistema Operacional Ubuntu 14.10 (x64);
    \item NVIDIA GeForce GTX 780 com 2304 CUDA cores.
\end{itemize}

\input{resultados/comandosFiguras.tex}

\input{resultados/sog_mog/index.tex}

\input{resultados/rvnd/index.tex}

\input{resultados/dvnd/index.tex}

\input{resultados/gdvnd/index.tex}


\chapter{Resultados} \label{cap:resultados}

Este capítulo exibe os resultados computacionais dos algoritmos propostos no Capítulo~\ref{cap:metodologia} para o caso do PML, para cada instância foi gerado um conjunto com 100 soluções iniciais aleatórias que foram submetidas aos métodos para comparação dos resultados.

Quando há referência à melhoria na solução (\textit{Imp}), esta melhoria pode ser calculada pelo quociente do valor da solução inicial pela solução final, ou seja:
\begin{equation}\label{eq:calculateImprovement}
Imp = \frac{f(\textrm{solução inicial})}{f(\textrm{solução final})}
\end{equation}

Desta forma quanto maior for o valor da melhoria ($Imp$) mais o método melhorou o valor da solução inicial.

\section{Instâncias} \label{sec:instancias}

Todas as instâncias usadas nos testes computacionais e cujas configurações de lançamento foram descritas na Tabela~\ref{tab:neighborhoodsLaunchConfigurarion} são as mesmas usadas em~\cite{wamca2016}.
Para o RVND foi feita uma implementação do algoritmo clássico (Algoritmo~\ref{alg:rvnd}) e também a implementação dataflow mencionada na Figura~\ref{fig:rvndGraph} fazendo uso de uma máquina.
Para o caso do DVND foi utilizada a implementação clássica (Algoritmo~\ref{alg:dvnd}) e a implementação dataflow proposta (Figura~\ref{fig:dvndGraph}), os resultados foram obtidos com diferentes números de máquinas e os mesmos são indicados conforme o caso.

\section{Implementação e ambiente computacional}\label{sec:amb}

A implementação para cada algoritmo proposto no Capítulo~\ref{cap:metodologia} utiliza a linguagem de programação \textit{C++11} em conjunto com a API CUDA\texttrademark, para a implementação dos grafos e do ambiente dataflow foi utilizada a biblioteca Sucuri~\cite{sucuri-original}\footnote{Disponível em \url{https://github.com/tiagoaoa/Sucuri}} implementada em Python, para a integração entre o dataflow e o código CUDA foi utilizada a biblioteca SimplePyCuda~\cite{simple-pycuda}\footnote{Disponível em \url{https://github.com/igormcoelho/simple-pycuda}}. As implementações com múltiplas threads usaram a biblioteca OpenMP.
%As implementações foram compiladas através do \textit{GCC} \textit{(GNU Compiler Collection)}\footnote{O GCC está disponível no seguinte sítio eletrônico: \url{https://gcc.gnu.org/}.} com a \textit{flag} de otimização $-O3$.
O ambiente computacional utilizado em todos os testes neste trabalho consiste de 4 máquinas com a seguinte configuração:

\begin{itemize}
    \item Processador Intel\textregistered Core\texttrademark i7-4820K 3.7 GHz (4 núcleos);
    \item 8 GB de memória RAM;
    \item Sistema Operacional Ubuntu 14.10 (x64);
    \item NVIDIA GeForce GTX 780 com 2304 CUDA cores.
\end{itemize}

\input{resultados/comandosFiguras.tex}

\input{resultados/sog_mog/index.tex}

\input{resultados/rvnd/index.tex}

\input{resultados/dvnd/index.tex}

\input{resultados/gdvnd/index.tex}



\chapter{Resultados} \label{cap:resultados}

Este capítulo exibe os resultados computacionais dos algoritmos propostos no Capítulo~\ref{cap:metodologia} para o caso do PML, para cada instância foi gerado um conjunto com 100 soluções iniciais aleatórias que foram submetidas aos métodos para comparação dos resultados.

Quando há referência à melhoria na solução (\textit{Imp}), esta melhoria pode ser calculada pelo quociente do valor da solução inicial pela solução final, ou seja:
\begin{equation}\label{eq:calculateImprovement}
Imp = \frac{f(\textrm{solução inicial})}{f(\textrm{solução final})}
\end{equation}

Desta forma quanto maior for o valor da melhoria ($Imp$) mais o método melhorou o valor da solução inicial.

\section{Instâncias} \label{sec:instancias}

Todas as instâncias usadas nos testes computacionais e cujas configurações de lançamento foram descritas na Tabela~\ref{tab:neighborhoodsLaunchConfigurarion} são as mesmas usadas em~\cite{wamca2016}.
Para o RVND foi feita uma implementação do algoritmo clássico (Algoritmo~\ref{alg:rvnd}) e também a implementação dataflow mencionada na Figura~\ref{fig:rvndGraph} fazendo uso de uma máquina.
Para o caso do DVND foi utilizada a implementação clássica (Algoritmo~\ref{alg:dvnd}) e a implementação dataflow proposta (Figura~\ref{fig:dvndGraph}), os resultados foram obtidos com diferentes números de máquinas e os mesmos são indicados conforme o caso.

\section{Implementação e ambiente computacional}\label{sec:amb}

A implementação para cada algoritmo proposto no Capítulo~\ref{cap:metodologia} utiliza a linguagem de programação \textit{C++11} em conjunto com a API CUDA\texttrademark, para a implementação dos grafos e do ambiente dataflow foi utilizada a biblioteca Sucuri~\cite{sucuri-original}\footnote{Disponível em \url{https://github.com/tiagoaoa/Sucuri}} implementada em Python, para a integração entre o dataflow e o código CUDA foi utilizada a biblioteca SimplePyCuda~\cite{simple-pycuda}\footnote{Disponível em \url{https://github.com/igormcoelho/simple-pycuda}}. As implementações com múltiplas threads usaram a biblioteca OpenMP.
%As implementações foram compiladas através do \textit{GCC} \textit{(GNU Compiler Collection)}\footnote{O GCC está disponível no seguinte sítio eletrônico: \url{https://gcc.gnu.org/}.} com a \textit{flag} de otimização $-O3$.
O ambiente computacional utilizado em todos os testes neste trabalho consiste de 4 máquinas com a seguinte configuração:

\begin{itemize}
    \item Processador Intel\textregistered Core\texttrademark i7-4820K 3.7 GHz (4 núcleos);
    \item 8 GB de memória RAM;
    \item Sistema Operacional Ubuntu 14.10 (x64);
    \item NVIDIA GeForce GTX 780 com 2304 CUDA cores.
\end{itemize}

\newcommand{\figureDvndOrRvndDcDd}[7]{
% #1 {box, scatter}, #2 {count, imp, time}, #3 {instance number}, #4 {Tempo, Melhoria}, #5 tamanho instância, #6 {DVND, RVND}, #7 {dvnd, rvnd}
\begin{figure}%
    \centering
    \includegraphics[scale=0.9]{figuras/#7/dc_dd/#1/#7_#1100sol_#2_in#3.png}
    \caption{#4 do #6 para a instância #3 de tamanho #5. $m$ indica o número de máquinas, \textit{DC} refere-se ao #6 clássico e \textit{DD} ao #6 implementado em dataflow.}%
    \label{fig:#2_#7DcDd_in#3}%
\end{figure}
}

\newcommand{\figureDvndDcDd}[5]{
% #1 {box, scatter}, #2 {count, imp, time}, #3 {instance number}, #4 {Tempo, Melhoria}, #5 tamanho instância
    \figureDvndOrRvndDcDd{#1}{#2}{#3}{#4}{#5}{DVND}{dvnd}
}

\newcommand{\figureRvndDcDd}[5]{
% #1 {box, scatter}, #2 {count, imp, time}, #3 {instance number}, #4 {Tempo, Melhoria}, #5 tamanho instância
    \figureDvndOrRvndDcDd{#1}{#2}{#3}{#4}{#5}{RVND}{rvnd}
}

\newcommand{\tabelaEstatisticasGeral}[6]{
% #1 Descrição, #2 label, #3 {dvnd, rvnd}, #4 {DVND, RVND}, #6 DD/DC, #6 Conteúdo
\begin{table}[ht]
    \centering
    \begin{tabular}{c|ccc|cc|ccc|cc|c}
        \hline \hline
        \# & Tipo & $m$ & $n$ & $min$ & $max$ & 1Q & 2Q & 3Q & $\overline{x}$ & $\sigma$ & $p-valor$ \\ \hline
        #6
    \end{tabular}
    \caption{#1 #4 #5
        Instância (\#), tipo de implementação (Tipo), número de máquinas ($m$), tamanho da instância ($n$), valor mínimo ($min$), máximo ($max$), primeiro quartil (1Q), mediana (2Q), terceiro quartil (3Q), média ($\overline{x}$), desvio padrão ($\sigma$) e p-valor para o teste de Wilcox entre as versões (valores em negrito quando $p-valor > 0.05$).
    }
    \label{tab:#3DcDd#2}
\end{table}
}

\newcommand{\tabelaEstatisticas}[5]{
    \tabelaEstatisticasGeral{#1}{#2}{#3}{#4}{na implementação clássica (DC) e a proposta de implementação usando dataflow (DD).}{#5}
}

\newcommand{\figureDvndSogMog}[7]{
% #1 {box, scatter}, #2 {count, imp, time}, #3 {instance number}, #4 {Tempo, Melhoria}, #5 tamanho instância, #6 {DVND, RVND}, #7 {dvnd, rvnd}
\begin{figure}%
    \centering
    \includegraphics[scale=0.9]{figuras/#7/sog_mog/#1/#7_#1100sol_#2_in#3.png}
    \caption{#4 do #6 para a instância #3 de tamanho #5. \textit{SOG} refere-se a uma porta de saída e \textit{MOG} a múltiplas portas de saída.}%
    \label{fig:#2_#7SogMog_in#3}%
\end{figure}
}

\newcommand{\figureDvndGdvnd}[9]{
% #1 {box, scatter}, #2 {count, imp, time}, #3 {instance number}, #4 {Tempo, Melhoria}, #5 tamanho instância, #6 {DVND, RVND}, #7 {dvnd, rvnd}, #8 {man_time, full_time}, #9 {man, dvnd} #10 descricao
\begin{figure}%
    \centering
    \includegraphics[scale=0.9]{figuras/#7/#8/#1/#9_#1100sol_#2_in#3.png}
    \caption{#4 do #6 para a instância #3 de tamanho #5. \textit{DVND} refere-se ao tempo gasto pelo algoritmo de mesmo nome, para \textit{GDVND} é análogo ao anterior, no caso do \textit{GDVND-MAN} este se refere ao tempo do GDVND subtraido do tempo para gerenciar os movimentos.}%
    \label{fig:#2_#7_#8_in#3}%
\end{figure}
}

\newcommand{\figureDvndGdvndTime}[8]{
    \figureDvndGdvnd{#1}{#2}{#3}{#4}{#5}{#6}{#7}{man_time}{man}
}

\newcommand{\figureGdvndDvndRvnd}[9]{
% #1 {box, scatter}, #2 {count, imp, time}, #3 {instance number}, #4 {Tempo, Melhoria}, #5 tamanho instância, #6 {DVND, RVND}, #7 {dvnd, rvnd}, #8 {man_time, full_time}, #9 {man, dvnd} #10 descricao
\begin{figure}%
    \centering
    \includegraphics[scale=0.9]{figuras/#7/#8/#1/#9_#1100sol_#2_in#3.png}
    \caption{#4 do #6 para a instância #3 de tamanho #5. \textit{DVND}, \textit{GDVND} e \textit{RVND} referem-se ao tempo gasto pelos algoritmos de mesmo nome.}%
    \label{fig:#2_#7_#8_in#3}%
\end{figure}
}

% \subfloat[$m=#1$]{{ %scale=0.225
%         \includegraphics[scale=0.425]{figuras/dvnd/n#1/time#2.png}
%         \label{fig:timeDvndRvnd_n#1in#2}
%     }}%
% #1 {dvnd, rvnd, gdvnd}, #2 {sog_mog, dc_dd}, #3 {time, imp}, #4 in, #5 tamanho, #6 {box, scatter}
\newcommand{\subFig}[6]{
    \subfloat[][Instância #4, $n=#5$]{
        \includegraphics[scale=0.425]{figuras/#1/#2/#6/#1_#6100sol_#3_in#4.png}
		\label{fig:#1_#2_#3_in#4}
    }
% 	\begin{subfigure}{0.45\textwidth} % dvnd_box100sol_imp_in0
% 		\includegraphics[scale=0.425]{figuras/#1/#2/#6/#1_#6100sol_#3_in#4.png}
% 		\caption{Instância #4, $n=#5$}
        % \label{fig:#1_#2_#3_in#4}
    % \end{subfigure}
}

\newcommand{\subFigBox}[5]{
	\subFig{#1}{#2}{#3}{#4}{#5}{box}
}

\newcommand{\subFigScatter}[5]{
	\subFig{#1}{#2}{#3}{#4}{#5}{scatter}
}

% #1 {dvnd, rvnd, gdvnd}, #2 {sog_mog, dc_dd}, #3 {time, imp}, #4 {box, scatter}, #5 {Tempo do DVND...}
\newcommand{\multiFigureInstanciasGeral}[5]{
	\begin{figure}[ht]
		\centering
		\subFig{#1}{#2}{#3}{0}{52}{#4}
		~
		\subFig{#1}{#2}{#3}{1}{100}{#4}
		
		\subFig{#1}{#2}{#3}{2}{226}{#4}
		~
		\subFig{#1}{#2}{#3}{3}{318}{#4}
		\caption{#5 Instâncias 0 a 3.}
		\label{fig:#1_#2_#3_in0_4}
	\end{figure}
	
	\begin{figure}[ht]
		\centering
		\subFig{#1}{#2}{#3}{4}{501}{#4}
		~
		\subFig{#1}{#2}{#3}{5}{657}{#4}
		
		\subFig{#1}{#2}{#3}{6}{783}{#4}
		~
		\subFig{#1}{#2}{#3}{7}{1001}{#4}
		\caption{#5 Instâncias 5 a 7.}
		\label{fig:#1_#2_#3_in5_7}
	\end{figure}
}

% #1 {dvnd, rvnd, gdvnd}, #2 {sog_mog, dc_dd}, #3 {time, imp}, #4 {Tempo do DVND...}
\newcommand{\multiFigureInstancias}[4]{
    \multiFigureInstanciasGeral{#1}{#2}{#3}{box}{#4}
}


\chapter{Resultados} \label{cap:resultados}

Este capítulo exibe os resultados computacionais dos algoritmos propostos no Capítulo~\ref{cap:metodologia} para o caso do PML, para cada instância foi gerado um conjunto com 100 soluções iniciais aleatórias que foram submetidas aos métodos para comparação dos resultados.

Quando há referência à melhoria na solução (\textit{Imp}), esta melhoria pode ser calculada pelo quociente do valor da solução inicial pela solução final, ou seja:
\begin{equation}\label{eq:calculateImprovement}
Imp = \frac{f(\textrm{solução inicial})}{f(\textrm{solução final})}
\end{equation}

Desta forma quanto maior for o valor da melhoria ($Imp$) mais o método melhorou o valor da solução inicial.

\section{Instâncias} \label{sec:instancias}

Todas as instâncias usadas nos testes computacionais e cujas configurações de lançamento foram descritas na Tabela~\ref{tab:neighborhoodsLaunchConfigurarion} são as mesmas usadas em~\cite{wamca2016}.
Para o RVND foi feita uma implementação do algoritmo clássico (Algoritmo~\ref{alg:rvnd}) e também a implementação dataflow mencionada na Figura~\ref{fig:rvndGraph} fazendo uso de uma máquina.
Para o caso do DVND foi utilizada a implementação clássica (Algoritmo~\ref{alg:dvnd}) e a implementação dataflow proposta (Figura~\ref{fig:dvndGraph}), os resultados foram obtidos com diferentes números de máquinas e os mesmos são indicados conforme o caso.

\section{Implementação e ambiente computacional}\label{sec:amb}

A implementação para cada algoritmo proposto no Capítulo~\ref{cap:metodologia} utiliza a linguagem de programação \textit{C++11} em conjunto com a API CUDA\texttrademark, para a implementação dos grafos e do ambiente dataflow foi utilizada a biblioteca Sucuri~\cite{sucuri-original}\footnote{Disponível em \url{https://github.com/tiagoaoa/Sucuri}} implementada em Python, para a integração entre o dataflow e o código CUDA foi utilizada a biblioteca SimplePyCuda~\cite{simple-pycuda}\footnote{Disponível em \url{https://github.com/igormcoelho/simple-pycuda}}. As implementações com múltiplas threads usaram a biblioteca OpenMP.
%As implementações foram compiladas através do \textit{GCC} \textit{(GNU Compiler Collection)}\footnote{O GCC está disponível no seguinte sítio eletrônico: \url{https://gcc.gnu.org/}.} com a \textit{flag} de otimização $-O3$.
O ambiente computacional utilizado em todos os testes neste trabalho consiste de 4 máquinas com a seguinte configuração:

\begin{itemize}
    \item Processador Intel\textregistered Core\texttrademark i7-4820K 3.7 GHz (4 núcleos);
    \item 8 GB de memória RAM;
    \item Sistema Operacional Ubuntu 14.10 (x64);
    \item NVIDIA GeForce GTX 780 com 2304 CUDA cores.
\end{itemize}

\input{resultados/comandosFiguras.tex}

\input{resultados/sog_mog/index.tex}

\input{resultados/rvnd/index.tex}

\input{resultados/dvnd/index.tex}

\input{resultados/gdvnd/index.tex}


\chapter{Resultados} \label{cap:resultados}

Este capítulo exibe os resultados computacionais dos algoritmos propostos no Capítulo~\ref{cap:metodologia} para o caso do PML, para cada instância foi gerado um conjunto com 100 soluções iniciais aleatórias que foram submetidas aos métodos para comparação dos resultados.

Quando há referência à melhoria na solução (\textit{Imp}), esta melhoria pode ser calculada pelo quociente do valor da solução inicial pela solução final, ou seja:
\begin{equation}\label{eq:calculateImprovement}
Imp = \frac{f(\textrm{solução inicial})}{f(\textrm{solução final})}
\end{equation}

Desta forma quanto maior for o valor da melhoria ($Imp$) mais o método melhorou o valor da solução inicial.

\section{Instâncias} \label{sec:instancias}

Todas as instâncias usadas nos testes computacionais e cujas configurações de lançamento foram descritas na Tabela~\ref{tab:neighborhoodsLaunchConfigurarion} são as mesmas usadas em~\cite{wamca2016}.
Para o RVND foi feita uma implementação do algoritmo clássico (Algoritmo~\ref{alg:rvnd}) e também a implementação dataflow mencionada na Figura~\ref{fig:rvndGraph} fazendo uso de uma máquina.
Para o caso do DVND foi utilizada a implementação clássica (Algoritmo~\ref{alg:dvnd}) e a implementação dataflow proposta (Figura~\ref{fig:dvndGraph}), os resultados foram obtidos com diferentes números de máquinas e os mesmos são indicados conforme o caso.

\section{Implementação e ambiente computacional}\label{sec:amb}

A implementação para cada algoritmo proposto no Capítulo~\ref{cap:metodologia} utiliza a linguagem de programação \textit{C++11} em conjunto com a API CUDA\texttrademark, para a implementação dos grafos e do ambiente dataflow foi utilizada a biblioteca Sucuri~\cite{sucuri-original}\footnote{Disponível em \url{https://github.com/tiagoaoa/Sucuri}} implementada em Python, para a integração entre o dataflow e o código CUDA foi utilizada a biblioteca SimplePyCuda~\cite{simple-pycuda}\footnote{Disponível em \url{https://github.com/igormcoelho/simple-pycuda}}. As implementações com múltiplas threads usaram a biblioteca OpenMP.
%As implementações foram compiladas através do \textit{GCC} \textit{(GNU Compiler Collection)}\footnote{O GCC está disponível no seguinte sítio eletrônico: \url{https://gcc.gnu.org/}.} com a \textit{flag} de otimização $-O3$.
O ambiente computacional utilizado em todos os testes neste trabalho consiste de 4 máquinas com a seguinte configuração:

\begin{itemize}
    \item Processador Intel\textregistered Core\texttrademark i7-4820K 3.7 GHz (4 núcleos);
    \item 8 GB de memória RAM;
    \item Sistema Operacional Ubuntu 14.10 (x64);
    \item NVIDIA GeForce GTX 780 com 2304 CUDA cores.
\end{itemize}

\input{resultados/comandosFiguras.tex}

\input{resultados/sog_mog/index.tex}

\input{resultados/rvnd/index.tex}

\input{resultados/dvnd/index.tex}

\input{resultados/gdvnd/index.tex}


\chapter{Resultados} \label{cap:resultados}

Este capítulo exibe os resultados computacionais dos algoritmos propostos no Capítulo~\ref{cap:metodologia} para o caso do PML, para cada instância foi gerado um conjunto com 100 soluções iniciais aleatórias que foram submetidas aos métodos para comparação dos resultados.

Quando há referência à melhoria na solução (\textit{Imp}), esta melhoria pode ser calculada pelo quociente do valor da solução inicial pela solução final, ou seja:
\begin{equation}\label{eq:calculateImprovement}
Imp = \frac{f(\textrm{solução inicial})}{f(\textrm{solução final})}
\end{equation}

Desta forma quanto maior for o valor da melhoria ($Imp$) mais o método melhorou o valor da solução inicial.

\section{Instâncias} \label{sec:instancias}

Todas as instâncias usadas nos testes computacionais e cujas configurações de lançamento foram descritas na Tabela~\ref{tab:neighborhoodsLaunchConfigurarion} são as mesmas usadas em~\cite{wamca2016}.
Para o RVND foi feita uma implementação do algoritmo clássico (Algoritmo~\ref{alg:rvnd}) e também a implementação dataflow mencionada na Figura~\ref{fig:rvndGraph} fazendo uso de uma máquina.
Para o caso do DVND foi utilizada a implementação clássica (Algoritmo~\ref{alg:dvnd}) e a implementação dataflow proposta (Figura~\ref{fig:dvndGraph}), os resultados foram obtidos com diferentes números de máquinas e os mesmos são indicados conforme o caso.

\section{Implementação e ambiente computacional}\label{sec:amb}

A implementação para cada algoritmo proposto no Capítulo~\ref{cap:metodologia} utiliza a linguagem de programação \textit{C++11} em conjunto com a API CUDA\texttrademark, para a implementação dos grafos e do ambiente dataflow foi utilizada a biblioteca Sucuri~\cite{sucuri-original}\footnote{Disponível em \url{https://github.com/tiagoaoa/Sucuri}} implementada em Python, para a integração entre o dataflow e o código CUDA foi utilizada a biblioteca SimplePyCuda~\cite{simple-pycuda}\footnote{Disponível em \url{https://github.com/igormcoelho/simple-pycuda}}. As implementações com múltiplas threads usaram a biblioteca OpenMP.
%As implementações foram compiladas através do \textit{GCC} \textit{(GNU Compiler Collection)}\footnote{O GCC está disponível no seguinte sítio eletrônico: \url{https://gcc.gnu.org/}.} com a \textit{flag} de otimização $-O3$.
O ambiente computacional utilizado em todos os testes neste trabalho consiste de 4 máquinas com a seguinte configuração:

\begin{itemize}
    \item Processador Intel\textregistered Core\texttrademark i7-4820K 3.7 GHz (4 núcleos);
    \item 8 GB de memória RAM;
    \item Sistema Operacional Ubuntu 14.10 (x64);
    \item NVIDIA GeForce GTX 780 com 2304 CUDA cores.
\end{itemize}

\input{resultados/comandosFiguras.tex}

\input{resultados/sog_mog/index.tex}

\input{resultados/rvnd/index.tex}

\input{resultados/dvnd/index.tex}

\input{resultados/gdvnd/index.tex}


\chapter{Resultados} \label{cap:resultados}

Este capítulo exibe os resultados computacionais dos algoritmos propostos no Capítulo~\ref{cap:metodologia} para o caso do PML, para cada instância foi gerado um conjunto com 100 soluções iniciais aleatórias que foram submetidas aos métodos para comparação dos resultados.

Quando há referência à melhoria na solução (\textit{Imp}), esta melhoria pode ser calculada pelo quociente do valor da solução inicial pela solução final, ou seja:
\begin{equation}\label{eq:calculateImprovement}
Imp = \frac{f(\textrm{solução inicial})}{f(\textrm{solução final})}
\end{equation}

Desta forma quanto maior for o valor da melhoria ($Imp$) mais o método melhorou o valor da solução inicial.

\section{Instâncias} \label{sec:instancias}

Todas as instâncias usadas nos testes computacionais e cujas configurações de lançamento foram descritas na Tabela~\ref{tab:neighborhoodsLaunchConfigurarion} são as mesmas usadas em~\cite{wamca2016}.
Para o RVND foi feita uma implementação do algoritmo clássico (Algoritmo~\ref{alg:rvnd}) e também a implementação dataflow mencionada na Figura~\ref{fig:rvndGraph} fazendo uso de uma máquina.
Para o caso do DVND foi utilizada a implementação clássica (Algoritmo~\ref{alg:dvnd}) e a implementação dataflow proposta (Figura~\ref{fig:dvndGraph}), os resultados foram obtidos com diferentes números de máquinas e os mesmos são indicados conforme o caso.

\section{Implementação e ambiente computacional}\label{sec:amb}

A implementação para cada algoritmo proposto no Capítulo~\ref{cap:metodologia} utiliza a linguagem de programação \textit{C++11} em conjunto com a API CUDA\texttrademark, para a implementação dos grafos e do ambiente dataflow foi utilizada a biblioteca Sucuri~\cite{sucuri-original}\footnote{Disponível em \url{https://github.com/tiagoaoa/Sucuri}} implementada em Python, para a integração entre o dataflow e o código CUDA foi utilizada a biblioteca SimplePyCuda~\cite{simple-pycuda}\footnote{Disponível em \url{https://github.com/igormcoelho/simple-pycuda}}. As implementações com múltiplas threads usaram a biblioteca OpenMP.
%As implementações foram compiladas através do \textit{GCC} \textit{(GNU Compiler Collection)}\footnote{O GCC está disponível no seguinte sítio eletrônico: \url{https://gcc.gnu.org/}.} com a \textit{flag} de otimização $-O3$.
O ambiente computacional utilizado em todos os testes neste trabalho consiste de 4 máquinas com a seguinte configuração:

\begin{itemize}
    \item Processador Intel\textregistered Core\texttrademark i7-4820K 3.7 GHz (4 núcleos);
    \item 8 GB de memória RAM;
    \item Sistema Operacional Ubuntu 14.10 (x64);
    \item NVIDIA GeForce GTX 780 com 2304 CUDA cores.
\end{itemize}

\input{resultados/comandosFiguras.tex}

\input{resultados/sog_mog/index.tex}

\input{resultados/rvnd/index.tex}

\input{resultados/dvnd/index.tex}

\input{resultados/gdvnd/index.tex}




\chapter{Resultados} \label{cap:resultados}

Este capítulo exibe os resultados computacionais dos algoritmos propostos no Capítulo~\ref{cap:metodologia} para o caso do PML, para cada instância foi gerado um conjunto com 100 soluções iniciais aleatórias que foram submetidas aos métodos para comparação dos resultados.

Quando há referência à melhoria na solução (\textit{Imp}), esta melhoria pode ser calculada pelo quociente do valor da solução inicial pela solução final, ou seja:
\begin{equation}\label{eq:calculateImprovement}
Imp = \frac{f(\textrm{solução inicial})}{f(\textrm{solução final})}
\end{equation}

Desta forma quanto maior for o valor da melhoria ($Imp$) mais o método melhorou o valor da solução inicial.

\section{Instâncias} \label{sec:instancias}

Todas as instâncias usadas nos testes computacionais e cujas configurações de lançamento foram descritas na Tabela~\ref{tab:neighborhoodsLaunchConfigurarion} são as mesmas usadas em~\cite{wamca2016}.
Para o RVND foi feita uma implementação do algoritmo clássico (Algoritmo~\ref{alg:rvnd}) e também a implementação dataflow mencionada na Figura~\ref{fig:rvndGraph} fazendo uso de uma máquina.
Para o caso do DVND foi utilizada a implementação clássica (Algoritmo~\ref{alg:dvnd}) e a implementação dataflow proposta (Figura~\ref{fig:dvndGraph}), os resultados foram obtidos com diferentes números de máquinas e os mesmos são indicados conforme o caso.

\section{Implementação e ambiente computacional}\label{sec:amb}

A implementação para cada algoritmo proposto no Capítulo~\ref{cap:metodologia} utiliza a linguagem de programação \textit{C++11} em conjunto com a API CUDA\texttrademark, para a implementação dos grafos e do ambiente dataflow foi utilizada a biblioteca Sucuri~\cite{sucuri-original}\footnote{Disponível em \url{https://github.com/tiagoaoa/Sucuri}} implementada em Python, para a integração entre o dataflow e o código CUDA foi utilizada a biblioteca SimplePyCuda~\cite{simple-pycuda}\footnote{Disponível em \url{https://github.com/igormcoelho/simple-pycuda}}. As implementações com múltiplas threads usaram a biblioteca OpenMP.
%As implementações foram compiladas através do \textit{GCC} \textit{(GNU Compiler Collection)}\footnote{O GCC está disponível no seguinte sítio eletrônico: \url{https://gcc.gnu.org/}.} com a \textit{flag} de otimização $-O3$.
O ambiente computacional utilizado em todos os testes neste trabalho consiste de 4 máquinas com a seguinte configuração:

\begin{itemize}
    \item Processador Intel\textregistered Core\texttrademark i7-4820K 3.7 GHz (4 núcleos);
    \item 8 GB de memória RAM;
    \item Sistema Operacional Ubuntu 14.10 (x64);
    \item NVIDIA GeForce GTX 780 com 2304 CUDA cores.
\end{itemize}

\newcommand{\figureDvndOrRvndDcDd}[7]{
% #1 {box, scatter}, #2 {count, imp, time}, #3 {instance number}, #4 {Tempo, Melhoria}, #5 tamanho instância, #6 {DVND, RVND}, #7 {dvnd, rvnd}
\begin{figure}%
    \centering
    \includegraphics[scale=0.9]{figuras/#7/dc_dd/#1/#7_#1100sol_#2_in#3.png}
    \caption{#4 do #6 para a instância #3 de tamanho #5. $m$ indica o número de máquinas, \textit{DC} refere-se ao #6 clássico e \textit{DD} ao #6 implementado em dataflow.}%
    \label{fig:#2_#7DcDd_in#3}%
\end{figure}
}

\newcommand{\figureDvndDcDd}[5]{
% #1 {box, scatter}, #2 {count, imp, time}, #3 {instance number}, #4 {Tempo, Melhoria}, #5 tamanho instância
    \figureDvndOrRvndDcDd{#1}{#2}{#3}{#4}{#5}{DVND}{dvnd}
}

\newcommand{\figureRvndDcDd}[5]{
% #1 {box, scatter}, #2 {count, imp, time}, #3 {instance number}, #4 {Tempo, Melhoria}, #5 tamanho instância
    \figureDvndOrRvndDcDd{#1}{#2}{#3}{#4}{#5}{RVND}{rvnd}
}

\newcommand{\tabelaEstatisticasGeral}[6]{
% #1 Descrição, #2 label, #3 {dvnd, rvnd}, #4 {DVND, RVND}, #6 DD/DC, #6 Conteúdo
\begin{table}[ht]
    \centering
    \begin{tabular}{c|ccc|cc|ccc|cc|c}
        \hline \hline
        \# & Tipo & $m$ & $n$ & $min$ & $max$ & 1Q & 2Q & 3Q & $\overline{x}$ & $\sigma$ & $p-valor$ \\ \hline
        #6
    \end{tabular}
    \caption{#1 #4 #5
        Instância (\#), tipo de implementação (Tipo), número de máquinas ($m$), tamanho da instância ($n$), valor mínimo ($min$), máximo ($max$), primeiro quartil (1Q), mediana (2Q), terceiro quartil (3Q), média ($\overline{x}$), desvio padrão ($\sigma$) e p-valor para o teste de Wilcox entre as versões (valores em negrito quando $p-valor > 0.05$).
    }
    \label{tab:#3DcDd#2}
\end{table}
}

\newcommand{\tabelaEstatisticas}[5]{
    \tabelaEstatisticasGeral{#1}{#2}{#3}{#4}{na implementação clássica (DC) e a proposta de implementação usando dataflow (DD).}{#5}
}

\newcommand{\figureDvndSogMog}[7]{
% #1 {box, scatter}, #2 {count, imp, time}, #3 {instance number}, #4 {Tempo, Melhoria}, #5 tamanho instância, #6 {DVND, RVND}, #7 {dvnd, rvnd}
\begin{figure}%
    \centering
    \includegraphics[scale=0.9]{figuras/#7/sog_mog/#1/#7_#1100sol_#2_in#3.png}
    \caption{#4 do #6 para a instância #3 de tamanho #5. \textit{SOG} refere-se a uma porta de saída e \textit{MOG} a múltiplas portas de saída.}%
    \label{fig:#2_#7SogMog_in#3}%
\end{figure}
}

\newcommand{\figureDvndGdvnd}[9]{
% #1 {box, scatter}, #2 {count, imp, time}, #3 {instance number}, #4 {Tempo, Melhoria}, #5 tamanho instância, #6 {DVND, RVND}, #7 {dvnd, rvnd}, #8 {man_time, full_time}, #9 {man, dvnd} #10 descricao
\begin{figure}%
    \centering
    \includegraphics[scale=0.9]{figuras/#7/#8/#1/#9_#1100sol_#2_in#3.png}
    \caption{#4 do #6 para a instância #3 de tamanho #5. \textit{DVND} refere-se ao tempo gasto pelo algoritmo de mesmo nome, para \textit{GDVND} é análogo ao anterior, no caso do \textit{GDVND-MAN} este se refere ao tempo do GDVND subtraido do tempo para gerenciar os movimentos.}%
    \label{fig:#2_#7_#8_in#3}%
\end{figure}
}

\newcommand{\figureDvndGdvndTime}[8]{
    \figureDvndGdvnd{#1}{#2}{#3}{#4}{#5}{#6}{#7}{man_time}{man}
}

\newcommand{\figureGdvndDvndRvnd}[9]{
% #1 {box, scatter}, #2 {count, imp, time}, #3 {instance number}, #4 {Tempo, Melhoria}, #5 tamanho instância, #6 {DVND, RVND}, #7 {dvnd, rvnd}, #8 {man_time, full_time}, #9 {man, dvnd} #10 descricao
\begin{figure}%
    \centering
    \includegraphics[scale=0.9]{figuras/#7/#8/#1/#9_#1100sol_#2_in#3.png}
    \caption{#4 do #6 para a instância #3 de tamanho #5. \textit{DVND}, \textit{GDVND} e \textit{RVND} referem-se ao tempo gasto pelos algoritmos de mesmo nome.}%
    \label{fig:#2_#7_#8_in#3}%
\end{figure}
}

% \subfloat[$m=#1$]{{ %scale=0.225
%         \includegraphics[scale=0.425]{figuras/dvnd/n#1/time#2.png}
%         \label{fig:timeDvndRvnd_n#1in#2}
%     }}%
% #1 {dvnd, rvnd, gdvnd}, #2 {sog_mog, dc_dd}, #3 {time, imp}, #4 in, #5 tamanho, #6 {box, scatter}
\newcommand{\subFig}[6]{
    \subfloat[][Instância #4, $n=#5$]{
        \includegraphics[scale=0.425]{figuras/#1/#2/#6/#1_#6100sol_#3_in#4.png}
		\label{fig:#1_#2_#3_in#4}
    }
% 	\begin{subfigure}{0.45\textwidth} % dvnd_box100sol_imp_in0
% 		\includegraphics[scale=0.425]{figuras/#1/#2/#6/#1_#6100sol_#3_in#4.png}
% 		\caption{Instância #4, $n=#5$}
        % \label{fig:#1_#2_#3_in#4}
    % \end{subfigure}
}

\newcommand{\subFigBox}[5]{
	\subFig{#1}{#2}{#3}{#4}{#5}{box}
}

\newcommand{\subFigScatter}[5]{
	\subFig{#1}{#2}{#3}{#4}{#5}{scatter}
}

% #1 {dvnd, rvnd, gdvnd}, #2 {sog_mog, dc_dd}, #3 {time, imp}, #4 {box, scatter}, #5 {Tempo do DVND...}
\newcommand{\multiFigureInstanciasGeral}[5]{
	\begin{figure}[ht]
		\centering
		\subFig{#1}{#2}{#3}{0}{52}{#4}
		~
		\subFig{#1}{#2}{#3}{1}{100}{#4}
		
		\subFig{#1}{#2}{#3}{2}{226}{#4}
		~
		\subFig{#1}{#2}{#3}{3}{318}{#4}
		\caption{#5 Instâncias 0 a 3.}
		\label{fig:#1_#2_#3_in0_4}
	\end{figure}
	
	\begin{figure}[ht]
		\centering
		\subFig{#1}{#2}{#3}{4}{501}{#4}
		~
		\subFig{#1}{#2}{#3}{5}{657}{#4}
		
		\subFig{#1}{#2}{#3}{6}{783}{#4}
		~
		\subFig{#1}{#2}{#3}{7}{1001}{#4}
		\caption{#5 Instâncias 5 a 7.}
		\label{fig:#1_#2_#3_in5_7}
	\end{figure}
}

% #1 {dvnd, rvnd, gdvnd}, #2 {sog_mog, dc_dd}, #3 {time, imp}, #4 {Tempo do DVND...}
\newcommand{\multiFigureInstancias}[4]{
    \multiFigureInstanciasGeral{#1}{#2}{#3}{box}{#4}
}


\chapter{Resultados} \label{cap:resultados}

Este capítulo exibe os resultados computacionais dos algoritmos propostos no Capítulo~\ref{cap:metodologia} para o caso do PML, para cada instância foi gerado um conjunto com 100 soluções iniciais aleatórias que foram submetidas aos métodos para comparação dos resultados.

Quando há referência à melhoria na solução (\textit{Imp}), esta melhoria pode ser calculada pelo quociente do valor da solução inicial pela solução final, ou seja:
\begin{equation}\label{eq:calculateImprovement}
Imp = \frac{f(\textrm{solução inicial})}{f(\textrm{solução final})}
\end{equation}

Desta forma quanto maior for o valor da melhoria ($Imp$) mais o método melhorou o valor da solução inicial.

\section{Instâncias} \label{sec:instancias}

Todas as instâncias usadas nos testes computacionais e cujas configurações de lançamento foram descritas na Tabela~\ref{tab:neighborhoodsLaunchConfigurarion} são as mesmas usadas em~\cite{wamca2016}.
Para o RVND foi feita uma implementação do algoritmo clássico (Algoritmo~\ref{alg:rvnd}) e também a implementação dataflow mencionada na Figura~\ref{fig:rvndGraph} fazendo uso de uma máquina.
Para o caso do DVND foi utilizada a implementação clássica (Algoritmo~\ref{alg:dvnd}) e a implementação dataflow proposta (Figura~\ref{fig:dvndGraph}), os resultados foram obtidos com diferentes números de máquinas e os mesmos são indicados conforme o caso.

\section{Implementação e ambiente computacional}\label{sec:amb}

A implementação para cada algoritmo proposto no Capítulo~\ref{cap:metodologia} utiliza a linguagem de programação \textit{C++11} em conjunto com a API CUDA\texttrademark, para a implementação dos grafos e do ambiente dataflow foi utilizada a biblioteca Sucuri~\cite{sucuri-original}\footnote{Disponível em \url{https://github.com/tiagoaoa/Sucuri}} implementada em Python, para a integração entre o dataflow e o código CUDA foi utilizada a biblioteca SimplePyCuda~\cite{simple-pycuda}\footnote{Disponível em \url{https://github.com/igormcoelho/simple-pycuda}}. As implementações com múltiplas threads usaram a biblioteca OpenMP.
%As implementações foram compiladas através do \textit{GCC} \textit{(GNU Compiler Collection)}\footnote{O GCC está disponível no seguinte sítio eletrônico: \url{https://gcc.gnu.org/}.} com a \textit{flag} de otimização $-O3$.
O ambiente computacional utilizado em todos os testes neste trabalho consiste de 4 máquinas com a seguinte configuração:

\begin{itemize}
    \item Processador Intel\textregistered Core\texttrademark i7-4820K 3.7 GHz (4 núcleos);
    \item 8 GB de memória RAM;
    \item Sistema Operacional Ubuntu 14.10 (x64);
    \item NVIDIA GeForce GTX 780 com 2304 CUDA cores.
\end{itemize}

\newcommand{\figureDvndOrRvndDcDd}[7]{
% #1 {box, scatter}, #2 {count, imp, time}, #3 {instance number}, #4 {Tempo, Melhoria}, #5 tamanho instância, #6 {DVND, RVND}, #7 {dvnd, rvnd}
\begin{figure}%
    \centering
    \includegraphics[scale=0.9]{figuras/#7/dc_dd/#1/#7_#1100sol_#2_in#3.png}
    \caption{#4 do #6 para a instância #3 de tamanho #5. $m$ indica o número de máquinas, \textit{DC} refere-se ao #6 clássico e \textit{DD} ao #6 implementado em dataflow.}%
    \label{fig:#2_#7DcDd_in#3}%
\end{figure}
}

\newcommand{\figureDvndDcDd}[5]{
% #1 {box, scatter}, #2 {count, imp, time}, #3 {instance number}, #4 {Tempo, Melhoria}, #5 tamanho instância
    \figureDvndOrRvndDcDd{#1}{#2}{#3}{#4}{#5}{DVND}{dvnd}
}

\newcommand{\figureRvndDcDd}[5]{
% #1 {box, scatter}, #2 {count, imp, time}, #3 {instance number}, #4 {Tempo, Melhoria}, #5 tamanho instância
    \figureDvndOrRvndDcDd{#1}{#2}{#3}{#4}{#5}{RVND}{rvnd}
}

\newcommand{\tabelaEstatisticasGeral}[6]{
% #1 Descrição, #2 label, #3 {dvnd, rvnd}, #4 {DVND, RVND}, #6 DD/DC, #6 Conteúdo
\begin{table}[ht]
    \centering
    \begin{tabular}{c|ccc|cc|ccc|cc|c}
        \hline \hline
        \# & Tipo & $m$ & $n$ & $min$ & $max$ & 1Q & 2Q & 3Q & $\overline{x}$ & $\sigma$ & $p-valor$ \\ \hline
        #6
    \end{tabular}
    \caption{#1 #4 #5
        Instância (\#), tipo de implementação (Tipo), número de máquinas ($m$), tamanho da instância ($n$), valor mínimo ($min$), máximo ($max$), primeiro quartil (1Q), mediana (2Q), terceiro quartil (3Q), média ($\overline{x}$), desvio padrão ($\sigma$) e p-valor para o teste de Wilcox entre as versões (valores em negrito quando $p-valor > 0.05$).
    }
    \label{tab:#3DcDd#2}
\end{table}
}

\newcommand{\tabelaEstatisticas}[5]{
    \tabelaEstatisticasGeral{#1}{#2}{#3}{#4}{na implementação clássica (DC) e a proposta de implementação usando dataflow (DD).}{#5}
}

\newcommand{\figureDvndSogMog}[7]{
% #1 {box, scatter}, #2 {count, imp, time}, #3 {instance number}, #4 {Tempo, Melhoria}, #5 tamanho instância, #6 {DVND, RVND}, #7 {dvnd, rvnd}
\begin{figure}%
    \centering
    \includegraphics[scale=0.9]{figuras/#7/sog_mog/#1/#7_#1100sol_#2_in#3.png}
    \caption{#4 do #6 para a instância #3 de tamanho #5. \textit{SOG} refere-se a uma porta de saída e \textit{MOG} a múltiplas portas de saída.}%
    \label{fig:#2_#7SogMog_in#3}%
\end{figure}
}

\newcommand{\figureDvndGdvnd}[9]{
% #1 {box, scatter}, #2 {count, imp, time}, #3 {instance number}, #4 {Tempo, Melhoria}, #5 tamanho instância, #6 {DVND, RVND}, #7 {dvnd, rvnd}, #8 {man_time, full_time}, #9 {man, dvnd} #10 descricao
\begin{figure}%
    \centering
    \includegraphics[scale=0.9]{figuras/#7/#8/#1/#9_#1100sol_#2_in#3.png}
    \caption{#4 do #6 para a instância #3 de tamanho #5. \textit{DVND} refere-se ao tempo gasto pelo algoritmo de mesmo nome, para \textit{GDVND} é análogo ao anterior, no caso do \textit{GDVND-MAN} este se refere ao tempo do GDVND subtraido do tempo para gerenciar os movimentos.}%
    \label{fig:#2_#7_#8_in#3}%
\end{figure}
}

\newcommand{\figureDvndGdvndTime}[8]{
    \figureDvndGdvnd{#1}{#2}{#3}{#4}{#5}{#6}{#7}{man_time}{man}
}

\newcommand{\figureGdvndDvndRvnd}[9]{
% #1 {box, scatter}, #2 {count, imp, time}, #3 {instance number}, #4 {Tempo, Melhoria}, #5 tamanho instância, #6 {DVND, RVND}, #7 {dvnd, rvnd}, #8 {man_time, full_time}, #9 {man, dvnd} #10 descricao
\begin{figure}%
    \centering
    \includegraphics[scale=0.9]{figuras/#7/#8/#1/#9_#1100sol_#2_in#3.png}
    \caption{#4 do #6 para a instância #3 de tamanho #5. \textit{DVND}, \textit{GDVND} e \textit{RVND} referem-se ao tempo gasto pelos algoritmos de mesmo nome.}%
    \label{fig:#2_#7_#8_in#3}%
\end{figure}
}

% \subfloat[$m=#1$]{{ %scale=0.225
%         \includegraphics[scale=0.425]{figuras/dvnd/n#1/time#2.png}
%         \label{fig:timeDvndRvnd_n#1in#2}
%     }}%
% #1 {dvnd, rvnd, gdvnd}, #2 {sog_mog, dc_dd}, #3 {time, imp}, #4 in, #5 tamanho, #6 {box, scatter}
\newcommand{\subFig}[6]{
    \subfloat[][Instância #4, $n=#5$]{
        \includegraphics[scale=0.425]{figuras/#1/#2/#6/#1_#6100sol_#3_in#4.png}
		\label{fig:#1_#2_#3_in#4}
    }
% 	\begin{subfigure}{0.45\textwidth} % dvnd_box100sol_imp_in0
% 		\includegraphics[scale=0.425]{figuras/#1/#2/#6/#1_#6100sol_#3_in#4.png}
% 		\caption{Instância #4, $n=#5$}
        % \label{fig:#1_#2_#3_in#4}
    % \end{subfigure}
}

\newcommand{\subFigBox}[5]{
	\subFig{#1}{#2}{#3}{#4}{#5}{box}
}

\newcommand{\subFigScatter}[5]{
	\subFig{#1}{#2}{#3}{#4}{#5}{scatter}
}

% #1 {dvnd, rvnd, gdvnd}, #2 {sog_mog, dc_dd}, #3 {time, imp}, #4 {box, scatter}, #5 {Tempo do DVND...}
\newcommand{\multiFigureInstanciasGeral}[5]{
	\begin{figure}[ht]
		\centering
		\subFig{#1}{#2}{#3}{0}{52}{#4}
		~
		\subFig{#1}{#2}{#3}{1}{100}{#4}
		
		\subFig{#1}{#2}{#3}{2}{226}{#4}
		~
		\subFig{#1}{#2}{#3}{3}{318}{#4}
		\caption{#5 Instâncias 0 a 3.}
		\label{fig:#1_#2_#3_in0_4}
	\end{figure}
	
	\begin{figure}[ht]
		\centering
		\subFig{#1}{#2}{#3}{4}{501}{#4}
		~
		\subFig{#1}{#2}{#3}{5}{657}{#4}
		
		\subFig{#1}{#2}{#3}{6}{783}{#4}
		~
		\subFig{#1}{#2}{#3}{7}{1001}{#4}
		\caption{#5 Instâncias 5 a 7.}
		\label{fig:#1_#2_#3_in5_7}
	\end{figure}
}

% #1 {dvnd, rvnd, gdvnd}, #2 {sog_mog, dc_dd}, #3 {time, imp}, #4 {Tempo do DVND...}
\newcommand{\multiFigureInstancias}[4]{
    \multiFigureInstanciasGeral{#1}{#2}{#3}{box}{#4}
}


\chapter{Resultados} \label{cap:resultados}

Este capítulo exibe os resultados computacionais dos algoritmos propostos no Capítulo~\ref{cap:metodologia} para o caso do PML, para cada instância foi gerado um conjunto com 100 soluções iniciais aleatórias que foram submetidas aos métodos para comparação dos resultados.

Quando há referência à melhoria na solução (\textit{Imp}), esta melhoria pode ser calculada pelo quociente do valor da solução inicial pela solução final, ou seja:
\begin{equation}\label{eq:calculateImprovement}
Imp = \frac{f(\textrm{solução inicial})}{f(\textrm{solução final})}
\end{equation}

Desta forma quanto maior for o valor da melhoria ($Imp$) mais o método melhorou o valor da solução inicial.

\section{Instâncias} \label{sec:instancias}

Todas as instâncias usadas nos testes computacionais e cujas configurações de lançamento foram descritas na Tabela~\ref{tab:neighborhoodsLaunchConfigurarion} são as mesmas usadas em~\cite{wamca2016}.
Para o RVND foi feita uma implementação do algoritmo clássico (Algoritmo~\ref{alg:rvnd}) e também a implementação dataflow mencionada na Figura~\ref{fig:rvndGraph} fazendo uso de uma máquina.
Para o caso do DVND foi utilizada a implementação clássica (Algoritmo~\ref{alg:dvnd}) e a implementação dataflow proposta (Figura~\ref{fig:dvndGraph}), os resultados foram obtidos com diferentes números de máquinas e os mesmos são indicados conforme o caso.

\section{Implementação e ambiente computacional}\label{sec:amb}

A implementação para cada algoritmo proposto no Capítulo~\ref{cap:metodologia} utiliza a linguagem de programação \textit{C++11} em conjunto com a API CUDA\texttrademark, para a implementação dos grafos e do ambiente dataflow foi utilizada a biblioteca Sucuri~\cite{sucuri-original}\footnote{Disponível em \url{https://github.com/tiagoaoa/Sucuri}} implementada em Python, para a integração entre o dataflow e o código CUDA foi utilizada a biblioteca SimplePyCuda~\cite{simple-pycuda}\footnote{Disponível em \url{https://github.com/igormcoelho/simple-pycuda}}. As implementações com múltiplas threads usaram a biblioteca OpenMP.
%As implementações foram compiladas através do \textit{GCC} \textit{(GNU Compiler Collection)}\footnote{O GCC está disponível no seguinte sítio eletrônico: \url{https://gcc.gnu.org/}.} com a \textit{flag} de otimização $-O3$.
O ambiente computacional utilizado em todos os testes neste trabalho consiste de 4 máquinas com a seguinte configuração:

\begin{itemize}
    \item Processador Intel\textregistered Core\texttrademark i7-4820K 3.7 GHz (4 núcleos);
    \item 8 GB de memória RAM;
    \item Sistema Operacional Ubuntu 14.10 (x64);
    \item NVIDIA GeForce GTX 780 com 2304 CUDA cores.
\end{itemize}

\input{resultados/comandosFiguras.tex}

\input{resultados/sog_mog/index.tex}

\input{resultados/rvnd/index.tex}

\input{resultados/dvnd/index.tex}

\input{resultados/gdvnd/index.tex}


\chapter{Resultados} \label{cap:resultados}

Este capítulo exibe os resultados computacionais dos algoritmos propostos no Capítulo~\ref{cap:metodologia} para o caso do PML, para cada instância foi gerado um conjunto com 100 soluções iniciais aleatórias que foram submetidas aos métodos para comparação dos resultados.

Quando há referência à melhoria na solução (\textit{Imp}), esta melhoria pode ser calculada pelo quociente do valor da solução inicial pela solução final, ou seja:
\begin{equation}\label{eq:calculateImprovement}
Imp = \frac{f(\textrm{solução inicial})}{f(\textrm{solução final})}
\end{equation}

Desta forma quanto maior for o valor da melhoria ($Imp$) mais o método melhorou o valor da solução inicial.

\section{Instâncias} \label{sec:instancias}

Todas as instâncias usadas nos testes computacionais e cujas configurações de lançamento foram descritas na Tabela~\ref{tab:neighborhoodsLaunchConfigurarion} são as mesmas usadas em~\cite{wamca2016}.
Para o RVND foi feita uma implementação do algoritmo clássico (Algoritmo~\ref{alg:rvnd}) e também a implementação dataflow mencionada na Figura~\ref{fig:rvndGraph} fazendo uso de uma máquina.
Para o caso do DVND foi utilizada a implementação clássica (Algoritmo~\ref{alg:dvnd}) e a implementação dataflow proposta (Figura~\ref{fig:dvndGraph}), os resultados foram obtidos com diferentes números de máquinas e os mesmos são indicados conforme o caso.

\section{Implementação e ambiente computacional}\label{sec:amb}

A implementação para cada algoritmo proposto no Capítulo~\ref{cap:metodologia} utiliza a linguagem de programação \textit{C++11} em conjunto com a API CUDA\texttrademark, para a implementação dos grafos e do ambiente dataflow foi utilizada a biblioteca Sucuri~\cite{sucuri-original}\footnote{Disponível em \url{https://github.com/tiagoaoa/Sucuri}} implementada em Python, para a integração entre o dataflow e o código CUDA foi utilizada a biblioteca SimplePyCuda~\cite{simple-pycuda}\footnote{Disponível em \url{https://github.com/igormcoelho/simple-pycuda}}. As implementações com múltiplas threads usaram a biblioteca OpenMP.
%As implementações foram compiladas através do \textit{GCC} \textit{(GNU Compiler Collection)}\footnote{O GCC está disponível no seguinte sítio eletrônico: \url{https://gcc.gnu.org/}.} com a \textit{flag} de otimização $-O3$.
O ambiente computacional utilizado em todos os testes neste trabalho consiste de 4 máquinas com a seguinte configuração:

\begin{itemize}
    \item Processador Intel\textregistered Core\texttrademark i7-4820K 3.7 GHz (4 núcleos);
    \item 8 GB de memória RAM;
    \item Sistema Operacional Ubuntu 14.10 (x64);
    \item NVIDIA GeForce GTX 780 com 2304 CUDA cores.
\end{itemize}

\input{resultados/comandosFiguras.tex}

\input{resultados/sog_mog/index.tex}

\input{resultados/rvnd/index.tex}

\input{resultados/dvnd/index.tex}

\input{resultados/gdvnd/index.tex}


\chapter{Resultados} \label{cap:resultados}

Este capítulo exibe os resultados computacionais dos algoritmos propostos no Capítulo~\ref{cap:metodologia} para o caso do PML, para cada instância foi gerado um conjunto com 100 soluções iniciais aleatórias que foram submetidas aos métodos para comparação dos resultados.

Quando há referência à melhoria na solução (\textit{Imp}), esta melhoria pode ser calculada pelo quociente do valor da solução inicial pela solução final, ou seja:
\begin{equation}\label{eq:calculateImprovement}
Imp = \frac{f(\textrm{solução inicial})}{f(\textrm{solução final})}
\end{equation}

Desta forma quanto maior for o valor da melhoria ($Imp$) mais o método melhorou o valor da solução inicial.

\section{Instâncias} \label{sec:instancias}

Todas as instâncias usadas nos testes computacionais e cujas configurações de lançamento foram descritas na Tabela~\ref{tab:neighborhoodsLaunchConfigurarion} são as mesmas usadas em~\cite{wamca2016}.
Para o RVND foi feita uma implementação do algoritmo clássico (Algoritmo~\ref{alg:rvnd}) e também a implementação dataflow mencionada na Figura~\ref{fig:rvndGraph} fazendo uso de uma máquina.
Para o caso do DVND foi utilizada a implementação clássica (Algoritmo~\ref{alg:dvnd}) e a implementação dataflow proposta (Figura~\ref{fig:dvndGraph}), os resultados foram obtidos com diferentes números de máquinas e os mesmos são indicados conforme o caso.

\section{Implementação e ambiente computacional}\label{sec:amb}

A implementação para cada algoritmo proposto no Capítulo~\ref{cap:metodologia} utiliza a linguagem de programação \textit{C++11} em conjunto com a API CUDA\texttrademark, para a implementação dos grafos e do ambiente dataflow foi utilizada a biblioteca Sucuri~\cite{sucuri-original}\footnote{Disponível em \url{https://github.com/tiagoaoa/Sucuri}} implementada em Python, para a integração entre o dataflow e o código CUDA foi utilizada a biblioteca SimplePyCuda~\cite{simple-pycuda}\footnote{Disponível em \url{https://github.com/igormcoelho/simple-pycuda}}. As implementações com múltiplas threads usaram a biblioteca OpenMP.
%As implementações foram compiladas através do \textit{GCC} \textit{(GNU Compiler Collection)}\footnote{O GCC está disponível no seguinte sítio eletrônico: \url{https://gcc.gnu.org/}.} com a \textit{flag} de otimização $-O3$.
O ambiente computacional utilizado em todos os testes neste trabalho consiste de 4 máquinas com a seguinte configuração:

\begin{itemize}
    \item Processador Intel\textregistered Core\texttrademark i7-4820K 3.7 GHz (4 núcleos);
    \item 8 GB de memória RAM;
    \item Sistema Operacional Ubuntu 14.10 (x64);
    \item NVIDIA GeForce GTX 780 com 2304 CUDA cores.
\end{itemize}

\input{resultados/comandosFiguras.tex}

\input{resultados/sog_mog/index.tex}

\input{resultados/rvnd/index.tex}

\input{resultados/dvnd/index.tex}

\input{resultados/gdvnd/index.tex}


\chapter{Resultados} \label{cap:resultados}

Este capítulo exibe os resultados computacionais dos algoritmos propostos no Capítulo~\ref{cap:metodologia} para o caso do PML, para cada instância foi gerado um conjunto com 100 soluções iniciais aleatórias que foram submetidas aos métodos para comparação dos resultados.

Quando há referência à melhoria na solução (\textit{Imp}), esta melhoria pode ser calculada pelo quociente do valor da solução inicial pela solução final, ou seja:
\begin{equation}\label{eq:calculateImprovement}
Imp = \frac{f(\textrm{solução inicial})}{f(\textrm{solução final})}
\end{equation}

Desta forma quanto maior for o valor da melhoria ($Imp$) mais o método melhorou o valor da solução inicial.

\section{Instâncias} \label{sec:instancias}

Todas as instâncias usadas nos testes computacionais e cujas configurações de lançamento foram descritas na Tabela~\ref{tab:neighborhoodsLaunchConfigurarion} são as mesmas usadas em~\cite{wamca2016}.
Para o RVND foi feita uma implementação do algoritmo clássico (Algoritmo~\ref{alg:rvnd}) e também a implementação dataflow mencionada na Figura~\ref{fig:rvndGraph} fazendo uso de uma máquina.
Para o caso do DVND foi utilizada a implementação clássica (Algoritmo~\ref{alg:dvnd}) e a implementação dataflow proposta (Figura~\ref{fig:dvndGraph}), os resultados foram obtidos com diferentes números de máquinas e os mesmos são indicados conforme o caso.

\section{Implementação e ambiente computacional}\label{sec:amb}

A implementação para cada algoritmo proposto no Capítulo~\ref{cap:metodologia} utiliza a linguagem de programação \textit{C++11} em conjunto com a API CUDA\texttrademark, para a implementação dos grafos e do ambiente dataflow foi utilizada a biblioteca Sucuri~\cite{sucuri-original}\footnote{Disponível em \url{https://github.com/tiagoaoa/Sucuri}} implementada em Python, para a integração entre o dataflow e o código CUDA foi utilizada a biblioteca SimplePyCuda~\cite{simple-pycuda}\footnote{Disponível em \url{https://github.com/igormcoelho/simple-pycuda}}. As implementações com múltiplas threads usaram a biblioteca OpenMP.
%As implementações foram compiladas através do \textit{GCC} \textit{(GNU Compiler Collection)}\footnote{O GCC está disponível no seguinte sítio eletrônico: \url{https://gcc.gnu.org/}.} com a \textit{flag} de otimização $-O3$.
O ambiente computacional utilizado em todos os testes neste trabalho consiste de 4 máquinas com a seguinte configuração:

\begin{itemize}
    \item Processador Intel\textregistered Core\texttrademark i7-4820K 3.7 GHz (4 núcleos);
    \item 8 GB de memória RAM;
    \item Sistema Operacional Ubuntu 14.10 (x64);
    \item NVIDIA GeForce GTX 780 com 2304 CUDA cores.
\end{itemize}

\input{resultados/comandosFiguras.tex}

\input{resultados/sog_mog/index.tex}

\input{resultados/rvnd/index.tex}

\input{resultados/dvnd/index.tex}

\input{resultados/gdvnd/index.tex}



\chapter{Resultados} \label{cap:resultados}

Este capítulo exibe os resultados computacionais dos algoritmos propostos no Capítulo~\ref{cap:metodologia} para o caso do PML, para cada instância foi gerado um conjunto com 100 soluções iniciais aleatórias que foram submetidas aos métodos para comparação dos resultados.

Quando há referência à melhoria na solução (\textit{Imp}), esta melhoria pode ser calculada pelo quociente do valor da solução inicial pela solução final, ou seja:
\begin{equation}\label{eq:calculateImprovement}
Imp = \frac{f(\textrm{solução inicial})}{f(\textrm{solução final})}
\end{equation}

Desta forma quanto maior for o valor da melhoria ($Imp$) mais o método melhorou o valor da solução inicial.

\section{Instâncias} \label{sec:instancias}

Todas as instâncias usadas nos testes computacionais e cujas configurações de lançamento foram descritas na Tabela~\ref{tab:neighborhoodsLaunchConfigurarion} são as mesmas usadas em~\cite{wamca2016}.
Para o RVND foi feita uma implementação do algoritmo clássico (Algoritmo~\ref{alg:rvnd}) e também a implementação dataflow mencionada na Figura~\ref{fig:rvndGraph} fazendo uso de uma máquina.
Para o caso do DVND foi utilizada a implementação clássica (Algoritmo~\ref{alg:dvnd}) e a implementação dataflow proposta (Figura~\ref{fig:dvndGraph}), os resultados foram obtidos com diferentes números de máquinas e os mesmos são indicados conforme o caso.

\section{Implementação e ambiente computacional}\label{sec:amb}

A implementação para cada algoritmo proposto no Capítulo~\ref{cap:metodologia} utiliza a linguagem de programação \textit{C++11} em conjunto com a API CUDA\texttrademark, para a implementação dos grafos e do ambiente dataflow foi utilizada a biblioteca Sucuri~\cite{sucuri-original}\footnote{Disponível em \url{https://github.com/tiagoaoa/Sucuri}} implementada em Python, para a integração entre o dataflow e o código CUDA foi utilizada a biblioteca SimplePyCuda~\cite{simple-pycuda}\footnote{Disponível em \url{https://github.com/igormcoelho/simple-pycuda}}. As implementações com múltiplas threads usaram a biblioteca OpenMP.
%As implementações foram compiladas através do \textit{GCC} \textit{(GNU Compiler Collection)}\footnote{O GCC está disponível no seguinte sítio eletrônico: \url{https://gcc.gnu.org/}.} com a \textit{flag} de otimização $-O3$.
O ambiente computacional utilizado em todos os testes neste trabalho consiste de 4 máquinas com a seguinte configuração:

\begin{itemize}
    \item Processador Intel\textregistered Core\texttrademark i7-4820K 3.7 GHz (4 núcleos);
    \item 8 GB de memória RAM;
    \item Sistema Operacional Ubuntu 14.10 (x64);
    \item NVIDIA GeForce GTX 780 com 2304 CUDA cores.
\end{itemize}

\newcommand{\figureDvndOrRvndDcDd}[7]{
% #1 {box, scatter}, #2 {count, imp, time}, #3 {instance number}, #4 {Tempo, Melhoria}, #5 tamanho instância, #6 {DVND, RVND}, #7 {dvnd, rvnd}
\begin{figure}%
    \centering
    \includegraphics[scale=0.9]{figuras/#7/dc_dd/#1/#7_#1100sol_#2_in#3.png}
    \caption{#4 do #6 para a instância #3 de tamanho #5. $m$ indica o número de máquinas, \textit{DC} refere-se ao #6 clássico e \textit{DD} ao #6 implementado em dataflow.}%
    \label{fig:#2_#7DcDd_in#3}%
\end{figure}
}

\newcommand{\figureDvndDcDd}[5]{
% #1 {box, scatter}, #2 {count, imp, time}, #3 {instance number}, #4 {Tempo, Melhoria}, #5 tamanho instância
    \figureDvndOrRvndDcDd{#1}{#2}{#3}{#4}{#5}{DVND}{dvnd}
}

\newcommand{\figureRvndDcDd}[5]{
% #1 {box, scatter}, #2 {count, imp, time}, #3 {instance number}, #4 {Tempo, Melhoria}, #5 tamanho instância
    \figureDvndOrRvndDcDd{#1}{#2}{#3}{#4}{#5}{RVND}{rvnd}
}

\newcommand{\tabelaEstatisticasGeral}[6]{
% #1 Descrição, #2 label, #3 {dvnd, rvnd}, #4 {DVND, RVND}, #6 DD/DC, #6 Conteúdo
\begin{table}[ht]
    \centering
    \begin{tabular}{c|ccc|cc|ccc|cc|c}
        \hline \hline
        \# & Tipo & $m$ & $n$ & $min$ & $max$ & 1Q & 2Q & 3Q & $\overline{x}$ & $\sigma$ & $p-valor$ \\ \hline
        #6
    \end{tabular}
    \caption{#1 #4 #5
        Instância (\#), tipo de implementação (Tipo), número de máquinas ($m$), tamanho da instância ($n$), valor mínimo ($min$), máximo ($max$), primeiro quartil (1Q), mediana (2Q), terceiro quartil (3Q), média ($\overline{x}$), desvio padrão ($\sigma$) e p-valor para o teste de Wilcox entre as versões (valores em negrito quando $p-valor > 0.05$).
    }
    \label{tab:#3DcDd#2}
\end{table}
}

\newcommand{\tabelaEstatisticas}[5]{
    \tabelaEstatisticasGeral{#1}{#2}{#3}{#4}{na implementação clássica (DC) e a proposta de implementação usando dataflow (DD).}{#5}
}

\newcommand{\figureDvndSogMog}[7]{
% #1 {box, scatter}, #2 {count, imp, time}, #3 {instance number}, #4 {Tempo, Melhoria}, #5 tamanho instância, #6 {DVND, RVND}, #7 {dvnd, rvnd}
\begin{figure}%
    \centering
    \includegraphics[scale=0.9]{figuras/#7/sog_mog/#1/#7_#1100sol_#2_in#3.png}
    \caption{#4 do #6 para a instância #3 de tamanho #5. \textit{SOG} refere-se a uma porta de saída e \textit{MOG} a múltiplas portas de saída.}%
    \label{fig:#2_#7SogMog_in#3}%
\end{figure}
}

\newcommand{\figureDvndGdvnd}[9]{
% #1 {box, scatter}, #2 {count, imp, time}, #3 {instance number}, #4 {Tempo, Melhoria}, #5 tamanho instância, #6 {DVND, RVND}, #7 {dvnd, rvnd}, #8 {man_time, full_time}, #9 {man, dvnd} #10 descricao
\begin{figure}%
    \centering
    \includegraphics[scale=0.9]{figuras/#7/#8/#1/#9_#1100sol_#2_in#3.png}
    \caption{#4 do #6 para a instância #3 de tamanho #5. \textit{DVND} refere-se ao tempo gasto pelo algoritmo de mesmo nome, para \textit{GDVND} é análogo ao anterior, no caso do \textit{GDVND-MAN} este se refere ao tempo do GDVND subtraido do tempo para gerenciar os movimentos.}%
    \label{fig:#2_#7_#8_in#3}%
\end{figure}
}

\newcommand{\figureDvndGdvndTime}[8]{
    \figureDvndGdvnd{#1}{#2}{#3}{#4}{#5}{#6}{#7}{man_time}{man}
}

\newcommand{\figureGdvndDvndRvnd}[9]{
% #1 {box, scatter}, #2 {count, imp, time}, #3 {instance number}, #4 {Tempo, Melhoria}, #5 tamanho instância, #6 {DVND, RVND}, #7 {dvnd, rvnd}, #8 {man_time, full_time}, #9 {man, dvnd} #10 descricao
\begin{figure}%
    \centering
    \includegraphics[scale=0.9]{figuras/#7/#8/#1/#9_#1100sol_#2_in#3.png}
    \caption{#4 do #6 para a instância #3 de tamanho #5. \textit{DVND}, \textit{GDVND} e \textit{RVND} referem-se ao tempo gasto pelos algoritmos de mesmo nome.}%
    \label{fig:#2_#7_#8_in#3}%
\end{figure}
}

% \subfloat[$m=#1$]{{ %scale=0.225
%         \includegraphics[scale=0.425]{figuras/dvnd/n#1/time#2.png}
%         \label{fig:timeDvndRvnd_n#1in#2}
%     }}%
% #1 {dvnd, rvnd, gdvnd}, #2 {sog_mog, dc_dd}, #3 {time, imp}, #4 in, #5 tamanho, #6 {box, scatter}
\newcommand{\subFig}[6]{
    \subfloat[][Instância #4, $n=#5$]{
        \includegraphics[scale=0.425]{figuras/#1/#2/#6/#1_#6100sol_#3_in#4.png}
		\label{fig:#1_#2_#3_in#4}
    }
% 	\begin{subfigure}{0.45\textwidth} % dvnd_box100sol_imp_in0
% 		\includegraphics[scale=0.425]{figuras/#1/#2/#6/#1_#6100sol_#3_in#4.png}
% 		\caption{Instância #4, $n=#5$}
        % \label{fig:#1_#2_#3_in#4}
    % \end{subfigure}
}

\newcommand{\subFigBox}[5]{
	\subFig{#1}{#2}{#3}{#4}{#5}{box}
}

\newcommand{\subFigScatter}[5]{
	\subFig{#1}{#2}{#3}{#4}{#5}{scatter}
}

% #1 {dvnd, rvnd, gdvnd}, #2 {sog_mog, dc_dd}, #3 {time, imp}, #4 {box, scatter}, #5 {Tempo do DVND...}
\newcommand{\multiFigureInstanciasGeral}[5]{
	\begin{figure}[ht]
		\centering
		\subFig{#1}{#2}{#3}{0}{52}{#4}
		~
		\subFig{#1}{#2}{#3}{1}{100}{#4}
		
		\subFig{#1}{#2}{#3}{2}{226}{#4}
		~
		\subFig{#1}{#2}{#3}{3}{318}{#4}
		\caption{#5 Instâncias 0 a 3.}
		\label{fig:#1_#2_#3_in0_4}
	\end{figure}
	
	\begin{figure}[ht]
		\centering
		\subFig{#1}{#2}{#3}{4}{501}{#4}
		~
		\subFig{#1}{#2}{#3}{5}{657}{#4}
		
		\subFig{#1}{#2}{#3}{6}{783}{#4}
		~
		\subFig{#1}{#2}{#3}{7}{1001}{#4}
		\caption{#5 Instâncias 5 a 7.}
		\label{fig:#1_#2_#3_in5_7}
	\end{figure}
}

% #1 {dvnd, rvnd, gdvnd}, #2 {sog_mog, dc_dd}, #3 {time, imp}, #4 {Tempo do DVND...}
\newcommand{\multiFigureInstancias}[4]{
    \multiFigureInstanciasGeral{#1}{#2}{#3}{box}{#4}
}


\chapter{Resultados} \label{cap:resultados}

Este capítulo exibe os resultados computacionais dos algoritmos propostos no Capítulo~\ref{cap:metodologia} para o caso do PML, para cada instância foi gerado um conjunto com 100 soluções iniciais aleatórias que foram submetidas aos métodos para comparação dos resultados.

Quando há referência à melhoria na solução (\textit{Imp}), esta melhoria pode ser calculada pelo quociente do valor da solução inicial pela solução final, ou seja:
\begin{equation}\label{eq:calculateImprovement}
Imp = \frac{f(\textrm{solução inicial})}{f(\textrm{solução final})}
\end{equation}

Desta forma quanto maior for o valor da melhoria ($Imp$) mais o método melhorou o valor da solução inicial.

\section{Instâncias} \label{sec:instancias}

Todas as instâncias usadas nos testes computacionais e cujas configurações de lançamento foram descritas na Tabela~\ref{tab:neighborhoodsLaunchConfigurarion} são as mesmas usadas em~\cite{wamca2016}.
Para o RVND foi feita uma implementação do algoritmo clássico (Algoritmo~\ref{alg:rvnd}) e também a implementação dataflow mencionada na Figura~\ref{fig:rvndGraph} fazendo uso de uma máquina.
Para o caso do DVND foi utilizada a implementação clássica (Algoritmo~\ref{alg:dvnd}) e a implementação dataflow proposta (Figura~\ref{fig:dvndGraph}), os resultados foram obtidos com diferentes números de máquinas e os mesmos são indicados conforme o caso.

\section{Implementação e ambiente computacional}\label{sec:amb}

A implementação para cada algoritmo proposto no Capítulo~\ref{cap:metodologia} utiliza a linguagem de programação \textit{C++11} em conjunto com a API CUDA\texttrademark, para a implementação dos grafos e do ambiente dataflow foi utilizada a biblioteca Sucuri~\cite{sucuri-original}\footnote{Disponível em \url{https://github.com/tiagoaoa/Sucuri}} implementada em Python, para a integração entre o dataflow e o código CUDA foi utilizada a biblioteca SimplePyCuda~\cite{simple-pycuda}\footnote{Disponível em \url{https://github.com/igormcoelho/simple-pycuda}}. As implementações com múltiplas threads usaram a biblioteca OpenMP.
%As implementações foram compiladas através do \textit{GCC} \textit{(GNU Compiler Collection)}\footnote{O GCC está disponível no seguinte sítio eletrônico: \url{https://gcc.gnu.org/}.} com a \textit{flag} de otimização $-O3$.
O ambiente computacional utilizado em todos os testes neste trabalho consiste de 4 máquinas com a seguinte configuração:

\begin{itemize}
    \item Processador Intel\textregistered Core\texttrademark i7-4820K 3.7 GHz (4 núcleos);
    \item 8 GB de memória RAM;
    \item Sistema Operacional Ubuntu 14.10 (x64);
    \item NVIDIA GeForce GTX 780 com 2304 CUDA cores.
\end{itemize}

\input{resultados/comandosFiguras.tex}

\input{resultados/sog_mog/index.tex}

\input{resultados/rvnd/index.tex}

\input{resultados/dvnd/index.tex}

\input{resultados/gdvnd/index.tex}


\chapter{Resultados} \label{cap:resultados}

Este capítulo exibe os resultados computacionais dos algoritmos propostos no Capítulo~\ref{cap:metodologia} para o caso do PML, para cada instância foi gerado um conjunto com 100 soluções iniciais aleatórias que foram submetidas aos métodos para comparação dos resultados.

Quando há referência à melhoria na solução (\textit{Imp}), esta melhoria pode ser calculada pelo quociente do valor da solução inicial pela solução final, ou seja:
\begin{equation}\label{eq:calculateImprovement}
Imp = \frac{f(\textrm{solução inicial})}{f(\textrm{solução final})}
\end{equation}

Desta forma quanto maior for o valor da melhoria ($Imp$) mais o método melhorou o valor da solução inicial.

\section{Instâncias} \label{sec:instancias}

Todas as instâncias usadas nos testes computacionais e cujas configurações de lançamento foram descritas na Tabela~\ref{tab:neighborhoodsLaunchConfigurarion} são as mesmas usadas em~\cite{wamca2016}.
Para o RVND foi feita uma implementação do algoritmo clássico (Algoritmo~\ref{alg:rvnd}) e também a implementação dataflow mencionada na Figura~\ref{fig:rvndGraph} fazendo uso de uma máquina.
Para o caso do DVND foi utilizada a implementação clássica (Algoritmo~\ref{alg:dvnd}) e a implementação dataflow proposta (Figura~\ref{fig:dvndGraph}), os resultados foram obtidos com diferentes números de máquinas e os mesmos são indicados conforme o caso.

\section{Implementação e ambiente computacional}\label{sec:amb}

A implementação para cada algoritmo proposto no Capítulo~\ref{cap:metodologia} utiliza a linguagem de programação \textit{C++11} em conjunto com a API CUDA\texttrademark, para a implementação dos grafos e do ambiente dataflow foi utilizada a biblioteca Sucuri~\cite{sucuri-original}\footnote{Disponível em \url{https://github.com/tiagoaoa/Sucuri}} implementada em Python, para a integração entre o dataflow e o código CUDA foi utilizada a biblioteca SimplePyCuda~\cite{simple-pycuda}\footnote{Disponível em \url{https://github.com/igormcoelho/simple-pycuda}}. As implementações com múltiplas threads usaram a biblioteca OpenMP.
%As implementações foram compiladas através do \textit{GCC} \textit{(GNU Compiler Collection)}\footnote{O GCC está disponível no seguinte sítio eletrônico: \url{https://gcc.gnu.org/}.} com a \textit{flag} de otimização $-O3$.
O ambiente computacional utilizado em todos os testes neste trabalho consiste de 4 máquinas com a seguinte configuração:

\begin{itemize}
    \item Processador Intel\textregistered Core\texttrademark i7-4820K 3.7 GHz (4 núcleos);
    \item 8 GB de memória RAM;
    \item Sistema Operacional Ubuntu 14.10 (x64);
    \item NVIDIA GeForce GTX 780 com 2304 CUDA cores.
\end{itemize}

\input{resultados/comandosFiguras.tex}

\input{resultados/sog_mog/index.tex}

\input{resultados/rvnd/index.tex}

\input{resultados/dvnd/index.tex}

\input{resultados/gdvnd/index.tex}


\chapter{Resultados} \label{cap:resultados}

Este capítulo exibe os resultados computacionais dos algoritmos propostos no Capítulo~\ref{cap:metodologia} para o caso do PML, para cada instância foi gerado um conjunto com 100 soluções iniciais aleatórias que foram submetidas aos métodos para comparação dos resultados.

Quando há referência à melhoria na solução (\textit{Imp}), esta melhoria pode ser calculada pelo quociente do valor da solução inicial pela solução final, ou seja:
\begin{equation}\label{eq:calculateImprovement}
Imp = \frac{f(\textrm{solução inicial})}{f(\textrm{solução final})}
\end{equation}

Desta forma quanto maior for o valor da melhoria ($Imp$) mais o método melhorou o valor da solução inicial.

\section{Instâncias} \label{sec:instancias}

Todas as instâncias usadas nos testes computacionais e cujas configurações de lançamento foram descritas na Tabela~\ref{tab:neighborhoodsLaunchConfigurarion} são as mesmas usadas em~\cite{wamca2016}.
Para o RVND foi feita uma implementação do algoritmo clássico (Algoritmo~\ref{alg:rvnd}) e também a implementação dataflow mencionada na Figura~\ref{fig:rvndGraph} fazendo uso de uma máquina.
Para o caso do DVND foi utilizada a implementação clássica (Algoritmo~\ref{alg:dvnd}) e a implementação dataflow proposta (Figura~\ref{fig:dvndGraph}), os resultados foram obtidos com diferentes números de máquinas e os mesmos são indicados conforme o caso.

\section{Implementação e ambiente computacional}\label{sec:amb}

A implementação para cada algoritmo proposto no Capítulo~\ref{cap:metodologia} utiliza a linguagem de programação \textit{C++11} em conjunto com a API CUDA\texttrademark, para a implementação dos grafos e do ambiente dataflow foi utilizada a biblioteca Sucuri~\cite{sucuri-original}\footnote{Disponível em \url{https://github.com/tiagoaoa/Sucuri}} implementada em Python, para a integração entre o dataflow e o código CUDA foi utilizada a biblioteca SimplePyCuda~\cite{simple-pycuda}\footnote{Disponível em \url{https://github.com/igormcoelho/simple-pycuda}}. As implementações com múltiplas threads usaram a biblioteca OpenMP.
%As implementações foram compiladas através do \textit{GCC} \textit{(GNU Compiler Collection)}\footnote{O GCC está disponível no seguinte sítio eletrônico: \url{https://gcc.gnu.org/}.} com a \textit{flag} de otimização $-O3$.
O ambiente computacional utilizado em todos os testes neste trabalho consiste de 4 máquinas com a seguinte configuração:

\begin{itemize}
    \item Processador Intel\textregistered Core\texttrademark i7-4820K 3.7 GHz (4 núcleos);
    \item 8 GB de memória RAM;
    \item Sistema Operacional Ubuntu 14.10 (x64);
    \item NVIDIA GeForce GTX 780 com 2304 CUDA cores.
\end{itemize}

\input{resultados/comandosFiguras.tex}

\input{resultados/sog_mog/index.tex}

\input{resultados/rvnd/index.tex}

\input{resultados/dvnd/index.tex}

\input{resultados/gdvnd/index.tex}


\chapter{Resultados} \label{cap:resultados}

Este capítulo exibe os resultados computacionais dos algoritmos propostos no Capítulo~\ref{cap:metodologia} para o caso do PML, para cada instância foi gerado um conjunto com 100 soluções iniciais aleatórias que foram submetidas aos métodos para comparação dos resultados.

Quando há referência à melhoria na solução (\textit{Imp}), esta melhoria pode ser calculada pelo quociente do valor da solução inicial pela solução final, ou seja:
\begin{equation}\label{eq:calculateImprovement}
Imp = \frac{f(\textrm{solução inicial})}{f(\textrm{solução final})}
\end{equation}

Desta forma quanto maior for o valor da melhoria ($Imp$) mais o método melhorou o valor da solução inicial.

\section{Instâncias} \label{sec:instancias}

Todas as instâncias usadas nos testes computacionais e cujas configurações de lançamento foram descritas na Tabela~\ref{tab:neighborhoodsLaunchConfigurarion} são as mesmas usadas em~\cite{wamca2016}.
Para o RVND foi feita uma implementação do algoritmo clássico (Algoritmo~\ref{alg:rvnd}) e também a implementação dataflow mencionada na Figura~\ref{fig:rvndGraph} fazendo uso de uma máquina.
Para o caso do DVND foi utilizada a implementação clássica (Algoritmo~\ref{alg:dvnd}) e a implementação dataflow proposta (Figura~\ref{fig:dvndGraph}), os resultados foram obtidos com diferentes números de máquinas e os mesmos são indicados conforme o caso.

\section{Implementação e ambiente computacional}\label{sec:amb}

A implementação para cada algoritmo proposto no Capítulo~\ref{cap:metodologia} utiliza a linguagem de programação \textit{C++11} em conjunto com a API CUDA\texttrademark, para a implementação dos grafos e do ambiente dataflow foi utilizada a biblioteca Sucuri~\cite{sucuri-original}\footnote{Disponível em \url{https://github.com/tiagoaoa/Sucuri}} implementada em Python, para a integração entre o dataflow e o código CUDA foi utilizada a biblioteca SimplePyCuda~\cite{simple-pycuda}\footnote{Disponível em \url{https://github.com/igormcoelho/simple-pycuda}}. As implementações com múltiplas threads usaram a biblioteca OpenMP.
%As implementações foram compiladas através do \textit{GCC} \textit{(GNU Compiler Collection)}\footnote{O GCC está disponível no seguinte sítio eletrônico: \url{https://gcc.gnu.org/}.} com a \textit{flag} de otimização $-O3$.
O ambiente computacional utilizado em todos os testes neste trabalho consiste de 4 máquinas com a seguinte configuração:

\begin{itemize}
    \item Processador Intel\textregistered Core\texttrademark i7-4820K 3.7 GHz (4 núcleos);
    \item 8 GB de memória RAM;
    \item Sistema Operacional Ubuntu 14.10 (x64);
    \item NVIDIA GeForce GTX 780 com 2304 CUDA cores.
\end{itemize}

\input{resultados/comandosFiguras.tex}

\input{resultados/sog_mog/index.tex}

\input{resultados/rvnd/index.tex}

\input{resultados/dvnd/index.tex}

\input{resultados/gdvnd/index.tex}



\chapter{Resultados} \label{cap:resultados}

Este capítulo exibe os resultados computacionais dos algoritmos propostos no Capítulo~\ref{cap:metodologia} para o caso do PML, para cada instância foi gerado um conjunto com 100 soluções iniciais aleatórias que foram submetidas aos métodos para comparação dos resultados.

Quando há referência à melhoria na solução (\textit{Imp}), esta melhoria pode ser calculada pelo quociente do valor da solução inicial pela solução final, ou seja:
\begin{equation}\label{eq:calculateImprovement}
Imp = \frac{f(\textrm{solução inicial})}{f(\textrm{solução final})}
\end{equation}

Desta forma quanto maior for o valor da melhoria ($Imp$) mais o método melhorou o valor da solução inicial.

\section{Instâncias} \label{sec:instancias}

Todas as instâncias usadas nos testes computacionais e cujas configurações de lançamento foram descritas na Tabela~\ref{tab:neighborhoodsLaunchConfigurarion} são as mesmas usadas em~\cite{wamca2016}.
Para o RVND foi feita uma implementação do algoritmo clássico (Algoritmo~\ref{alg:rvnd}) e também a implementação dataflow mencionada na Figura~\ref{fig:rvndGraph} fazendo uso de uma máquina.
Para o caso do DVND foi utilizada a implementação clássica (Algoritmo~\ref{alg:dvnd}) e a implementação dataflow proposta (Figura~\ref{fig:dvndGraph}), os resultados foram obtidos com diferentes números de máquinas e os mesmos são indicados conforme o caso.

\section{Implementação e ambiente computacional}\label{sec:amb}

A implementação para cada algoritmo proposto no Capítulo~\ref{cap:metodologia} utiliza a linguagem de programação \textit{C++11} em conjunto com a API CUDA\texttrademark, para a implementação dos grafos e do ambiente dataflow foi utilizada a biblioteca Sucuri~\cite{sucuri-original}\footnote{Disponível em \url{https://github.com/tiagoaoa/Sucuri}} implementada em Python, para a integração entre o dataflow e o código CUDA foi utilizada a biblioteca SimplePyCuda~\cite{simple-pycuda}\footnote{Disponível em \url{https://github.com/igormcoelho/simple-pycuda}}. As implementações com múltiplas threads usaram a biblioteca OpenMP.
%As implementações foram compiladas através do \textit{GCC} \textit{(GNU Compiler Collection)}\footnote{O GCC está disponível no seguinte sítio eletrônico: \url{https://gcc.gnu.org/}.} com a \textit{flag} de otimização $-O3$.
O ambiente computacional utilizado em todos os testes neste trabalho consiste de 4 máquinas com a seguinte configuração:

\begin{itemize}
    \item Processador Intel\textregistered Core\texttrademark i7-4820K 3.7 GHz (4 núcleos);
    \item 8 GB de memória RAM;
    \item Sistema Operacional Ubuntu 14.10 (x64);
    \item NVIDIA GeForce GTX 780 com 2304 CUDA cores.
\end{itemize}

\newcommand{\figureDvndOrRvndDcDd}[7]{
% #1 {box, scatter}, #2 {count, imp, time}, #3 {instance number}, #4 {Tempo, Melhoria}, #5 tamanho instância, #6 {DVND, RVND}, #7 {dvnd, rvnd}
\begin{figure}%
    \centering
    \includegraphics[scale=0.9]{figuras/#7/dc_dd/#1/#7_#1100sol_#2_in#3.png}
    \caption{#4 do #6 para a instância #3 de tamanho #5. $m$ indica o número de máquinas, \textit{DC} refere-se ao #6 clássico e \textit{DD} ao #6 implementado em dataflow.}%
    \label{fig:#2_#7DcDd_in#3}%
\end{figure}
}

\newcommand{\figureDvndDcDd}[5]{
% #1 {box, scatter}, #2 {count, imp, time}, #3 {instance number}, #4 {Tempo, Melhoria}, #5 tamanho instância
    \figureDvndOrRvndDcDd{#1}{#2}{#3}{#4}{#5}{DVND}{dvnd}
}

\newcommand{\figureRvndDcDd}[5]{
% #1 {box, scatter}, #2 {count, imp, time}, #3 {instance number}, #4 {Tempo, Melhoria}, #5 tamanho instância
    \figureDvndOrRvndDcDd{#1}{#2}{#3}{#4}{#5}{RVND}{rvnd}
}

\newcommand{\tabelaEstatisticasGeral}[6]{
% #1 Descrição, #2 label, #3 {dvnd, rvnd}, #4 {DVND, RVND}, #6 DD/DC, #6 Conteúdo
\begin{table}[ht]
    \centering
    \begin{tabular}{c|ccc|cc|ccc|cc|c}
        \hline \hline
        \# & Tipo & $m$ & $n$ & $min$ & $max$ & 1Q & 2Q & 3Q & $\overline{x}$ & $\sigma$ & $p-valor$ \\ \hline
        #6
    \end{tabular}
    \caption{#1 #4 #5
        Instância (\#), tipo de implementação (Tipo), número de máquinas ($m$), tamanho da instância ($n$), valor mínimo ($min$), máximo ($max$), primeiro quartil (1Q), mediana (2Q), terceiro quartil (3Q), média ($\overline{x}$), desvio padrão ($\sigma$) e p-valor para o teste de Wilcox entre as versões (valores em negrito quando $p-valor > 0.05$).
    }
    \label{tab:#3DcDd#2}
\end{table}
}

\newcommand{\tabelaEstatisticas}[5]{
    \tabelaEstatisticasGeral{#1}{#2}{#3}{#4}{na implementação clássica (DC) e a proposta de implementação usando dataflow (DD).}{#5}
}

\newcommand{\figureDvndSogMog}[7]{
% #1 {box, scatter}, #2 {count, imp, time}, #3 {instance number}, #4 {Tempo, Melhoria}, #5 tamanho instância, #6 {DVND, RVND}, #7 {dvnd, rvnd}
\begin{figure}%
    \centering
    \includegraphics[scale=0.9]{figuras/#7/sog_mog/#1/#7_#1100sol_#2_in#3.png}
    \caption{#4 do #6 para a instância #3 de tamanho #5. \textit{SOG} refere-se a uma porta de saída e \textit{MOG} a múltiplas portas de saída.}%
    \label{fig:#2_#7SogMog_in#3}%
\end{figure}
}

\newcommand{\figureDvndGdvnd}[9]{
% #1 {box, scatter}, #2 {count, imp, time}, #3 {instance number}, #4 {Tempo, Melhoria}, #5 tamanho instância, #6 {DVND, RVND}, #7 {dvnd, rvnd}, #8 {man_time, full_time}, #9 {man, dvnd} #10 descricao
\begin{figure}%
    \centering
    \includegraphics[scale=0.9]{figuras/#7/#8/#1/#9_#1100sol_#2_in#3.png}
    \caption{#4 do #6 para a instância #3 de tamanho #5. \textit{DVND} refere-se ao tempo gasto pelo algoritmo de mesmo nome, para \textit{GDVND} é análogo ao anterior, no caso do \textit{GDVND-MAN} este se refere ao tempo do GDVND subtraido do tempo para gerenciar os movimentos.}%
    \label{fig:#2_#7_#8_in#3}%
\end{figure}
}

\newcommand{\figureDvndGdvndTime}[8]{
    \figureDvndGdvnd{#1}{#2}{#3}{#4}{#5}{#6}{#7}{man_time}{man}
}

\newcommand{\figureGdvndDvndRvnd}[9]{
% #1 {box, scatter}, #2 {count, imp, time}, #3 {instance number}, #4 {Tempo, Melhoria}, #5 tamanho instância, #6 {DVND, RVND}, #7 {dvnd, rvnd}, #8 {man_time, full_time}, #9 {man, dvnd} #10 descricao
\begin{figure}%
    \centering
    \includegraphics[scale=0.9]{figuras/#7/#8/#1/#9_#1100sol_#2_in#3.png}
    \caption{#4 do #6 para a instância #3 de tamanho #5. \textit{DVND}, \textit{GDVND} e \textit{RVND} referem-se ao tempo gasto pelos algoritmos de mesmo nome.}%
    \label{fig:#2_#7_#8_in#3}%
\end{figure}
}

% \subfloat[$m=#1$]{{ %scale=0.225
%         \includegraphics[scale=0.425]{figuras/dvnd/n#1/time#2.png}
%         \label{fig:timeDvndRvnd_n#1in#2}
%     }}%
% #1 {dvnd, rvnd, gdvnd}, #2 {sog_mog, dc_dd}, #3 {time, imp}, #4 in, #5 tamanho, #6 {box, scatter}
\newcommand{\subFig}[6]{
    \subfloat[][Instância #4, $n=#5$]{
        \includegraphics[scale=0.425]{figuras/#1/#2/#6/#1_#6100sol_#3_in#4.png}
		\label{fig:#1_#2_#3_in#4}
    }
% 	\begin{subfigure}{0.45\textwidth} % dvnd_box100sol_imp_in0
% 		\includegraphics[scale=0.425]{figuras/#1/#2/#6/#1_#6100sol_#3_in#4.png}
% 		\caption{Instância #4, $n=#5$}
        % \label{fig:#1_#2_#3_in#4}
    % \end{subfigure}
}

\newcommand{\subFigBox}[5]{
	\subFig{#1}{#2}{#3}{#4}{#5}{box}
}

\newcommand{\subFigScatter}[5]{
	\subFig{#1}{#2}{#3}{#4}{#5}{scatter}
}

% #1 {dvnd, rvnd, gdvnd}, #2 {sog_mog, dc_dd}, #3 {time, imp}, #4 {box, scatter}, #5 {Tempo do DVND...}
\newcommand{\multiFigureInstanciasGeral}[5]{
	\begin{figure}[ht]
		\centering
		\subFig{#1}{#2}{#3}{0}{52}{#4}
		~
		\subFig{#1}{#2}{#3}{1}{100}{#4}
		
		\subFig{#1}{#2}{#3}{2}{226}{#4}
		~
		\subFig{#1}{#2}{#3}{3}{318}{#4}
		\caption{#5 Instâncias 0 a 3.}
		\label{fig:#1_#2_#3_in0_4}
	\end{figure}
	
	\begin{figure}[ht]
		\centering
		\subFig{#1}{#2}{#3}{4}{501}{#4}
		~
		\subFig{#1}{#2}{#3}{5}{657}{#4}
		
		\subFig{#1}{#2}{#3}{6}{783}{#4}
		~
		\subFig{#1}{#2}{#3}{7}{1001}{#4}
		\caption{#5 Instâncias 5 a 7.}
		\label{fig:#1_#2_#3_in5_7}
	\end{figure}
}

% #1 {dvnd, rvnd, gdvnd}, #2 {sog_mog, dc_dd}, #3 {time, imp}, #4 {Tempo do DVND...}
\newcommand{\multiFigureInstancias}[4]{
    \multiFigureInstanciasGeral{#1}{#2}{#3}{box}{#4}
}


\chapter{Resultados} \label{cap:resultados}

Este capítulo exibe os resultados computacionais dos algoritmos propostos no Capítulo~\ref{cap:metodologia} para o caso do PML, para cada instância foi gerado um conjunto com 100 soluções iniciais aleatórias que foram submetidas aos métodos para comparação dos resultados.

Quando há referência à melhoria na solução (\textit{Imp}), esta melhoria pode ser calculada pelo quociente do valor da solução inicial pela solução final, ou seja:
\begin{equation}\label{eq:calculateImprovement}
Imp = \frac{f(\textrm{solução inicial})}{f(\textrm{solução final})}
\end{equation}

Desta forma quanto maior for o valor da melhoria ($Imp$) mais o método melhorou o valor da solução inicial.

\section{Instâncias} \label{sec:instancias}

Todas as instâncias usadas nos testes computacionais e cujas configurações de lançamento foram descritas na Tabela~\ref{tab:neighborhoodsLaunchConfigurarion} são as mesmas usadas em~\cite{wamca2016}.
Para o RVND foi feita uma implementação do algoritmo clássico (Algoritmo~\ref{alg:rvnd}) e também a implementação dataflow mencionada na Figura~\ref{fig:rvndGraph} fazendo uso de uma máquina.
Para o caso do DVND foi utilizada a implementação clássica (Algoritmo~\ref{alg:dvnd}) e a implementação dataflow proposta (Figura~\ref{fig:dvndGraph}), os resultados foram obtidos com diferentes números de máquinas e os mesmos são indicados conforme o caso.

\section{Implementação e ambiente computacional}\label{sec:amb}

A implementação para cada algoritmo proposto no Capítulo~\ref{cap:metodologia} utiliza a linguagem de programação \textit{C++11} em conjunto com a API CUDA\texttrademark, para a implementação dos grafos e do ambiente dataflow foi utilizada a biblioteca Sucuri~\cite{sucuri-original}\footnote{Disponível em \url{https://github.com/tiagoaoa/Sucuri}} implementada em Python, para a integração entre o dataflow e o código CUDA foi utilizada a biblioteca SimplePyCuda~\cite{simple-pycuda}\footnote{Disponível em \url{https://github.com/igormcoelho/simple-pycuda}}. As implementações com múltiplas threads usaram a biblioteca OpenMP.
%As implementações foram compiladas através do \textit{GCC} \textit{(GNU Compiler Collection)}\footnote{O GCC está disponível no seguinte sítio eletrônico: \url{https://gcc.gnu.org/}.} com a \textit{flag} de otimização $-O3$.
O ambiente computacional utilizado em todos os testes neste trabalho consiste de 4 máquinas com a seguinte configuração:

\begin{itemize}
    \item Processador Intel\textregistered Core\texttrademark i7-4820K 3.7 GHz (4 núcleos);
    \item 8 GB de memória RAM;
    \item Sistema Operacional Ubuntu 14.10 (x64);
    \item NVIDIA GeForce GTX 780 com 2304 CUDA cores.
\end{itemize}

\input{resultados/comandosFiguras.tex}

\input{resultados/sog_mog/index.tex}

\input{resultados/rvnd/index.tex}

\input{resultados/dvnd/index.tex}

\input{resultados/gdvnd/index.tex}


\chapter{Resultados} \label{cap:resultados}

Este capítulo exibe os resultados computacionais dos algoritmos propostos no Capítulo~\ref{cap:metodologia} para o caso do PML, para cada instância foi gerado um conjunto com 100 soluções iniciais aleatórias que foram submetidas aos métodos para comparação dos resultados.

Quando há referência à melhoria na solução (\textit{Imp}), esta melhoria pode ser calculada pelo quociente do valor da solução inicial pela solução final, ou seja:
\begin{equation}\label{eq:calculateImprovement}
Imp = \frac{f(\textrm{solução inicial})}{f(\textrm{solução final})}
\end{equation}

Desta forma quanto maior for o valor da melhoria ($Imp$) mais o método melhorou o valor da solução inicial.

\section{Instâncias} \label{sec:instancias}

Todas as instâncias usadas nos testes computacionais e cujas configurações de lançamento foram descritas na Tabela~\ref{tab:neighborhoodsLaunchConfigurarion} são as mesmas usadas em~\cite{wamca2016}.
Para o RVND foi feita uma implementação do algoritmo clássico (Algoritmo~\ref{alg:rvnd}) e também a implementação dataflow mencionada na Figura~\ref{fig:rvndGraph} fazendo uso de uma máquina.
Para o caso do DVND foi utilizada a implementação clássica (Algoritmo~\ref{alg:dvnd}) e a implementação dataflow proposta (Figura~\ref{fig:dvndGraph}), os resultados foram obtidos com diferentes números de máquinas e os mesmos são indicados conforme o caso.

\section{Implementação e ambiente computacional}\label{sec:amb}

A implementação para cada algoritmo proposto no Capítulo~\ref{cap:metodologia} utiliza a linguagem de programação \textit{C++11} em conjunto com a API CUDA\texttrademark, para a implementação dos grafos e do ambiente dataflow foi utilizada a biblioteca Sucuri~\cite{sucuri-original}\footnote{Disponível em \url{https://github.com/tiagoaoa/Sucuri}} implementada em Python, para a integração entre o dataflow e o código CUDA foi utilizada a biblioteca SimplePyCuda~\cite{simple-pycuda}\footnote{Disponível em \url{https://github.com/igormcoelho/simple-pycuda}}. As implementações com múltiplas threads usaram a biblioteca OpenMP.
%As implementações foram compiladas através do \textit{GCC} \textit{(GNU Compiler Collection)}\footnote{O GCC está disponível no seguinte sítio eletrônico: \url{https://gcc.gnu.org/}.} com a \textit{flag} de otimização $-O3$.
O ambiente computacional utilizado em todos os testes neste trabalho consiste de 4 máquinas com a seguinte configuração:

\begin{itemize}
    \item Processador Intel\textregistered Core\texttrademark i7-4820K 3.7 GHz (4 núcleos);
    \item 8 GB de memória RAM;
    \item Sistema Operacional Ubuntu 14.10 (x64);
    \item NVIDIA GeForce GTX 780 com 2304 CUDA cores.
\end{itemize}

\input{resultados/comandosFiguras.tex}

\input{resultados/sog_mog/index.tex}

\input{resultados/rvnd/index.tex}

\input{resultados/dvnd/index.tex}

\input{resultados/gdvnd/index.tex}


\chapter{Resultados} \label{cap:resultados}

Este capítulo exibe os resultados computacionais dos algoritmos propostos no Capítulo~\ref{cap:metodologia} para o caso do PML, para cada instância foi gerado um conjunto com 100 soluções iniciais aleatórias que foram submetidas aos métodos para comparação dos resultados.

Quando há referência à melhoria na solução (\textit{Imp}), esta melhoria pode ser calculada pelo quociente do valor da solução inicial pela solução final, ou seja:
\begin{equation}\label{eq:calculateImprovement}
Imp = \frac{f(\textrm{solução inicial})}{f(\textrm{solução final})}
\end{equation}

Desta forma quanto maior for o valor da melhoria ($Imp$) mais o método melhorou o valor da solução inicial.

\section{Instâncias} \label{sec:instancias}

Todas as instâncias usadas nos testes computacionais e cujas configurações de lançamento foram descritas na Tabela~\ref{tab:neighborhoodsLaunchConfigurarion} são as mesmas usadas em~\cite{wamca2016}.
Para o RVND foi feita uma implementação do algoritmo clássico (Algoritmo~\ref{alg:rvnd}) e também a implementação dataflow mencionada na Figura~\ref{fig:rvndGraph} fazendo uso de uma máquina.
Para o caso do DVND foi utilizada a implementação clássica (Algoritmo~\ref{alg:dvnd}) e a implementação dataflow proposta (Figura~\ref{fig:dvndGraph}), os resultados foram obtidos com diferentes números de máquinas e os mesmos são indicados conforme o caso.

\section{Implementação e ambiente computacional}\label{sec:amb}

A implementação para cada algoritmo proposto no Capítulo~\ref{cap:metodologia} utiliza a linguagem de programação \textit{C++11} em conjunto com a API CUDA\texttrademark, para a implementação dos grafos e do ambiente dataflow foi utilizada a biblioteca Sucuri~\cite{sucuri-original}\footnote{Disponível em \url{https://github.com/tiagoaoa/Sucuri}} implementada em Python, para a integração entre o dataflow e o código CUDA foi utilizada a biblioteca SimplePyCuda~\cite{simple-pycuda}\footnote{Disponível em \url{https://github.com/igormcoelho/simple-pycuda}}. As implementações com múltiplas threads usaram a biblioteca OpenMP.
%As implementações foram compiladas através do \textit{GCC} \textit{(GNU Compiler Collection)}\footnote{O GCC está disponível no seguinte sítio eletrônico: \url{https://gcc.gnu.org/}.} com a \textit{flag} de otimização $-O3$.
O ambiente computacional utilizado em todos os testes neste trabalho consiste de 4 máquinas com a seguinte configuração:

\begin{itemize}
    \item Processador Intel\textregistered Core\texttrademark i7-4820K 3.7 GHz (4 núcleos);
    \item 8 GB de memória RAM;
    \item Sistema Operacional Ubuntu 14.10 (x64);
    \item NVIDIA GeForce GTX 780 com 2304 CUDA cores.
\end{itemize}

\input{resultados/comandosFiguras.tex}

\input{resultados/sog_mog/index.tex}

\input{resultados/rvnd/index.tex}

\input{resultados/dvnd/index.tex}

\input{resultados/gdvnd/index.tex}


\chapter{Resultados} \label{cap:resultados}

Este capítulo exibe os resultados computacionais dos algoritmos propostos no Capítulo~\ref{cap:metodologia} para o caso do PML, para cada instância foi gerado um conjunto com 100 soluções iniciais aleatórias que foram submetidas aos métodos para comparação dos resultados.

Quando há referência à melhoria na solução (\textit{Imp}), esta melhoria pode ser calculada pelo quociente do valor da solução inicial pela solução final, ou seja:
\begin{equation}\label{eq:calculateImprovement}
Imp = \frac{f(\textrm{solução inicial})}{f(\textrm{solução final})}
\end{equation}

Desta forma quanto maior for o valor da melhoria ($Imp$) mais o método melhorou o valor da solução inicial.

\section{Instâncias} \label{sec:instancias}

Todas as instâncias usadas nos testes computacionais e cujas configurações de lançamento foram descritas na Tabela~\ref{tab:neighborhoodsLaunchConfigurarion} são as mesmas usadas em~\cite{wamca2016}.
Para o RVND foi feita uma implementação do algoritmo clássico (Algoritmo~\ref{alg:rvnd}) e também a implementação dataflow mencionada na Figura~\ref{fig:rvndGraph} fazendo uso de uma máquina.
Para o caso do DVND foi utilizada a implementação clássica (Algoritmo~\ref{alg:dvnd}) e a implementação dataflow proposta (Figura~\ref{fig:dvndGraph}), os resultados foram obtidos com diferentes números de máquinas e os mesmos são indicados conforme o caso.

\section{Implementação e ambiente computacional}\label{sec:amb}

A implementação para cada algoritmo proposto no Capítulo~\ref{cap:metodologia} utiliza a linguagem de programação \textit{C++11} em conjunto com a API CUDA\texttrademark, para a implementação dos grafos e do ambiente dataflow foi utilizada a biblioteca Sucuri~\cite{sucuri-original}\footnote{Disponível em \url{https://github.com/tiagoaoa/Sucuri}} implementada em Python, para a integração entre o dataflow e o código CUDA foi utilizada a biblioteca SimplePyCuda~\cite{simple-pycuda}\footnote{Disponível em \url{https://github.com/igormcoelho/simple-pycuda}}. As implementações com múltiplas threads usaram a biblioteca OpenMP.
%As implementações foram compiladas através do \textit{GCC} \textit{(GNU Compiler Collection)}\footnote{O GCC está disponível no seguinte sítio eletrônico: \url{https://gcc.gnu.org/}.} com a \textit{flag} de otimização $-O3$.
O ambiente computacional utilizado em todos os testes neste trabalho consiste de 4 máquinas com a seguinte configuração:

\begin{itemize}
    \item Processador Intel\textregistered Core\texttrademark i7-4820K 3.7 GHz (4 núcleos);
    \item 8 GB de memória RAM;
    \item Sistema Operacional Ubuntu 14.10 (x64);
    \item NVIDIA GeForce GTX 780 com 2304 CUDA cores.
\end{itemize}

\input{resultados/comandosFiguras.tex}

\input{resultados/sog_mog/index.tex}

\input{resultados/rvnd/index.tex}

\input{resultados/dvnd/index.tex}

\input{resultados/gdvnd/index.tex}



\chapter{Resultados} \label{cap:resultados}

Este capítulo exibe os resultados computacionais dos algoritmos propostos no Capítulo~\ref{cap:metodologia} para o caso do PML, para cada instância foi gerado um conjunto com 100 soluções iniciais aleatórias que foram submetidas aos métodos para comparação dos resultados.

Quando há referência à melhoria na solução (\textit{Imp}), esta melhoria pode ser calculada pelo quociente do valor da solução inicial pela solução final, ou seja:
\begin{equation}\label{eq:calculateImprovement}
Imp = \frac{f(\textrm{solução inicial})}{f(\textrm{solução final})}
\end{equation}

Desta forma quanto maior for o valor da melhoria ($Imp$) mais o método melhorou o valor da solução inicial.

\section{Instâncias} \label{sec:instancias}

Todas as instâncias usadas nos testes computacionais e cujas configurações de lançamento foram descritas na Tabela~\ref{tab:neighborhoodsLaunchConfigurarion} são as mesmas usadas em~\cite{wamca2016}.
Para o RVND foi feita uma implementação do algoritmo clássico (Algoritmo~\ref{alg:rvnd}) e também a implementação dataflow mencionada na Figura~\ref{fig:rvndGraph} fazendo uso de uma máquina.
Para o caso do DVND foi utilizada a implementação clássica (Algoritmo~\ref{alg:dvnd}) e a implementação dataflow proposta (Figura~\ref{fig:dvndGraph}), os resultados foram obtidos com diferentes números de máquinas e os mesmos são indicados conforme o caso.

\section{Implementação e ambiente computacional}\label{sec:amb}

A implementação para cada algoritmo proposto no Capítulo~\ref{cap:metodologia} utiliza a linguagem de programação \textit{C++11} em conjunto com a API CUDA\texttrademark, para a implementação dos grafos e do ambiente dataflow foi utilizada a biblioteca Sucuri~\cite{sucuri-original}\footnote{Disponível em \url{https://github.com/tiagoaoa/Sucuri}} implementada em Python, para a integração entre o dataflow e o código CUDA foi utilizada a biblioteca SimplePyCuda~\cite{simple-pycuda}\footnote{Disponível em \url{https://github.com/igormcoelho/simple-pycuda}}. As implementações com múltiplas threads usaram a biblioteca OpenMP.
%As implementações foram compiladas através do \textit{GCC} \textit{(GNU Compiler Collection)}\footnote{O GCC está disponível no seguinte sítio eletrônico: \url{https://gcc.gnu.org/}.} com a \textit{flag} de otimização $-O3$.
O ambiente computacional utilizado em todos os testes neste trabalho consiste de 4 máquinas com a seguinte configuração:

\begin{itemize}
    \item Processador Intel\textregistered Core\texttrademark i7-4820K 3.7 GHz (4 núcleos);
    \item 8 GB de memória RAM;
    \item Sistema Operacional Ubuntu 14.10 (x64);
    \item NVIDIA GeForce GTX 780 com 2304 CUDA cores.
\end{itemize}

\newcommand{\figureDvndOrRvndDcDd}[7]{
% #1 {box, scatter}, #2 {count, imp, time}, #3 {instance number}, #4 {Tempo, Melhoria}, #5 tamanho instância, #6 {DVND, RVND}, #7 {dvnd, rvnd}
\begin{figure}%
    \centering
    \includegraphics[scale=0.9]{figuras/#7/dc_dd/#1/#7_#1100sol_#2_in#3.png}
    \caption{#4 do #6 para a instância #3 de tamanho #5. $m$ indica o número de máquinas, \textit{DC} refere-se ao #6 clássico e \textit{DD} ao #6 implementado em dataflow.}%
    \label{fig:#2_#7DcDd_in#3}%
\end{figure}
}

\newcommand{\figureDvndDcDd}[5]{
% #1 {box, scatter}, #2 {count, imp, time}, #3 {instance number}, #4 {Tempo, Melhoria}, #5 tamanho instância
    \figureDvndOrRvndDcDd{#1}{#2}{#3}{#4}{#5}{DVND}{dvnd}
}

\newcommand{\figureRvndDcDd}[5]{
% #1 {box, scatter}, #2 {count, imp, time}, #3 {instance number}, #4 {Tempo, Melhoria}, #5 tamanho instância
    \figureDvndOrRvndDcDd{#1}{#2}{#3}{#4}{#5}{RVND}{rvnd}
}

\newcommand{\tabelaEstatisticasGeral}[6]{
% #1 Descrição, #2 label, #3 {dvnd, rvnd}, #4 {DVND, RVND}, #6 DD/DC, #6 Conteúdo
\begin{table}[ht]
    \centering
    \begin{tabular}{c|ccc|cc|ccc|cc|c}
        \hline \hline
        \# & Tipo & $m$ & $n$ & $min$ & $max$ & 1Q & 2Q & 3Q & $\overline{x}$ & $\sigma$ & $p-valor$ \\ \hline
        #6
    \end{tabular}
    \caption{#1 #4 #5
        Instância (\#), tipo de implementação (Tipo), número de máquinas ($m$), tamanho da instância ($n$), valor mínimo ($min$), máximo ($max$), primeiro quartil (1Q), mediana (2Q), terceiro quartil (3Q), média ($\overline{x}$), desvio padrão ($\sigma$) e p-valor para o teste de Wilcox entre as versões (valores em negrito quando $p-valor > 0.05$).
    }
    \label{tab:#3DcDd#2}
\end{table}
}

\newcommand{\tabelaEstatisticas}[5]{
    \tabelaEstatisticasGeral{#1}{#2}{#3}{#4}{na implementação clássica (DC) e a proposta de implementação usando dataflow (DD).}{#5}
}

\newcommand{\figureDvndSogMog}[7]{
% #1 {box, scatter}, #2 {count, imp, time}, #3 {instance number}, #4 {Tempo, Melhoria}, #5 tamanho instância, #6 {DVND, RVND}, #7 {dvnd, rvnd}
\begin{figure}%
    \centering
    \includegraphics[scale=0.9]{figuras/#7/sog_mog/#1/#7_#1100sol_#2_in#3.png}
    \caption{#4 do #6 para a instância #3 de tamanho #5. \textit{SOG} refere-se a uma porta de saída e \textit{MOG} a múltiplas portas de saída.}%
    \label{fig:#2_#7SogMog_in#3}%
\end{figure}
}

\newcommand{\figureDvndGdvnd}[9]{
% #1 {box, scatter}, #2 {count, imp, time}, #3 {instance number}, #4 {Tempo, Melhoria}, #5 tamanho instância, #6 {DVND, RVND}, #7 {dvnd, rvnd}, #8 {man_time, full_time}, #9 {man, dvnd} #10 descricao
\begin{figure}%
    \centering
    \includegraphics[scale=0.9]{figuras/#7/#8/#1/#9_#1100sol_#2_in#3.png}
    \caption{#4 do #6 para a instância #3 de tamanho #5. \textit{DVND} refere-se ao tempo gasto pelo algoritmo de mesmo nome, para \textit{GDVND} é análogo ao anterior, no caso do \textit{GDVND-MAN} este se refere ao tempo do GDVND subtraido do tempo para gerenciar os movimentos.}%
    \label{fig:#2_#7_#8_in#3}%
\end{figure}
}

\newcommand{\figureDvndGdvndTime}[8]{
    \figureDvndGdvnd{#1}{#2}{#3}{#4}{#5}{#6}{#7}{man_time}{man}
}

\newcommand{\figureGdvndDvndRvnd}[9]{
% #1 {box, scatter}, #2 {count, imp, time}, #3 {instance number}, #4 {Tempo, Melhoria}, #5 tamanho instância, #6 {DVND, RVND}, #7 {dvnd, rvnd}, #8 {man_time, full_time}, #9 {man, dvnd} #10 descricao
\begin{figure}%
    \centering
    \includegraphics[scale=0.9]{figuras/#7/#8/#1/#9_#1100sol_#2_in#3.png}
    \caption{#4 do #6 para a instância #3 de tamanho #5. \textit{DVND}, \textit{GDVND} e \textit{RVND} referem-se ao tempo gasto pelos algoritmos de mesmo nome.}%
    \label{fig:#2_#7_#8_in#3}%
\end{figure}
}

% \subfloat[$m=#1$]{{ %scale=0.225
%         \includegraphics[scale=0.425]{figuras/dvnd/n#1/time#2.png}
%         \label{fig:timeDvndRvnd_n#1in#2}
%     }}%
% #1 {dvnd, rvnd, gdvnd}, #2 {sog_mog, dc_dd}, #3 {time, imp}, #4 in, #5 tamanho, #6 {box, scatter}
\newcommand{\subFig}[6]{
    \subfloat[][Instância #4, $n=#5$]{
        \includegraphics[scale=0.425]{figuras/#1/#2/#6/#1_#6100sol_#3_in#4.png}
		\label{fig:#1_#2_#3_in#4}
    }
% 	\begin{subfigure}{0.45\textwidth} % dvnd_box100sol_imp_in0
% 		\includegraphics[scale=0.425]{figuras/#1/#2/#6/#1_#6100sol_#3_in#4.png}
% 		\caption{Instância #4, $n=#5$}
        % \label{fig:#1_#2_#3_in#4}
    % \end{subfigure}
}

\newcommand{\subFigBox}[5]{
	\subFig{#1}{#2}{#3}{#4}{#5}{box}
}

\newcommand{\subFigScatter}[5]{
	\subFig{#1}{#2}{#3}{#4}{#5}{scatter}
}

% #1 {dvnd, rvnd, gdvnd}, #2 {sog_mog, dc_dd}, #3 {time, imp}, #4 {box, scatter}, #5 {Tempo do DVND...}
\newcommand{\multiFigureInstanciasGeral}[5]{
	\begin{figure}[ht]
		\centering
		\subFig{#1}{#2}{#3}{0}{52}{#4}
		~
		\subFig{#1}{#2}{#3}{1}{100}{#4}
		
		\subFig{#1}{#2}{#3}{2}{226}{#4}
		~
		\subFig{#1}{#2}{#3}{3}{318}{#4}
		\caption{#5 Instâncias 0 a 3.}
		\label{fig:#1_#2_#3_in0_4}
	\end{figure}
	
	\begin{figure}[ht]
		\centering
		\subFig{#1}{#2}{#3}{4}{501}{#4}
		~
		\subFig{#1}{#2}{#3}{5}{657}{#4}
		
		\subFig{#1}{#2}{#3}{6}{783}{#4}
		~
		\subFig{#1}{#2}{#3}{7}{1001}{#4}
		\caption{#5 Instâncias 5 a 7.}
		\label{fig:#1_#2_#3_in5_7}
	\end{figure}
}

% #1 {dvnd, rvnd, gdvnd}, #2 {sog_mog, dc_dd}, #3 {time, imp}, #4 {Tempo do DVND...}
\newcommand{\multiFigureInstancias}[4]{
    \multiFigureInstanciasGeral{#1}{#2}{#3}{box}{#4}
}


\chapter{Resultados} \label{cap:resultados}

Este capítulo exibe os resultados computacionais dos algoritmos propostos no Capítulo~\ref{cap:metodologia} para o caso do PML, para cada instância foi gerado um conjunto com 100 soluções iniciais aleatórias que foram submetidas aos métodos para comparação dos resultados.

Quando há referência à melhoria na solução (\textit{Imp}), esta melhoria pode ser calculada pelo quociente do valor da solução inicial pela solução final, ou seja:
\begin{equation}\label{eq:calculateImprovement}
Imp = \frac{f(\textrm{solução inicial})}{f(\textrm{solução final})}
\end{equation}

Desta forma quanto maior for o valor da melhoria ($Imp$) mais o método melhorou o valor da solução inicial.

\section{Instâncias} \label{sec:instancias}

Todas as instâncias usadas nos testes computacionais e cujas configurações de lançamento foram descritas na Tabela~\ref{tab:neighborhoodsLaunchConfigurarion} são as mesmas usadas em~\cite{wamca2016}.
Para o RVND foi feita uma implementação do algoritmo clássico (Algoritmo~\ref{alg:rvnd}) e também a implementação dataflow mencionada na Figura~\ref{fig:rvndGraph} fazendo uso de uma máquina.
Para o caso do DVND foi utilizada a implementação clássica (Algoritmo~\ref{alg:dvnd}) e a implementação dataflow proposta (Figura~\ref{fig:dvndGraph}), os resultados foram obtidos com diferentes números de máquinas e os mesmos são indicados conforme o caso.

\section{Implementação e ambiente computacional}\label{sec:amb}

A implementação para cada algoritmo proposto no Capítulo~\ref{cap:metodologia} utiliza a linguagem de programação \textit{C++11} em conjunto com a API CUDA\texttrademark, para a implementação dos grafos e do ambiente dataflow foi utilizada a biblioteca Sucuri~\cite{sucuri-original}\footnote{Disponível em \url{https://github.com/tiagoaoa/Sucuri}} implementada em Python, para a integração entre o dataflow e o código CUDA foi utilizada a biblioteca SimplePyCuda~\cite{simple-pycuda}\footnote{Disponível em \url{https://github.com/igormcoelho/simple-pycuda}}. As implementações com múltiplas threads usaram a biblioteca OpenMP.
%As implementações foram compiladas através do \textit{GCC} \textit{(GNU Compiler Collection)}\footnote{O GCC está disponível no seguinte sítio eletrônico: \url{https://gcc.gnu.org/}.} com a \textit{flag} de otimização $-O3$.
O ambiente computacional utilizado em todos os testes neste trabalho consiste de 4 máquinas com a seguinte configuração:

\begin{itemize}
    \item Processador Intel\textregistered Core\texttrademark i7-4820K 3.7 GHz (4 núcleos);
    \item 8 GB de memória RAM;
    \item Sistema Operacional Ubuntu 14.10 (x64);
    \item NVIDIA GeForce GTX 780 com 2304 CUDA cores.
\end{itemize}

\input{resultados/comandosFiguras.tex}

\input{resultados/sog_mog/index.tex}

\input{resultados/rvnd/index.tex}

\input{resultados/dvnd/index.tex}

\input{resultados/gdvnd/index.tex}


\chapter{Resultados} \label{cap:resultados}

Este capítulo exibe os resultados computacionais dos algoritmos propostos no Capítulo~\ref{cap:metodologia} para o caso do PML, para cada instância foi gerado um conjunto com 100 soluções iniciais aleatórias que foram submetidas aos métodos para comparação dos resultados.

Quando há referência à melhoria na solução (\textit{Imp}), esta melhoria pode ser calculada pelo quociente do valor da solução inicial pela solução final, ou seja:
\begin{equation}\label{eq:calculateImprovement}
Imp = \frac{f(\textrm{solução inicial})}{f(\textrm{solução final})}
\end{equation}

Desta forma quanto maior for o valor da melhoria ($Imp$) mais o método melhorou o valor da solução inicial.

\section{Instâncias} \label{sec:instancias}

Todas as instâncias usadas nos testes computacionais e cujas configurações de lançamento foram descritas na Tabela~\ref{tab:neighborhoodsLaunchConfigurarion} são as mesmas usadas em~\cite{wamca2016}.
Para o RVND foi feita uma implementação do algoritmo clássico (Algoritmo~\ref{alg:rvnd}) e também a implementação dataflow mencionada na Figura~\ref{fig:rvndGraph} fazendo uso de uma máquina.
Para o caso do DVND foi utilizada a implementação clássica (Algoritmo~\ref{alg:dvnd}) e a implementação dataflow proposta (Figura~\ref{fig:dvndGraph}), os resultados foram obtidos com diferentes números de máquinas e os mesmos são indicados conforme o caso.

\section{Implementação e ambiente computacional}\label{sec:amb}

A implementação para cada algoritmo proposto no Capítulo~\ref{cap:metodologia} utiliza a linguagem de programação \textit{C++11} em conjunto com a API CUDA\texttrademark, para a implementação dos grafos e do ambiente dataflow foi utilizada a biblioteca Sucuri~\cite{sucuri-original}\footnote{Disponível em \url{https://github.com/tiagoaoa/Sucuri}} implementada em Python, para a integração entre o dataflow e o código CUDA foi utilizada a biblioteca SimplePyCuda~\cite{simple-pycuda}\footnote{Disponível em \url{https://github.com/igormcoelho/simple-pycuda}}. As implementações com múltiplas threads usaram a biblioteca OpenMP.
%As implementações foram compiladas através do \textit{GCC} \textit{(GNU Compiler Collection)}\footnote{O GCC está disponível no seguinte sítio eletrônico: \url{https://gcc.gnu.org/}.} com a \textit{flag} de otimização $-O3$.
O ambiente computacional utilizado em todos os testes neste trabalho consiste de 4 máquinas com a seguinte configuração:

\begin{itemize}
    \item Processador Intel\textregistered Core\texttrademark i7-4820K 3.7 GHz (4 núcleos);
    \item 8 GB de memória RAM;
    \item Sistema Operacional Ubuntu 14.10 (x64);
    \item NVIDIA GeForce GTX 780 com 2304 CUDA cores.
\end{itemize}

\input{resultados/comandosFiguras.tex}

\input{resultados/sog_mog/index.tex}

\input{resultados/rvnd/index.tex}

\input{resultados/dvnd/index.tex}

\input{resultados/gdvnd/index.tex}


\chapter{Resultados} \label{cap:resultados}

Este capítulo exibe os resultados computacionais dos algoritmos propostos no Capítulo~\ref{cap:metodologia} para o caso do PML, para cada instância foi gerado um conjunto com 100 soluções iniciais aleatórias que foram submetidas aos métodos para comparação dos resultados.

Quando há referência à melhoria na solução (\textit{Imp}), esta melhoria pode ser calculada pelo quociente do valor da solução inicial pela solução final, ou seja:
\begin{equation}\label{eq:calculateImprovement}
Imp = \frac{f(\textrm{solução inicial})}{f(\textrm{solução final})}
\end{equation}

Desta forma quanto maior for o valor da melhoria ($Imp$) mais o método melhorou o valor da solução inicial.

\section{Instâncias} \label{sec:instancias}

Todas as instâncias usadas nos testes computacionais e cujas configurações de lançamento foram descritas na Tabela~\ref{tab:neighborhoodsLaunchConfigurarion} são as mesmas usadas em~\cite{wamca2016}.
Para o RVND foi feita uma implementação do algoritmo clássico (Algoritmo~\ref{alg:rvnd}) e também a implementação dataflow mencionada na Figura~\ref{fig:rvndGraph} fazendo uso de uma máquina.
Para o caso do DVND foi utilizada a implementação clássica (Algoritmo~\ref{alg:dvnd}) e a implementação dataflow proposta (Figura~\ref{fig:dvndGraph}), os resultados foram obtidos com diferentes números de máquinas e os mesmos são indicados conforme o caso.

\section{Implementação e ambiente computacional}\label{sec:amb}

A implementação para cada algoritmo proposto no Capítulo~\ref{cap:metodologia} utiliza a linguagem de programação \textit{C++11} em conjunto com a API CUDA\texttrademark, para a implementação dos grafos e do ambiente dataflow foi utilizada a biblioteca Sucuri~\cite{sucuri-original}\footnote{Disponível em \url{https://github.com/tiagoaoa/Sucuri}} implementada em Python, para a integração entre o dataflow e o código CUDA foi utilizada a biblioteca SimplePyCuda~\cite{simple-pycuda}\footnote{Disponível em \url{https://github.com/igormcoelho/simple-pycuda}}. As implementações com múltiplas threads usaram a biblioteca OpenMP.
%As implementações foram compiladas através do \textit{GCC} \textit{(GNU Compiler Collection)}\footnote{O GCC está disponível no seguinte sítio eletrônico: \url{https://gcc.gnu.org/}.} com a \textit{flag} de otimização $-O3$.
O ambiente computacional utilizado em todos os testes neste trabalho consiste de 4 máquinas com a seguinte configuração:

\begin{itemize}
    \item Processador Intel\textregistered Core\texttrademark i7-4820K 3.7 GHz (4 núcleos);
    \item 8 GB de memória RAM;
    \item Sistema Operacional Ubuntu 14.10 (x64);
    \item NVIDIA GeForce GTX 780 com 2304 CUDA cores.
\end{itemize}

\input{resultados/comandosFiguras.tex}

\input{resultados/sog_mog/index.tex}

\input{resultados/rvnd/index.tex}

\input{resultados/dvnd/index.tex}

\input{resultados/gdvnd/index.tex}


\chapter{Resultados} \label{cap:resultados}

Este capítulo exibe os resultados computacionais dos algoritmos propostos no Capítulo~\ref{cap:metodologia} para o caso do PML, para cada instância foi gerado um conjunto com 100 soluções iniciais aleatórias que foram submetidas aos métodos para comparação dos resultados.

Quando há referência à melhoria na solução (\textit{Imp}), esta melhoria pode ser calculada pelo quociente do valor da solução inicial pela solução final, ou seja:
\begin{equation}\label{eq:calculateImprovement}
Imp = \frac{f(\textrm{solução inicial})}{f(\textrm{solução final})}
\end{equation}

Desta forma quanto maior for o valor da melhoria ($Imp$) mais o método melhorou o valor da solução inicial.

\section{Instâncias} \label{sec:instancias}

Todas as instâncias usadas nos testes computacionais e cujas configurações de lançamento foram descritas na Tabela~\ref{tab:neighborhoodsLaunchConfigurarion} são as mesmas usadas em~\cite{wamca2016}.
Para o RVND foi feita uma implementação do algoritmo clássico (Algoritmo~\ref{alg:rvnd}) e também a implementação dataflow mencionada na Figura~\ref{fig:rvndGraph} fazendo uso de uma máquina.
Para o caso do DVND foi utilizada a implementação clássica (Algoritmo~\ref{alg:dvnd}) e a implementação dataflow proposta (Figura~\ref{fig:dvndGraph}), os resultados foram obtidos com diferentes números de máquinas e os mesmos são indicados conforme o caso.

\section{Implementação e ambiente computacional}\label{sec:amb}

A implementação para cada algoritmo proposto no Capítulo~\ref{cap:metodologia} utiliza a linguagem de programação \textit{C++11} em conjunto com a API CUDA\texttrademark, para a implementação dos grafos e do ambiente dataflow foi utilizada a biblioteca Sucuri~\cite{sucuri-original}\footnote{Disponível em \url{https://github.com/tiagoaoa/Sucuri}} implementada em Python, para a integração entre o dataflow e o código CUDA foi utilizada a biblioteca SimplePyCuda~\cite{simple-pycuda}\footnote{Disponível em \url{https://github.com/igormcoelho/simple-pycuda}}. As implementações com múltiplas threads usaram a biblioteca OpenMP.
%As implementações foram compiladas através do \textit{GCC} \textit{(GNU Compiler Collection)}\footnote{O GCC está disponível no seguinte sítio eletrônico: \url{https://gcc.gnu.org/}.} com a \textit{flag} de otimização $-O3$.
O ambiente computacional utilizado em todos os testes neste trabalho consiste de 4 máquinas com a seguinte configuração:

\begin{itemize}
    \item Processador Intel\textregistered Core\texttrademark i7-4820K 3.7 GHz (4 núcleos);
    \item 8 GB de memória RAM;
    \item Sistema Operacional Ubuntu 14.10 (x64);
    \item NVIDIA GeForce GTX 780 com 2304 CUDA cores.
\end{itemize}

\input{resultados/comandosFiguras.tex}

\input{resultados/sog_mog/index.tex}

\input{resultados/rvnd/index.tex}

\input{resultados/dvnd/index.tex}

\input{resultados/gdvnd/index.tex}



