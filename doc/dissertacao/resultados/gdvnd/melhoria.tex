\subsection{Melhoria no valor da solução}

A Tabela~\ref{tab:gdvndDcDdImp} mostra e as Figuras~\ref{fig:gdvnd_full_time_imp_in0}-\ref{fig:gdvnd_full_time_imp_in7} evidenciam que de forma geral não há grande diferença na qualidade da solução encontrada pelos métodos.
De fato, ao analisarmos o $p-valor$ encontrado pelo teste de Wilcox realizado da amostra do GDVND com as demais, do DVND e RVND podemos ver que para muitos casos não há significância estatística para afirmar a diferença entre as amostras, valores destacados em negrito na coluna $p-valor$ da Tabela~\ref{tab:gdvndDcDdImp}.

\tabelaEstatisticasGeral{Comparativos de melhoria na solução para o}{Imp}{gdvnd}{GDVND}{com DVND e RVND.}{
    \multirow{3}{*}{0} & DVND & 4 & \multirow{3}{*}{52} & 4,27 & 6,401 & 4,954 & 5,287 & 5,564 & 5,264 & 0,475 & 2,447e-08 \\
     & RVND & 1 &  & 3,595 & 6,305 & 4,693 & 5,126 & 5,473 & 5,048 & 0,547 & 0,0002838 \\
     & GDVND & 4 &  & 2,706 & 6,457 & 4,024 & 4,836 & 5,273 & 4,609 & 0,871 &  \\ \hline
    \multirow{3}{*}{1} & DVND & 4 & \multirow{3}{*}{100} & 6,884 & 9,227 & 7,743 & 8,025 & 8,411 & 8,05 & 0,526 & 0,00738 \\
     & RVND & 1 &  & 6,163 & 9,343 & 7,686 & 8,125 & 8,317 & 8,049 & 0,58 & 0,008289 \\
     & GDVND & 4 &  & 4,012 & 9,348 & 7,256 & 7,838 & 8,311 & 7,627 & 1,04 &  \\ \hline
    \multirow{3}{*}{2} & DVND & 4 & \multirow{3}{*}{226} & 22,47 & 31,19 & 25,48 & 26,69 & 27,78 & 26,7 & 1,77 & \textbf{0,06525} \\
     & RVND & 1 &  & 21,46 & 31,22 & 25,24 & 26,51 & 27,78 & 26,46 & 1,93 & \textbf{0,1826} \\
     & GDVND & 4 &  & 5,265 & 30,89 & 24,97 & 26,17 & 27,62 & 25,29 & 4,31 &  \\ \hline
    \multirow{3}{*}{3} & DVND & 4 & \multirow{3}{*}{318} & 13,39 & 16,69 & 14,86 & 15,16 & 15,63 & 15,19 & 0,585 & \textbf{0,1916} \\
     & RVND & 1 &  & 13,8 & 17,03 & 14,79 & 15,24 & 15,61 & 15,21 & 0,587 & \textbf{0,1484} \\
     & GDVND & 4 &  & 5,163 & 16,94 & 14,67 & 15,07 & 15,5 & 14,38 & 2,68 &  \\ \hline
    \multirow{3}{*}{4} & DVND & 4 & \multirow{3}{*}{501} & 15,23 & 17,47 & 16,14 & 16,4 & 16,62 & 16,37 & 0,399 & \textbf{0,185} \\
     & RVND & 1 &  & 15,58 & 17,42 & 16,2 & 16,49 & 16,74 & 16,47 & 0,411 & 0,005406 \\
     & GDVND & 4 &  & 6,112 & 17,56 & 16,02 & 16,32 & 16,59 & 15,9 & 2,03 &  \\ \hline
    \multirow{3}{*}{5} & DVND & 4 & \multirow{3}{*}{657} & 18,22 & 20,49 & 19,08 & 19,42 & 19,71 & 19,38 & 0,476 & \textbf{0,3123} \\
     & RVND & 1 &  & 18,1 & 20,84 & 19,07 & 19,4 & 19,79 & 19,42 & 0,52 & \textbf{0,1671} \\
     & GDVND & 4 &  & 17,63 & 20,49 & 18,96 & 19,26 & 19,71 & 19,3 & 0,515 &  \\ \hline
    \multirow{3}{*}{6} & DVND & 4 & \multirow{3}{*}{783} & 19,44 & 21,83 & 20,16 & 20,48 & 20,85 & 20,52 & 0,469 & \textbf{0,1648} \\
     & RVND & 1 &  & 19,53 & 21,7 & 20,25 & 20,52 & 20,92 & 20,56 & 0,457 & \textbf{0,05223} \\
     & GDVND & 4 &  & 18,73 & 21,45 & 20,11 & 20,37 & 20,74 & 20,4 & 0,513 &  \\ \hline
    \multirow{3}{*}{7} & DVND & 4 & \multirow{3}{*}{1001} & 22,29 & 24,33 & 23,04 & 23,35 & 23,62 & 23,35 & 0,441 & 0,02061 \\
     & RVND & 1 &  & 22,23 & 24,88 & 23,08 & 23,37 & 23,7 & 23,38 & 0,503 & 0,009164 \\ 
     & GDVND & 4 &  & 20,66 & 24,09 & 22,91 & 23,17 & 23,5 & 23,17 & 0,493 &  \\ \hline
}

% \figureGdvndDvndRvnd{box}{imp}{0}{Melhoria no valor da solução}{52}{GDVND}{gdvnd}{full_time}{gdvnd}

% \figureGdvndDvndRvnd{box}{imp}{1}{Melhoria no valor da solução}{100}{GDVND}{gdvnd}{full_time}{dvnd}

% \figureGdvndDvndRvnd{box}{imp}{2}{Melhoria no valor da solução}{226}{GDVND}{gdvnd}{full_time}{dvnd}

% \figureGdvndDvndRvnd{box}{imp}{3}{Melhoria no valor da solução}{318}{GDVND}{gdvnd}{full_time}{dvnd}

% \figureGdvndDvndRvnd{box}{imp}{4}{Melhoria no valor da solução}{501}{GDVND}{gdvnd}{full_time}{dvnd}

% \figureGdvndDvndRvnd{box}{imp}{5}{Melhoria no valor da solução}{657}{GDVND}{gdvnd}{full_time}{dvnd}

% \figureGdvndDvndRvnd{box}{imp}{6}{Melhoria no valor da solução}{783}{GDVND}{gdvnd}{full_time}{dvnd}

% \figureGdvndDvndRvnd{box}{imp}{7}{Melhoria no valor da solução}{1001}{GDVND}{gdvnd}{full_time}{dvnd}

\multiFigureInstancias{gdvnd}{full_time}{imp}{Melhoria no valor da solução para os algoritmos GDVND, DVND e RVND, $n$ representa o tamanho da instância, $m$ indica o número de máquinas.}
