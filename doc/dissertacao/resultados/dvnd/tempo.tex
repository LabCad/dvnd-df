\subsection{Tempo}

Pode-se ver pela Tabela~\ref{tab:dvndDcDdTime} que o DVND clássico apresenta melhores tempos para as menores instâncias.
Até a instância 4, de tamanho 501, os tempos do DVND clássico são sensivelmente melhores que os tempos do DVND em dataflow, contudo a partir da instância 5, de tamanho 657, a implementação em dataflow alcança os tempos da implementação clássica quando usa mais de uma máquina.

\tabelaEstatisticas{Tempos comparativos do}{Time}{dvnd}{DVND}{
    \multirow{5}{*}{0} & \multirow{4}{*}{DD} & 1 & \multirow{5}{*}{52} & 1,487 & 1,82 & 1,566 & 1,603 & 1,648 & 1,61 & 0,067 & 1,238e-14 \\
     &  & 2 &  & 1,425 & 2,787 & 1,569 & 2,621 & 2,684 & 2,218 & 0,551 & 1,238e-14 \\
     &  & 3 &  & 1,504 & 2,95 & 2,495 & 2,643 & 2,727 & 2,497 & 0,386 & 1,238e-14 \\
     &  & 4 &  & 1,706 & 3,98 & 2,698 & 2,81 & 2,926 & 2,91 & 0,44 & 1,238e-14 \\
     & DC & 1 &  & 1,137 & 1,428 & 1,214 & 1,254 & 1,29 & 1,253 & 0,0554 &  \\ \hline
    \multirow{5}{*}{1} & \multirow{4}{*}{DD} & 1 & \multirow{5}{*}{100} & 1,711 & 2,195 & 1,839 & 1,903 & 1,981 & 1,918 & 0,104 & 1,238e-14 \\
     &  & 2 &  & 1,572 & 3,055 & 1,778 & 2,745 & 2,844 & 2,422 & 0,532 & 1,238e-14 \\
     &  & 3 &  & 1,702 & 3,018 & 2,601 & 2,716 & 2,85 & 2,606 & 0,369 & 1,238e-14 \\
     &  & 4 &  & 1,86 & 4,111 & 2,76 & 2,873 & 2,966 & 2,968 & 0,417 & 1,238e-14 \\
     & DC & 1 &  & 1,224 & 1,557 & 1,315 & 1,354 & 1,398 & 1,358 & 0,0583 &  \\ \hline
    \multirow{5}{*}{2} & \multirow{4}{*}{DD} & 1 & \multirow{5}{*}{226} & 2,253 & 3,011 & 2,517 & 2,605 & 2,74 & 2,621 & 0,155 & 1,238e-14 \\
     &  & 2 &  & 1,835 & 3,497 & 2,149 & 2,879 & 3,262 & 2,74 & 0,543 & 1,238e-14 \\
     &  & 3 &  & 1,969 & 3,465 & 2,738 & 2,921 & 3,17 & 2,881 & 0,389 & 1,238e-14 \\
     &  & 4 &  & 1,991 & 4,351 & 3,053 & 3,206 & 3,31 & 3,212 & 0,315 & 1,238e-14 \\
     & DC & 1 &  & 1,258 & 1,735 & 1,476 & 1,526 & 1,587 & 1,526 & 0,0937 &  \\ \hline
    \multirow{5}{*}{3} & \multirow{4}{*}{DD} & 1 & \multirow{5}{*}{318} & 2,929 & 3,909 & 3,247 & 3,345 & 3,487 & 3,352 & 0,2 & 1,238e-14 \\
     &  & 2 &  & 2,341 & 4,047 & 2,659 & 3,295 & 3,764 & 3,233 & 0,542 & 1,238e-14 \\
     &  & 3 &  & 2,474 & 4,032 & 3,181 & 3,434 & 3,691 & 3,39 & 0,37 & 1,238e-14 \\
     &  & 4 &  & 2,693 & 4,767 & 3,497 & 3,66 & 3,834 & 3,662 & 0,425 & 1,238e-14 \\
     & DC & 1 &  & 1,625 & 2,725 & 1,874 & 1,952 & 2,07 & 1,975 & 0,175 &  \\ \hline
    \multirow{5}{*}{4} & \multirow{4}{*}{DD} & 1 & \multirow{5}{*}{501} & 4,237 & 5,701 & 4,628 & 4,812 & 4,99 & 4,818 & 0,291 & 1,91e-14 \\
     &  & 2 &  & 3,115 & 5,038 & 3,597 & 3,958 & 4,626 & 4,066 & 0,553 & 1,91e-14 \\
     &  & 3 &  & 3,169 & 4,963 & 3,956 & 4,383 & 4,575 & 4,264 & 0,44 & 1,91e-14 \\
     &  & 4 &  & 3,386 & 5,563 & 4,351 & 4,534 & 4,766 & 4,539 & 0,433 & 1,91e-14 \\
     & DC & 1 &  & 1,76 & 3,549 & 2,632 & 2,811 & 3,014 & 2,842 & 0,315 &  \\ \hline
    \multirow{5}{*}{5} & \multirow{4}{*}{DD} & 1 & \multirow{5}{*}{657} & 5,512 & 7,375 & 6,173 & 6,41 & 6,677 & 6,425 & 0,369 & 0,03197 \\
     &  & 2 &  & 3,923 & 6,032 & 4,451 & 4,639 & 5,2 & 4,822 & 0,525 & 0,03197 \\
     &  & 3 &  & 4,038 & 6,053 & 4,77 & 5,235 & 5,482 & 5,12 & 0,482 & 0,03197 \\
     &  & 4 &  & 3,988 & 6,683 & 5,204 & 5,453 & 5,706 & 5,421 & 0,498 & 0,03197 \\
     & DC & 1 &  & 3,555 & 7,213 & 4,529 & 4,856 & 5,474 & 5,018 & 0,745 &  \\ \hline
    \multirow{5}{*}{6} & \multirow{4}{*}{DD} & 1 & \multirow{5}{*}{783} & 7,211 & 9,492 & 7,923 & 8,174 & 8,651 & 8,281 & 0,507 & 0,0001566 \\
     &  & 2 &  & 5,016 & 7,217 & 5,598 & 5,841 & 6,42 & 5,99 & 0,523 & 0,0001566 \\
     &  & 3 &  & 5,296 & 7,37 & 5,898 & 6,421 & 6,672 & 6,308 & 0,479 & 0,0001566 \\
     &  & 4 &  & 5,237 & 8,034 & 6,134 & 6,598 & 6,815 & 6,496 & 0,588 & 0,0001566 \\
     & DC & 1 &  & 3,031 & 8,138 & 5,537 & 6,066 & 6,543 & 6,105 & 0,848 &  \\ \hline
    \multirow{5}{*}{7} & \multirow{4}{*}{DD} & 1 & \multirow{5}{*}{1001} & 10,93 & 14,49 & 11,89 & 12,44 & 13,02 & 12,48 & 0,764 & 3,915e-11 \\
     &  & 2 &  & 7,773 & 10,46 & 8,416 & 8,933 & 9,448 & 8,964 & 0,688 & 3,915e-11 \\
     &  & 3 &  & 7,584 & 10,26 & 8,801 & 9,191 & 9,526 & 9,146 & 0,585 & 3,915e-11 \\
     &  & 4 &  & 7,516 & 10,46 & 8,647 & 9,008 & 9,373 & 8,99 & 0,578 & 3,915e-11 \\
     & DC & 1 &  & 3,024 & 17,28 & 12,19 & 13,34 & 14,39 & 13,28 & 1,92 &  \\ \hline
}

Na maior instância, de tamanho 1001, pode ser visto que o resultado da implementação dataflow para uma máquina é sutilmente melhor que da implementação clássica, o que se torna mais evidente ao utilizar mais de uma máquina, quando os tempos melhoram sensivelmente em relação à implementação clássica.
Conforme se vê na Tabela~\ref{tab:dvndDcDdTime} pela coluna $p-valor$ há significância estatística para se verificar a diferença entre as amostragens.

% \figureDvndDcDd{box}{time}{0}{Tempos}{52}

Podemos ver na Figura~\ref{fig:dvnd_dc_dd_time_in0} que o DVND clássico possui tempos bem menores que o DVND em dataflow e o uso de mais máquinas não consegue melhorar os tempos do procedimento.

% \figureDvndDcDd{box}{time}{1}{Tempos}{100}

A Figura~\ref{fig:dvnd_dc_dd_time_in1} é bem parecida com a anterior, inclusive com tempos bastante próximos, indicando que o aumento de 52 para 100 no tamanho da solução não é suficiente para causar um grande aumento no tempos de solução pelo método.

% \figureDvndDcDd{box}{time}{2}{Tempos}{226}

Para a Figura~\ref{fig:dvnd_dc_dd_time_in2} percebe-se que os tempos aumentam um pouco mas o comportamento é bastante semelhante, o DVND clássico é mais rápido para resolver o problema e aumentar o número de máquinas não melhora razoavelmente o desempenho.

% \figureDvndDcDd{box}{time}{3}{Tempos}{318}

% \figureDvndDcDd{box}{time}{4}{Tempos}{501}

Para a Figura~\ref{fig:dvnd_dc_dd_time_in4}, que representa a instância 4 de tamanho 501, percebe-se pela primeira vez uma melhoria razoável no tempo do DVND em dataflow pelo uso de mais de uma máquina, contudo ainda não sendo suficiente para melhorar os resultado do DVND clássico.

% \figureDvndDcDd{box}{time}{5}{Tempos}{657}

Na instância 5, de tamanho 657, ilustrada na Figura~\ref{fig:dvnd_dc_dd_time_in5}, o tempo do DVND em dataflow, quando usa mais de uma máquina, melhora alcançando ao DVND clássico.

% \figureDvndDcDd{box}{time}{6}{Tempos}{783}

Na instância 6, de tamanho 783, ilustrada na Figura~\ref{fig:dvnd_dc_dd_time_in6}, os resultados são bastante parecidos com a instância anterior.

% \figureDvndDcDd{box}{time}{7}{Tempos}{1001}

Na instância 7, de tamanho 1001, ilustrada na Figura~\ref{fig:dvnd_dc_dd_time_in7}, os tempos alcançados pelo DVND dataflow são menores que o DVND clássico para mais de uma máquina alcançando assim melhores tempos para a maior instância.

\multiFigureInstancias{dvnd}{dc_dd}{time}{Tempo do DVND, $n$ representa o tamanho da instância, $m$ indica o número de máquinas, \textit{DC} refere-se ao DVND clássico e \textit{DD} ao DVND implementado em dataflow.}
