% Igor coloquei isso aqui como um Teorema, não sei se é muito ambicioso da minha parte
% 
\begin{theorem}[Teorema da independência de movimentos]\label{teo:independenciaMovimentos}
Seja um conjunto $M'$ com movimentos livres de contexto em que todos os movimentos são independentes entre si dois a dois, ou seja $m_i \parallel m_j \mid \forall m_i, m_j \in M'$, com $m_i \ne m_j$.
E seja $s'$ a solução dada pela aplicação de todos os movimentos à solução inicial $s$, logo o valor da solução resultante $s'$ será dado pelo somatório do valor de cada movimento. Assim, pode-se escrever:

% \begin{align*} \label{eq:teo:independenciaMovimentos}
% m_1 \parallel m_2 \implies& s' = m_1 \circ m_2 \circ s \\
% m_1 \parallel m_2 \implies& f(s') = \widehat{m_1}(s) + \widehat{m_2}(s) + f(s) \\
% \end{align*}

% \begin{align*}
\begin{equation}
    \label{eq:teo:independenciaMovimentos}
    \begin{split}
        m_i \parallel m_j, \forall m_i, m_j \in M' \implies& s' = m_1 \circ m_2 \circ \dots \circ m_n \circ s \\
        m_i \parallel m_j, \forall m_i, m_j \in M' \implies& f(s') = f(s) + \widehat{m_1}(s) + \widehat{m_2}(s)+ \widehat{m_3}(s) + \dots + \widehat{m_n}(s) \\
        m_i \parallel m_j, \forall m_i, m_j \in M' \implies& f(s') = f(s) + \sum_i^n{\widehat{m_i}(s)}
    \end{split}
\end{equation}
% \end{align*}
\end{theorem}

\begin{proof}\label{proof:independenciaMovimentos}
% ---
% Prova por contradição
% ---
Suponhamos um conjunto $M = \{ m_1, m_2, \dots, m_n \}$ com $n$ movimentos independentes dois a dois mas $f(m_1 \circ m_2 \circ \dots \circ m_n \circ s) \neq f(s) + \widehat{m_1}(s) + \widehat{m_2}(s)+ \widehat{m_3}(s) + \dots + \widehat{m_n}(s)$.

\begin{align*}
    f(m_1 \circ m_2 \circ \dots \circ m_n \circ s) \neq f(s) + \widehat{m_1}(s) + \widehat{m_2}(s) + \dots + \widehat{m_n}(s) & \\
    \widehat{m_1}(m_2 \circ \dots \circ m_n \circ s) + f(m_2 \circ \dots \circ m_n \circ s) \neq f(s) + \widehat{m_1}(s) + \widehat{m_2}(s) + \dots + \widehat{m_n}(s) & \ [1] \\
    \widehat{m_1}(s) + f(m_2 \circ \dots \circ m_n \circ s) \neq f(s) + \widehat{m_1}(s) + \widehat{m_2}(s) + \dots + \widehat{m_n}(s) & \\
    \widehat{m_1}(s) + \widehat{m_2}(m_3 \circ \dots \circ m_n \circ s) + f(m_3 \circ \dots \circ m_n \circ s) \neq f(s) + \widehat{m_1}(s) + \widehat{m_2}(s) + \dots + \widehat{m_n}(s) & \ [2] \\
    \widehat{m_1}(s) + \widehat{m_2}(s) + f(m_3 \circ \dots \circ m_n \circ s) \neq f(s) + \widehat{m_1}(s) + \widehat{m_2}(s) + \dots + \widehat{m_n}(s) & \\
    \vdots & \\
    \widehat{m_1}(s) + \widehat{m_2}(s) + \dots + \widehat{m_n}(s) + f(s) \neq f(s) + \widehat{m_1}(s) + \widehat{m_2}(s) + \dots + \widehat{m_n}(s) & \ [3] \\
\end{align*}

Os passos acima se dão conforme:
\begin{align*}
[1] \ & m_1 \parallel m_i \mid \forall m_i \in M \\
[2] \ & m_2 \parallel m_i \mid \forall m_i \in M \\
[3] \ & \textrm{Logo temos uma contradição} \bot
\end{align*}

Desta forma, por \textit{reductio ad absurdum} demonstramos que se os movimentos são independentes então o valor da solução resultante se dará pelo somatório da solução anterior com o valor dos movimentos.

% ---
% Indução
% ---
% Pela Equação~\ref{eq:movimentoCustoSomaDois} temos que $f(m_1 \circ m_2 \circ s) = \widehat{m_1}(s) + \widehat{m_2}(s) + f(s)$, supondo por hipótese de indução que $f(m_1 \circ m_2 \circ \dots \circ m_n \circ s) = f(s) + \widehat{m_1}(s) + \widehat{m_2}(s) + \dots + \widehat{m_n}(s)$

% \begin{align*}
%     f(m_1 \circ m_2 \circ \dots \circ m_n \circ m_{n+1} \circ s) = & f(m_{n+1} \circ m_1 \circ m_2 \circ \dots \circ m_n \circ s) \\
%     f(m_1 \circ m_2 \circ \dots \circ m_n \circ m_{n+1} \circ s) = & \widehat{m_{n+1}}(m_1 \circ m_2 \circ \dots \circ m_n \circ s) + f(m_1 \circ m_2 \circ \dots \circ m_n \circ s) \\
% \end{align*}

% % \widehat{m_1}(s) + \widehat{m_2}(s) + \dots + \widehat{m_n}(s) + f(s)
% Fazendo $s_2 = m_2 \circ \dots \circ m_n \circ s$.
% \begin{align*}
%     f(m_1 \circ m_2 \circ \dots \circ m_n \circ m_{n+1} \circ s) = & \widehat{m_{n+1}}(m_1 s_2) + f(m_1 \circ m_2 \circ \dots \circ m_n \circ s) \\
%     f(m_1 \circ m_2 \circ \dots \circ m_n \circ m_{n+1} \circ s) = & \widehat{m_{n+1}}(s_2) + f(m_1 \circ m_2 \circ \dots \circ m_n \circ s)
% \end{align*}

% Pois $m_{n+1} \parallel m_1$, fazendo agora $s_3 = m_3 \circ \dots \circ m_n \circ s$.
% \begin{align*}
%     f(m_1 \circ m_2 \circ \dots \circ m_n \circ m_{n+1} \circ s) = & \widehat{m_{n+1}}(m_2 \circ \dots \circ m_n \circ s) + f(m_1 \circ m_2 \circ \dots \circ m_n \circ s)\\
%     f(m_1 \circ m_2 \circ \dots \circ m_n \circ m_{n+1} \circ s) = & \widehat{m_{n+1}}(m_2 \circ s_3) + f(m_1 \circ m_2 \circ \dots \circ m_n \circ s)\\
%     f(m_1 \circ m_2 \circ \dots \circ m_n \circ m_{n+1} \circ s) = & \widehat{m_{n+1}}(s_3) + f(m_1 \circ m_2 \circ \dots \circ m_n \circ s)\\
%     \vdots \\
%     f(m_1 \circ m_2 \circ \dots \circ m_n \circ m_{n+1} \circ s) = & \widehat{m_{n+1}}(s) + f(m_1 \circ m_2 \circ \dots \circ m_n \circ s)\\
%     f(m_1 \circ m_2 \circ \dots \circ m_n \circ m_{n+1} \circ s) = & \widehat{m_{n+1}}(s) + \widehat{m_1}(s) + \widehat{m_2}(s) + \dots + \widehat{m_n}(s) + f(s)
% \end{align*}

% Logo, por indução finita temos que $m_i \parallel m_j, \forall m_i, m_j \in M' \implies f(s') = f(s) + \sum_i^n{\widehat{m_i}(s)}$.

\end{proof}
