\section{Problemas de Decisão} \label{sec:problemaDecisao}

Um problema $\Pi$ é dito um problema de decisão quando seu conjunto solução é composto apenas pelos elementos \textit{Sim} e \textit{Não}, ou seja:
\begin{equation} \label{eq:problemaDecisao}
\begin{split}
\Pi: D \rightarrow Im  \\
Im = \{ Sim, N\widetilde{a}o \}
\end{split}
\end{equation}

São exemplos de problemas de decisão:
\begin{itemize}
    \item Seja $x \in C$, sendo $x$ um número pertencente ao conjunto $C$, $x$ é o menor número deste conjunto?
    \item Seja um grafo $G(V,A)$ denotado pelos arestas $A$ e vértices $V$, existe um caminho do vértice $x$ para o vértice $y$ com custo menor que $c$?
\end{itemize}
