\section{Ótimo global vs Ótimo local} \label{sec:otimoLocalGlobal}

O ótimo global e ótimo local são exemplificados na Figura~\ref{fig:espacoDeBusca}.
Uma solução $s^* \in S$ é dita \textbf{ótimo global} para um problema $\Pi$ quando não existe outra solução viável $s'$ com melhor valor de função objetivo, formalmente temos que $s^*$ é ótimo global quando:
\begin{itemize}
    \item $\forall s' \in S \mid f(s') \le f(s^*), s' \neq s^* $ para um problema de maximização;
    \item $\forall s' \in S \mid f(s') \ge f(s^*), s' \neq s^* $ para um problema de minimização.
\end{itemize}

Considere uma busca local de \textit{Best Improvement} $H$ para o problema $\Pi$ sobre a estrutura de vizinhança $N^x$, após aplicar $H$ a uma solução inicial $s^0 \in S$ é obtido um conjunto de soluções $N^x(s^0)$ vizinhas, assim o \textbf{ótimo local} (\textit{mínimo local}) segundo a vizinhança $N^x$ para a solução $s^0$ é dado por:
\begin{itemize}
    \item $s'' \in N^x(s^0) \mid f(s'') < f(s'), \forall s' \in N^x(s^0), s'' \neq s'$, para um problema de minimização;
    \item $s'' \in N^x(s^0) \mid f(s'') > f(s'), \forall s' \in N^x(s^0), s'' \neq s'$, para um problema de maximização.
\end{itemize}

Em linhas gerais um \textbf{ótimo local} é a solução com melhor valor de função objetivo para um contexto local, seja uma vizinhança ou o conjunto imagem de uma heurística.
