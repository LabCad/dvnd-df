\section{Busca local} \label{sec:buscaLocal}

\subsection{Vizinhança} \label{subsec:vizinhanca}

Seja $S$ o espaço de soluções de um problema de otimização $\Pi$, e $s \in S$ uma solução qualquer, considere a função objetivo $f: S \rightarrow \mathbb{R}$ que atribui um valor para cada solução.
Denotamos então por $N^x(s)$ o conjunto de soluções vizinhas de $s$ para a vizinhança $x$ com $N^x(s) \subset S$, em que as soluções dessa vizinhança podem ser obtidas de $s$ a partir da aplicação de determinadas operações.
Uma solução $s'$ é vizinha de $s$ (isto é $s' \in N^x(s)$) segundo uma vizinhança $s$ se $s'$ é alcançável a partir de $s$ fazendo uso de uma pequena perturbação nesta última.

\subsection{Movimento} \label{subsec:movimento}

% Seja $S$ o espaço de soluções de um problema de otimização $\Pi$, e $s \in S$ uma solução qualquer, considere a função objetivo $f: S \rightarrow \mathbb{R}$ que atribui um valor para cada solução.
Considere $m^x: S \rightarrow S$ a função que representa um movimento que leva uma solução $s$ a uma solução $s'$ ou de forma equivalente $s' = m^x \circ s$.
Designamos então o movimento $m^x \in M^x$ no conjunto de movimentos da vizinhança $x$.

Adicionalmente podemos definir o custo do movimento $m^x(s)$ em relação a solução $s$ como sendo a diferença entre o valor da solução $s'$ gerada ao aplicar este movimento à solução $s$ e o valor da solução $s$, conforme pode ser visto na Equação~\ref{eq:movimentoCusto}, iremos omitir a função $f$ e a vizinhança $x$ quando estiverem claros no contexto. Formalmente, a notação circunflexo representa a função $\widehat{m}: S \rightarrow \mathbb{R}$.
\begin{equation} \label{eq:movimentoCusto}
\widehat{m^x_f}(s) = f(s') - f(s)
\end{equation}

\subsubsection{Movimentos livres de contexto} \label{subsubsec:movimentosLivresDeContexto}

Um movimento $m$ é livre de contexto se ele sempre pode ser aplicado a uma solução $s$ sem gerar uma solução inválida.

\begin{equation} \label{eq:movimentoLivreDeContexto}
m \ \textrm{é livre de contexto} \  \iff m \circ s \in S, \forall s \in S
\end{equation}

Quando uma classe de movimentos é livre de contexto para um determinado problema então se pode dizer que a função a representar este movimento é fechada para o conjunto de soluções $S$.

A caracterização de um movimento como livre de contexto depende de suas restrições e da sua representação.
Desta forma, para o Problema do Caixeiro Viajante em que o grafo com as distâncias entre as cidades é completo, o movimento \textit{Swap} será livre de contexto contudo num grafo incompleto uma aplicação do \textit{Swap} pode gerar uma solução inviável pois pode não existir um determinado percurso após a alteração na solução.

\subsubsection{Movimentos parcialmente independentes} \label{subsubsec:movimentosParcialmenteIndependentes}

Movimentos parcialmente independentes são aqueles que podem ser aplicados simultaneamente a uma solução sem que o valor da solução obtida após a aplicação dos movimentos seja alterado, a independência parcial de movimentos pode ser definida para movimentos de vizinhanças diferentes.
Formalmente temos que dois movimentos $m_1$ e $m_2$ são parcialmente independentes se e somente se:
\begin{equation}
% \begin{align}
m_1 \textrm{ parcialmente independente } m_2 \iff f(m_1 \circ m_2 \circ s) = f(m_2 \circ m_1 \circ s) \label{eq:movimentosParcialmenteIndependentes}
% \end{align}
\end{equation}

\subsubsection{Movimentos independentes} \label{subsubsec:movimentosIndependentes}

Movimentos independentes são aqueles que a aplicação de um não altera o valor do outro, o valor do movimento $m_2$ aplicado à solução $m_1 \circ s$ é igual ao valor deste aplicado à solução $s$, a independência de movimentos pode ser definida para movimentos de vizinhanças diferentes.
Formalmente temos que dois movimentos $m_1$ e $m_2$ são independentes $m_1 \parallel m_2$ se e somente se:
\begin{equation}
% \begin{align}
m_1 \parallel m_2 \iff \widehat{m_1}(m_2 \circ s) = \widehat{m_2}(s) \land \widehat{m_2}(m_1 \circ s) = \widehat{m_1}(s)
% \end{align}
\end{equation}

\begin{theorem}%[Teorema da independência de movimentos]\label{teo:independenciaMovimentos}
Dois movimentos $m_1$ e $m_s$ são independentes então também são parcialmente independentes.
\begin{equation}
m_1 \parallel m_2 \implies m_1 \textrm{ parcialmente independente } m_2
\end{equation}

\begin{proof}
    Suponhamos que $m_1$ e $m_2$ sejam independentes mas não parcialmente independentes então $\widehat{m_1}(m_2 \circ s) = \widehat{m_2}(s) \land \widehat{m_2}(m_1 \circ s) = \widehat{m_1}(s)$ mas $f(m_1 \circ m_2 \circ s) \neq f(m_1 \circ m_1 \circ s)$
    \begin{align*}
        f(m_1 \circ m_2 \circ s) \neq & f(m_2 \circ m_1 \circ s) & \textrm{ De }(\ref{eq:movimentoCusto}) \\
        \widehat{m_1}(m_2 \circ s) + f(m_2 \circ s) \neq & \widehat{m_2}(m_1 \circ s) + f(m_1 \circ s) & \textrm{ De }\ (\ref{eq:movimentoCusto}) \\
        \widehat{m_1}(m_2 \circ s) + \widehat{m_2} + f(s) \neq & \widehat{m_2}(m_1 \circ s) + \widehat{m_1} + f(s) & \textrm{ Como } m_1 \parallel m_2 \\
        \widehat{m_1}(s) + \widehat{m_2} + f(s) \neq & \widehat{m_2}(s) + \widehat{m_1} + f(s) & \textrm{ Logo uma contradição } \bot
    \end{align*}
    
    Assim, por \textit{reductio ad absurdum}, temos que se $m_1$ e $m_2$ são independentes então também são parcialmente independentes.
\end{proof}
\end{theorem}

Outra propriedade interessante pode ser vista a seguir:
\begin{theorem}[Teorema da independência dos movimentos dois a dois]\label{teo:independenciaMovimentos2a2}
Se $m_1 \parallel m_2$ então o valor da solução gerada pela aplicação destes movimentos será igual ao valor anterior da solução somado ao valor dos movimentos.
\begin{equation}
\label{eq:movimentoCustoSomaDois}
m_1 \parallel m_2 \implies f(m_1 \circ m_2 \circ s) = \widehat{m_1}(s) + \widehat{m_2}(s) + f(s)
\end{equation}
\begin{proof}
    Suponhamos que $m_1$ e $m_2$ sejam independentes mas $f(m_1 \circ m_2 \circ s) \neq \widehat{m_1}(s) + \widehat{m_2}(s) + f(s)$.
    \begin{align*}
        f(m_1 \circ m_2 \circ s) \neq & \widehat{m_1}(s) + \widehat{m_2}(s) + f(s) & \textrm{ De } (\ref{eq:movimentoCusto}) \\
        \widehat{m_1}(m_2 \circ s) + f(m_2 \circ s) \neq & \widehat{m_1}(s) + \widehat{m_2}(s) + f(s) & \textrm{ De } (\ref{eq:movimentoCusto}) \\
        \widehat{m_1}(m_2 \circ s) + \widehat{m_2}(s) + f(s) \neq & \widehat{m_1}(s) + \widehat{m_2}(s) + f(s) & \textrm{ Como } m_1 \parallel m_2 \\
        \widehat{m_1}(s) + \widehat{m_2}(s) + f(s) \neq & \widehat{m_1}(s) + \widehat{m_2}(s) + f(s) & \textrm{ Logo uma contradição } \bot
    \end{align*}
    
    Assim, por \textit{reductio ad absurdum}, temos que se $m_1$ e $m_2$ são independentes então $f(m_1 \circ m_2 \circ s) = \widehat{m_1}(s) + \widehat{m_2}(s) + f(s)$.
\end{proof}
\end{theorem}

\subsubsection{Movimentos estritamente independentes} \label{subsubsec:movimentosEstritamenteIndependentes}

Dois movimentos $m_1$ e $m_2$ são estritamente independentes quando são independentes e podem ser aplicados simultaneamente a uma solução sem que a alteração feita por um gere algum conflito na causada pelo outro, a independência de movimentos pode ser definida para movimentos de vizinhanças diferentes.
Formalmente temos que dois movimentos $m_1$ e $m_2$ são estritamente independentes $m_1 \parallel_e m_2$ se e somente se:
\begin{equation}  \label{eq:movimentosIndependentes}
m_1 \parallel_e m_2 \iff m_1(m_2(s)) = m_2(m_1(s)) = m_1 \circ m_2 \circ s = m_2 \circ m_1 \circ s
\end{equation}

Pela Equação~\ref{eq:movimentosParcialmenteIndependentes} e pela definição de custo do movimento (Equação~\ref{eq:movimentoCusto}) podemos ver que um movimento estritamente independente também será um movimento independente.

A mesma ideia pode ser aplicada para um conjunto de movimentos $M = \{ m_1, m_2, m_3, ...\}$, são ditos estritamente independentes se para uma solução $s$ qualquer e para todo subconjunto não-vazio $M' = \{ m_1, m_2, m_3, \dots, m_k \} \subseteq M$ temos $m_1 \circ m_2 \circ m_3 \circ ...\circ m_k \circ s = m_2 \circ m_1 \circ m_3 \circ ...\circ m_k \circ s$ para qualquer permutação dos movimentos em $M'$.
% $\widehat{m_1}(s) + \widehat{m_2}(s) + \widehat{m_3}(s) + \dots + \widehat{m_k}(s) = \widehat{m_1}(m_2 \circ m_3 \circ ...\circ m_k \circ s)$. 

Pela Equação~\ref{eq:movimentosIndependentes} pode-se perceber que movimentos estritamente independentes são operações comutativas, pela própria definição.


Assim podemos estender a definição da vizinhança $x$ para:
\begin{equation}  \label{eq:vizinhanca}
N(s) = \{ m_i \circ s \mid \forall m_i \in M \}
\end{equation}

Cabe aqui destacar que a independência de movimentos é uma relação dada dois a dois entre os movimentos, logo não existe transitividade na relação de independência de destes, ou seja, se temos dois movimentos independentes $m_1 \parallel m_2$ e outro movimento $m_3$ tal que $m_3 \parallel m_2$ então \textbf{não} implica que $m_1 \parallel m_3$.
Pode existir um conflito, logo os movimentos $m_1 \nparallel m_3$ ou seja, seriam conflitantes.
Assim em termos de conflitos entre movimentos podemos escrever:
\begin{equation}
m_1 \parallel m_2 \land m_2 \parallel m_3 \centernot\implies m_1 \parallel m_3
\end{equation}

Tenhamos como exemplo o caso a seguir para a vizinhança de Swap, sendo eles $Swap(2,3), Swap(3,6), Swap(4,5)$, neste caso podemos ver que $Swap(2,3) \parallel Swap(4,5)$ e que $Swap(4,5) \parallel Swap(3,6)$ contudo $Swap(2,3) \nparallel Swap(3,6)$.

Para fins dessa dissertação, por questão de simplificação de notação, deste ponto em diante as referências a \textit{movimentos independentes} estão referenciando \textit{movimentos estritamente independentes}.

% Estou achadno a prova do teorema muito fraca
% % Igor coloquei isso aqui como um Teorema, não sei se é muito ambicioso da minha parte
% 
\begin{theorem}[Teorema da independência de movimentos]\label{teo:independenciaMovimentos}
Seja um conjunto $M'$ com movimentos livres de contexto em que todos os movimentos são independentes entre si dois a dois, ou seja $m_i \parallel m_j \mid \forall m_i, m_j \in M'$, com $m_i \ne m_j$.
E seja $s'$ a solução dada pela aplicação de todos os movimentos à solução inicial $s$, logo o valor da solução resultante $s'$ será dado pelo somatório do valor de cada movimento. Assim, pode-se escrever:

% \begin{align*} \label{eq:teo:independenciaMovimentos}
% m_1 \parallel m_2 \implies& s' = m_1 \circ m_2 \circ s \\
% m_1 \parallel m_2 \implies& f(s') = \widehat{m_1}(s) + \widehat{m_2}(s) + f(s) \\
% \end{align*}

% \begin{align*}
\begin{equation}
    \label{eq:teo:independenciaMovimentos}
    \begin{split}
        m_i \parallel m_j, \forall m_i, m_j \in M' \implies& s' = m_1 \circ m_2 \circ \dots \circ m_n \circ s \\
        m_i \parallel m_j, \forall m_i, m_j \in M' \implies& f(s') = f(s) + \widehat{m_1}(s) + \widehat{m_2}(s)+ \widehat{m_3}(s) + \dots + \widehat{m_n}(s) \\
        m_i \parallel m_j, \forall m_i, m_j \in M' \implies& f(s') = f(s) + \sum_i^n{\widehat{m_i}(s)}
    \end{split}
\end{equation}
% \end{align*}
\end{theorem}

\begin{proof}\label{proof:independenciaMovimentos}
% ---
% Prova por contradição
% ---
Suponhamos um conjunto $M = \{ m_1, m_2, \dots, m_n \}$ com $n$ movimentos independentes dois a dois mas $f(m_1 \circ m_2 \circ \dots \circ m_n \circ s) \neq f(s) + \widehat{m_1}(s) + \widehat{m_2}(s)+ \widehat{m_3}(s) + \dots + \widehat{m_n}(s)$.

\begin{align*}
    f(m_1 \circ m_2 \circ \dots \circ m_n \circ s) \neq f(s) + \widehat{m_1}(s) + \widehat{m_2}(s) + \dots + \widehat{m_n}(s) & \\
    \widehat{m_1}(m_2 \circ \dots \circ m_n \circ s) + f(m_2 \circ \dots \circ m_n \circ s) \neq f(s) + \widehat{m_1}(s) + \widehat{m_2}(s) + \dots + \widehat{m_n}(s) & \ [1] \\
    \widehat{m_1}(s) + f(m_2 \circ \dots \circ m_n \circ s) \neq f(s) + \widehat{m_1}(s) + \widehat{m_2}(s) + \dots + \widehat{m_n}(s) & \\
    \widehat{m_1}(s) + \widehat{m_2}(m_3 \circ \dots \circ m_n \circ s) + f(m_3 \circ \dots \circ m_n \circ s) \neq f(s) + \widehat{m_1}(s) + \widehat{m_2}(s) + \dots + \widehat{m_n}(s) & \ [2] \\
    \widehat{m_1}(s) + \widehat{m_2}(s) + f(m_3 \circ \dots \circ m_n \circ s) \neq f(s) + \widehat{m_1}(s) + \widehat{m_2}(s) + \dots + \widehat{m_n}(s) & \\
    \vdots & \\
    \widehat{m_1}(s) + \widehat{m_2}(s) + \dots + \widehat{m_n}(s) + f(s) \neq f(s) + \widehat{m_1}(s) + \widehat{m_2}(s) + \dots + \widehat{m_n}(s) & \ [3] \\
\end{align*}

Os passos acima se dão conforme:
\begin{align*}
[1] \ & m_1 \parallel m_i \mid \forall m_i \in M \\
[2] \ & m_2 \parallel m_i \mid \forall m_i \in M \\
[3] \ & \textrm{Logo temos uma contradição} \bot
\end{align*}

Desta forma, por \textit{reductio ad absurdum} demonstramos que se os movimentos são independentes então o valor da solução resultante se dará pelo somatório da solução anterior com o valor dos movimentos.

% ---
% Indução
% ---
% Pela Equação~\ref{eq:movimentoCustoSomaDois} temos que $f(m_1 \circ m_2 \circ s) = \widehat{m_1}(s) + \widehat{m_2}(s) + f(s)$, supondo por hipótese de indução que $f(m_1 \circ m_2 \circ \dots \circ m_n \circ s) = f(s) + \widehat{m_1}(s) + \widehat{m_2}(s) + \dots + \widehat{m_n}(s)$

% \begin{align*}
%     f(m_1 \circ m_2 \circ \dots \circ m_n \circ m_{n+1} \circ s) = & f(m_{n+1} \circ m_1 \circ m_2 \circ \dots \circ m_n \circ s) \\
%     f(m_1 \circ m_2 \circ \dots \circ m_n \circ m_{n+1} \circ s) = & \widehat{m_{n+1}}(m_1 \circ m_2 \circ \dots \circ m_n \circ s) + f(m_1 \circ m_2 \circ \dots \circ m_n \circ s) \\
% \end{align*}

% % \widehat{m_1}(s) + \widehat{m_2}(s) + \dots + \widehat{m_n}(s) + f(s)
% Fazendo $s_2 = m_2 \circ \dots \circ m_n \circ s$.
% \begin{align*}
%     f(m_1 \circ m_2 \circ \dots \circ m_n \circ m_{n+1} \circ s) = & \widehat{m_{n+1}}(m_1 s_2) + f(m_1 \circ m_2 \circ \dots \circ m_n \circ s) \\
%     f(m_1 \circ m_2 \circ \dots \circ m_n \circ m_{n+1} \circ s) = & \widehat{m_{n+1}}(s_2) + f(m_1 \circ m_2 \circ \dots \circ m_n \circ s)
% \end{align*}

% Pois $m_{n+1} \parallel m_1$, fazendo agora $s_3 = m_3 \circ \dots \circ m_n \circ s$.
% \begin{align*}
%     f(m_1 \circ m_2 \circ \dots \circ m_n \circ m_{n+1} \circ s) = & \widehat{m_{n+1}}(m_2 \circ \dots \circ m_n \circ s) + f(m_1 \circ m_2 \circ \dots \circ m_n \circ s)\\
%     f(m_1 \circ m_2 \circ \dots \circ m_n \circ m_{n+1} \circ s) = & \widehat{m_{n+1}}(m_2 \circ s_3) + f(m_1 \circ m_2 \circ \dots \circ m_n \circ s)\\
%     f(m_1 \circ m_2 \circ \dots \circ m_n \circ m_{n+1} \circ s) = & \widehat{m_{n+1}}(s_3) + f(m_1 \circ m_2 \circ \dots \circ m_n \circ s)\\
%     \vdots \\
%     f(m_1 \circ m_2 \circ \dots \circ m_n \circ m_{n+1} \circ s) = & \widehat{m_{n+1}}(s) + f(m_1 \circ m_2 \circ \dots \circ m_n \circ s)\\
%     f(m_1 \circ m_2 \circ \dots \circ m_n \circ m_{n+1} \circ s) = & \widehat{m_{n+1}}(s) + \widehat{m_1}(s) + \widehat{m_2}(s) + \dots + \widehat{m_n}(s) + f(s)
% \end{align*}

% Logo, por indução finita temos que $m_i \parallel m_j, \forall m_i, m_j \in M' \implies f(s') = f(s) + \sum_i^n{\widehat{m_i}(s)}$.

\end{proof}


\subsection{First Improvement vs Best Improvement} \label{subsec:firstBestImprovement}

As estratégias \textit{First Improvement} (Primeira melhora) e \textit{Best Improvement} (Melhor melhora) recebem como parâmetro a solução da iteração corrente para gerar seus vizinhos e escolhem uma solução a ser retornada conforme um critério específico, a saber, a primeira solução a melhorar a atual e a melhor solução encontrada na vizinhança, respectivamente.

\begin{algorithm}[htpb]
\caption{First Improvement para um problema de minimização}
\label{alg:firstImprovement}
\begin{algorithmic}[1]
    \Function{FirstImprovement}{Solução: $s$, Operador de vizinhança: $x$}
        % \For{$s' \in N^x(s)$} \Comment{Para cada solução $s'$ vizinha de $s$}
        %     \If{$f(s') < f(s)$} \Comment{Se a solução for melhor que a atual}
        %         \Return{$s'$}
        %     \EndIf
        % \EndFor
        \For{$m_i \in M$} \Comment{Para cada movimento $m_i \in M$}
            \If{$\widehat{m_i}(s) < 0$} \Comment{Se a solução for melhor que a atual}
                \Return{$m_i \circ s$}
            \EndIf
        \EndFor
        \Return{$s$} \Comment{Caso não consiga melhorar retorna a própria solução}
    \EndFunction
\end{algorithmic}
\end{algorithm}

Podemos ver o pseudocódigo do \textit{First Improvement} no Algoritmo~\ref{alg:firstImprovement} que consiste de enumerar os vizinhos até encontrar o primeiro que seja melhor que a solução atual, este então é retornado como resposta do método.
O método de \textit{Best Improvement} (Algoritmo~\ref{alg:bestImprovement}) consiste em enumerar toda a vizinhança guardando a informação do melhor encontrado até o momento, e então retornar o melhor resultado encontrado.

\begin{algorithm}[htpb]
\caption{Best Improvement para um problema de minimização}
\label{alg:bestImprovement}
\begin{algorithmic}[1]
    \Function{BestImprovement}{Solução: $s$, Operador de vizinhança: $x$}
        % \Let{$s^{best}$}{$s$} \Comment{Melhor solução encontrada}
        % \For{$s' \in N^x(s)$} \Comment{Para cada solução $s'$ vizinha de $s$}
        %     \If{$f(s') < f(s)$} \Comment{Se a solução for melhor que a atual altera a melhor solução encontrada}
        %         \Let{$s^{best}$}{$s'$}
        %     \EndIf
        % \EndFor
        \Let{$s^{best}$}{$s$} \Comment{Melhor solução encontrada}
        \For{$m_i \in M$} \Comment{Para cada movimento $m_i \in M$}
            \If{$\widehat{m_i}(s) < f(s^{best}) - f(s)$} \Comment{Se a solução for melhor que a atual altera a melhor solução encontrada}
                \Let{$s^{best}$}{$m_i \circ s$}
            \EndIf
        \EndFor
        \Return{$s^{best}$} \Comment{Retorna a melhor solução encontrada}
    \EndFunction
\end{algorithmic}
\end{algorithm}

O \textit{First Improvement} pode ser uma opção ao método de \textit{Best Improvement} quando a enumeração de toda a vizinhança é uma atividade muito custosa.
% Posso afirmar isso?
Embora não haja um paralelo para a definição matemática formal da solução $s'$ retornada pelo \textit{First Improvement} esta pode ser definida para o \textit{Best Improvement} de maneira simples por $s' \in N^x(s) \mid f(s') < f(s), \forall s \in N^x(s)$, o que, como veremos a seguir na seção~\ref{sec:otimoLocalGlobal}, corresponde ao ótimo local para a solução $s$ segundo a vizinhança $x$.
Em termos de movimento temos $s' = m' \circ$ com $\widehat{m'}(s) < \widehat{m_i}(s) \mid \forall m_i \in M$.

\subsection{Random Selection}

Nesta estratégia \textit{Random Selection} (Escolha Aleatória) é selecionada uma solução aleatoriamente entre aquelas que melhoram a solução atual.

\begin{algorithm}[htpb]
\caption{Random Selection para um problema de minimização}
\label{alg:randomSelection}
\begin{algorithmic}[1]
    \Function{RandomSelection}{Solução: $s$, Operador de vizinhança: $x$}
        \Let{$S_{imp}$}{$\emptyset$} \Comment{Conjunto com soluções de melhora}
        \For{$m_i \in M$} \Comment{Para cada movimento $m_i \in M$}
            \If{$\widehat{m_i}(s) < 0$} \Comment{Se a solução for melhor que a atual}
                \Let{$S_{imp}$}{$S_{imp} \cup \{ m_i \circ s\}$} \label{alg:randomSelection:salvaMelhora} \Comment{Adiciona ao conjunto de soluções de melhora}
            \EndIf
        \EndFor
        \Return{$Any(S_{imp})$} \Comment{Retorna uma das soluções de melhora}
    \EndFunction
\end{algorithmic}
\end{algorithm}

A estratégia \textit{Random Selection} (mostrada no Algoritmo~\ref{alg:randomSelection}) navega pelas soluções e na mantém as melhores soluções que melhoram a solução atual, conforme linha~\ref{alg:randomSelection:salvaMelhora}, para ao final retornam uma deste grupo.

\subsection{Multi Improvement}

Uma alternativa ao \emph{Best Improvement}, \emph{First Improvement} e ao \emph{Random Selection} é o \emph{Multi Improvement}~\cite{rios2015}.
A ideia é combinar um conjunto de movimentos independentes e executá-los simultaneamente sobre a solução de entrada.
Note que, embora consista na aplicação de diversos movimentos, somente uma única solução vizinha é gerada.
O \emph{Multi Improvement} pode ser utilizado em qualquer contexto que o \emph{Best Improvement} ou \emph{First Improvement} se encaixe (etapa de Exploração de Vizinhança ou {\it Neighborhood Exploration}), porém caso só exista um único movimento independente na vizinhança, ele terá comportamento equivalente ao Best Improvement.
Assim a solução $s'$ retornada por uma iteração do \emph{Multi Improvement} após ser aplicado a uma solução $s$ é dada por $s' = m_1 \circ m_2 \circ \dots m_k \circ s$ com os movimentos independentes $\{ m_1, m_2, \dots, m_k \} \subset M$.

O \emph{Multi Improvement} se encaixa particularmente bem com o conceito de \emph{SIMD} (\emph{Single Instruction Multiple Data}), presente nas GPUs, sendo sua complexidade similar ao \emph{Best Improvement} (todos movimentos da vizinhança são enumerados), seguido de uma etapa de junção (ou {\it merge}) dos movimentos independentes.
Podem existir cenários em que o \emph{Best Improvement} seja mais eficiente (com poucos movimentos independentes), embora já tenha sido demonstrado na literatura que mesmo casos com apenas dois movimentos independentes acabam mais promissores no \emph{Multi Improvement} do que no \emph{Best Improvement}. % Não seria importante colocar a referência disso?

\subsection{Passo iterativo} \label{subsec:passoIterativo}
% Posso fazer essa definição que estou fazendo aqui?
% Pensei em fazer isso para facilitar explicar algumas coisas mais pra frente

Em geral, um algoritmo de busca local é um processo iterativo pesado que tem como objetivo encontrar uma solução melhor que a atual dentro de um espaço de busca.
A solução recebida como entrada pode ser aleatória ou advinda de alguma heurística construtiva, a intenção do processo é aprimorar o resultado encontrado.

Cada iteração da busca local tenta encontrar a melhor solução mediante alguma alteração na solução atual, então o processo se repete na solução gerada até que nenhuma melhora seja possível.

\begin{algorithm}[htpb]
\caption{Busca local definida de forma genérica}
\label{alg:localSearch}
\begin{algorithmic}[1]
    \Function{LocalSearch}{Solução: $s$}
        \While{$f(Alterar(s)) < f(s)$} \Comment{Cada iteração corresponde a um passo iterativo da busca local}
            \Let{s}{$Alterar(s)$}
        \EndWhile
        \Return{$s$} \Comment{Retorna a melhor solução encontrada}
    \EndFunction
\end{algorithmic}
\end{algorithm}

Supondo que $Alterar(s)$ (apresentado do Algoritmo~\ref{alg:localSearch}) retorna uma solução melhor que a atual segundo alguma alteração, convencionemos então chamar de \textbf{passo iterativo} cada iteração da busca local em que o processo obtém uma solução melhor que a atual e salva o melhor resultado encontrado até o momento.
Assim para uma solução $s$ o passo iterativo é dado pela Equação~\ref{eq:passoIterativo}.

\begin{equation} \label{eq:passoIterativo}
\rho(s) = min(s, Alterar(s)) \implies f(\rho(s)) \le f(s)
\end{equation}
