\newcommand{\tabelaEstatisticasGeral}[6]{
% #1 Descrição, #2 label, #3 {dvnd, rvnd}, #4 {DVND, RVND}, #6 DD/DC, #6 Conteúdo
\begin{table}[ht]
    \centering
    \begin{tabular}{c|ccc|cc|ccc|cc|c}
        \hline \hline
        \# & Tipo & $m$ & $n$ & $min$ & $max$ & 1Q & 2Q & 3Q & $\overline{x}$ & $\sigma$ & $p-valor$ \\ \hline
        #6
    \end{tabular}
    \caption{#1 #4 #5
        Instância (\#), tipo de implementação (Tipo), número de máquinas ($m$), tamanho da instância ($n$), valor mínimo ($min$), máximo ($max$), primeiro quartil (1Q), mediana (2Q), terceiro quartil (3Q), média ($\overline{x}$), desvio padrão ($\sigma$) e p-valor para o teste de Wilcox entre as versões (valores em negrito quando $p-valor > 0.05$).
    }
    \label{tab:#3DcDd#2}
\end{table}
}

\newcommand{\tabelaEstatisticas}[5]{
    \tabelaEstatisticasGeral{#1}{#2}{#3}{#4}{na implementação clássica (DC) e a proposta de implementação usando dataflow (DD).}{#5}
}

\newcommand{\subFig}[6]{
%     \subfloat[][Instância #4, $n=#5$]{
%         \includegraphics[width=.5\linewidth]{figuras/#1/#2/#6/#1_#6100sol_#3_in#4.png}
% 		\label{fig:#1_#2_#3_in#4}
%     }
%     \subfigure[Instância #4, $n=#5$]{%
%         \includegraphics[width=.5\linewidth]{figuras/#1/#2/#6/#1_#6100sol_#3_in#4.png}
% 		\label{fig:#1_#2_#3_in#4}
%     }
	\begin{subfigure}[t]{0.475\textwidth} % dvnd_box100sol_imp_in0
        \includegraphics[width=1\linewidth]{figuras/#1/#2/#6/#1_#6100sol_#3_in#4.png}
        \caption{Instância #4, $n=#5$}
        \label{fig:#1_#2_#3_in#4}
    \end{subfigure}
}

% #1 {dvnd, rvnd, gdvnd}, #2 {sog_mog, dc_dd}, #3 {time, imp}, #4 {box, scatter}, #5 {Tempo do DVND...}
\newcommand{\multiFigureInstanciasGeral}[5]{
	\begin{figure}[ht]
		\centering
		\subFig{#1}{#2}{#3}{0}{52}{#4}
		~
		\subFig{#1}{#2}{#3}{1}{100}{#4}
		
		\subFig{#1}{#2}{#3}{2}{226}{#4}
		~
		\subFig{#1}{#2}{#3}{3}{318}{#4}

        \includegraphics[width=.9\linewidth]{figuras/#1/#2/#4/#1_#4100sol_#3_legenda.png}
		\caption{#5 Instâncias 0 a 3.}
		\label{fig:#1_#2_#3_in0_4}
	\end{figure}
	
	\begin{figure}[ht]
		\centering
		\subFig{#1}{#2}{#3}{4}{501}{#4}
		~
		\subFig{#1}{#2}{#3}{5}{657}{#4}
		
		\subFig{#1}{#2}{#3}{6}{783}{#4}
		~
		\subFig{#1}{#2}{#3}{7}{1001}{#4}

		\includegraphics[width=.9\linewidth]{figuras/#1/#2/#4/#1_#4100sol_#3_legenda.png}
		\caption{#5 Instâncias 4 a 7.}
		\label{fig:#1_#2_#3_in5_7}
	\end{figure}
}

% #1 {dvnd, rvnd, gdvnd}, #2 {sog_mog, dc_dd}, #3 {time, imp}, #4 {Tempo do DVND...}
\newcommand{\multiFigureInstancias}[4]{
    \multiFigureInstanciasGeral{#1}{#2}{#3}{box}{#4}
}
